\section{On-the-job training}\label{sec:ch13sec3}

As is clear from Figure~\ref{fig:earningseducationlevel}, earnings are
raised both by education and experience. Learning on the job is central to
the age-earnings profiles of the better educated, and is less important for
those with lower levels of education. \terminology{On-the-job training}
improves human capital through work experience. If on-the-job training
increases worker productivity, who should pay for this learning -- firms or
workers? To understand who should pay, we distinguish between two kinds of
skills: \terminology{Firm-specific skills} that raise a worker's
productivity in a particular firm, and \terminology{general skills} that
enhance productivity in many jobs or firms.

Firm-specific HK could involve knowing how particular components of a
somewhat unique production structure functions, whereas general human
capital might involve an understanding of engineering or architectural
principles that can be applied universally. As for who should pay for the
accumulation of skills: An employer should be willing to undertake most of
the cost of firm-specific skills, because they are of less value to the
worker should she go elsewhere. Firms offering general or transferable
training try to pass the cost on to the workers, perhaps by offering a
wage-earnings profile that starts very low, but that rises over time.
Low-wage apprenticeships are examples. Hence, whether an employee is a
medical doctor in residence, a plumber in an apprenticeship or a young
lawyer in a law partnership, she `pays' for the accumulation of her portable
HK by facing a low wage when young. Workers are willing to accept such an
earnings profile because their projected future earnings will compensate for
lower initial earning.

\begin{DefBox}
	\textbf{On-the-job training} improves human capital through work experience.
	
	\textbf{Firm-specific skills} raise a worker's productivity in a particular firm.
	
	\textbf{General skills} enhance productivity in many jobs or firms.
\end{DefBox}