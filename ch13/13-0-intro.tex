\chapter{Human capital and the income distribution}\label{chap:humancapital}

\begin{topics}
	\textbf{In this chapter we will explore:}
	\begin{description}
		\item[\ref{sec:ch13sec1}] The concept of human capital
		\item[\ref{sec:ch13sec2}] Productivity and education
		\item[\ref{sec:ch13sec3}] On-the-job training
		\item[\ref{sec:ch13sec4}] Education as signalling
		\item[\ref{sec:ch13sec5}] Returns to education and education quality
		\item[\ref{sec:ch13sec6}] Discrimination
		\item[\ref{sec:ch13sec7}] The income and earnings distribution in Canada
		\item[\ref{sec:ch13sec8}] Wealth and capitalism
	\end{description}
\end{topics}

Individuals with different characteristics earn different amounts because
their productivity levels differ. While it is convenient to work with a
single marginal productivity of labour function to illustrate the
functioning of the labour market, as we did in Chapter~\ref{chap:marketlabourcapital},
for the most part wages and earnings vary by
education level and experience, and sometimes by ethnicity and gender. In
this chapter we develop an understanding of the sources of these
differentials, and how they are reflected in the distribution of income.