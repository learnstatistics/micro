\newpage
	\section*{Key Terms}
\begin{keyterms}
\textbf{Human capital} is the stock of expertise accumulated by a worker that determines future productivity and earnings.

\textbf{Age-earnings profiles} define the pattern of earnings over time for individuals with different characteristics.

\textbf{Education premium}: the difference in earnings between the more and less highly educated.

\textbf{On-the-job training} improves human capital through work experience.

\textbf{Firm-specific skills} raise a worker's productivity in a particular firm.

\textbf{General skills} enhance productivity in many jobs or firms.

\textbf{Signalling} is the decision to undertake an action in order to reveal information.

\textbf{Screening} is the process of obtaining information by observing differences in behaviour.

\textbf{Discrimination} implies an earnings differential that is attributable to a trait other than human capital.

\textbf{Lorenz curve} describes the cumulative percentage of the income distribution going to different quantiles of the population.

\textbf{Gini index}: a measure of how far the Lorenz curve lies from the line of equality. Its maximum value is one; its minimum value is zero.
\end{keyterms}