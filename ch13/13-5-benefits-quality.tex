\section{Education returns and quality}\label{sec:ch13sec5}

How can we be sure that further education really does generate the returns,
in the form of higher future incomes, to justify the investment? For many
years econometricians proposed that an extra year of schooling might offer a
return in the region of 10\% -- quite a favourable return in comparison with
what is frequently earned on physical capital. Doubters then asked if the
econometric estimation might be subject to bias -- what if the additional
earnings of those with more education are simply attributable to the fact
that it is the innately more capable individuals who both earn more and who
have more schooling? And since we cannot observe who has more innate
ability, how can we be sure that it is the education itself, rather than
just differences in ability, that generate the extra income?

This is a classical problem in inference: Does correlation imply causation?
The short answer to this question is that education economists are convinced
that the time invested in additional schooling does indeed produce
additional rewards, even if it is equally true that individuals who are
innately smarter do choose to invest in that way. Furthermore, it appears
that the returns to graduate education are higher than the returns to
undergraduate education.


What can be said of the quality of different educational systems? Are
educational institutions in different countries equally good at producing
knowledgeable students? Or, viewed another way: Has a grade nine student in
Canada the same skill set as a grade nine student in France or Hong Kong? An
answer to this question is presented in Table~\ref{table:sciencereading},
which contains results from the Program for International Student Assessment
(PISA) -- an international survey of 15-year old student abilities in
mathematics, science and literacy. This particular table presents the
results for a sample of the countries that were surveyed. The results
indicate that Canadian students perform well in all three dimensions of the
test. 


%Table 13.1
\begin{Table}{caption={Mean scores in PISA tests \label{table:sciencereading}},description={\textit{Source}: ``PISA 2012 Results in	Focus.'' \ \url{http://goo.gl/YmTeMc}},descwidth={23em}}
	\begin{tabu} to \linewidth {|X[2,l]X[1,c]X[1,c]X[1,c]|} \hline 
		\rowcolor{rowcolour}	\textbf{Country} & \textbf{Math} & \textbf{Science} & \textbf{Reading} \\ \hline
		Australia & 504 & 521 & 512 \\ 
		\rowcolor{rowcolour}	Austria & 506 & 506 & 490 \\ 
		Belgium & 515 & 505 & 509 \\ 
		\rowcolor{rowcolour}	Canada & 518 & 525 & 523 \\ 
		Denmark & 500 & 498 & 496 \\ 
		\rowcolor{rowcolour}	Finland & 519 & 545 & 524 \\ 
		France & 495 & 499 & 505 \\ 
		\rowcolor{rowcolour}	Germany & 514 & 524 & 508 \\ 
		Greece & 453 & 467 & 477 \\ 
		\rowcolor{rowcolour}	Hong Kong & 561 & 555 & 545 \\ 
		Ireland & 501 & 522 & 523 \\ 
		\rowcolor{rowcolour}	Italy & 485 & 494 & 490 \\ 
		Japan & 563 & 547 & 538 \\ 
		\rowcolor{rowcolour}	Korea & 554 & 538 & 536 \\ 
		Mexico & 413 & 415 & 424 \\ 
		\rowcolor{rowcolour}	New Zealand & 500 & 516 & 512 \\ 
		Norway & 489 & 495 & 504 \\ 
		\rowcolor{rowcolour}	Spain & 484 & 496 & 488 \\ 
		Sweden & 478 & 485 & 483 \\ 
		\rowcolor{rowcolour}	Switzerland  & 531 & 515 & 509 \\ 
		Turkey & 448 & 463 & 475 \\ 
		\rowcolor{rowcolour}	United States & 481 & 497 & 498 \\ 
		United Kingdom & 494 & 514 & 499 \\ 
		\rowcolor{rowcolour}	\textbf{OECD Average} & \textbf{494} & \textbf{501} & \textbf{496} \\ \hline  
	\end{tabu}
\end{Table}

\newhtmlpage


An interesting paradox arises at this point: If productivity growth in
Canada has lagged behind some other economies in recent decades, as many
economists believe, how can this be explained if Canada produces 
many well-educated high-skill workers? The answer may
be that there is a considerable time lag before high participation rates in
third-level education and high quality make themselves felt on the national
stage in the form of elevated productivity. The evidence suggests a good
productivity future, in so far as it depends upon human capital. 
