\section{The income distribution}\label{sec:ch13sec7}

How does all of our preceding discussion play out when it comes to the
income distribution? That is, when we examine the incomes of all individuals
or households in the economy, how equally or unequally are they distributed?

The study of inequality is a critical part of economic analysis. It
recognizes that income differences that are in some sense `too large' are
not good for society. Inordinately large differences can reflect poverty and
foster social exclusion and crime. Economic growth that is concentrated in
the hands of the few can increase social tensions, and these can have
economic as well as social or psychological costs. Crime is one reflection
of the divide between `haves' and `have-nots'. It is economically costly;
but so too is child poverty. Impoverished children rarely achieve their
social or economic potential and this is a loss both to the individual and the
economy at large.

In this section we will first describe a subset of the basic statistical
tools that economists use to measure inequality. Second, we will examine how
income inequality has evolved in recent decades. We shall see that, while
the picture is complex, market income inequality has indeed increased in
Canada. Third, we shall investigate some of the proposed reasons for the
observed increase in inequality. Finally we will examine if the government
offsets the inequality that arises from the marketplace through its taxation
and redistribution policies.

It is to be emphasized that income inequality is just one proximate measure
of the distribution of wellbeing. The extent of poverty is another such
measure. Income is not synonymous with happiness but, that being said,
income inequality can be computed reliably, and it provides a good measure
of households' control over economic resources.

\newhtmlpage

\subsection*{Theory and measurement}

Let us rank the market incomes of
all households in the economy from poor to rich, and categorize this
ordering into different quantiles or groups. With five such quantiles the
shares are called \textit{quintiles}. The richest group forms the highest
quintile, while the poorest group forms the lowest quintile. Such a
representation is given in Table~\ref{table:quintileincome}. The first
numerical column displays the income in each quintile as a percentage of
total income. If we wanted a finer breakdown, we could opt for decile (ten),
or even vintile (twenty) shares, rather than quintile shares. These data can
be graphed in a variety of ways. Since the data are in share, or percentage,
form, we can compare, in a meaningful manner, distributions from economies
that have different average income levels.

\begin{Table}{caption={Quintile shares of total family income in Canada, 2006 \label{table:quintileincome}},description={\textit{Source}: Statistics Canada, CANSIM Matrix 2020405. These combinations are represented by the circles in the figure.},descwidth={30em}}
	\begin{tabu} to \linewidth {|X[0.75,c]X[1,c]X[1,c]|} \hline 
		\rowcolor{rowcolour}	& \textbf{Quintile share of total income} & \textbf{Cumulative share} \\[0.25em]
		\textbf{First quintile}  & 4.1 & 4.1 \\
		\rowcolor{rowcolour}	\textbf{Second quintile} & 9.6 & 13.7 \\
		\textbf{Third quintile} & 15.3 & 29.0 \\
		\rowcolor{rowcolour}	\textbf{Fourth quintile} & 23.8 & 52.8 \\ 
		\textbf{Fifth quintile} & 47.2 & 100.0 \\
		\rowcolor{rowcolour}	\textbf{Total} & 100 &  \\ \hline 
	\end{tabu}
\end{Table}

% Figure 13.3 
\begin{TikzFigure}{xscale=1,yscale=1,descwidth=25em,caption={Gini index and Lorenz curve \label{fig:ginilorenzcurve}},description={The more equal are the income shares, the closer is the Lorenz curve to the diagonal line of equality. The Gini index is the ratio of the area A to the area (A+B). The Lorenz curve plots the cumulative percentage of total income against the cumulative percentage of the population.}}
\begin{axis}[
	axis line style=thick,
	every tick label/.append style={font=\footnotesize},
	every node near coord/.append style={font=\scriptsize},
	xticklabel=\pgfmathparse{100*\tick}\pgfmathprintnumber{\pgfmathresult}\,\%,
	xticklabel style={anchor=north,/pgf/number format/1000 sep=},
	scaled y ticks=false,
	x=1.25cm/0.1,
	y=0.87cm/0.1,
	yticklabel=\pgfmathparse{100*\tick}\pgfmathprintnumber{\pgfmathresult}\,\%,
	yticklabel style={/pgf/number format/fixed,/pgf/number format/1000 sep = \thinspace},
	xmin=0,xmax=1,ymin=0,ymax=1,
	xlabel={Cumulative share of population},
	ylabel={Cumulative share of total income},
]
\addplot[ultra thick,black,mark=none] coordinates { % absolute equality
	(0,0)
	(1,1)
} node [black,mynode,pos=0.6,above left] {Line of\\absolute\\equality};
\addplot[ultra thick,datasetcolourthree,mark=*] coordinates { % middle section of Lorenz curve
	(0.2,0.041)
	(0.4,0.138)
	(0.6,.294)
	(0.8,.531)
} node [black,mynode,pos=0.6,below right] {Lorenz\\curve};
\addplot[ultra thick,datasetcolourthree,mark=none] coordinates { %first section of Lorenz curve
	(0,0)
	(0.2,0.041)
};
\addplot[ultra thick,datasetcolourthree,mark=none] coordinates { %last section of Lorenz curve
	(0.8,0.531)
	(1,1)
};
\addplot[mark=none] coordinates {
	(0.7,0.1)
} node [mynode,above] {B};
\addplot[mark=none] coordinates {
	(0.5,0.3)
} node [mynode,above] {A};
\end{axis}
\end{TikzFigure}

\newhtmlpage

An interesting way of presenting these data graphically is to plot the
cumulative share of income against the cumulative share of the population.
This is given in the final column, and also presented graphically in Figure~%
\ref{fig:ginilorenzcurve}. The bottom quintile has 4.1\% of total income.
The bottom two quintiles together have $13.7\%(4.1\%+9.6\%)$, and so forth.
By joining the coordinate pairs represented by the circles, a %
\terminology{Lorenz curve} is obtained. Relative to the diagonal line it is
a measure of how unequally incomes are distributed: If everyone had the same
income, each 20\% of the population would have 20\% of total income and by
joining the points for such a distribution we would get a straight diagonal
line joining the corners of the box. In consequence, if the Lorenz curve is
further from the line of equality the distribution is less equal than if the
Lorenz curve is close to the line of equality.

\begin{DefBox}
	\textbf{Lorenz curve} describes the cumulative percentage of the income distribution going to different quantiles of the population.
\end{DefBox}

This suggests that the area A relative to the area (A + B) forms a measure
of inequality in the income distribution. This fraction obviously lies
between zero and one, and it is called the \terminology{Gini index}. A
larger value of the Gini index indicates that inequality is greater. We will
not delve into the mathematical formula underlying the Gini, but for this
set of numbers its value is 0.4.

\begin{DefBox}
	\textbf{Gini index}: a measure of how far the Lorenz curve lies from the line of equality. Its maximum value is one; its minimum value is zero.
\end{DefBox}

The Gini index is what is termed \textit{summary index} of inequality -- it
encompasses a lot of information in one number. There exist very many other
such summary statistics.

It is important to recognize that very different Gini index values emerge
for a given economy by using different income definitions of the variable
going into the calculations. For example, the quintile shares of the 
\textit{earnings of individuals} rather than the \textit{incomes of households}
could be very different. Similarly, the shares of income \textit{post tax}
and \textit{post transfers} will differ from their shares on a \textit{pre-tax}, 
\textit{pre-transfer} basis.

\newhtmlpage

% Figure 13.4
\begin{TikzFigure}{xscale=1,yscale=1,descwidth=19em,caption={Gini index Canada 1976-2010 \label{fig:giniindexcanada}},description={\textit{Source:} Statistics Canada, CANSIM Table 202-0709}}
\begin{axis}[
	axis line style=thick,
	every tick label/.append style={font=\footnotesize},
	every node near coord/.append style={font=\scriptsize},
	xticklabel style={rotate=90,anchor=east,/pgf/number format/1000 sep=},
	scaled y ticks=false,
	yticklabel style={/pgf/number format/fixed,/pgf/number format/1000 sep = \thinspace},
	ymin=0,ymax=0.5,
	x=1.35cm/5,
	y=1.35cm/0.1,
	x label style={at={(axis description cs:0.5,-0.05)},anchor=north},
	y label style={at={(axis description cs:0.05,0.5)},anchor=south},
	xlabel={Year},
	ylabel={Gini Index},
	legend entries={Market income,Post-tax disposable income},
	legend style={at={(axis cs:1982,0.1)},anchor=south west},
]
\addplot[ultra thick,black,dashed,mark=none] table { % market income
X			Y
1976	0.384
1977	0.368
1978	0.375
1979	0.365
1980	0.37
1981	0.369
1982	0.388
1983	0.403
1984	0.401
1985	0.395
1986	0.395
1987	0.392
1988	0.391
1989	0.388
1990	0.403
1991	0.422
1992	0.429
1993	0.429
1994	0.432
1995	0.43
1996	0.439
1997	0.438
1998	0.446
1999	0.437
2000	0.439
2001	0.44
2002	0.439
2003	0.437
2004	0.442
2005	0.435
2006	0.434
2007	0.434
2008	0.436
2009	0.442
2010	0.445
2011	0.436
};
\addplot[ultra thick,datasetcolourthree,mark=none] table { % after-tax income
X			Y
1976	0.3
1977	0.286
1978	0.291
1979	0.286
1980	0.286
1981	0.285
1982	0.288
1983	0.296
1984	0.293
1985	0.29
1986	0.29
1987	0.287
1988	0.282
1989	0.281
1990	0.286
1991	0.292
1992	0.291
1993	0.289
1994	0.29
1995	0.293
1996	0.301
1997	0.304
1998	0.311
1999	0.31
2000	0.317
2001	0.318
2002	0.318
2003	0.316
2004	0.322
2005	0.317
2006	0.316
2007	0.315
2008	0.318
2009	0.318
2010	0.317
2011	0.313
};
\end{axis}
\end{TikzFigure}

Figure~\ref{fig:giniindexcanada} contains Gini index values for two
different definitions of income from 1976 to 2011. The upper line
represents the Gini index values for households where the income measure is
market income; the lower line defines the Gini values when income is defined
as post-tax and post-transfer incomes. The latter income measure deducts
taxes paid and adds income such as Employment Insurance or Social Assistance
benefits. Two messages emerge from this graphic: The first is that the
distribution of market incomes displays more inequality than the
distribution of incomes after the government has intervened. In the latter
case incomes are more equally distributed than in the former. The second
message to emerge is that inequality has increased over time -- the Gini
values are larger in the later years than in the earlier years, although the
increase in market income inequality is greater than the increase in income
inequality based on a `post-government' measure of income.

\newhtmlpage

This is a very brief description of recent events. It is also possible to
analyze inequality among women and men, for example, as well as among
individuals and households. But the essential message remains clear:
Definitions are important; in particular the distinction between incomes
generated in the market place and incomes after the government has
intervened through its tax and transfer policies.

\begin{ApplicationBox}{caption={The very rich \label{app:veryrich}}}
	McMaster University Professor Michael Veall and his colleague Emmanuel Saez, from University of California, Berkeley, have examined the evolution of the top end of the Canadian earnings distribution in the twentieth century. Using individual earnings from a database built upon tax returns, they show how the share of the very top of the distribution declined in the nineteen thirties and forties, remained fairly stable in the decades following World War II, and then increased from the eighties to the present time. The increase in share is particularly strong for the top 1\% and even stronger for the top one tenth of the top 1\%. These changes are driven primarily by changes in earnings, not on stock options awarded to high-level corporate employees. The authors conclude that the change in this region of the distribution is attributable to changes in social norms. Whereas, in the nineteen eighties, it was expected that a top executive would earn perhaps a half million dollars, the `norm' has become several million dollars in the present day. Such high remuneration became a focal point of public discussion after so many banks in the United States in 2008 and 2009 required government loans and support in order to avoid collapse. It also motivated the many `occupy' movements of 2011 and 2012.

	Saez, E. and M. Veall. ``The evolution of high incomes in Canada, 1920-2000.'' Department of Economics research paper, McMaster University, March 2003.
\end{ApplicationBox}

\newhtmlpage

\subsection*{Economic forces}

The increase in inequality of earnings in the market place in Canada has
been reflected in many other developed economies -- to a greater degree in
the US and to a lesser extent in some European economies. Economists have
devoted much energy to studying why, and as a result there are several
accepted reasons.

Younger workers and those with lower skill levels have faired poorly in the
last three decades. Globalization and out-sourcing
have put pressure on low-end wages. In effect the workers in the lower tail of
the distribution are increasingly competing with workers from low-wage
less-developed economies. While this is a plausible causation, the critics
of the perspective point out that wages at the bottom have fallen not only
for those workers who compete with overseas workers in manufacturing, but
also in the domestic services sector right across the economy. Obviously the
workers at \textit{McDonalds} have not the same competition from low-wage economies
as workers who assemble toys.

A competing perspective is that it is technological change that has enabled
some workers to do better than others. In explaining why high wage workers
in many economies have seen their wages increase, whereas low-wage workers
have seen a relative decline, the technological change hypothesis proposes
that the form of recent technological change is critical: Change has been
such as to require other complementary skills and education in order to
benefit from it. For example, the introduction of computer-aided design
technology is a benefit to workers who are already skilled and earning a
high wage: \textit{Existing high skills and technological change are
	complementary}. Such technological change is therefore different from the
type underlying the production line. Automation in the early twentieth
century in Henry Ford's plants improved the wages of lower skilled workers.
But in the modern economy it is the highly skilled rather than the low
skilled that benefit most from innovation.

A third perspective is that key institutional changes manifested themselves
in the eighties and nineties, and these had independent impacts on the
distribution. In particular, changes in the extent of unionization and
changes in the minimum wage had significant impacts on earnings in the
middle and bottom of the distribution: If unionization declines or the
minimum wage fails to keep up with inflation, these workers will suffer. An
alternative `institutional' player is the government: In Canada the federal
government became slightly less supportive, or `generous', with its array of
programs that form Canada's social safety net in the nineteen nineties. This
tightening goes some way to explaining the modest inequality increase in the
post-government income distribution in Figure~\ref{fig:ginilorenzcurve} at
this time.

We conclude this overview of distributional issues by pointing out that we
have not analyzed the distribution of wealth. Wealth too represents
purchasing power, and it is wealth rather than income flows that primarily
distinguishes Warren Buffet, Mark Zuckerberg and Bill Gates from the rest of
us mortals. A detailed treatment of wealth inequality is beyond the scope of
this book. However, we describe briefly, in the final section, the recent
contritution of Thomas Piketty to the inequality debate.