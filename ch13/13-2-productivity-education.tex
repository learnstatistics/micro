\section{Productivity and education}\label{sec:ch13sec2}

Human capital is the result of past investment that raises future incomes. A
critical choice for individuals is to decide upon exactly how much
additional human capital to accumulate. The cost of investing in another
year of school is the \textit{direct cost}, such as school fees, plus the 
\textit{indirect}, or \textit{opportunity}, \textit{cost}, which can be measured by the
foregone earnings during that extra year. The benefit of the additional
investment is that the future flow of earnings is augmented. Consequently,
wage differentials should reflect different degrees of education-dependent
productivity.

\newhtmlpage

\subsection*{Age-earnings profiles}

Figure~\ref{fig:earningseducationlevel} illustrates two typical %
\terminology{age-earnings profiles} for individuals with different levels of
education. These profiles define the typical pattern of earnings over time,
and are usually derived by examining averages across individuals in surveys.
Two aspects are clear: People with more education not only earn more, but
the spread tends to grow with time. Less educated, healthy young individuals
who work hard may earn a living wage but, unlike their more educated
counterparts, they cannot look forward to a wage that rises substantially
over time. More highly-educated individuals go into jobs and occupations
that take a longer time to master: Lawyers, doctors and most professionals
not only undertake more schooling than truck drivers, they also spend many
years learning on the job, building up a clientele and accumulating
expertise.

\begin{DefBox}
	\textbf{Age-earnings profiles} define the pattern of earnings over time for individuals with different characteristics.
\end{DefBox}

% Figure 13.1
\begin{TikzFigure}{xscale=0.15,yscale=0.25,descwidth=25em,caption={Age-Earnings profiles by education level \label{fig:earningseducationlevel}},description={Individuals with a higher level of education earn more than individuals with a `standard' level of education. In addition, the differential grows over time.}}
\draw [dotted,thick]
	(18,0) node [mynode,below] {18 (High school grad)} -- +(0,25)
	(65,0) node [mynode,below] {65 (Retirement)} -- +(0,25);
\draw [datasetcolourthree,ultra thick] (18,5) -- (25,7) -- (35,8.5) -- (45,9.5) node [black,mynode,above left] {High school\\education} -- (55,10) -- (65,10.25);
\draw [datasetcolourthree!50,ultra thick] (21,9) -- (25,11.5) -- (35,15) -- (45,17) node [black,mynode,above left] {Third level\\education} -- (55,18) -- (65,18.5);
\draw [thick, -] (0,25) node [mynode1,above] {Average\\earnings\\in \$} -- (0,0) -- (70,0) node [mynode1,right] {Age};
\end{TikzFigure}



\newhtmlpage

\subsection*{The education premium}

Individuals with different education levels earn different wages. The %
\terminology{education premium} is the difference in earnings between the
more and less highly educated. Quantitatively, Professors Kelly Foley and
David Green have recently proposed that the completion of a college or trade
certification adds about 15\% to one's income, relative to an individual who
has completed high school. A Bachelor's degree brings a premium of 20-25\%,
and a graduate degree several percentage points more\footnote{Foley, K. and D. Green, 2015, 
``Why more education will not solve rising inequality (and may make it worse)'' , 
\textit{Institute for Research in Public Policy}, Montreal, Canada.}. The failure to
complete high school penalizes individuals to the extent of about 10\%.
These are average numbers, and they vary depending upon the province of
residence, time period and gender. Nonetheless the findings underline that
more human capital is associated with higher earnings. The earnings premium
depends upon both the supply and demand of high HK individuals. 
\textit{Ceteris paribus}, if high-skill workers are heavily in demand by employers,
then the premium should be greater than if lower-skill workers are more in
demand.

The distribution of earnings has become more unequal in Canada and the US in
recent decades, and one reason that has been proposed for this development
is that the modern economy demands more high-skill workers; in particular
that technological change has a bigger impact on productivity when combined
with high-skill workers than with low-skill workers. Consider Figure~\ref{fig:edskillpremium}
which contains supply and demand functions with a
twist. We imagine that there are two types of labour: One with a high level
of human capital, the other with a lower level. The vertical axis measures
the wage \textit{premium} of the high-education group (which can be measured
in dollars or percentage terms), and the horizontal axis measures the 
\textit{fraction of the total labour force that is of the high-skill type}. 
$D$ is the \textit{relative} demand for the high skill workers, in this
example for the economy as a whole. There is some degree of substitution
between high and low-skill workers in the modern economy. We do not propose
that several low-skill workers can perform the work of one neuro-surgeon;
but several individual households (low-skill) could complete their income tax
submissions in the same time as one skilled tax specialist. In this example
there is a degree of substitutability. In a production environment, a
high-skill manager, equipped with technology and capital, can perform the
tasks of several line workers.

% Figure 13.2
\begin{TikzFigure}{xscale=0.28,yscale=0.22,descwidth=25em,caption={The education/skill premium \label{fig:edskillpremium}},description={A shift in demand increases the wage premium in the short run (from $E_0$ to $E_1$) by more than in the long run (to $E_2$). In the short run, the percentage of the labour force ($S_S$) that is highly skilled is fixed. In the long run it ($S_L$) is variable and responds to the wage premium.}}
% demand lines
\draw [demandcolour,ultra thick,name path=D] (0,15) -- (15,0) node [black,mynode,above right,pos=0.95] {$D$};
\draw [demandcolour,ultra thick,name path=D1] (0,20) -- (27,0) node [black,mynode,above right,pos=0.95] {$D_1$};
% supply lines
\draw [supplycolour,ultra thick,name path=SS] (8.2353,0) -- (8.2353,24) node [black,mynode,above] {$S_S$};
\draw [supplycolour,ultra thick,name path=SL] (0,1) -- (30,22) node [black,mynode,above] {$S_L$};
% axes
\draw [thick] (0,25) node (yaxis) [mynode1,above] {Wage\\premium} -- (0,0) node [mynode,below left] {0} -- (30,0) node [mynode1,right] {Fraction of labour\\force with high skill\\(max=1)};
% intersection of supply and demand lines
\draw [name intersections={of=SS and D, by=E0},name intersections={of=SS and D1, by=E1},name intersections={of=SL and D1, by=E2}]
	[dotted,thick] (yaxis |- E0) node [mynode,left] {$W_{p0}$} -- (E0) node [mynode,right=0cm and 0.1cm] {$E_0$}
	[dotted,thick] (yaxis |- E1) node [mynode,left] {$W_{p1}$} -- (E1) node [mynode,above right] {$E_1$}
	[dotted,thick] (yaxis |- E2) node [mynode,left] {$W_{p2}$} -- (E2) node [mynode,right=0cm and 0.1cm] {$E_2$};
\end{TikzFigure}

\newhtmlpage

The demand curve $D$ defines the premium that demanders are willing to pay
to the higher skill group. The negative slope indicates that if demanders
were to employ a high proportion of skilled workers, the premium they would
be willing to pay would be less than if they demanded a smaller share of
high-skilled workers, and a larger share of lower-skilled workers. The wage
premium for high HK individuals at any given time is determined by the
intersection of supply and demand. 

\begin{DefBox}
	\textbf{Education premium}: the difference in earnings between the more and less highly educated.
\end{DefBox}

In the short run the make-up of the labour force is fixed, and this is
reflected in the vertical supply curve $S_s$. The equilibrium is at $E_0$,
and $W_{p0}$ is the premium, or excess, paid to the higher-skill worker over
the lower-skill worker. In the long run it is possible for the economy to
change the composition of its labour supply: If the wage premium increases,
more individuals will find it profitable to train as high-skill workers.
That is to say, the fraction of the total that is high-skill increases. It
follows that the long-run supply curve slopes upwards.

So what happens when there is an increase in the demand for high-skill
workers relative to low-skill workers? The demand curve shifts upward to 
$D_{1}$, and the new equilibrium is at $E_{1}$. The supply mix is fixed in
the short run, so there is an increase in the wage premium. But over time,
some individuals who might have been just indifferent between educating
themselves more and going into the workplace with lower skill levels now
find it worthwhile to pursue further education. Their higher anticipated
returns to the additional human capital they invest in now exceed the
additional costs of more schooling, whereas before the premium increase
these additional costs and benefits were in balance. In Figure~\ref{fig:edskillpremium}
the new short-run equilibrium at $E_{1}$ has a
corresponding wage premium of $W_{p1}$. In the long run, after additional
supply has reached the market, the increased premium is moderated to $W_{p2}$
at the equilibrium $E_{2}$. 

This figure displays what many economists believe has happened in North
America in recent decades: The demand for high HK individuals has increased,
and the additional supply has not been as great. Consequently the wage
premium for the high-skill workers has increased. As we describe later in
this chapter, that is not the only perspective on what has happened.

\newhtmlpage

\subsection*{Are students credit-constrained or culture-constrained?}

The foregoing analysis assumes that students and potential students make
rational decisions on the costs and benefits of further education and act
accordingly. It also assumes implicitly that individuals can borrow the
funds necessary to build their human capital: If the additional returns to
further education are worthwhile, individuals should borrow the money to
make the investment, just as entrepreneurs do with physical capital.

However, there is a key difference in the credit markets. If an entrepreneur
fails in her business venture the lender will have a claim on the physical
capital. But a bank cannot repossess a human being who drops out of school
without having accumulated the intended human capital. Accordingly, the
traditional lending institutions are frequently reluctant to lend the amount
that students might like to borrow---students are credit constrained. The
sons and daughters of affluent families therefore find it easier to attend
university, because they are more likely to have a supply of funds
domestically. Governments customarily step into the breach and supply loans
and bursaries to students who have limited resources. While funding
frequently presents an obstacle to attending a third-level institution, a
stronger determinant of attendance is the education of the parents, as
detailed in Application Box~\ref{app:parentedcanada}.

\newhtmlpage

\newpage

\begin{ApplicationBox}{caption={Parental education and  university attendance in Canada \label{app:parentedcanada}}}
	The biggest single determinant of university attendance in the modern era is parental education. A recent study* of who goes to university examined the level of parental education of young people `in transition' -- at the end of their high school -- for the years 1991 and 2000. 

	For the year 2000 they found that, if a parent had not completed high school, there was just a 12\% chance that their son would attend university and an 18\% chance that a daughter would attend.  In contrast, for parents who themselves had completed a university degree, the probability that a son would also attend university was 53\% and for a daughter 62\%. Hence, the probability of a child attending university was roughly four times higher if the parent came from the top educational category rather than the bottom category! Furthermore the authors found that this probability gap opened wider between 1991 and 2000.

	In the United States, Professor Sear Reardon of Stanford University has followed the performance of children from low-income households and compared their achievement with children from high-income households. He has found that the achievement gap between these groups of children has increased substantially over the last three decades. The reason for this growing separation is not because children from low-income households are performing worse in school, it is because high-income parents invest much more of their time and resources in educating their children, both formally in the school environment, and also in extra-school activities.

	*Finnie, R., C. Laporte and E. Lascelles. ``Family Background and Access to Post-Secondary Education: What Happened in the Nineties?'' Statistics Canada Research Paper, Catalogue number 11F0019MIE-226, 2004
	
	Reardon, Sean, ``The Great Divide'', New York Times, April 8, 2015.
\end{ApplicationBox}
