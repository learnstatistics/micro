\section{Discrimination}\label{sec:ch13sec6}

Wage differences are a natural response to differences in human capital. But
we frequently observe wage differences that might be discriminatory. For
example, women on average earn less than men with similar qualifications;
older workers may be paid less than those in their prime years; immigrants
may be paid less than native-born Canadians, and ethnic minorities may be
paid less than traditional white workers. The term %
\terminology{discrimination} describes an earnings differential that is
attributable to a trait other than human capital.

If two individuals have the same HK, in the broadest sense of having the
same capability to perform a particular task, then a wage premium paid to
one represents discrimination. Correctly measured then, the discrimination
premium between individuals from these various groups is the differential in
earnings after correcting for HK differences. Thousands of studies have been
undertaken on discrimination, and most conclude that discrimination abounds.
Women, particularly those who have children, are paid less than men, and
frequently face a `glass ceiling' -- a limit on their promotion
possibilities within organizations.

\begin{DefBox}
	\textbf{Discrimination} implies an earnings differential that is attributable to a trait other than human capital.
\end{DefBox}

In contrast, women no longer face discrimination in university and college
admissions, and form a much higher percentage of the student population than
men in many of the higher paying professions such as medicine and law.
Immigrants to Canada also suffer from a wage deficit. This is especially
true for the most recent cohorts of working migrants who now come
predominantly, not from Europe, as was once the case, but from China, South
Asia, Africa and the Caribbean. For similarly-measured HK as Canadian-born
individuals, these migrants frequently have an initial wage deficit of 30\%,
and require a period of more than twenty years to catch-up.