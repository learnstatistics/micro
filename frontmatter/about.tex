\chapter*{About the Authors}

\textbf{Doug Curtis} is a specialist in macroeconomics. He is the author of numerous research papers on fiscal policy, monetary policy, and economic growth and structural change. He has also prepared research reports for Canadian industry and government agencies and authored numerous working papers. He completed his PhD at McGill University, and has held visiting appointments at the University of Cambridge and the University of York in the United Kingdom. His current research interests are monetary and fiscal policy rules, and the relationship between economic growth and structural change. He is Professor Emeritus of Economics at Trent University in Peterborough, Ontario, and also held an appointment as Sessional Adjunct Professor in the Department of Economics at Queen's University in Kingston, Ontario from 2003 until 2013.

\textbf{Ian Irvine} is a specialist in microeconomics, public economics, economic inequality and health economics. He is the author of numerous research papers in these fields. He completed his PhD at the University of Western Ontario, has been a visitor at the London School of Economics, the University of Sydney, the University of Colorado, University College Dublin and the Economic and Social Research Institute. His current research interests are in tobacco use and taxation, and Canada's Employment Insurance and Welfare systems. He has done numerous studies for the Government of Canada, and is currently a Professor of Economics at Concordia University in Montreal.

\subsection*{Our Philosophy}

\emph{\booktitle} focuses upon the material that students need to cover in a first introductory course. It is slightly more compact than the majority of principles books in the Canadian marketplace. Decades of teaching experience and textbook writing has led the authors to avoid the encyclopedic approach that characterizes the recent trends in textbooks.

Consistent with this approach, there are no appendices or `afterthought' chapters. No material is relegated elsewhere for a limited audience; the text makes choices on what issues and topics are important in an introductory course. This philosophy has resulted in a Micro book of just 15 chapters, of which Chapters~\ref{chap:intro} through ~\ref{chap:classical} are common to both Micro and Macro, and a Macro book of 13 chapters.

Examples are domestic and international in their subject matter and are of the modern era -- consumers buy iPods, snowboards and jazz, not so much coffee and hamburgers. Globalization is a recurring theme.

While this book avoids calculus, and uses equations sparingly, it still aims to be rigorous. In contrast to many books on the market, that simply insert diagrams and discuss concepts in a diagrammatic framework, our books almost invariably analyze the key issues in each chapter by introducing a numerical example or case study at the outset. Students are introduced immediately to the practice of taking a set of data, examining it numerically, plotting it, and again analyzing the material in that form. This process is not difficult, but it is rigorous, and stresses that economics is about data analysis as well as ideas and theories. The end-of-chapter problems also involve a considerable amount of numerical and graphical analysis. A small number of problems in each chapter involve solving simple linear equations (intersecting straight lines); but we provide a sufficient number of questions for the student to test his or her understanding of the material without working through that subset of questions. 

\subsection*{Structure of the Text}

\emph{\booktitle} provides a concise, yet complete, coverage of introductory microeconomic theory, application and policy in a Canadian and global environment. Our beginning is orthodox: We explain and develop the standard tools of analysis in the discipline.

Economic policy is about the well-being of the economy's participants, and economic theory should inform economic policy. So we investigate the meaning of `well-being' in the context of an efficient use of the economy's resources early in the text.

We next develop an understanding of individual optimizing behaviour. This behaviour in turn is used to link household decisions on savings with firms' decisions on production, expansion and investment. A natural progression is to explain production and cost structures.

From the individual level of household and firm decision making, the text then explores behaviour in a variety of different market structures from perfect competition to monopoly.

Markets for the inputs in the productive process -- capital and labour -- are a natural component of firm-level decisions. But education and human capital are omnipresent concepts and concerns in the modern economy, so we devote a complete chapter to them.

The book then examines the role of a major and important non-market player in the economy -- the government, and progresses to develop the key elements in the modern theory of international trade.