\section{Technological change and globalization}\label{sec:ch8sec7}

\terminology{Technological change} represents innovation that can reduce the
cost of production or bring new products on line. As stated earlier, the
very long run is a period that is sufficiently long for new technology to
evolve and be implemented.

\begin{DefBox}
	\textbf{Technological change} represents innovation that can reduce the cost of production or bring new products on line.
\end{DefBox}

Technological change has had an enormous impact on economic life for several
centuries. It is not something that is defined in terms of the recent
telecommunications revolution. The industrial revolution began in eighteenth
century Britain. It was accompanied by a less well-recognized, but equally
important, agricultural revolution. The improvement in cultivation
technology, and ensuing higher yields, freed up enough labour to populate
the factories that were the core of the industrial revolution\footnote{
Two English farmers changed the lives of millions in the 18\textsuperscript{th} 
century through their technological genius. The first
was Charles Townsend, who introduced the concept of crop rotation. He
realized that continual use of soil for a single purpose drained the land of
key nutrients. This led him ultimately to propose a five-year rotation that
included tillage, vegetables, and sheep. This rotation reduced the time
during which land would lie fallow, and therefore increased the productivity
of land. A second game-changing technological innovation saw the
introduction of the seed plough, a tool that facilitated seed sowing in
rows, rather than scattering it randomly. This line sowing meant that weeds
could be controlled more easily---weeds that would otherwise smother much of
the crop. The genius here was a gentleman by the name of Jethro Tull.}. The
development and spread of mechanical power dominated the nineteenth century,
and the mass production line of Henry Ford heralded in the twentieth
century. The modern communications revolution has reduced costs, just like
its predecessors. But it has also greatly sped up \terminology{globalization},
the increasing integration of national markets.

\begin{DefBox}
	\textbf{Globalization} is the tendency for international markets to be ever more integrated.
\end{DefBox}

Globalization has several drivers, the most important of which are lower
transportation and communication costs, reduced barriers to trade and
capital mobility, and the spread of new technologies that facilitate cost
and quality control. New technology and better communications have been
critical in both increasing the minimum efficient scale of operation and
reducing diseconomies of scale; they facilitate the efficient management of
large companies.

The continued reduction in trade barriers in the post-World War II era has
also meant that the effective marketplace has become the globe rather than
the national economy for many products. Companies like \textit{Apple}, \textit{Microsoft}, and
\textit{Dell} are visible worldwide. Globalization has been accompanied by the
collapse of the Soviet Union, the adoption of an outward looking philosophy
on the part of China, and an increasing role for the market place in India.
These developments together have facilitated the outsourcing of much of the
West's manufacturing to lower-wage economies.

\newhtmlpage

But new technology not only helps existing companies grow large; it also
enables new ones to start up. It is now cheaper for small producers to
manage their inventories and maintain contact with their own suppliers.
Contrary to what is often claimed, new technology has not led to a greater
concentration of the economy's output in the hands of a small number of big
producers; the competitive forces that the new technology has unleashed are
too strong.

The impact of technology on typical cost curves is illustrated in Figure~\ref{fig:techchange}.
It both reduces the whole cost structure of production,
and at the same time increases the minimum efficient scale.

% Figure 8.7
\begin{TikzFigure}{xscale=0.3,yscale=0.34,descwidth=25em,caption={Technological change and LAC \label{fig:techchange}},description={Technological change reduces the unit production cost for any output produced and may also increase the minimum efficient scale ($MES$) threshold.}}
\draw [latccolour,ultra thick,-] (3,13) to [out=-30,in=180] (15,9) to [out=0,in=200] (24,11) node [black,mynode,above right] {$LAC$\\pre-technological\\change};
\draw [dashed,latccolour,ultra thick,-] (6,10) to [out=-30,in=180] (18,6) to [out=0,in=200] (27,8) node [black,mynode,right] {$LAC$\\post-technological\\change};
\draw [thick, -] (0,20) node [above] {Cost (\$)} |- (30,0) node [right] {Output};
\draw [->] (16,3) node [mynode,left] {$MES$ increases} -- (19,5);
\end{TikzFigure}