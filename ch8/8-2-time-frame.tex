\section{The time frame}\label{sec:ch8sec2}

We distinguish initially between the \terminology{short run} and the 
\terminology{long run}. When discussing technological change, we use the
term \terminology{very long run}. These concepts have little to do with
clocks or calendars; rather, they are defined by the degree of flexibility
an entrepreneur or manager has in her production process. A key decision
variable is capital.

A customary assumption is that a producer can hire more labour immediately,
if necessary, either by taking on new workers (since there are usually some
who are unemployed and looking for work), or by getting the existing workers
to work longer hours. In contrast, getting new capital in place is generally
more time consuming: The entrepreneur may have to place an order for new
machinery, which will involve a production and delivery time lag. Or she may
have to move to a more spacious location in order to accommodate the added
capital. Whether this calendar time is one week, one month, or one year is
of no concern to us. We define the long run as a period of sufficient length
to enable the entrepreneur to adjust her capital stock, whereas in the short
run at least one factor of production is fixed. Note that it matters little
whether it is labour or capital that is fixed in the short run. A software
development company may be able to install new capital (computing power)
instantaneously but have to train new developers. In such a case capital is
variable and labour is fixed in the short run. The definition of the short
run is that one of the factors is fixed, and in our examples we will assume
that it is capital.

\begin{DefBox}
	\textbf{Short run}: a period during which at least one factor of production is fixed. If capital is fixed, then more output is produced by using additional labour.
	
	\textbf{Long run}: a period of time that is sufficient to enable all factors of production to be adjusted.
	
	\textbf{Very long run}: a period sufficiently long for new technology to develop.
\end{DefBox}