\section{Production in the short run}\label{sec:ch8sec3}

Black Diamond Snowboards (BDS) is a start-up snowboard producing enterprise.
Its founder has invented a new lamination process that gives extra strength
to his boards. He has set up a production line in his garage that has four
workstations: Laminating, attaching the steel edge, waxing, and packing.

With this process in place, he must examine how productive his firm can be.
After extensive testing, he has determined exactly how his productivity
depends upon the number of workers. If he employs only one worker, then that
worker must perform several tasks, and will encounter `down time' between
workstations. Extra workers would therefore not only increase the total
output; they could, in addition, increase output \textit{per worker}. He
also realizes that once he has employed a critical number of workers,
additional workers may not be so productive: Because they will have to share
the fixed amount of machinery in his garage, they may have to wait for
another worker to finish using a machine. At such a point, the productivity
of his plant will begin to fall off, and he may want to consider capital
expansion. But for the moment he is constrained to using this particular
assembly plant. Testing leads him to formulate the relationship between
workers and output that is described in Table~\ref{table:snowprod}.

\begin{Table}{caption={Snowboard production and productivity \label{table:snowprod}}}
\begin{tabu} to \linewidth {|X[1,c]X[1,c]X[1,c]X[1,c]X[3,c]|} \hline
	\rowcolor{rowcolour}	\textbf{1}	&	\textbf{2}	&	\textbf{3}	&	\textbf{4}	&	\textbf{5}	\\[-0.1em]
	\rowcolor{rowcolour}	\textbf{Workers} & \textbf{Output} & \textbf{Marginal} & \textbf{Average} & \textbf{Stages of} \\[-0.35em]
	\rowcolor{rowcolour}		&	 \textbf{($TP$)}	&	\textbf{product}	&	\textbf{product}	&	\textbf{production}	\\[-0.35em]
	\rowcolor{rowcolour}		&	&	\textbf{($MP_L$)}	&	\textbf{($AP_L$)}	&	\\	\hline
	0 & 0 &  &  & \multirow{5}{*}{\textit{$MP_L$ increasing}} \\ 
	\cellcolor{rowcolour}	1 & \cellcolor{rowcolour}15 & \cellcolor{rowcolour}15 & \cellcolor{rowcolour}15 & \\
	2 & 40 & 25 & 20 & 		\\
	\cellcolor{rowcolour}	3 & \cellcolor{rowcolour}70 & \cellcolor{rowcolour}30 & \cellcolor{rowcolour}23.3 & \\
	4 & 110 & 40 & 27.5 &	\\
	\rowcolor{rowcolour}	5 & 145 & 35 & 29 	&	\\[-0.1em]
	6 & 175 & 30 & 29.2 &	\cellcolor{rowcolour}		\\[-0.1em]
	\rowcolor{rowcolour}	7 & 200 & 25 & 28.6 &	\\[-0.1em]
	8 & 220 & 20 & 27.5 &	\cellcolor{rowcolour}		\\[-0.1em]
	\rowcolor{rowcolour}	9 & 235 & 15 & 26.1 &	\\[-0.1em]
	10 & 240 & 5 & 24.0 & \multirow{-6}{*}{\cellcolor{rowcolour}\textit{$MP_L$ positive and declining}}\\
	\cellcolor{rowcolour}	11 & \cellcolor{rowcolour}235 & \cellcolor{rowcolour}-5 & \cellcolor{rowcolour}21.4 & \textit{$MP_L$ negative} \\ \hline 
\end{tabu}
\end{Table}

\newhtmlpage

By increasing the number of workers in the plant, BDS produces more boards.
The relationship between these two variables in columns 1 and 2 in the table
is plotted in Figure~\ref{fig:TPcurve}. This is called the 
\terminology{total product function} ($TP$), and it defines the output
produced with different amounts of labour in a plant of fixed size.

% Figure 8.1
\begin{TikzFigure}{xscale=1,yscale=1,descwidth=25em,caption={Total product curve \label{fig:TPcurve}},description={Output increases with the amount of labour used. Initially the increase in output due to using more labour is high, subsequently it is lower. The initial phase characterizes increasing productivity, the later phase defines declining productivity.}}
\begin{axis}[
	axis line style=thick,
	every tick label/.append style={font=\footnotesize},
	every node near coord/.append style={font=\scriptsize},
	xticklabel style={anchor=north,/pgf/number format/1000 sep=},
	scaled y ticks=false,
	x=0.9cm/1,
	y=1cm/50,
	yticklabel style={/pgf/number format/fixed,/pgf/number format/1000 sep = \thinspace},
	xmin=0,xmax=12,ymin=0,ymax=325,
	xlabel={Labour},
	ylabel={Output},
]
\addplot[ultra thick,tpcolour,mark=none] coordinates {
	(0,0)
	(1,15)
	(2,40)
	(3,70)
	(4,110)
	(5,145)
	(6,175)
	(7,200)
	(8,220)
	(9,235)
	(10,240)
	(11,235)
};
\end{axis}
\end{TikzFigure}

\begin{DefBox}
	\textbf{Total product} is the relationship between total output produced and the number of workers employed, for a given amount of capital.
\end{DefBox}

\newhtmlpage

This relationship is positive, indicating that more workers produce more
boards. But the curve has an interesting pattern. In the initial expansion
of employment it becomes progressively steeper -- its curvature is slightly
convex; following this phase the function's increase becomes progressively
less steep -- its curvature is concave. These different stages in the $TP$
curve tell us a great deal about productivity in BDS. To see this, consider
the additional number of boards produced by each worker. The first worker
produces 15. When a second worker is hired, the total product rises to 40,
so the additional product attributable to the second worker is 25. A third
worker increases output by 30 units, and so on. We refer to this additional
output as the marginal product ($MP$) of an additional worker, because it
defines the incremental, or marginal, contribution of the worker. These
values are entered in column 3.

More generally the $MP$ of labour is defined as the change in output divided
by the change in the number of units of labour employed. Using, as before,
the Greek capital delta ($\Delta$) to denote a change, we can define

\begin{equation*}
MP_{L}=\frac{\text{Change in output produced}}{\text{Change in labour employed}}=\frac{\Delta Q}{\Delta L}
\end{equation*}

In this example the change in labour is one unit at each stage and hence the %
\terminology{marginal product of labour} is simply the corresponding change
in output. It is also the case that the $MP_L$ is the slope of the $TP$
curve -- the change in the value on the vertical axis due to a change in the
value of the variable on the horizontal axis.

\begin{DefBox}
	\textbf{Marginal product of labour} is the addition to output produced by each additional worker. It is also the slope of the total product curve.
\end{DefBox}

\newhtmlpage

% Figure 8.2
\begin{TikzFigure}{xscale=1,yscale=1,descwidth=25em,caption={Average and marginal product curves \label{fig:AMPcurve}},description={The productivity curves initially rise and then decline, reflecting increasing and decreasing productivity. The $MP_L$ curves must intersect the $AP_L$ curve at the maximum of the $AP_L$: The average must increase if the marginal exceeds the average and must decline if the marginal is less than the average.}}
\begin{axis}[
	axis line style=thick,
	every tick label/.append style={font=\footnotesize},
	every node near coord/.append style={font=\scriptsize},
	xticklabel style={anchor=north,/pgf/number format/1000 sep=},
	scaled y ticks=false,
	x=0.9cm/1,
	y=0.75cm/5,
	yticklabel style={/pgf/number format/fixed,/pgf/number format/1000 sep = \thinspace},
	xmin=0,xmax=12,ymin=0,ymax=45,
	xlabel={Labour},
	ylabel={Output},
]
\addplot[ultra thick,apcolour,mark=none] coordinates { % average product curve
	(1,15)
	(2,20)
	(3,23.3)
	(4,27.5)
	(5,29)
	(6,29.2)
	(7,28.6)
	(8,27.5)
	(9,26.1)
	(10,24.0)
	(11,21.4)
} node [black,mynode,pos=0.8,above right] {Average\\product\\of labour};
\addplot[dashed,ultra thick,mpcolour,mark=none] coordinates { % marginal product curve
	(1,15)
	(2,25)
	(3,30)
	(4,40)
	(5,35)
	(6,30)
	(7,25)
	(8,20)
	(9,15)
	(10,5)
	(11,-5)
} node [black,mynode,pos=0.7,below left] {Marginal\\product\\of labour};
\end{axis}
\end{TikzFigure}


During the initial stage of production expansion, the marginal product of
each worker is increasing. It increases from 15 to 40 as BDS moves from
having one employee to four employees. This increasing $MP$ is made possible
by the fact that each worker is able to spend more time at his workstation,
and less time moving between tasks. But, at a certain point in the
employment expansion, the $MP$ reaches a maximum and then begins to tail
off. At this stage -- in the concave region of the $TP$ curve -- additional
workers continue to produce additional output, but at a diminishing rate.
For example, while the fourth worker adds 40 units to output, the fifth
worker adds 35, the sixth worker 30, and so on. This declining $MP$ is due
to the constraint of a fixed number of machines: All workers must share the
same capital. The $MP$ function is plotted in Figure~\ref{fig:AMPcurve}.

\newhtmlpage

The phenomenon we have just described has the status of a law in economics:
The \terminology{law of diminishing returns} states that, in the face of a
fixed amount of capital, the contribution of additional units of a variable
factor must eventually decline.

\begin{DefBox}
	\textbf{Law of diminishing returns}: when increments of a variable factor (labour) are added to a fixed amount of another factor (capital), the marginal product of the variable factor must eventually decline.
\end{DefBox}

The relationship between Figures~\ref{fig:TPcurve} and \ref{fig:AMPcurve}
should be noted. First, the $MP_L$ reaches a maximum at an output of 4 units
-- where the slope of the $TP$ curve is greatest. The $MP_L$ curve remains
positive beyond this output, but declines: The $TP$ curve reaches a maximum
when the tenth unit of labour is employed. An eleventh unit actually reduces
total output; therefore, the $MP$ of this eleventh worker is negative! In
Figure~\ref{fig:AMPcurve}, the $MP$ curve becomes negative at this point.
The garage is now so crowded with workers that they are beginning to
obstruct the operation of the production process. Thus the producer would
never employ an eleventh unit of labour.

Next, consider the information in the fourth column of the table. It defines
the average product of labour ($AP_L$)---the amount of output produced, on
average, by workers at different employment levels:
\begin{equation*}
AP_{L}=\frac{\text{Total output produced}}{\text{Total amount of labour employed}}=\frac{Q}{L}.
\end{equation*}

This function is also plotted in Figure~\ref{fig:AMPcurve}. Referring to the
table: The $AP$ column indicates, for example, that when two units of labour
are employed and forty units of output are produced, the average production
level of each worker is 20 units ($=40/2$). When three workers produce 70
units, their average production is 23.3 ($=70/3$), and so forth. Like the $MP$
function, this one also increases and subsequently decreases, reflecting
exactly the same productivity forces that are at work on the $MP$ curve.

\begin{DefBox}
	\textbf{Average product of labour} is the number of units of output produced per unit of labour at different levels of employment.
\end{DefBox}

\newhtmlpage

The $AP$ and $MP$ functions intersect at the point where the $AP$ is at its
peak. This is no accident, and has a simple explanation. Imagine a baseball
player who is batting .280 coming into today's game---he has been hitting
his way onto base 28 percent of the time when he goes up to bat, so far this
season. This is his average product, $AP$.

In today's game, if he bats .500 (he hits his way to base on half of his
at-bats), then he will improve his average. Today's batting (his $MP$) at
.500 therefore pulls up his season's $AP$. Accordingly, whenever the $MP$
exceeds the $AP$, the $AP$ is pulled up. By the same reasoning, if his $MP$
is less than his season average, his average will be pulled down. It follows
that the two functions must intersect at the peak of the $AP$ curve. To
summarize:

\begin{quote}
	If the $MP$ exceeds the $AP$, then the $AP$ increases; \\
	If the $MP$ is less than the $AP$, then the $AP$ declines.
\end{quote}

While the owner of BDS may understand his productivity relations, his
ultimate goal is to make profit, and for this he must figure out how
productivity translates into cost.
