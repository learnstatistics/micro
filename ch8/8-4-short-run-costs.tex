\section{Costs in the short run}\label{sec:ch8sec4}

The cost structure for the production of snowboards at Black Diamond is
illustrated in Table~\ref{table:snowprodcost}. Employees are skilled and are
paid a weekly wage of \$1,000. The cost of capital is \$3,000 and it is fixed,
which means that it does not vary with output. As in Table~\ref{table:snowprod},
the number of employees and the output are given in the
first two columns. The following three columns define the capital costs, the
labour costs, and the sum of these in producing different levels of output.
We use the terms \terminology{fixed}, \terminology{variable}, and 
\terminology{total costs} to define the cost structure of a firm. Fixed
costs do not vary with output, whereas variable costs do, and total costs
are the sum of fixed and variable costs. To keep this example as simple as
possible, we will ignore the cost of raw materials. We could add an
additional column of costs, but doing so will not change the conclusions.

% Table 8.2
\begin{Table}{caption={Snowboard production costs \label{table:snowprodcost}}}
	\begin{tabu} to \linewidth {|X[1,c]X[0.9,c]X[1,c]X[1,c]X[0.9,c]X[1,c]X[1,c]X[1,c]X[1.1,c]|} \hline 
		\rowcolor{rowcolour}	\textbf{Workers} & \textbf{Output} & \textbf{Capital} & \textbf{Labour} & \textbf{Total} & \textbf{Average} & \textbf{Average} & \textbf{Average} & \textbf{Marginal} \\[-0.1em]
		\rowcolor{rowcolour}	&	&	\textbf{cost}	&	\textbf{cost}	&	\textbf{costs}	&	\textbf{fixed}	&	\textbf{variable}	&	\textbf{total}	&	\textbf{cost}	\\[-0.1em]
		\rowcolor{rowcolour}	&	&	\textbf{fixed}	&	\textbf{variable}	&	&	\textbf{cost}	&	\textbf{cost}	&	\textbf{cost}	&	\\	\hline
		0 & 0 & 3,000 & 0 & 3,000 &  &  &  &  \\
		\rowcolor{rowcolour}	1 & 15 & 3,000 & 1,000 & 4,000 & 200.0 & 66.7 & 266.7 & 66.7 \\
		2 & 40 & 3,000 & 2,000 & 5,000 & 75.0 & 50.0 & 125.0 & 40.0 \\ 
		\rowcolor{rowcolour}	3 & 70 & 3,000 & 3,000 & 6,000 & 42.9 & 42.9 & 85.7 & 33.3 \\ 
		4 & 110 & 3,000 & 4,000 & 7,000 & 27.3 & 36.4 & 63.6 & 25.0 \\ 
		\rowcolor{rowcolour}	5 & 145 & 3,000 & 5,000 & 8,000 & 20.7 & 34.5 & 55.2 & 28.6 \\
		6 & 175 & 3,000 & 6,000 & 9,000 & 17.1 & 34.3 & 51.4 & 33.3 \\ 
		\rowcolor{rowcolour}	7 & 200 & 3,000 & 7,000 & 10,000 & 15.0 & 35.0 & 50.0 & 40.0 \\
		8 & 220 & 3,000 & 8,000 & 11,000 & 13.6 & 36.4 & 50.0 & 50.0 \\
		\rowcolor{rowcolour}	9 & 235 & 3,000 & 9,000 & 12,000 & 12.8 & 38.3 & 51.1 & 66.7 \\ 
		10 & 240 & 3,000 & 10,000 & 13,000 & 12.5 & 41.7 & 54.2 & 200.0 \\ \hline 
	\end{tabu}
\end{Table}

\newhtmlpage

\begin{DefBox}
	\textbf{Fixed costs} are costs that are independent of the level of output.
	
	\textbf{Variable costs} are related to the output produced.
	
	\textbf{Total cost} is the sum of fixed cost and variable cost.
\end{DefBox}

Total costs are illustrated in Figure~\ref{fig:TCcurve} as the vertical sum
of variable and fixed costs. For example, Table~\ref{table:snowprodcost}
indicates that the total cost of producing 220 units of output is the sum of
\$3,000 in fixed costs plus \$8,000 in variable costs. Therefore, at the
output level 220 on the horizontal axis in Figure~\ref{fig:TCcurve}, the sum
of the cost components yields a value of \$11,000 that forms one point on
the total cost curve. Performing a similar calculation for every possible
output yields a series of points that together form the complete total cost
curve.

% Figure 8.3
\begin{TikzFigure}{xscale=1,yscale=1,descwidth=20em,caption={Total cost curves \label{fig:TCcurve}},description={Total cost is the vertical sum of the variable and fixed costs.}}
\begin{axis}[
	axis line style=thick,
	every tick label/.append style={font=\footnotesize},
	every node near coord/.append style={font=\scriptsize},
	extra y ticks={3000},
	xticklabel style={anchor=north,/pgf/number format/1000 sep=},
	scaled y ticks=false,
	x=0.85cm/25,
	y=0.5cm/1000,
	yticklabel style={/pgf/number format/fixed,/pgf/number format/1000 sep = \thinspace},
	xmin=0,xmax=325,ymin=0,ymax=14000,
	xlabel={Output},
	ylabel={Cost (\$)},
]
\addplot[tccolour,ultra thick,mark=none] coordinates { % Total cost
	(0,3000)
	(15,4000)
	(40,5000)
	(70,6000)
	(110,7000)
	(145,8000)
	(175,9000)
	(200,10000)
	(220,11000)
	(235,12000)
	(240,13000)
} node [black,mynode,pos=0.9,right] {Total cost};
\addplot[vccolour,ultra thick,mark=none] coordinates { % Variable cost
	(0,0)
	(15,1000)
	(40,2000)
	(70,3000)
	(110,4000)
	(145,5000)
	(175,6000)
	(200,7000)
	(220,8000)
	(235,9000)
	(240,10000)
} node [black,mynode,pos=0.9,right] {Variable cost};
\addplot[fccolour,ultra thick,mark=none] coordinates { % Fixed cost
	(0,3000)
	(240,3000)
} node [black,mynode,pos=0.9,above] {Fixed cost ($FC$)};
\end{axis}
\end{TikzFigure}

\newhtmlpage

Average costs are given in the next three columns of Table~\ref{table:snowprodcost}.
Average cost is the cost per unit of output, and we
can define an average cost corresponding to each of the fixed, variable, and
total costs defined above. \terminology{Average fixed cost} ($AFC$) is the
total fixed cost divided by output; \terminology{average variable cost}
($AVC$) is the total variable cost divided by output; and 
\terminology{average total cost} ($ATC$) is the total cost divided by output.
\begin{align*}
	AFC& =\text{(Fixed cost)}/Q=FC/Q \\
	AVC& =\text{(Total variable costs)}/Q=TVC/Q \\
	ATC& =AFC+AVC
\end{align*}

\begin{DefBox}
	\textbf{Average fixed cost} is the total fixed cost per unit of output.
	
	\textbf{Average variable cost} is the total variable cost per unit of output.
	
	\textbf{Average total cost} is the sum of all costs per unit of output.
\end{DefBox}

\newhtmlpage

\subsection*{The productivity-cost relationship}

Consider the \textbf{average variable cost - average product relationship},
as developed in column 7 of Table~\ref{table:snowprodcost}; its
corresponding variable cost curve is plotted in Figure~\ref{fig:AMCcurve}.
In this example, $AVC$ first decreases and then increases. The intuition
behind its shape is straightforward (and realistic) if you have understood
why productivity varies in the short run: The variable cost, which
represents the cost of labour, is constant per unit of labour, because the wage paid to each
worker does not change. However, each worker's productivity varies. Initially,
when we hire more workers, they become more productive, perhaps because they
have less `down time' in switching between tasks. This means that the labour
costs per snowboard must decline. At some point, however, the law of
diminishing returns sets in: As before, each additional worker is paid a
constant amount, but as productivity declines the labour cost per snowboard
increases.

% Figure 8.4
\begin{TikzFigure}{xscale=1,yscale=1,descwidth=25em,caption={Average and marginal cost curves \label{fig:AMCcurve}},description={The $MC$ intersects the $ATC$ and $AVC$ at their minimum values. The $AFC$ declines indefinitely as fixed costs are spread over a greater output.}}
\begin{axis}[
	axis line style=thick,
	every tick label/.append style={font=\footnotesize},
	every node near coord/.append style={font=\scriptsize},
	xticklabel style={anchor=north,/pgf/number format/1000 sep=},
	scaled y ticks=false,
	x=1cm/25,
	y=1.25cm/50,
	yticklabel style={/pgf/number format/fixed,/pgf/number format/1000 sep = \thinspace},
	xmin=0,xmax=275,ymin=0,ymax=275,
	xlabel={Output},
	ylabel={Cost (\$)},
]
\addplot[atccolour,ultra thick,mark=none] coordinates { % ATC
	(15,266.7)
	(40,125)
	(70,85.7)
	(110,63.6)
	(145,55.2)
	(175,51.4)
	(200,50)
	(220,50)
	(235,51.1)
	(240,54.2)
} node [black,mynode,pos=1,right] {$ATC$};
\addplot[dashed,mccolour,ultra thick,mark=none] coordinates { % MC
	(15,66.7)
	(40,40)
	(70,33.3)
	(110,25)
	(145,28.6)
	(175,33.3)
	(200,40)
	(220,50)
	(235,66.7)
} node [black,mynode,pos=1,right] {$MC$};
\addplot[avccolour,ultra thick,mark=none] coordinates { % AVC
	(15,66.7)
	(40,50)
	(70,42.9)
	(110,36.4)
	(145,34.5)
	(175,34.3)
	(200,35)
	(220,36.4)
	(235,38.3)
	(240,41.7)
} node [black,mynode,pos=1,right] {$AVC$};
\addplot[afccolour,ultra thick,mark=none] coordinates { % AFC
	(15,200)
	(40,75)
	(70,42.9)
	(110,27.3)
	(145,20.7)
	(175,17.1)
	(200,15)
	(220,13.6)
	(235,12.8)
	(240,12.5)
} node [black,mynode,pos=1,right] {$AFC$};
\end{axis}
\end{TikzFigure}


\newhtmlpage

In this numerical example the $AP$ is at a maximum when six units of labour
are employed and output is 175. This is also the point where the $AVC$ is at
a minimum. This maximum/minimum relationship is also illustrated in
Figures~\ref{fig:AMPcurve} and \ref{fig:AMCcurve}.

Now consider the \textbf{marginal cost - marginal product relationship}. The 
\terminology{marginal cost} ($MC$) defines the cost of producing one more
unit of output. In Table~\ref{table:snowprodcost}, the marginal cost of
output is given in the final column. It is the additional cost of production
divided by the additional number of units produced. For example, in going
from 15 units of output to 40, total costs increase from \$4,000 to \$5,000.
The $MC$ is the cost of those additional units divided by the number of
additional units. In this range of output, $MC$ is $\$1,000/25=\$40$. We could
also calculate the $MC$ as the addition to variable costs rather than the
addition to total costs, because the \textit{addition} to each is the
same---fixed costs are fixed. Hence:
\begin{align*}
	MC &=\frac{\text{Change in total costs}}{\text{Change in output produced}}=\frac{\Delta TC}{\Delta Q} \\
	&=\frac{\text{Change in variable costs}}{\text{Change in output produced}}=\frac{\Delta TVC}{\Delta Q}.
\end{align*}

\begin{DefBox}
	\textbf{Marginal cost} of production is the cost of producing each additional unit of output.
\end{DefBox}

Just as the behaviour of the $AVC$ curve is determined by the $AP$ curve, so
too the behaviour of the $MC$ is determined by the $MP$ curve. When the $MP$
of an additional worker exceeds the $MP$ of the previous worker, this
implies that the cost of the additional output produced by the last worker
hired must be declining. To summarize:
\begin{quote}
	If the marginal product of labour increases, then the marginal cost of output declines; \\
	If the marginal product of labour declines, then the marginal cost of output increases.
\end{quote}

In our example, the $MP_{L}$ reaches a maximum when the fourth unit of
labour is employed (or 110 units of output are produced), and this also is
where the $MC$ is at a minimum. This illustrates that the \textit{marginal 
cost reaches a minimum at the output level where the marginal product 
reaches a maximum}.

\newhtmlpage

The average total cost is the sum of the fixed cost per unit of output and
the variable cost per unit of output. Typically, fixed costs are the
dominant component of total costs at low output levels, but become less
dominant at higher output levels. Unlike average variable costs, note that
the average fixed cost must always decline with output, because a fixed cost
is being spread over more units of output. Hence, when the $ATC$ curve
eventually increases, it is because the increasing variable cost component
eventually dominates the declining $AFC$ component. In our example, this
occurs when output increases from 220 units (8 workers) to 235 (9
workers).

Finally, observe the interrelationship between the $MC$ curve on the one
hand and the $ATC$ and $AVC$ on the other. Note from Figure~\ref{fig:AMCcurve}
that the $MC$ cuts the $AVC$ and the $ATC$ at the minimum
point of each of the latter. The logic behind this pattern is analogous to
the logic of the relationship between marginal and average product curves:
When the cost of an additional unit of output is less than the average, this
reduces the average cost; whereas, if the cost of an additional unit of
output is above the average, this raises the average cost. This must hold
true regardless of whether we relate the $MC$ to the $ATC$ or the $AVC$.

\begin{quote}
	When the marginal cost is less than the average cost, the average cost must decline; \\
	When the marginal cost exceeds the average cost, the averge cost must increase.
\end{quote}

\textit{Notation: We use both the abbreviations }$\mathit{ATC}$\textit{ and 
}$\mathit{AC}$\textit{ to denote average total cost. The term `average
cost' is understood in economics to include both fixed and variable costs.}