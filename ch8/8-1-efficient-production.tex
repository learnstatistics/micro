\section{Efficient production}\label{sec:ch8sec1}

Firms that fail to operate efficiently seldom survive. They are dominated by
their competitors because the latter produce more efficiently and can sell
at a lower price. The drive for profitability is everywhere present in the
modern economy. Companies that promise more profit, by being more efficient,
are valued more highly on the stock exchange. For example: In July of 2015
\textit{Google} announced that, going forward, it would be more attentive to cost
management in its numerous research endeavours that aim to bring new
products to the marketplace. This policy, put in place by the Company's new
Chief Financial Officer, was welcomed by investors who, as a result, bought
up the stock. The Company's stock increased in value by 16\% in one day -- equivalent
to about \$50 billion.

The remuneration of managers in virtually all corporations is linked to
profitability. Efficient production, \textit{a.k.a.} cost reduction, is
critical to achieving this goal. In this chapter we will examine cost
management and efficient production from the ground up -- by exploring how a
small entrepreneur brings his or her product to market in the most efficient
way possible. As we shall see, efficient production and cost minimization
amount to the same thing: Cost minimization is the financial reflection of
efficient production.

Efficient production is critical in any budget-driven organization, not just
in the private sector. Public institutions equally are, and should be,
concerned with costs and efficiency.

Entrepreneurs employ factors of production (capital and labour) in order to
transform raw materials and other inputs into goods or services. The
relationship between output and the inputs used in the production process is
called a \terminology{production function}. It specifies how much output can
be produced with given combinations of inputs. A production function is not
restricted to profit-driven organizations. Municipal road repairs are
carried out with labour and capital. Students are educated with teachers,
classrooms, computers, and books. Each of these is a production process.

\begin{DefBox}
	\textbf{Production function}: a technological relationship that specifies how much output can be produced with specific amounts of inputs.
\end{DefBox}

\newhtmlpage

Economists distinguish
between two concepts of efficiency: One is \terminology{technological efficiency};
the other is \terminology{economic efficiency}. To illustrate
the difference, consider the case of auto assembly in Oshawa Megamobile Inc.,
an auto manufacturer. Megamobile could assemble its vehicles either by
using a large number of assembly workers and a plant that has a relatively
small amount of machinery, or it could use fewer workers accompanied by more
machinery in the form of robots. Each of these processes could be deemed
technologically efficient, provided that there is no waste. If the workers
without robots are combined with their capital to produce as much as
possible, then that production process is technologically efficient.
Likewise, in the scenario with robots, if the workers and capital are
producing as much as possible, then that process too is efficient in the
technological sense.

\begin{DefBox}
	\textbf{Technological efficiency} means that the maximum output is produced with the given set of inputs.
\end{DefBox}

Economic efficiency is concerned with more than just technological
efficiency. Since the entrepreneur's goal is to make profit, she must
consider which technologically efficient process best achieves that
objective. More broadly, any budget-driven process should focus on being
economically efficient, whether in the public or private sector. An
economically efficient production structure is the one that produces output
at least cost.

\begin{DefBox}
	\textbf{Economic efficiency} defines a production structure that produces output at least cost.
\end{DefBox}

Auto-assembly plants the world over have moved to using robots during the
last two decades. Why? The reason is not that robots were invented 20 years
ago; they were invented long before that. The real reason is that, until
recently, this technology was not economically efficient. Robots were too
expensive; they were not capable of high-precision assembly. But once their
cost declined and their accuracy increased they became economically
efficient. The development of robots represented technological progress.
When this progress reached a critical point, entrepreneurs embraced it.

\newhtmlpage

To illustrate the point further, consider the case of garment assembly.
There is no doubt that our engineers could make robots capable of joining
the pieces of fabric that form garments. This is not beyond our
technological abilities. Why, then, do we not have such capital-intensive
production processes for garment making, similar to the production process
chosen by Oshawa Megamobile? The answer is that, while such a concept could
be technologically efficient, it would not be economically efficient. It is
more profitable to use large amounts of labour and relatively traditional
machines to assemble garments, particularly when labour in Asia costs less and
the garments can be shipped back to Canada inexpensively (in contrast to
automobiles). Containerization and scale economies in shipping mean that a
garment can be shipped to Canada from Asia for a few cents per unit.

Efficiency in production is not limited to the manufacturing sector. Farmers
must choose the optimal combination of labour, capital and fertilizer to
use. In the health and education sectors, efficient supply involves choices
on how many high- and low-skill workers to employ, how much traditional
physical capital to use, how much information technology to use, based upon
the productivity and cost of each. Professors and physicians are costly
inputs. When they work with new technology (capital) they become more
efficient at performing their tasks: It is less costly to have a single
professor teach in a 300-seat classroom that is equipped with the latest
technology, than have several professors each teaching 60-seat classes with
chalk and a blackboard.