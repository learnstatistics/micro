\section{Long-run production and costs}\label{sec:ch8sec6}

The snowboard manufacturer we portray produces a relatively low level of
output; in reality, millions of snowboards are produced each year in the
global market. Black Diamond Snowboards may have hoped to get a start by
going after a local market---the ``free-ride'' teenagers at Mont Sainte Anne
in Quebec or at Fernie in British Columbia. If this business takes off, the
owner must increase production, take the business out of his garage and set
up a larger-scale operation. But how will this affect his cost structure?
Will he be able to produce boards at a lower cost than when he was producing
a very limited number of boards each season? Real-world experience would
indicate yes.

Production costs almost always decline when the \textit{scale} of the
operation initially increases. We refer to this phenomenon simply as %
\terminology{economies of scale}. There are several reasons why scale
economies are encountered. One is that production flows can be organized in
a more efficient manner when more is being produced. Another is that the
opportunity to make greater use of task specialization presents itself; for
example, Black Diamond Snowboards may be able to subdivide tasks within the
laminating and packaging stations. If scale economies do define the real
world, then a bigger plant---one that is geared to produce a higher level of
output---should have an average total cost curve that is ``lower'' than the
cost curve corresponding to the smaller scale of operation we considered in
the example above.

\newhtmlpage

\subsection*{Average costs in the long run}

Figure~\ref{fig:longshortavg} illustrates a possible relationship between
the $ATC$ curves for four different scales of operation. $ATC_{1}$ is the
average total cost curve associated with a small-sized plant; think of it as
the plant built in the entrepreneur's garage. $ATC_{2}$ is associated with a
somewhat larger plant, perhaps one she has put together in a rented
industrial or commercial space. The further a cost curve is located to the
right of the diagram the larger the production facility it defines, given
that output is measured on the horizontal axis. If there are economies
associated with a larger scale of operation, then the average costs
associated with producing larger outputs in a larger plant should be lower
than the average costs associated with lower outputs in a smaller plant,
assuming that the plants are producing the output levels they were designed
to produce. For this reason, the cost curve $ATC_{2}$ and the cost curve 
$ATC_{3}$ each have a segment that is lower than the lowest segment on 
$ATC_{1}$. However, in Figure~\ref{fig:longshortavg} the cost curve $ATC_{4}$
has moved upwards. What behaviours are implied here?

% Figure 8.5
\begin{TikzFigure}{xscale=0.38,yscale=0.32,descwidth=25em,caption={Long-run and short-run average costs \label{fig:longshortavg}},description={The long-run $ATC$ curve, $LATC$, is the lower envelope of all short-run $ATC$ curves. It defines the least cost per unit of output when all inputs are variable. Minimum efficient scale is that output level at which the $LATC$ is a minimum, indicating that further increases in the scale of production will not reduce unit costs.}}
	\draw [latccolour,ultra thick,name path=LATC] plot [smooth, tension=0.75] coordinates {(1.5,14) (6,7) (20,7) (24.5,14)} node [black,mynode,below right] {$LATC$};
	\draw [atccolour,ultra thick] plot [smooth,tension=1.75] coordinates {(2,13.5) (4,10.5) (7,13.5)} node [black,mynode,above] {$ATC_1$};
	\draw [atccolour,ultra thick] plot [smooth,tension=1.75] coordinates {(5,10) (7.25,6.75) (10,10)} node [black,mynode,above] {$ATC_2$};
	\draw [atccolour,ultra thick] plot [smooth,tension=1.71] coordinates {(15,10) (18.75,6.75) (21.75,10)};
	\draw [atccolour,ultra thick] plot [smooth,tension=1.75] coordinates {(19,13.5) (22,10.5) (24,13.5)};
	% axis
	\draw [thick] (0,20) node [above] {Cost (\$)} |- (25,0) node [right] {Quantity};
	\node [mynode] at (2,4) {Region\\of $IRS$};
	\node [mynode] at (22,4) {Region\\of $DRS$};
	\node [mynode] at (13,4) {Region\\of $CRS$};
	\draw [thick,<-,shorten <=0.5em] (10,6) -- +(-2,-3) node [mynode,below] {Minimum efficient\\scale (MES)};
	\node [black,mynode,above] at (15,10) {$ATC_3$};
	\node [black,mynode,above] at (19,13.5) {$ATC_4$};
\end{TikzFigure}

\newhtmlpage

In many production environments, beyond some large scale of operation, it
becomes increasingly difficult to reap further cost reductions from
specialization, organizational economies, or marketing economies. At such a
point, the scale economies are effectively exhausted, and larger plant sizes
no longer give rise to lower (short-run) $ATC$ curves. This is reflected in the
similarity of the $ATC_{2}$ and the $ATC_{3}$ curves. The pattern suggests
that we have almost exhausted the possibilities of further scale advantages once we
build a plant size corresponding to $ATC_{2}$. Consider next what is implied by the
position of the $ATC_{4}$ curve relative to the $ATC_{2}$ and $ATC_{3}$
curves. The relatively higher position of the $ATC_{4}$ curve implies that
unit costs will be higher in a yet larger plant. Stated differently: If we
increase the scale of this firm to extremely high output levels, we are
actually encountering \terminology{diseconomies of scale}. Diseconomies of
scale imply that unit costs increase as a result of the firm's becoming too
large: Perhaps co-ordination difficulties have set in at the very high
output levels, or quality-control monitoring costs have risen. These
coordination and management difficulties are reflected in increasing unit
costs in the long run. Corporations in the modern era are at times broken up
into separate operating units and then sold as independent units. Because of
the diseconomies of scale, the components of multi-unit corporations may be
more valuable independently than when grouped together: When together
coordination problems arise; when independent the coordination challenges
vanish.

The terms \terminology{increasing}, \terminology{constant}, and 
\terminology{decreasing returns to scale} underlie the concepts of scale
economies and diseconomies: Increasing returns to scale (IRS) implies that,
when all inputs are increased by a given proportion, output increases more
than proportionately. Constant returns to scale (CRS) implies that output
increases in direct proportion to an equal proportionate increase in all
inputs. Decreasing returns to scale (DRS) implies that an equal
proportionate increase in all inputs leads to a less than proportionate
increase in output.

\begin{DefBox}
	\textbf{Increasing returns to scale} implies that, when all inputs are increased by a given proportion, output increases more than proportionately. 
	
	\textbf{Constant returns to scale} implies that output increases in direct proportion to an equal proportionate increase in all inputs.
	
	\textbf{Decreasing returns to scale} implies that an equal proportionate increase in all inputs leads to a less than proportionate increase in output.
\end{DefBox}

\newhtmlpage

These are pure production function relationships, but, if the prices of
inputs are fixed for producers, they translate directly into the various
cost structures illustrated in Figure~\ref{fig:longshortavg}. For example,
if a 40\% increase in capital and labour use allows for better production
flows than when in the smaller plant, and therefore yields more than a 40\%
increase in output, this implies that the cost per snowboard produced must
fall in the new plant. In contrast, if a 40\% increase in capital and labour
leads to say just a 30\% increase in output, then the cost per snowboard in
the new larger plant must be higher. Between these extremes, there may be a
range of relatively constant unit costs, corresponding to where the
production relation is subject to constant returns to scale. In 
Figure~\ref{fig:longshortavg}, the falling unit costs output region has increasing
returns to scale, the region that has relatively constant unit costs has
constant returns to scale, and the increasing cost region has decreasing
returns to scale.

Increasing returns to scale characterize businesses with large initial costs
and relatively low costs of producing each unit of output. Computer chip
manufacturers, pharmaceutical manufacturers, even brewers all appear to
benefit from scale economies. In the beer market, brewing, bottling and
shipping are all low-cost operations relative to the capital cost of setting
up a brewery. Consequently, we observe surprisingly few breweries in any
brewing company, even in large land-mass economies such as Canada or the US.

\begin{ApplicationBox}{caption={Decreasing returns to scale \label{app:decretscale}}}
	The CEO of Hewlett Packard announced in October 2012 that the company would reduce its labour force by 29,000 workers (out of a total of 350,000). The problem was that communications within the company were so complex and strained as to increase unit costs. In addition, the company was producing an excessive product variety -- 2,100 variants of laser printer!
\end{ApplicationBox}


In addition to the four short-run average total cost curves, Figure~\ref{fig:longshortavg} 
contains a curve that forms an envelope around the bottom
of these short-run average cost curves. This envelope is the 
\terminology{long-run average total cost} ($LATC$) curve, because it defines
average cost as we move from one plant size to another. Remember that in the
long run both labour and capital are variable, and as we move from one
short-run average cost curve to another, that is exactly what happens---all
factors of production are variable. Hence, the collection of short-run cost
curves in Figure~\ref{fig:longshortavg} provides the ingredients for a
long-run average total cost curve\footnote{Note that the long-run average 
total cost is not the collection of minimum	points from each short-run 
average cost curve. The envelope of the short-run curves will pick up 
mainly points that are not at the minimum, as you will see if you try to 
draw the outcome. The intuition behind the definition is this: With increasing 
returns to scale, it may be better to build a plant	size that operates 
with some spare capacity than to build one that is geared to producing 
a smaller output level. In building the larger plant, we can take greater 
advantage of the scale economies, and it may prove less costly to produce 
in such a plant than to produce with a smaller plant that has less unused 
capacity and does not exploit the underlying scale economies. Conversely, 
in the presence of decreasing returns to scale, it may be less costly to 
produce output in a plant that is used ``overtime'' than to use a larger 
plant that suffers from	scale diseconomies.}.

\newhtmlpage



\begin{equation*}
	LATC=\text{(Long-run total costs)}/Q=LTC/Q
\end{equation*}

\begin{DefBox}
	\textbf{Long-run average total cost} is the lower envelope of all the short-run ATC curves.
\end{DefBox}

The particular range of output on the $LATC$ where it begins to flatten out is
called the range of \terminology{minimum efficient scale}. This is an
important concept in industrial policy, as we shall see in later chapters.
At such an output level, the producer has expanded sufficiently to take
advantage of virtually all the scale economies available.

\begin{DefBox}
	\textbf{Minimum efficient scale} defines a threshold size of operation such that scale economies are almost exhausted.
\end{DefBox}

\newhtmlpage

In view of this discussion and the shape of the $LATC$ in 
Figure~\ref{fig:longshortavg}, it is obvious that economies of scale can also be
defined in terms of the curvature of the $LATC$. Where the $LATC$ declines
there are IRS, where the $LATC$ is flat there are $CRS$, where the $LATC$
slopes upward there are $DRS$.

%Table 8.3
\begin{Table}{caption={LATC elements for two plants (thousands \$) \label{table:latcelementstwoplants}},description={Plant 1 $FC=\$1$m. Plant 2 $FC=\$2$m. For $Q<200$, $ATC_{1}<ATC_{2}$; for $Q>200$, $ATC_{1}>ATC_{2}$; and for $Q=200$, $ATC_1=ATC_2$. $LATC$ defined by data in bold font.},descwidth={30em}}
	\begin{tabu} to 40em {|ccccccc|}	\hline	% 8 columns
		\rowcolor{rowcolour}$Q$ & $AFC_{1}$ & $MC_{1}=AVC_{1}$ & $ATC_{1}$ & $AFC_{2}$ & $MC_{2}=AVC_{2}$ & $ATC_{2}$ \\ \hline
		20 & 50 & 30 & \textbf{80} & 100 & 25 & 125 \\ 
		\rowcolor{rowcolour}40 & 25 & 30 & \textbf{55} & 50 & 25 & 75 \\ 
		60 & 16.67 & 30 & \textbf{46.67} & 33.33 & 25 & 58.33 \\ 
		\rowcolor{rowcolour}80 & 12.5 & 30 & \textbf{42.5} & 25 & 25 & 50 \\ 
		100 & 10 & 30 & \textbf{40} & 20 & 25 & 45 \\ 
		\rowcolor{rowcolour}120 & 8.33 & 30 & \textbf{38.33} & 16.67 & 25 & 41.67 \\ 
		140 & 7.14 & 30 & \textbf{37.14} & 14.29 & 25 & 39.29 \\ 
		\rowcolor{rowcolour}160 & 6.25 & 30 & \textbf{36.25} & 12.5 & 25 & 37.5 \\ 
		180 & 5.56 & 30 & \textbf{35.56} & 11.11 & 25 & 36.11 \\ 
		\rowcolor{rowcolour}\textit{200} & \textit{5} & \textit{30} & \textbf{35} & \textit{10} & \textit{25} & \textbf{35} \\ 
		220 & 4.55 & 30 & 34.55 & 9.09 & 25 & \textbf{34.09} \\ 
		\rowcolor{rowcolour}240 & 4.17 & 30 & 34.17 & 8.33 & 25 & \textbf{33.33} \\ 
		260 & 3.85 & 30 & 33.85 & 7.69 & 25 & \textbf{32.69} \\ 
		\rowcolor{rowcolour}280 & 3.57 & 30 & 33.57 & 7.14 & 25 & \textbf{32.14} \\ \hline
	\end{tabu}
\end{Table}

\newhtmlpage

\subsection*{Long-run costs -- a simple numerical example}

Kitt is an automobile designer specializing in the production of off-road
vehicles sold to a small clientele. He has a choice of two (and only two)
plant sizes; one involving mainly labour and the other employing robots
extensively. The set-up (i.e. fixed) costs of these two assembly
plants are \$1 million and \$2 million respectively. The advantage to having
the more costly plant is that the pure production costs (variable costs) are
less. The cost components are defined in Table~\ref{table:latcelementstwoplants}. The variable cost (equal
to the marginal cost here) is \$30,000 in the plant that relies primarily on
labour, and \$25,000 in the plant that has robots. The $ATC$ for each plant
size is the sum of $AFC$ and $AVC$. The $AFC$ declines as the fixed cost is
spread over more units produced. The variable cost per unit is constant in
each case. By comparing the fourth and final columns, it is clear
that the robot-intensive plant has lower costs if it produces a large number of
vehicles. At an output of 200 vehicles the average costs in each plant are
identical: The higher fixed costs associated with the robots are exactly
offset by the lower variable costs at this output level.

The $ATC$ curve corresponding to each plant size is given in 
Figure~\ref{fig:ATCforTwoPlants}. There are two short-run $ATC$
curves. The positions of these curves indicate that if the manufacturer
believes he can produce at least 200 vehicles his unit costs will be less
with the plant involving robots; but at output levels less than this his
unit costs would be less in the labour-intensive plant.

% Figure 8.6
\begin{TikzFigure}{xscale=1,yscale=1,caption={LATC for two plants in \$000 \label{fig:ATCforTwoPlants}}}
\begin{axis}[
	axis line style=thick,
	every tick label/.append style={font=\footnotesize},
	every node near coord/.append style={font=\scriptsize},
	xticklabel style={anchor=north,/pgf/number format/1000 sep=},
	scaled y ticks=false,
	x=1.4cm/50,
	y=1.15cm/10,
	yticklabel style={/pgf/number format/fixed,/pgf/number format/1000 sep = \thinspace},
	xmin=0,xmax=400,ymin=0,ymax=60,
	xlabel={Quantity},
	ylabel={Cost},
]
\addplot[ultra thick,atccolour,mark=none] coordinates { %
	(20,80)
	(40,55)
	(60,46.67)
	(80,42.5)
	(100,40)
	(120,38.33)
	(140,37.14)
	(160,36.25)
	(180,35.56)
	(200,35)
	(220,34.55)
	(240,34.17)
	(260,33.85)
	(280,33.57)
	(300,33.333)
	(320,33.125)
	(340,32.941)
	(360,32.778)
	(380,32.632)
	(400,32.5)
} node [mynode,above,black,pos=0.8] {$ATC_1$};
\addplot[ultra thick,atccolouralt,mark=none] coordinates { %
	(20,125)
	(40,75)
	(60,58.33)
	(80,50)
	(100,45)
	(120,41.67)
	(140,39.29)
	(160,37.5)
	(180,36.11)
	(200,35)
	(220,34.09)
	(240,33.33)
	(260,32.69)
	(280,32.14)
	(300,31.667)
	(320,31.25)
	(340,30.882)
	(360,30.556)
	(380,30.263)
	(400,30)
} node [mynode,below,black,pos=0.8] {$ATC_2$};
\end{axis}
\end{TikzFigure}

The long-run average cost curve for this producer is the lower envelope of
these two cost curves: $ATC_1$ up to output 200 and $ATC_2$ thereafter.
Two features of this example are to be noted. First we do not encounter
decreasing returns -- the $LATC$ curve never increases. Second, in the
interests of simplicity we have assumed just two plant sizes are possible.
With more possibilities on the introduction of robots we could imagine more
short-run ATC curves which would form the lower-envelope $LATC$.
