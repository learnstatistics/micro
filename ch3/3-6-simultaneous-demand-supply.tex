\section{Simultaneous supply and demand impacts}\label{sec:ch3sec6}

In the real world, demand and supply frequently shift at the same time. We
present such a case in Figure~\ref{fig:housedemandsupply}. It is based upon
real estate data describing the housing market in a small Montreal
municipality. Vertical curves define the supply side of the market. Such
vertical curves mean that a fixed number of homeowners decide to put their
homes on the market, and these suppliers just take whatever price results in
the market. In this example, fewer houses were offered for sale in 2002
(less than 50) than in 1997 (more than 70). We are assuming in this market
that the houses traded were similar; that is, we are not lumping together
mansions with row houses.

During this time period household incomes increased substantially and, also,
mortgage rates fell. Both of these developments shifted the demand curve
upward/outward: Buyers were willing to pay more for housing in 2002 than in
1997, both because their incomes were on average higher and because they
could borrow more cheaply.

The shifts on both sides of the market resulted in a higher average price.
And each of these shifts compounded the other: The outward shift in demand
would lead to a higher price on its own, and a reduction in supply would do
likewise. Hence both forces acted to push up the price in 2002. If, instead,
the supply had been greater in 2002 than in 1997 this would have acted to
reduce the equilibrium price. And with the demand and supply shifts
operating in opposing directions, it is not possible to say in general
whether the price would increase or decrease. If the demand shift were
strong and the supply shift weak then the demand forces would have dominated
and led to a higher price. Conversely, if the supply forces were stronger
than the demand forces.

% Figure 3.5
\begin{TikzFigure}{xscale=0.4,yscale=0.35,descwidth=25em,caption={A model of the housing market with shifts in demand and supply \label{fig:housedemandsupply}},description={The vertical supply denotes a fixed number of houses supplied each year. Demand was stronger in 2002 than in 1997 both on account of higher incomes and lower mortgage rates. Thus the higher price in 2002 is due to both a reduction in supply and an increase in demand.}}
\draw [demandcolour,ultra thick,name path=demand2002] (5,18) node [black,mynode,left] {$D_{2002}$} -- (17,12);
\draw [demandcolour,ultra thick,name path=demand1997] (5,11.25) node [black,mynode,left] {$D_{1997}$} -- (16,8.5);
\draw [supplycolour,ultra thick,name path=supply2002] (9,0) -- (9,19) node [black,mynode,above] {$S_{2002}$};
\draw [supplycolour,ultra thick,name path=supply1997] (14,0) -- (14,19) node [black,mynode,above] {$S_{1997}$};
\draw [thick, -] (0,20) node (yaxis) [mynode1,above] {Price in\\ \$000} |- (20,0) node (xaxis) [right] {Quantity};
% intersection of supply and demand lines
\draw [name intersections={of=demand2002 and supply2002,by=E2002},name intersections={of=demand1997 and supply1997,by=E1997}]
	[dotted,thick] (yaxis |- E2002) -- (E2002) node [mynode,above right] {$E_{2002}$}
	[dotted,thick] (yaxis |- E1997) -- (E1997) node [mynode,above right] {$E_{1997}$};
% axis markers
\draw [thick] (5,0) node [mynode,below] {25} -- +(0,0.2) -- +(0,-0.2);
\draw [thick] (10,0) node [mynode,below] {50} -- +(0,0.2) -- +(0,-0.2);
\draw [thick] (15,0) node [mynode,below] {75} -- +(0,0.2) -- +(0,-0.2);
\draw [thick] (0,5) node [mynode,left] {100} -- +(0.2,0) -- +(-0.2,0);
\draw [thick] (0,10) node [mynode,left] {200} -- +(0.2,0) -- +(-0.2,0);
\draw [thick] (0,15) node [mynode,left] {300} -- +(0.2,0) -- +(-0.2,0);
\end{TikzFigure}
