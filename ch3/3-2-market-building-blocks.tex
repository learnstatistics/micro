\section{The market's building blocks}\label{sec:ch3sec2}

In economics we use the terminology that describes trade in a particular
manner. Non-economists frequently describe microeconomics by saying ``it's
all about supply and demand''. While this is largely true we need to define
exactly what we mean by these two central words.
\terminology{Demand} is the quantity of a good or service that buyers wish
to purchase at each conceivable price, with all other influences on demand
remaining unchanged. It reflects a multitude of values, not a single value.
It is not a single or unique quantity such as two cell phones, but rather a
full description of the quantity of a good or service that buyers would
purchase at various prices.

\begin{DefBox}
\textbf{Demand} is the quantity of a good or service that buyers wish to purchase at each possible price, with all other influences on demand remaining unchanged.
\end{DefBox}

\newhtmlpage

As a hypothetical example, the first column of Table~\ref{table:dsnaturalgas} shows the
price of natural gas per cubic foot. The second column shows the quantity
that would be purchased in a given time period at each price. It is
therefore a schedule of quantities demanded at various prices. For example, at a price \$6 per unit, buyers
would like to purchase 4 units, whereas at the lower price of \$3 buyers
would like to purchase 7 units. Note also that this is a homogeneous good. A
cubit foot of natural gas is considered to be the same product no matter
which supplier brings it to the market. In contrast, accomodations supplied
through \textit{Airbnb} are heterogeneous -- they vary in size and quality.

% Table 3.1
\begin{Table}{caption={Demand and supply for natural gas \label{table:dsnaturalgas}}}
\begin{tabu} to \linewidth {|X[0.5,c]X[1.25,c]X[1.25,c]X[1.25,c]|} \hline 
	\rowcolor{rowcolour} \textbf{Price (\$)} & \textbf{Demand (thousands} & \textbf{Supply (thousands} & \textbf{Excess} \\[-0.3em]
	\rowcolor{rowcolour} 					 & \textbf{of cu feet)}	&	\textbf{of cu feet)}	&	\\	\hline
	10 & 0 & 18 & \multirow{6}{*}{\textbf{Excess Supply}}	\\
	\rowcolor{rowcolour}	9 & 1 & 16 &	\cellcolor{white}	\\
	8 & 2 & 14	&	\\
	\cellcolor{rowcolour}7 & \cellcolor{rowcolour}3 & \cellcolor{rowcolour}12 &	\\
	6 & 4 & 10 &	\\
	\cellcolor{rowcolour}5 & \cellcolor{rowcolour}5 & \cellcolor{rowcolour}8 &	\\
	4 & 6 & 6 &	\cellcolor{rowcolour}\textbf{Equilibrium}	\\
	\cellcolor{rowcolour}3 & \cellcolor{rowcolour}7 & \cellcolor{rowcolour}4 & \multirow{4}{*}{\textbf{Excess Demand}} \\
	2 & 8 & 2 &		\\
	\cellcolor{rowcolour}1 & \cellcolor{rowcolour}9 & \cellcolor{rowcolour}0 &	\\
	0 & 10 & 0 &	\\	\hline
\end{tabu}
\end{Table}

\newhtmlpage

Supply is interpreted in a similar manner. It is not a single value; we 
say that \terminology{supply} is the quantity of a good or service
that sellers are willing to sell at each possible price, with all other
influences on supply remaining unchanged. Such a supply schedule is defined
in the third column of the table. It is assumed that no supplier can make a
profit (on account of their costs) unless the price is at least \$2 per
unit, and therefore a zero quantity is supplied below that price. The higher
price is more profitable, and therefore induces a greater quantity supplied,
perhaps by attracting more suppliers. This is reflected in the data. For
example, at a price of \$3 suppliers are willing to supply 4 units, whereas
with a price of \$7 they are willing to supply 12 units. There is thus a
positive relationship between price and quantity for the supplier -- a higher
price induces a greater quantity; whereas on the demand side of the market a
higher price induces a lower quantity demanded -- a negative relationship.

\begin{DefBox}
\textbf{Supply} is the quantity of a good or service that sellers are willing to sell at each possible price, with all other influences on supply remaining unchanged.
\end{DefBox}

We can now identify a key difference in terminology -- between the words
demand and quantity demanded, and between supply and quantity supplied.
While the words demand and supply refer to the complete schedules of demand
and supply, the terms \terminology{quantity demanded} and %
\terminology{quantity supplied} each define a single value of demand or
supply at a particular price.

\begin{DefBox}
\textbf{Quantity demanded} defines the amount purchased at a particular price.

\textbf{Quantity supplied} refers to the amount supplied at a particular price.
\end{DefBox}

Thus while the non-economist may say that when some fans did not get tickets
to the Stanley Cup it was a case of demand exceeding supply, as economists
we say that the quantity demanded exceeded the quantity supplied \textit{at
the going price of tickets}. In this instance, had every ticket been offered
at a sufficiently high price, the market could have generated an excess
supply rather than an excess demand. A higher ticket price would reduce the 
\textit{quantity demanded}; yet would not change \textit{demand}, because
demand refers to the whole schedule of possible quantities demanded at
different prices.

\newhtmlpage

\subsection*{Other things equal -- ceteris paribus}

The demand and supply schedules rest on the assumption that all other
influences on supply and demand remain the same as we move up and down the
possible price values. The expression \textit{other things being equal%
}, or its Latin counterpart \textit{ceteris paribus}, describes this
constancy of other influences. For example, we assume on the demand side
that the prices of other goods remain constant, and that tastes and incomes are
unchanging. On the supply side we assume, for example, that there is no
technological change in production methods. If any of these elements change
then the market supply or demand schedules will reflect such changes. For
example, if coal or oil prices increase (decline) then some buyers may switch
to (away from) gas. This will be reflected in the data: At any given price
more (or less) will be demanded. We will illustrate this in graphic form
presently.

\newhtmlpage

\subsection*{Market equilibrium}

Let us now bring the demand and supply schedules together in an attempt to
analyze what the marketplace will produce -- will a single price emerge
that will equate supply and demand? We will keep other things constant for
the moment, and explore what materializes at different prices. At low
prices, the data in Table~\ref{table:dsnaturalgas} indicate that the
quantity demanded exceeds the quantity supplied -- for example, verify what
happens when the price is \$3 per unit. The opposite occurs when the price
is high -- what would happen if the price were \$8? Evidently, there exists
an intermediate price, where the quantity demanded equals the quantity
supplied. At this point we say that the market is in equilibrium. The %
\terminology{equilibrium price} equates demand and supply -- it clears the
market.

\begin{DefBox}
The \textbf{equilibrium price} equilibrates the market. It is the price at which quantity demanded equals the quantity supplied.
\end{DefBox}

In Table~\ref{table:dsnaturalgas} the equilibrium price is \$4, and the
equilibrium quantity is 6 thousand cubic feet of gas (we will use the
notation `k' to denote thousands). At higher prices there is an %
\terminology{excess supply}---suppliers wish to sell more than buyers wish
to buy. Conversely, at lower prices there is an \terminology{excess demand}.
Only at the equilibrium price is the quantity supplied equal to the quantity demanded.

\begin{DefBox}
\textbf{Excess supply} exists when the quantity supplied exceeds the quantity demanded at the going price.

\textbf{Excess demand} exists when the quantity demanded exceeds the quantity supplied at the going price.
\end{DefBox}

Does the market automatically reach equilibrium? To answer this question,
suppose initially that the sellers choose a price of \$10. Here suppliers
would like to supply 18k cubic feet, but there are no buyers---a situation
of extreme excess supply. At the price of \$7 the excess supply is reduced
to 9k, because both the quantity demanded is now higher at 3k units, and the
quantity supplied is lower at 12k. But excess supply means that there are
suppliers willing to supply at a lower price, and this willingness exerts
continual downward pressure on any price above the price that equates demand
and supply.

\newhtmlpage

At prices below the equilibrium there is, conversely, an excess demand. In
this situation, suppliers could force the price upward, knowing that buyers
will continue to buy at a price at which the suppliers are willing to sell.
Such upward pressure would continue until the excess demand is eliminated.

In general then, above the equilibrium price excess supply exerts downward
pressure on price, and below the equilibrium excess demand exerts upward
pressure on price. This process implies that the buyers and sellers have
information on the various elements that make up the marketplace.

We will explore later in this chapter some specific circumstances in which
trading could take place at prices above or below the equilibrium price. In
such situations the quantity actually traded always corresponds to the short
side of the market: At high prices the quantity demanded is less than
the quantity supplied, and it is the quantity demanded that is traded because buyers will
not buy the amount suppliers would like to supply. At low prices the
quantity demanded exceeds quantity supplied, and it is the amount that
suppliers are willing to sell that is traded. In sum, when trading takes
place at prices other than the equilibrium price it is always the lesser of
the quantity demanded or supplied that is traded. Hence we say that at non-equilibrium
prices the \terminology{short side} dominates. We will return to
this in a series of examples later in this chapter.

\begin{DefBox}
The \textbf{short side of the market} determines outcomes at prices other than the equilibrium.
\end{DefBox}

\subsection*{Supply and the nature of costs}

Before progressing to a graphical analysis, we should add a word about
costs. The supply schedules are based primarily on the cost of producing the
product in question, and we frequently assume that all of the costs
associated with supply are incorporated in the supply schedules. In
Chapter~\ref{chap:individualchoice} we will explore cases where costs additional to
those incurred by producers may be relevant. For example, coal burning power
plants emit pollutants into the atmosphere; but the individual supplier may
not take account of these pollutants, which are costs to society at large,
in deciding how much to supply at different prices. Stated another way, the
private costs of production would not reflect the total, or full social
costs of production. For the moment the assumption is that no such
additional costs are associated with the markets we analyze.