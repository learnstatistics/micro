\section{Demand and supply curves}\label{sec:ch3sec3}

The \terminology{demand curve} is a graphical expression of the relationship
between price and quantity demanded, holding other things constant. Figure~\ref{fig:measurepricequantity}
measures price on the vertical axis and quantity on the
horizontal axis. The curve $D$ represents the data from the first two
columns of Table~\ref{table:dsnaturalgas}. Each combination of price and
quantity demanded lies on the curve. In this case the curve is \textit{linear%
}---it is a straight line. The demand curve slopes downward (technically we
say that its slope is negative), reflecting the fact that buyers wish to
purchase more when the price is less.

% figure 3.1 (called 3.1a in Ian's email)
\begin{TikzFigure}{xscale=0.62,yscale=0.52,caption={Measuring price \& quantity \label{fig:measurepricequantity}}}
% axis and axis marks
\draw [thick, -] (0,12) node (yaxis) [above] {Price} |- (12,0) node (xaxis) [right] {Quantity};
\draw \foreach \p in {0,2,4,6,8,10,12} {(\p,0) -- +(0,-0.25) node [below,mynode] {\p} (0,\p) -- +(-0.25,0) node [left,mynode] {\p}};
\draw \foreach \p in {1,3,5,7,9,11} {(\p,0) -- +(0,-0.125) (0,\p) -- +(-0.125,0)};
% circles at certain points
\draw [demandcolour,mark options={fill=demandcolour},only marks] plot[mark=*,mark size=+4pt] coordinates {(0,10) (1,9) (2,8) (3,7) (4,6) (5,5) (6,4) (7,3) (8,2) (9,1) (10,0)};
% line through points
\draw [thick,demandcolour,name path=D] (0,10) -- coordinate[pos=0.85] (Dpoint) (10,0);
\draw [thick,<-,shorten <=0.5mm,shorten >=-1mm] (Dpoint) -- +(1.5,1.5) node [mynode,above right] {$D$};
% nodes at (2,8) and (7,3)
\node [mynode,above right] at (2,8) {$h$};
\node [mynode,above right] at (7,3) {$l$};
\end{TikzFigure}

\newhtmlpage

To derive this demand curve we take each price-quantity combination from the
demand schedule in Table~\ref{table:dsnaturalgas} and insert a point that corresponds to those
combinations. For example, point $h$ defines the combination $%
\{P=\$8,Q_d=2\}$, the point $l$ denotes the combination $\{P=\$3,Q_d=7\}$.
If we join all such points we obtain the demand curve in Figure~\ref{fig:sdeq}. In
this particular case the demand schedule results in a straight-line, or
linear, demand curve. The same process yields the supply curve in Figure~\ref{fig:sdeq}.

\begin{DefBox}
The \textbf{demand curve} is a graphical expression of the relationship between price and quantity demanded, with other influences remaining unchanged.
\end{DefBox}

The \terminology{supply curve} is a graphical representation of the relationship
between price and quantity supplied, holding other things constant.
The supply curve $S$ in Figure~\ref{fig:sdeq} is based on the data from
columns 1 and 3 in Table~\ref{table:dsnaturalgas}. It, too, is linear, but
has a positive slope indicating that suppliers wish to supply more at higher
prices.

\begin{DefBox}
The \textbf{supply curve} is a graphical expression of the relationship between price and quantity supplied, with other influences remaining unchanged.
\end{DefBox}

\newhtmlpage

% figure 3.2
\begin{TikzFigure}{xscale=0.52,yscale=0.42,caption={Supply, demand, equilibrium \label{fig:sdeq}}}
\draw [demandcolour,ultra thick,name path=demand] (0,10) node [black,mynode,left] {10} -- node [mynode,black,above right,pos=0.25] {Demand} (10,0) node [black,mynode,below] {10};
\draw [supplycolour,ultra thick,name path=supply] (0,1) node [black,mynode,left] {1} -- (10,6) node [black,mynode,above] {Supply};
\draw [thick, -] (0,15) node (yaxis) [above] {Price} |- (15,0) node (xaxis) [right] {Quantity};
% intersection of demand and supply and dotted line
\draw [name intersections={of=demand and supply, by=E}]
	[dotted,thick] (yaxis |- E) node [mynode,left] {4} -- (E) node [mynode,above] {$E_0$} -- (xaxis -| E) node [mynode,below] {6};
\end{TikzFigure}

The demand and supply curves intersect at point $E_0$, corresponding to a
price of \$4 which, as illustrated above, is the equilibrium price for this
market. At any price below this the horizontal distance between the supply
and demand curves represents excess demand, because demand exceeds supply.
Conversely, at any price above \$4 there is an excess supply that is again
measured by the horizontal distance between the two curves. Market forces
tend to eliminate excess demand and excess supply as we explained above. In
the final section of the chapter we illustrate how the supply and demand
curves can be `solved' for the equilibrium price and quantity.