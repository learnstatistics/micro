\newpage
\section{Useful techniques -- demand and supply equations}\label{sec:ch3sec9}

The supply and demand functions, or equations, underlying Table~\ref{table:dsnaturalgas}
and Figure~\ref{fig:sdeq} can be written in their mathematical form:
\begin{equation*}
\text{Demand: }P=10-Q 
\end{equation*}
\begin{equation*}
\text{Supply: }P=1+(1/2)Q 
\end{equation*}

A straight line is represented completely by the intercept and slope. In
particular, if the variable $P$ is on the vertical axis and $Q$ on the
horizontal axis, the straight-line equation relating $P$ and $Q$ is defined
by $P=a+bQ$. Where the line is negatively sloped, as in the demand equation,
the parameter $b$ must take a negative value. By observing either the data in
Table~\ref{table:dsnaturalgas} or Figure~\ref{fig:sdeq} it is clear that the
vertical intercept, $a$, takes a value of \$10. The vertical intercept
corresponds to a zero-value for the $Q$ variable. Next we can see from
Figure~\ref{fig:sdeq} that the slope (given by the rise over the run) is 10/10
and hence has a value of $-1$. Accordingly the demand equation takes the form
$P=10-Q$.

On the supply side the price-axis intercept, from either the figure or the
table, is clearly 1. The slope is one half, because a two-unit change in
quantity is associated with a one-unit change in price. This is a positive
relationship obviously so the supply curve can be written as $P=1+(1/2)Q$.

Where the supply and demand curves intersect is the market equilibrium; that
is, the price-quantity combination is the same for both supply and demand
where the supply curve takes on the same values as the demand curve. This
unique price-quantity combination is obtained by equating the two curves: If
Demand$=$Supply, then
\begin{equation*}
10-Q=1+(1/2)Q. 
\end{equation*}
Gathering the terms involving $Q$ to one side and the numerical terms to the
other side of the equation results in $9=1.5Q.$ This implies that the
equilibrium quantity must be 6 units. And this quantity must trade at a
price of \$4. That is, when the price is \$4 both the quantity demanded
and the quantity supplied take a value of 6 units.

\newhtmlpage

\subsection*{Modelling market interventions using equations}

To illustrate the impact of market interventions examined in Section~\ref{sec:ch3sec7} on
our numerical market model for natural gas, suppose that the
government imposes a minimum price of \$6 -- above the equilibrium price
obviously. We can easily determine the quantity supplied and demanded at
such a price. Given the supply equation
\begin{equation*}
P=1+(1/2)Q, 
\end{equation*}
it follows that at $P=6$ the quantity supplied is 10. This follows by solving
the relationship $6=1+(1/2)Q$ for the value of $Q$. Accordingly, suppliers
\textit{would like to supply} 10 units at this price.

Correspondingly on the demand side, given the demand curve
\begin{equation*}
P=10-Q, 
\end{equation*}
with a price given by $P=\$6$, it must be the case that $Q=4$. So buyers 
\textit{would like to buy} 4 units at that price: There is excess supply.
But we know that the short side of the market will win out, and so the
actual amount traded at this restricted price will be 4 units.

\section*{Conclusion}

We have covered a lot of ground in this chapter. It is intended to open up
the vista of economics to the new student in the discipline. Economics is
powerful and challenging, and the ideas we have developed here will serve as
conceptual foundations for our exploration of the subject. Our next chapter
deals with measurement and responsiveness.
