\section{Non-price influences on supply}\label{sec:ch3sec5}

To date we have drawn supply curves with an upward slope. Is this a
reasonable representation of supply in view of what is frequently observed
in markets? We suggested earlier that the various producers of a particular
good or service may have different levels of efficiency. If so, only the
more efficient producers can make a profit at a low price, whereas at higher
prices more producers or suppliers enter the market -- producers who may not
be as lean and efficient as those who can survive in a lower-price
environment. This view of the world yields a positively-sloping supply curve.

As a second example, consider \textit{Uber} taxi drivers. Some drivers may
be in serious need of income and may be willing to drive for a low hourly
rate. For other individuals driving may be a secondary source of income, and
such drivers are less likely to want to drive unless the hourly wage is
higher. Consequently if \textit{Uber} needs a large number of drivers at any one time
it may be necessary to pay a higher wage -- \textit{and charge a higher fare
to passengers}, to induce more drivers to take their taxis onto the road.
This phenomenon corresponds to a positively-sloped supply curve.

In contrast to these two examples, some suppliers simply choose a unique
price and let buyers purchase as much as they want at that price. This is
the practice of most retailers. For example, the price of \textit{Samsung's
Galaxy} is typically fixed, no matter how many are purchased -- and tens of
millions are sold at a fixed price when a new model is launched. \textit{%
Apple} also sets a price, and buyers purchase as many as they desire at that
price. This practice corresponds to a horizontal supply cuve: The price does
not vary and the market equilibrium occurs where the demand curve intersects
this supply curve.

In yet other situations supply is fixed. This happens in auctions. Bidders
at the auction simply determine the price to be paid. At a real estate
auction a given property is put on the market and the price is determined by
the bidding process. In this case the supply of a single property is
represented by a vertical supply at a quantity of 1 unit.

Regardless of the type of market we encounter, however, it is safe to assume
that supply curves do not slope downward. So, for the moment, we adopt the
stance that supply curves are generally upward sloping -- somewhere between
the extremes of being vertical or horizontal -- as we have drawn them to
this point.

Next, we examine those other influences that underlie supply curves.
Technology, input costs, the prices of competing goods, expectations and the
number of suppliers are the most important.

\newhtmlpage

\subsection*{Technology -- computers and fracking}

A technological advance may involve an idea that allows more output to be
produced with the same inputs, or an equal output with fewer inputs. A good
example is \textit{just-in-time} technology. Before the modern era, auto
manufacturers kept large stocks of components in their production
facilities, but developments in communications and computers at that time
made it possible for manufacturers to link directly with their input
suppliers. Nowadays assembly plants place their order for, say, seat
delivery to their local seat supplier well ahead of assembly time. The seats
swing into the assembly area hours or minutes before assembly---just in
time. The result is that the assembler reduces his seat inventory (an input)
and thereby reduces production cost.

Such a technology-induced cost saving is represented by moving the supply
curve downward or outward: The supplier is now able and willing to supply
the same quantity at a lower price because of the technological innovation.
Or, saying the same thing slightly differently, suppliers will supply more
at a given price than before.

A second example relates to the extraction of natural gas. The development
of `fracking' means that companies involved in gas recovery can now do so at
a lower cost. Hence they are willing to supply any given quantity at a lower
price.

\subsection*{Input costs -- green power}

Input costs can vary independently of technology. For example, a wage
negotiation that grants workers a substantial pay raise will increase the 
cost of production. This is reflected in a leftward, or
upward, supply shift: Any quantity supplied is now priced higher;
alternatively, suppliers are willing to supply less at the going price.

As a further example, suppose the government decrees that power-generating
companies must provide a certain percentage of their power using `green'
sources -- from solar power or windmills. Since such sources are not yet as
cost efficient as more conventional power sources, the electricity they
generate comes at a higher cost.

\newhtmlpage

\subsection*{Competing products -- Airbnb versus hotels}

If competing products improve in quality or fall in price, a supplier may be
forced to follow suit. For example, \textit{Asus} and \textit{Dell} are
constantly watching each other's pricing policies. If \textit{Dell} brings
out a new generation of computers at a lower price, \textit{Asus} may lower
its prices in turn---which is to say that Asus' supply curve will shift
downward. Likewise, \textit{Samsung} and \textit{Apple} each responds to the
other's pricing and technology behaviours. The arrival of new products in
the marketplace also impacts the willingness of suppliers to supply goods at
a given price. New intermediaries such as \textit{Airbnb} and
\textit{Vacation Rentals by Owner} have shifted the supply curves of hotel
rooms downward.

These are some of the many factors that influence the position of the supply
curve in a given market.

\begin{ApplicationBox}{caption={The price of light \label{app:pricelight}}}
Technological developments have had a staggering impact on many price declines. Professor William Nordhaus of Yale University is an expert on measuring technological change. He has examined the trend in the real price of lighting. Originally, light was provided by whale oil and gas lamps and these sources of lumens (the scientific measure of the amount of light produced) were costly. In his research, Professor Nordhaus pieced together evidence on the actual historic cost of light produced at various times, going all the way back to 1800. He found that light in 1800 cost about 100 times more than in 1900, and light in the year 2000 was a fraction of its cost in 1900. A rough calculation suggests that light was five hundred times more expensive at the start of this 200-year period than at the end, and this was before the arrival of LEDs.

In terms of supply and demand analysis, light has been subject to very substantial downward supply shifts. Despite the long-term growth in demand, the technologically-induced supply changes have been the dominant factor in its price determination.

For further information, visit Professor Nordhaus's website in the Department of Economics at Yale University.
\end{ApplicationBox}

\newhtmlpage

\subsection*{Shifts in supply}

Whenever technology changes, or the costs of production change, or the
prices of competing products adjust, then one of our \textit{ceteris paribus}
assumptions is violated. Such changes are generally reflected by shifting
the supply curve. Figure~\ref{fig:supplyshift} illustrates the impact of the
arrival of just-in-time technology. The supply curve shifts, reflecting the
ability of suppliers to supply the same output at a reduced price. The
resulting new equilibrium price is lower, since production costs have
fallen. At this reduced price more gas is traded at a lower price.

% Figure 3.4
\begin{TikzFigure}{xscale=0.52,yscale=0.42,descwidth=25em,caption={Supply shift and new equilibrium \label{fig:supplyshift}},description={The supply curve shifts due to lower production costs. A new equilibrium $E_1$ is attained in the market at a lower price.}}
\draw [demandcolour,ultra thick,name path=demand] (0,10) node [black,mynode,left] {10} -- (10,0) node [black,mynode,below] {10};
\draw [supplycolour,ultra thick,name path=supply0] (0,1) node [black,mynode,left] {1} -- (10,6) node [black,mynode,right] {Supply};
\draw [supplycolour,ultra thick,name path=supply1] (0,1) -- (11,4) node [black,mynode,right] {New supply};
\draw [thick, -] (0,15) node (yaxis) [above] {Price} |- (15,0) node (xaxis) [right] {Quantity};
% intersection of supply lines with demand line
\draw [name intersections={of=demand and supply0, by=E0},name intersections={of=demand and supply1,by=E1}]
	[dotted,thick] (yaxis |- E0) node [mynode,left] {4} -- (E0) node [mynode,above] {$E_0$} -- (xaxis -| E0) node [mynode,below] {6}
	[dotted,thick] (yaxis |- E1) -- (E1) node [mynode,above] {$E_1$} -- (xaxis -| E1);
% arrow from old to new supply line
\path [name path=line1] (6,15) -- (12,0);
\draw [name intersections={of=line1 and supply0, by=S0},name intersections={of=line1 and supply1, by=S1}]
	[->,thick,shorten >=1mm,shorten <=1mm] (S0) -- (S1);
\end{TikzFigure}
