\section{Non-price influences on demand}\label{sec:ch3sec4}

We have emphasized several times the importance of the \textit{ceteris
paribus} assumption when exploring the impact of different prices on the
quantity demanded: We assume all other influences on the purchase decision
are unchanged (at least momentarily). These other influences fall into
several broad categories: The prices of related goods; the incomes of
buyers; buyer tastes; and expectations about the future. Before proceeding,
note that we are dealing with \textit{market} demand rather than demand by
one \textit{individual} (the precise relationship between the two is
developed later in this chapter).

\subsection*{The prices of related goods -- oil and gas, Kindle and paperbacks}

We expect that the price of other forms of energy would impact the price of
natural gas. For example, if electricity, oil or coal becomes less expensive
we would expect some buyers to switch to these other products.
Alternatively, if gas-burning furnaces experience a technological
breakthrough that makes them more efficient and cheaper we would expect some
users of other fuels to move to gas. Among these examples, it is clear that
oil and electricity are substitute fuels for gas; in contrast the efficient
new gas furnace complements the use of gas. We use these terms, %
\terminology{substitutes} and \terminology{complements}, to describe
products that influence the demand for the primary good.

\begin{DefBox}
\textbf{Substitute goods}: when a price reduction (rise) for a related product reduces (increases) the demand for a primary product, it is a substitute for the primary product.

\textbf{Complementary goods}: when a price reduction (rise) for a related product increases (reduces) the demand for a primary product, it is a complement for the primary product.
\end{DefBox}

Clearly electricity is a substitute for gas in the power market, whereas a
gas furnace is a complement for gas as a fuel. The words substitutes and
complements immediately suggest the nature of the relationships. Every
product has complements and substitutes. As another example: Electronic
readers such as Kindle, Nook and Kobo are substitutes for paper-form books;
a rise in the price of paper books should increase the demand for electronic
readers at any given price for electronic readers. In graphical terms, the
demand curve \textit{shifts} in response to changes in the prices of other
goods -- an increase in the price of paper-form books will shift the demand
for electronic readers outward, because more electronic readers will be
demanded at any price.

\newhtmlpage

\subsection*{Buyer incomes -- which goods to buy}

The demand for most goods increases in response to income increases. Given
this, the demand curve for gas will shift outward if household incomes in
the economy increase. Household incomes may increase either because there
are more households in the economy or because the incomes of the existing
households grow.

Most goods are demanded in greater quantity in response to higher incomes at
any given price. But there are exceptions. For example, public transit
demand may decline at any price when household incomes rise, because some
individuals move to cars. Or the demand for laundromats may decline in
response to higher incomes, as households purchase more of their own
consumer durables -- washers and driers. We use the term %
\terminology{inferior good} to define these cases: An inferior good is one
whose demand declines in response to increasing incomes, whereas a %
\terminology{normal good} experiences an increase in demand in response to
rising incomes.

\begin{DefBox}
An \textbf{inferior good} is one whose demand falls in response to higher incomes.

A \textbf{normal good} is one whose demand increases in response to higher incomes.
\end{DefBox}

There is a further sense in which consumer incomes influence demand, and
this relates to how the incomes are \textit{distributed} in the economy. In
the discussion above we stated that higher total incomes shift demand curves
outwards when goods are normal. But think of the difference in the demand
for electronic readers between Portugal and Saudi Arabia. These economies
have roughly the same average per-person income, but incomes are distributed
more unequally in Saudi Arabia. It does not have a large middle class that
can afford electronic readers or \textit{iPads}, despite the huge wealth
held by the elite. In contrast, Portugal has a relatively larger middle
class that can afford such goods. Consequently, the \textit{distribution of
income} can be an important determinant of the demand for many commodities
and services.

\newhtmlpage

\subsection*{Tastes and networks -- hemlines and homogeneity}

While demand functions are drawn on the assumption that tastes are constant,
in an evolving world they are not. We are all subject to peer pressure, the
fashion industry, marketing, and a desire to maintain our image. If the
fashion industry dictates that lapels or long skirts are \textit{de rigueur}
for the coming season, some fashion-conscious individuals will discard a
large segment of their wardrobe, even though the clothes may be in perfectly
good condition: Their demand is influenced by the dictates of current
fashion.

Correspondingly, the items that other individuals buy or use frequently
determine our own purchases. Businesses frequently decide that all of their
employees will have the same type of computer and software on account of 
\textit{network economies}: It is easier to communicate if equipment is
compatible, and it is less costly to maintain infrastructure where the
variety is less.

\subsection*{Expectations -- betting on the future}

In our natural gas example, if households expected that the price of natural
gas was going to stay relatively low for many years -- perhaps on account of
the discovery of large deposits -- then they would be tempted to purchase a
gas burning furnace rather than an oil burning furnace. In this example, it
is more than the current price that determines choices; \textit{the prices
that are expected to prevail in the future} also determine current demand.

Expectations are particularly important in stock markets. When investors
anticipate that corporations will earn high rewards in the future they will
buy a stock today. If enough people believe this, the price of the stock
will be driven upward on the market, even before profitable earnings are
registered.

\newhtmlpage

\subsection*{Shifts in demand}

The demand curve in Figure~\ref{fig:sdeq} is drawn for a given level of
other prices, incomes, tastes, and expectations. Movements along the demand
curve reflect solely the impact of different prices for the good in
question, holding other influences constant. But changes in any of these
other factors will change the position of the demand curve. Figure~\ref{fig:demandshift}
illustrates a shift in the demand curve. This shift could
result from a rise in household incomes that increase the quantity demanded 
\textit{at every price}. This is illustrated by an outward shift in the
demand curve. With supply conditions unchanged, there is a new equilibrium
at $E_{1}$, indicating a greater quantity of purchases accompanied by a
higher price. The new equilibrium reflects a \textit{change in quantity 
supplied and a change in demand}.

% Figure 3.3
\begin{TikzFigure}{xscale=0.52,yscale=0.42,descwidth=20em,caption={Demand shift and new equilibrium \label{fig:demandshift}},description={The outward shift in demand leads to a new equilibrium $E_1$.}}
\draw [demandcolour,ultra thick,name path=demand1] (0,10) node [black,mynode,left] {10} -- (10,0) node [black,mynode,below] {10};
\draw [demandcolour,ultra thick,name path=demand2] (0,13) -- (13,0);
\draw [supplycolour,ultra thick,name path=supply] (0,1) node [black,mynode,left] {1} -- (10,6) node [black,mynode,right] {Supply};
\draw [thick] (0,15) node (yaxis) [above] {Price} |- (15,0) node (xaxis) [right] {Quantity};
% intersection of demand lines with supply line
\draw [name intersections={of=demand1 and supply, by=E0},name intersections={of=demand2 and supply, by=E1}]
	[dotted,thick] (yaxis |- E0) node [mynode,left] {4} -- (E0) node [mynode,below left=0.25cm and 0cm] {$E_0$} -- (xaxis -| E0) node [mynode,below] {6}
	[dotted,thick] (yaxis |- E1) -- (E1) node [mynode,above] {$E_1$} -- (xaxis -| E1);
% arrow between demand lines
\path [name path=line1] (0,5) -- (8,15);
\draw [name intersections={of=line1 and demand1, by=I1},name intersections={of=line1 and demand2, by=I2}]
	[->,thick,shorten >=1mm,shorten <=1mm] (I1) -- (I2);
\end{TikzFigure}

\newhtmlpage

We may well ask why so much emphasis in our diagrams and analysis is placed
on the relationship between \textit{price} and quantity, rather than on the
relationship between quantity and its other determinants. The answer is that
we could indeed draw diagrams with quantity on the horizontal axis and a
measure of one of these other influences on the vertical axis. But the price
mechanism plays a very important role. \textit{Variations in price are what
equilibrate the market}. By focusing primarily upon the price, we see the
self-correcting mechanism by which the market reacts to excess supply or
excess demand.

In addition, this analysis illustrates the method of %
\terminology{comparative statics}---examining the impact of changing one of
the other things that are assumed constant in the supply and demand diagrams.

\begin{DefBox}
\textbf{Comparative static analysis} compares an initial equilibrium with a new equilibrium, where the difference is due to a change in one of the other things that lie behind the demand curve or the supply curve.
\end{DefBox}

`Comparative' obviously denotes the idea of a comparison, and static means
that we are not in a state of motion. Hence we use these words in
conjunction to indicate that we compare one outcome with another, without
being concerned too much about the transition from an initial equilibrium to
a new equilibrium. The transition would be concerned with dynamics rather
than statics. In Figure~\ref{fig:demandshift} we explain the difference
between the points $E_0$ and $E_1$ by indicating that there has been a
change in incomes or in the price of a substitute good. We do not attempt to
analyze the details of this move or the exact path from $E_0$ to $E_1$.

\newhtmlpage
\newpage

\begin{ApplicationBox}{caption={Corn prices and demand shifts \label{app:cornprice}}}
	In the middle of its second mandate, the Bush Administration in the US decided to encourage the production of ethanol -- a fuel that is less polluting than gasoline. The target production was 35 billion for 2017 -- from a base of 1 billion gallons in 2000. Corn is the principal input in ethanol production. It is also used as animal feed, as a sweetener and as a food for humans. The target was to be met with the help of a subsidy to producers and a tariff on imports of Brazil's sugar-cane based ethanol.
	
	The impact on corn prices was immediate; from a farm-gate price of \$2 per bushel in 2005, the price reached the \$4 range two years later, despite a significant increase in production. In 2012 the price was \$7. While other factors, such as growing incomes, have stimulated the demand for corn; ethanol is seen as the main price driver.
	
	The wider impact of these developments was that the prices of virtually all grains increased in tandem with corn. For example, the prices of sorghum and barley increased because of a switch in land use towards corn. Corn was seen as more profitable, less acreage was allocated to other grains, and the supply of these other grains fell.
	
	While producers benefited from the price rise, consumers -- particularly those in less developed economies -- experienced a dramatic increase in their basic living costs. Visit the site of the United Nations' Food and Agricultural Organization for an assessment.
	
	In terms of supply and demand shifts, the demand side has dominated. The ethanol drive, combined with secular growth in the demand for food, means that the demand for grains shifted outward faster than the supply.
\end{ApplicationBox}