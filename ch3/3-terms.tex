\newpage
	\section*{Key Terms}
\begin{keyterms}
\textbf{Demand} is the quantity of a good or service that buyers wish to purchase at each possible price, with all other influences on demand remaining unchanged.

\textbf{Supply} is the quantity of a good or service that sellers are willing to sell at each possible price, with all other influences on supply remaining unchanged.

\textbf{Quantity demanded} defines the amount purchased at a particular price.

\textbf{Quantity supplied} refers to the amount supplied at a particular price.

\textbf{Equilibrium price}: equilibrates the market. It is the price at which quantity demanded equals the quantity supplied.

\textbf{Excess supply} exists when the quantity supplied exceeds the quantity demanded at the going price.

\textbf{Excess demand} exists when the quantity demanded exceeds quantity supplied at the going price.

\textbf{Short side of the market} determines outcomes at prices other than the equilibrium.

\textbf{Demand curve} is a graphical expression of the relationship between price and quantity demanded, with other influences remaining unchanged.

\textbf{Supply curve} is a graphical expression of the relationship between price and quantity supplied, with other influences remaining unchanged.

\textbf{Substitute goods}: when a price reduction (rise) for a related product reduces (increases) the demand for a primary product, it is a substitute for the primary product.

\textbf{Complementary goods}: when a price reduction (rise) for a related product increases (reduces) the demand for a primary product, it is a complement for the primary product.

\textbf{Inferior good} is one whose demand falls in response to higher incomes.

\textbf{Normal good} is one whose demand increases in response to higher incomes.

\textbf{Comparative static analysis} compares an initial equilibrium with a new equilibrium, where the difference is due to a change in one of the other things that lie behind the demand curve or the supply curve.

\textbf{Price controls} are government rules or laws that inhibit the formation of market-determined prices.

\textbf{Quotas} are physical restrictions on output.

\textbf{Market demand}: the horizontal sum of individual demands.
\end{keyterms}