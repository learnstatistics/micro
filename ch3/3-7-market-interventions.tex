\section{Market interventions -- governments and interest groups}\label{sec:ch3sec7}

The freely functioning markets that we have developed certainly do not
describe all markets. For example, minimum wages characterize the labour
market, most agricultural markets have supply restrictions, apartments are
subject to rent controls, and blood is not a freely traded market commodity
in Canada. In short, price controls and quotas characterize many markets. %
\terminology{Price controls} are government rules or laws that inhibit the
formation of market-determined prices. \terminology{Quotas} are physical
restrictions on how much output can be brought to the market.

\begin{DefBox}
\textbf{Price controls} are government rules or laws that inhibit the formation of market-determined prices. 

\textbf{Quotas} are physical restrictions on output.
\end{DefBox}

Price controls come in the form of either \textit{floors} or \textit{ceilings}.
Price floors are frequently accompanied by \textit{marketing boards}.

\newhtmlpage

\subsection*{Price ceilings -- rental boards}

Ceilings mean that suppliers cannot legally charge more than a specific
price. Limits on apartment rents are one form of ceiling. In times of
emergency -- such as flooding or famine, price controls are frequently
imposed on foodstuffs, in conjunction with rationing, to ensure that access
is not determined by who has the most income. The problem with price
ceilings, however, is that they leave demand unsatisfied, and therefore they
must be accompanied by some other allocation mechanism.

Consider an environment where, for some reason -- perhaps a sudden and
unanticipated growth in population -- rents increase. Let the resulting
equilibrium be defined by the point $E_0$ in Figure~\ref{fig:priceceiling}.
If the government were to decide that this is an unfair price because it
places hardships on low- and middle-income households, it might impose a
price limit, or ceiling, of $P_c$. The problem with such a limit is that
excess demand results: Individuals want to rent more apartments than are
available in the city. In a free market the price would adjust upward to
eliminate the excess demand, but in this controlled environment it cannot.
So some other way of allocating the available supply between demanders must
evolve.

In reality, most apartments are allocated to those households already
occupying them. But what happens when such a resident household decides to
purchase a home or move to another city? In a free market, the landlord
could increase the rent in accordance with market pressures. But in a
controlled market a city's rental tribunal may restrict the annual rent
increase to just a couple of percent and the demand may continue to
outstrip supply. So how does the stock of apartments get allocated between
the potential renters? One allocation method is well known: The existing
tenant informs her friends of her plan to move, and the friends are the
first to apply to the landlord to occupy the apartment. But that still
leaves much unmet demand. If this is a student rental market, students whose
parents live nearby may simply return `home'. Others may chose to move to a
part of the city where rents are more affordable.

\newhtmlpage

% Figure 3.6
\begin{TikzFigure}{xscale=0.52,yscale=0.42,descwidth=25em,caption={The effect of a price ceiling \label{fig:priceceiling}},description={The free market equilibrium occurs at $E_0$. A price ceiling at $P_c$ holds down the price but leads to excess demand $E_c$B, because $Q_c$ is the quantity traded. A price ceiling above $P_0$ is irrelevant since the free market equilibrium $E_0$ can still be attained.}}
\draw [demandcolour,ultra thick,name path=demand] (0,12) -- node [mynode,black,above right,pos=0.9] {Demand} (12,0);
\draw [supplycolour,ultra thick,name path=supply] (0,1) -- (13,14) node [black,mynode,right] {Supply};
\draw [thick, -] (0,15) node (yaxis) [above] {Price} |- (15,0) node (xaxis) [right] {Quantity};
% intersection of supply and demand and dotted lines from axes
\draw [name intersections={of=supply and demand,by=E0}]
	[dotted,thick] (yaxis |- E0) node [mynode,left] {$P_0$} -- (E0) node [mynode,above] {$E_0$} -- (xaxis -| E0) node [mynode,below] {$Q_0$};
% horizontal path from P_c
\path [name path=PCline] ([yshift=-3cm]yaxis |- E0) -- +(15,0);
% intersection of PCline with demand and supply
\draw [name intersections={of=demand and PCline,by=B},name intersections={of=supply and PCline,by=Ec}]
	[dotted,thick] (yaxis |- Ec) node [mynode,left] {$P_c$} -- (Ec) node [mynode,above] {$E_c$} -- (xaxis -| Ec) node [mynode,below] {$Q_c$}
	[dotted,thick] (Ec) -- (B) node [mynode,above] {B} -- (xaxis -| B);
% arrow between Ec and B
\draw [<->,thick,shorten >=0.5mm,shorten <=0.5mm] ([yshift=-0.5cm]Ec) -- coordinate[pos=0.65] (A) ([yshift=-0.5cm]B);
% arrow pointing to near middle of <-> arrow
\draw [<-,thick,shorten >=1mm,shorten <=1mm] (A) to[out=-60,in=260] +(5,4) node [mynode] {Excess demand at $P_c$};
\end{TikzFigure}


However, rent controls sometimes yield undesirable outcomes. Rent
controls are widely studied in economics, and the consequences are well
understood: Landlords tend not to repair or maintain their rental units in
good condition if they cannot obtain the rent they believe they are entitled
to. Accordingly, the residential rental stock deteriorates. In addition,
builders realize that more money is to be made in building condominium units
than rental units, or in \textit{converting rental units to condominiums}.
The frequent consequence is thus a \textit{reduction} in supply and a
reduced quality. Market forces are hard to circumvent because, as we
emphasized in Chapter~\ref{chap:intro}, economic players react to the
incentives they face. These outcomes are examples of what we call the 
\textit{law of unintended consequences}.

\newhtmlpage

\subsection*{Price floors -- minimum wages}

An effective price floor sets the price \textit{above} the market-clearing
price. A minimum wage is the most widespread example in the Canadian
economy. Provinces each set their own minimum, and it is seen as a way of
protecting the well-being of low-skill workers. Such a floor is illustrated
in Figure~\ref{fig:pricefloor}. The free-market equilibrium is again $E_0$,
but the effective market outcome is the combination of price and quantity
corresponding to the point $E_f$ at the price floor, $P_f$. In this
instance, there is excess supply equal to the amount $E_f$C.

% Figure 3.7
\begin{TikzFigure}{xscale=0.52,yscale=0.42,descwidth=25em,caption={Price floor -- minimum wage \label{fig:pricefloor}},description={In a free market the equilibrium is $E_0$. A minimum wage of $P_f$ raises the hourly wage, but reduces the hours demanded to $Q_f$. Thus $E_f$C is the excess supply.}}
\draw [demandcolour,ultra thick,name path=demand] (0,12) -- node [mynode,black,above right,pos=0.9] {Demand} (12,0);
\draw [supplycolour,ultra thick,name path=supply] (0,1) -- (13,14) node [black,mynode,right] {Supply};
\draw [thick, -] (0,15) node (yaxis) [above] {Price=wage} |- (15,0) node (xaxis) [mynode1,right] {Quantity of\\labour=hours};
% intersection of demand and supply
\draw [name intersections={of=demand and supply,by=E0}]
	[dotted,thick] (yaxis |- E0) node [mynode,left] {$P_0$} -- (E0) node [mynode,above] {$E_0$} -- (xaxis -| E0) node [mynode,below] {$Q_0$};
% horizontal path from P_f
\path [name path=Pfline] ([yshift=3cm]yaxis |- E0) -- +(15,0);
% intersection of Pfline with demand and supply lines
\draw [name intersections={of=Pfline and demand,by=Ef},name intersections={of=Pfline and supply,by=C}]
	[dotted,thick] (yaxis |- Ef) node [mynode,left] {$P_f$} -- (Ef) node [mynode,above=0.15cm and 0cm] {$E_f$} -- (xaxis -| Ef) node [mynode,below] {$Q_f$}
	[dotted,thick] (Ef) -- (C) node [mynode,above=0.15cm and 0cm] {C};
% arrow between Ef and C
\draw [<->,thick,shorten >=0.5mm,shorten <=0.5mm] ([yshift=0.5cm]Ef) -- coordinate[pos=0.65] (A) ([yshift=0.5cm]C);
% arrow pointing to near middle of <-> arrow
\draw [<-,thick,shorten >=1mm,shorten <=0.5mm] (A) to[out=-75,in=180] +(5,-5) node [mynode,right=0cm and -0.15cm] {Excess supply};
\end{TikzFigure}

\newhtmlpage

Note that there is a similarity between the outcomes defined in the floor
and ceiling cases: The quantity actually traded is \textit{the lesser of the
supply quantity and demand quantity at the going price: The short side
dominates.}

If price floors, in the form of minimum wages, result in some workers going
unemployed, why do governments choose to put them in place? The excess
supply in this case corresponds to unemployment -- more individuals are
willing to work for the going wage than buyers (employers) wish to employ.
The answer really depends upon the magnitude of the excess supply. In
particular, suppose, in Figure~\ref{fig:pricefloor} that the supply and demand curves going
through the equilibrium $E_0$ were more `vertical'. This would result in a
smaller excess supply than is represented with the existing supply and
demand curves. This would mean in practice that a higher wage could go to
workers, making them better off, without causing substantial unemployment.
This is the tradeoff that governments face: With a view to increasing the
purchasing power of generally lower-skill individuals, a minimum wage is
set, hoping that the negative impact on employment will be small. We will
return to this in the next chapter, where we examine the responsiveness of
supply and demand curves to different prices.

\newhtmlpage

\subsection*{Quotas -- agricultural supply}

A quota represents the right to supply a specified quantity of a good to the
market. It is a means of keeping prices higher than the free-market
equilibrium price. As an alternative to imposing a price floor, the
government can generate a high price by restricting supply.

Agricultural markets abound with examples. In these markets, farmers can
supply only what they are permitted by the quota they hold, and there is
usually a market for these quotas. For example, in several Canadian
provinces it currently costs in the region of \$30,000 to purchase a quota
granting the right to sell the milk of one cow. The cost of purchasing
quotas can thus easily outstrip the cost of a farm and herd. Canadian cheese
importers must pay for the right to import cheese from abroad. Restrictions
also apply to poultry. The impact of all of these restrictions is to raise
the domestic price above the free market price.

In Figure~\ref{fig:quota}, the free-market equilibrium is at $E_0$. In
order to raise the price above $P_0$, the government restricts supply to $Q_q$
by granting quotas, which permit producers to supply a limited amount
of the good in question. This supply is purchased at the price equal to $P_q$.
From the standpoint of farmers, a higher price might be beneficial,
even if they get to supply a smaller quantity, provided the amount of
revenue they get as a result is as great as the revenue in the free market.

% Figure 3.8
\begin{TikzFigure}{xscale=0.52,yscale=0.42,descwidth=25em,caption={The effect of a quota \label{fig:quota}},description={The government decides that the equilibrium price $P_0$ is too low. It decides to boost price by reducing supply from $Q_0$ to $Q_q$. It achieves this by requiring producers to have a production quota. This is equivalent to fixing supply at $S_q$.}}
\draw [demandcolour,ultra thick,name path=demand] (2,14) -- node [mynode,black,above right,pos=0.9] {Demand} (9,0);
\draw [supplycolour,ultra thick,name path=supplyquota] (3.5,0) node [black,mynode,below] {$Q_q$} -- +(0,14) node [black,mynode,above] {$S_q$=supply\\with quota};
\draw [supplycolour,ultra thick,name path=supply] (0,1) -- (13,14) node [black,mynode,right] {Supply};
\draw [thick, -] (0,15) node (yaxis) [above] {Price} |- (15,0) node (xaxis) [right] {Quantity};
% intersection of supply and demand
\draw [name intersections={of=supply and demand,by=E0}]
	[dotted,thick] (yaxis |- E0) node [mynode,left] {$P_0$} -- (E0) node [mynode,above] {$E_0$} -- (xaxis -| E0) node [mynode,below] {$Q_0$};
% intersection of supplyquota and demand
\draw [name intersections={of=supplyquota and demand,by=Eq}]
	[dotted,thick] (yaxis |- Eq) node [mynode,left] {$P_q$} -- (Eq) node [mynode,above right] {$E_q$};
% horizontal path used to draw dotted line to point C
\path [name path=Cline] (yaxis |- Eq) -- +(15,0);
\draw [name intersections={of=Cline and supply,by=C}]
	[dotted,thick] (Eq) -- (C) node [mynode,above] {C};
\end{TikzFigure}

\newhtmlpage

\subsection*{Marketing boards -- milk and maple syrup}

A marketing board is a means of insuring that a quota or price floor can be
maintained. Quotas are frequent in the agriculture sector of the economy.
One example is maple syrup in Quebec. The Federation of Maple Syrup Producers
of Quebec has the sole right to market maple syrup. All producers must sell
their syrup through this marketing board. It is a \textit{de facto} monopoly. The
Federation increases the total revenue going to producers by artificially
restricting the supply to the market. The Federation calculates that by
reducing supply and selling it at a higher price, more revenue will accrue
to the producers. This is illustrated in Figure~\ref{fig:quota}. The market
equilibrium is given by $E_0$, but the Federation restricts supply to the
quantity $Q_q$, which is sold to buyers at price $P_q$. To make this possible
the total supply must be restricted; otherwise producers would supply the amount
given by the point C on the supply curve, and this would result in excess
supply in the amount $E_q$C. In order to restrict supply to $Q_q$ in total,
individual producers are limited in what they can sell to the Federation;
they have a quota, which gives them the right to produce and sell no more than
a specific limited amount. This system of quotas is necessary to eliminate
the excess supply that would emerge at the above-equilibrium price $P_q$.

We will return to this topic in Chapter~\ref{chap:elasticities}. For the moment,
to see that this type of revenue-increasing outcome is possible, examine
Table~\ref{table:dsnaturalgas} again. At this equilibrium price of \$4 the
quantity traded is 6 units, yielding a total expenditure by buyers (revenue
to suppliers) of \$24. However, if the supply were restricted and a price
of \$5 were set, the expenditure by buyers (revenue to suppliers) would
rise to \$25.
