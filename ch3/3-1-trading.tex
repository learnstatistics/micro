\section{The marketplace -- trading}\label{sec:ch3sec1}

The marketplace in today's economy has evolved from
earlier times. It no longer has a unique form -- one where buyers and
sellers physically come together for the purpose of exchange. Indeed,
supermarkets require individuals to be physically present to make their
purchases. But when purchasing an airline ticket, individuals simply go 
online and interact with perhaps a number of different airlines (suppliers)
simultaneously. Or again, individuals may simply give an instruction to
their stock broker, who will execute a purchase on their behalf -- the
broker performs the role of a middleman, who may additionally give advice to
the purchaser. Or a marketing agency may decide to subcontract work to a
translator or graphic artist who resides in Mumbai. In pure auctions (where
a single work of art or a single residence is offered for sale) buyers
compete one against the other for the single item supplied. Accomodations in
private homes are supplied to potential visitors (buyers) through \textit{%
Airbnb}. Taxi rides are mediated through \textit{Lyft} or \textit{Uber}.
These institutions are all different types of markets; they serve the
purpose of facilitating exchange and trade.

Not all goods and services in the modern
economy are obtained through the marketplace. Schooling and health care are
allocated in Canada primarily by government decree. In some instances the
market plays a supporting role: Universities and colleges may levy fees, and
most individuals must pay, at least in part, for their pharmaceuticals. In contrast,
broadcasting services may carry a price of zero -- as with the Canadian
Broadcasting Corporation.

The importance of the marketplace springs from its role as an allocating
mechanism. Elevated prices effectively send a signal to suppliers that the
buyers in the market place a high value on the product being traded;
conversely when prices are low. Accordingly, suppliers may decide to cease
supplying markets where prices do not remunerate them sufficiently, and
redirect their energies and the productive resources under their control to
other markets -- markets where the product being traded is more highly
valued, and where the buyer is willing to pay more.

Whatever their form, the marketplace is central to the economy we live in.
Not only does it facilitate trade, it also provides a means of earning a
livelihood. Suppliers must hire resources -- human and non-human in order to
bring their supplies to market and these resources must be paid a return --
income is generated.

In this chapter we will examine the process of price formation -- how the
prices that we observe in the marketplace come to be what they are. We will
illustrate that the price for a good is inevitably linked to the quantity of
a good; price and quantity are different sides of the same coin and cannot
generally be analyzed separately. To understand this process more fully, we
need to \textit{model} a typical market. The essentials are demand and
supply.