\section{Cartels: Acting like a monopolist}\label{sec:ch10sec6}

A \terminology{cartel} is a group of suppliers that colludes to operate like
a monopolist. The cartel formed by the members of the Organization of
Oil Exporting Countries (OPEC) is an example of a cartel that was successful
in achieving its objectives for a long period. This cartel first flexed its
muscles in 1973, by increasing the world price of oil from \$3 per barrel to
\$10 per barrel. The result was to transfer billions of dollars from the
energy-importing nations in Europe and North America to OPEC members -- the
demand for oil is relatively inelastic, hence an increase in price increases
total expenditures.

\begin{DefBox}
	A \textbf{cartel} is a group of suppliers that colludes to operate like a monopolist.
\end{DefBox}

A second renowned cartel is managed by De Beers, which controls a large part
of the world's diamond supply. A third is Major League Baseball in the US.
Lesser-known cartels, but in some cases very effective, are those formed by
the holders of taxi licenses in many cities throughout the world, and by
agricultural marketing boards in many developed economies. These cartels may
have thousands of members. By limiting entry, through requiring a production
`quota' or a license (taxi medallion), the incumbents can charge a higher
price than if entry to the industry were free.

Some cartels are sustained through violence, and frequently wars break out
between competing cartels or groups who want to sustain their market power.
Drug gangs frequently fight for hegemony over distribution. The recent drug
wars in Mexico between rival cartels have seen tens of thousands on
individuals killed as of 2015.

\newhtmlpage

To illustrate the dynamics of cartels consider Figure~\ref{fig:cartelindustry}.
Several producers, with given production capacities,
come together and agree to restrict output with a view to increasing price
and therefore profit. This may be done with the agreement of the government,
or it may be done secretively, and possibly against the law. Each firm has a 
$MC$ curve, and the industry supply is defined as the sum of these marginal
cost curves, as illustrated in Figure~\ref{fig:industrysupply}. The
resulting cartel is effectively one in which there is a single supplier with
many different plants -- a multi-plant monopolist. To maximize profits this
organization will choose an output level $Q_{m}$ where the $MR$ equals the 
$MC$. In contrast, if these firms act competitively the output chosen will be 
$Q_{c}$. The competitive output yields no supernormal profit, whereas the
monopoly/cartel output does.

% Figure 10.14
\begin{TikzFigure}{xscale=1.25,yscale=0.3,descwidth=25em,caption={Cartelizing a competitive industry \label{fig:cartelindustry}},description={A cartel is formed when individual suppliers come together and act like a monopolist in order to increase profit. If $MC$ is the joint supply curve of the cartel, profits are maximized at the output $Q_m$, where $MC=MR$. In contrast, if these firms operate competitively output increases to $Q_c$.}}
% MC line
\draw [dashed,mccolour,ultra thick,name path=MC] (0,1) -- (8,17) node [black,mynode,above] {$MC=S$};
% MR line
\draw [dashed,mrcolour,ultra thick,name path=MR] (0,16) -- (4,0) node [pos=0.9,black,mynode,above right] {$MR$};
% Demand line
\draw [demandcolour,ultra thick,name path=D] (0,16) -- (8,0) node [pos=0.9,black,mynode,above right] {$D$};
% axes
\draw [thick, -] (0,20) node (yaxis) [above] {\$} |- (8,0) node (xaxis) [right] {Quantity};
% intersection of MC and D
\draw [name intersections={of=MC and D, by=B}]
	[dotted,thick] (yaxis |- B) node [mynode,left] {$P_c$} -- (B) node [mynode,above] {B} -- (xaxis -| B) node [mynode,below] {$Q_c$};
% intersection of MC and MR
\draw [name intersections={of=MC and MR, by=F}]
	[dotted,thick] (F) node [mynode,right=0cm and 0.15cm] {F} -- (xaxis -| F) node [mynode,below] {$Q_m$};
% path to create dotted line from P_m -- A -- F
\path [name path=Fline] (xaxis -| F) -- +(0,20);
% intersection of Fline with D
\draw [name intersections={of=D and Fline, by=A}]
	[dotted,thick] (yaxis |- A) node [mynode,left] {$P_m$} -- (A) node [mynode,above] {A} -- (F);
\end{TikzFigure}

The cartel results in a deadweight loss equal to the area ABF, just as in
the standard monopoly model.

\newhtmlpage

\subsection*{Cartel instability}

Cartels tend to be unstable in the long run for a number of reasons. 

In the first instance, the degree of instability depends on the authority
that the governing body of the cartel can exercise over its members, and
upon the degree of information it has on the operations of its members. If a
cartel is simply an arrangement among producers to limit output, each
individual member of the cartel has an incentive to increase its output,
because the monopoly price that the cartel attempts to sustain exceeds the
cost of producing a marginal unit of output. In Figure~\ref{fig:cartelindustry}
each firm has a $MC$ of output equal to \$F when the
group collectively produces the output $Q_{m}$. Yet any firm that brings
output to market, \textit{beyond its agreed production limit}, at the price 
$P_{m}$ will make a profit of AF on that additional output -- 
\textit{provided the other members of the cartel agree to restrict their output}.
Since each firm faces the same incentive to increase output, it is difficult
to restrain all members from doing so.

Individual members are more likely to abide by the cartel rules if the
organization can sanction them for breaking the supply-restriction
agreement. Alternatively, if the actions of individual members are not
observable by the organization, then the incentive to break ranks may be too
strong for the cartel to sustain its monopoly power.

Cartels within individual economies are almost universally illegal. Yet at
the international level there exists no governing authority to limit such
behaviour. In practice, governments are unwilling to see their own citizens
and consumers being `gouged', but are relatively unconcerned if their
national or multinational corporations are willing and successful in gouging
the consumers of other economies! We will see in Chapter~\ref{chap:government}
that Canada's Competition Act forbids the formation of
cartels, as it forbids many other anti-competitive practices.

In the second instance, cartels may be undermined eventually by the
emergence of new products and new technologies. OPEC has lost much of its
power in the modern era because of technological developments in oil
recovery. Canada's `tar sands' yield oil, as a result of technological
developments that enabled producers to separate the oil from the earth it is
mixed with. Fracking technologies are the latest means of extracting oil
that is discovered in small pockets and encased in rock. These technologies
do not enable oil to be produced as cheaply as in traditional oil wells, but
they limit the degree to which cartels can raise prices. 

The new sharing economy has brought competition to many traditional
suppliers. Application Box~\ref{app:taxicartel} details how taxi cartels are 
in the process of being disrupted by new `ride-sharing' drivers.

\newhtmlpage

\begin{ApplicationBox}{caption={The taxi cartel \label{app:taxicartel}}}
	Cartels frequently contain thousands of members and are maintained by legal entry barriers. City taxis are an example of such a formation: Entry is restricted to drivers who hold a permit (medallion), and fares are maintained at a higher level as a consequence of the resulting lower supply. A secondary market then develops for licenses -- medallions, in which the city may offer new medallions through auction, or existing owners may exit and sell their medallions. Restricted entry characterizes most of Canada's major cities, with the result that new medallions frequently generate in excess of \$100,000 from the buyer. New York and Boston medallions traded at close to one million dollars in 2012.

	But a recent start-up company is aiming to change all of that: \textit{Uber} arrived on the international scene. This company developed a smart-phone app that links demanders for rides with drivers who are not part of the traditional taxi companies. \textit{Uber} has set up operations in many of the world's major cities and has succeeded in taking part of the taxi business away from the traditional operators. One result is that the price of taxi medallions on the open market in large US cities has fallen by about 20\%. Not surprisingly, the traditional taxi companies are charging that \textit{Uber} operators are violating the accepted rules governing the taxi business, and have launched legal suits against \textit{Uber}.


	\textit{Uber} is part of what is called the `sharing economy'. Participants operate with very little traditional capital. For example, suppliers may use an online site to rent a spare bedroom in their house to visitors to their city (Airbnb), and thus compete with hotels. The main capital in this business is in the form of the information technology that links potential buyers to potential sellers.

	See \url{http://www.fcpp.org} for Canada, and for New York:

	\href{http://www.nyc.gov/html/tlc/downloads/pdf/press_release_medallion_auction.pdf}{www.nyc.gov/html/tlc/downloads/pdf/press\_release\_medallion\_auction.pdf}
\end{ApplicationBox}
