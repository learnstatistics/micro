\section{Price discrimination}\label{sec:ch10sec5}

A common characteristic in the pricing of many goods is that different
individuals pay different prices for goods or services that are essentially
the same. Examples abound: Seniors get a reduced rate for coffee in 
\textit{Burger King}; hair salons charge women more than they charge men; bank
charges are frequently waived for juniors. \terminology{Price discrimination}
involves charging different prices to different consumers in order to
increase profit.

\begin{DefBox}
	\textbf{Price discrimination} involves charging different prices to different consumers in order to increase profit.
\end{DefBox}

A strict definition of discrimination involves different prices for 
\textit{identical products}. We all know of a school friend who has been willing to
take the midnight flight to make it home at school break at a price he can
afford. In contrast, the business executive prefers the seven a.m.
flight to arrive for a nine a.m. business meeting in the same city
at several times the price. These are very mild forms of price
discrimination, since a midnight flight (or a midday flight) is not a
perfect substitute for an early morning flight. Price discrimination is
practiced because buyers are willing to pay different amounts for a good or
service, and the supplier may have a means of profiting from this. Consider
the following example.

\newhtmlpage

\textit{Family Flicks} is the local movie theatre. It has two distinct
groups of customers -- those of prime age form one group; youth and seniors
form the other. Family Flicks has done its market research and determined
that each group accounts for 50 percent of the total market of 100 potential
viewers per screening. It has also established that the prime-age group
members are willing to pay \$12 to see a movie, while the seniors and youth
are willing to pay just \$5. How should the tickets be priced?

Family Flicks has no variable costs, only fixed costs. It must pay a \$100
royalty to the movie maker each time it shows the current movie, and must
pay a cashier and usher \$20 each. Total costs are therefore \$140,
regardless of how many people show up -- short-run $MC$ is zero. On the
pricing front, as illustrated in Table~\ref{table:pricediscrimination}
below, if Family Flicks charges \$12 per ticket it will attract 50 viewers,
generate \$600 in revenue and therefore make a profit of \$460.

\begin{Table}{caption={Price discrimination \label{table:pricediscrimination}}}
\begin{tabu} to 36em {|X[3,c]X[1,c]X[1,c]X[1,c]|} \hline 
	\rowcolor{rowcolour}	& \textbf{$P$=\$5}	& \textbf{$P$=\$12} & \textbf{Twin price} \\
	\textbf{No. of customers}		& 100	& 50	&	\\
	\rowcolor{rowcolour}	\textbf{Total revenue}	& \$500	& \$600 & \$850	\\ 
	\textbf{Total costs}	& \$140				& \$140 			& \$140	\\
	\rowcolor{rowcolour}	\textbf{Profit}		& \$360		& \$460 & \$710	\\ \hline 
\end{tabu}
\end{Table}

In contrast, if it charges \$5 it can fill the theatre, because each of the
prime-age individuals is willing to pay more than \$5, but the seniors and
youth are now offered a price they too are willing to pay. However, the
total revenue is now only \$500 ($100\times\$5=\$500$), and profits are
reduced to \$360. It therefore decides to charge the high price and leave
the theatre half-empty, because this strategy maximizes its profit.

Suppose finally that the theatre is able to segregate its customers. It can
ask the young and senior customers for identification upon entry, and in
this way charge them a lower price, \textit{while still maintaining the
	higher price to the prime-age customers}. If it can execute such a plan
Family Flicks can now generate \$850 in revenue -- \$600 from the prime-age
group and \$250 from the youth and seniors groups. Profit soars to \$710.

\newhtmlpage

There are two important conditions for this scheme to work:

\begin{enumerate}
	\item The seller must be able to \textit{segregate the market} at a
	reasonable cost. In the movie case this is achieved by asking for
	identification.
	
	\item The second condition is that \textit{resale must be impossible or
		impractical}. For example, we rule out the opportunity for young buyers to
	resell their tickets to the prime-age individuals. Sellers have many ways of
	achieving this -- they can require immediate entry to the movie theatre upon
	ticket purchase, they can stamp the customer's hand, they can demand the
	showing of ID with the ticket when entering the theatre area.
\end{enumerate}

Frequently we think of sellers who offer price reductions to specific groups
as being generous. For example, hotels may levy only a nominal fee for the
presence of a child, once the parents have paid a suitable rate for the room
or suite in which a family stays. The hotel knows that if it charges too
much for the child, it may lose the whole family as a paying unit. The
coffee shop offering cheap coffee to seniors is interested in getting a
price that will cover its variable cost and so contribute to its profit. It
is unlikely to be motivated by philanthropy, or to be concerned with the
financial circumstances of seniors.

\newhtmlpage

% Figure 10.11
\begin{TikzFigure}{xscale=0.38,yscale=0.33,descwidth=25em,caption={Price discrimination at the movies \label{fig:pricedismovie}},description={At $P=12$, 50 prime-age individuals demand movie tickets. At $P=5$, 50 more seniors and youths demand tickets. Since the $MC$ is zero the efficient output is where the demand curve takes a zero value -- where all 100 customers purchase tickets. Thus, any scheme that results in all 100 individuals buying ticket is efficient. Efficient output is at point C.}}
\draw [dotted,thick] (10,0) node [mynode,below] {50} node [mynode,above right] {D} -- (10,5);
% demand line
\draw [demandcolour,ultra thick,-] (0,12) node [black,mynode,left] {12} -- (10,12) node [black,mynode,midway,above] {Demand curve} -- (10,5) node [black,mynode,above right] {A} -- (20,5) node [black,mynode,above right] {B} -- (20,0) node [black,mynode,below] {100} node [black,mynode,above right] {C};
% axes
\draw [thick, -] (0,20) node [above] {\$} |- (25,0) node [right] {Quantity};
% MC line
\draw [dashed,mccolour,ultra thick] (0,0) -- coordinate [pos=0.2] (MCnamepoint) (25,0);
% arrow to MC line
\draw [<-,thick,shorten <=0.5mm] (MCnamepoint) -- +(0,3) node [mynode,above] {$MC$};
% point of price axis
\node [mynode,left] at (0,5) {5};
\end{TikzFigure}

Price discrimination has a further interesting feature that is illustrated
in Figure~\ref{fig:pricedismovie}: It frequently \textit{reduces the
deadweight loss} associated with a monopoly seller!
	
\newhtmlpage

In our Family Flicks example, the profit maximizing monopolist that did not,
or could not, price discriminate \textit{left 50 customers unsupplied who
	were willing to pay \$5 for a good that had a zero $MC$}. This is a
deadweight loss of \$250 because 50 seniors and youth valued a commodity at
\$5 that had a zero $MC$. Their demand was not met because, in the absence
of an ability to discriminate between consumer groups, Family Flicks made
more profit by satisfying the demand of the prime-age group alone. But in this
example, by segregating its customers, the firm's profit maximization
behaviour resulted in the DWL being eliminated, because it supplied the
product to those additional 50 individuals. In this instance \textit{price
	discrimination improves welfare}, because more of a good is supplied in a
situation where market valuation exceeds marginal cost.

In the preceding example we simplified the demand side of the market by
assuming that every individual in a given group was willing to pay the same
price -- either \$12 or \$5. More realistically each group can be defined by
a downward-sloping demand curve, reflecting the variety of prices that
buyers in a given market segment are willing to pay. It is valuable to
extend the analysis to include this reality. For example, a supplier may
face different demands from her domestic and foreign buyers, and if she can
segment these markets she can price discriminate effectively.

\newhtmlpage

Consider Figure~\ref{fig:pricesegregatedmarket} where two segmented demands
are displayed, $D_A$ and $D_B$, with their associated marginal revenue
curves, $MR_A$ and $MR_B$. We will assume that marginal costs are constant
for the moment. It should be clear by this point that the profit maximizing
solution for the monopoly supplier is to supply an amount to each market
where the $MC$ equals the $MR$ in each market: Since the buyers in one
market cannot resell to buyers in the other, the monopolist considers these
as two different markets and therefore maximizes profit by applying the
standard rule. She will maximize profit in market A by supplying the
quantity $Q_A$ and in market B by supplying $Q_B$. The prices at which these
quantities can be sold are $P_A$ and $P_B$. These prices, unsurprisingly,
are different -- the objective of segmenting markets is to increase profit
by treating the markets as distinct.

% Figure 10.12
\begin{TikzFigure}{xscale=0.26,yscale=0.14,descwidth=25em,caption={Pricing in segregated markets \label{fig:pricesegregatedmarket}},description={With two separate markets defined by $D_A$ and $D_B$, and their associated $MR$ curves $MR_A$ and $MR_B$, a profit maximizing strategy is to produce where $MC=MR_A=MR_B$, and \emph{discriminate} between the two markets by charging prices $P_A$ and $P_B$.}}
% demand lines
\draw [demandcolour,ultra thick,name path=DA] (0,40) -- node [mynode,black,above right,pos=0.15] {$D_A$} (40,0);
\draw [demandcolour,ultra thick,name path=DB] (0,15) -- node [mynode,black,above right,pos=0.15] {$D_B$} (15,0);
% MR lines
\draw [dashed,mrcolour,ultra thick,name path=MRA] (0,40) -- (20,0) node [mynode,black,above right] {$MR_A$};
\draw [dashed,mrcolour,ultra thick,name path=MRB] (0,15) -- (7.5,0) node [mynode,black,above right] {$MR_B$};
% MC line
\draw [dashed,mccolour,ultra thick,name path=MC] (0,5) -- (45,5) node [mynode,black,above] {$MC$};
% axes
\draw [thick, -] (0,45) node (yaxis) [above] {Price} |- (45,0) node (xaxis) [right] {Quantity};
% intersection of MRA and MC
\draw [name intersections={of=MRA and MC, by=QA}]
	[dotted,thick] (QA) -- (xaxis -| QA) node [mynode,below] {$Q_A$};
% path from QA up to DA
\path [name path=QAline] (xaxis -| QA) -- +(0,45);
% intersection of QAline and DA
\draw [name intersections={of=QAline and DA, by=PA}]
	[dotted,thick] (yaxis |- PA) node [mynode,left] {$P_A$} -| (QA);
% intersection of MRB and MC
\draw [name intersections={of=MRB and MC, by=QB}]
	[dotted,thick] (QB) -- (xaxis -| QB) node [mynode,below] {$Q_B$};
% path from QB up to DB
\path [name path=QBline] (xaxis -| QB) -- +(0,45);
% intersection of QBline and DB
\draw [name intersections={of=QBline and DB, by=PB}]
	[dotted,thick] (yaxis |- PB) node [mynode,left] {$P_B$} -| (QB);
\end{TikzFigure}

\newhtmlpage

The preceding examples involved two separable groups of customers and are
very real. This kind of group segregation is sometimes called \textit{third
	degree price discrimination}. But it may be possible to segregate customers
into several groups rather than just two. In the limit, if we could charge a
different price to every consumer in a market, or for every unit sold, the
revenue accruing to the monopolist would be the area under the demand curve
up to the output sold. Though primarily of theoretical interest, this is
illustrated in Figure~\ref{fig:perfectpricedis}. It is termed \textit{%
	perfect price discrimination}, and sometimes \textit{first degree price
	discrimination}. Such discrimination is not so unrealistic: A tax accountant
may charge different customers a different price for providing the same
service; home renovators may try to charge as much as any client appears
willing to pay.

% Figure 10.13
\begin{TikzFigure}{xscale=0.27,yscale=0.25,descwidth=25em,caption={Perfect price discrimination \label{fig:perfectpricedis}},description={A monopolist who can sell each unit at a different price maximizes profit by producing $Q^*$. With each consumer paying a different price the demand curve becomes the $MR$ curve. The result is that the monopoly DWL is eliminated because the efficient output is produced, and the monopolist appropriates all the consumer surplus. Total revenue for the perfect price discriminator is OAB$Q^*$.}}
% MC line
\draw [dashed,mccolour,ultra thick,name path=MC] (0,8) node [black,mynode,left] {$P_0$} -- (35,8) node [black,mynode,above] {$MC$};
% Demand line
\draw [demandcolour,ultra thick,name path=D] (0,20) node [mynode,black,left] {A} -- node [mynode,above right,black,pos=0.25] {$D=MR$} (34,3);
% axes
\draw [thick] (0,25) node (yaxis) [above] {\$} -- (0,0) node [below left=0cm and 0cm] {O} -- (35,0) node (xaxis) [right] {Quantity};
% intersection of D and MC
\draw [name intersections={of=D and MC, by=B}]
	[dotted,thick] (B) node [mynode,above] {B} -- (xaxis -| B) node [mynode,below] {$Q^{*}$};
\end{TikzFigure}

\newhtmlpage

\textit{Second degree price discrimination} is based on a different concept
of buyer identifiability. In the cases we have developed above, the seller
is able to distinguish the buyers by \textit{observing} a vital
characteristic that signals their type. It is also possible that, while
individuals might have defining traits which influence their demands, such
traits might not be detectable by the supplier. Nonetheless, it is
frequently possible for the supplier to offer different pricing options
(corresponding to different uses of a product) that buyers would choose
from, with the result that her profit would be greater than under a uniform
price with no variation in the use of the service. Different cell phone
`plans', or different internet plans that users can choose from are examples
of this second-degree discrimination.