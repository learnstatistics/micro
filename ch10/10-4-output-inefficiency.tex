\section{Output inefficiency}\label{sec:ch10sec4}

A characteristic of perfect competition is that it secures an efficient
allocation of resources when there are no externalities in the market:
Resources are used up to the point where their marginal cost equals their
marginal value -- as measured by the price that consumers are willing to
pay. But a monopoly structure does not yield this output. Consider Figure~\ref{fig:monopolyinefficiency}.

% Figure 10.10
\begin{TikzFigure}{xscale=0.28,yscale=0.25,descwidth=25em,caption={Monopoly output inefficiency \label{fig:monopolyinefficiency}},description={A monopolist maximizes profit at $Q_M$. Here the value of marginal output exceeds cost. If output expands to $Q^*$ a gain arises equal to the area ABF. This is the deadweight loss associated with the output $Q_M$ rather than $Q^*$. If the monopolist's long-run $MC$ is equivalent to a competitive industry's supply curve, then the deadweight loss is the cost of having a monopoly rather than a perfectly competitive market.}}
% MC curve
\draw [dashed,mccolour,ultra thick,name path=MC] (15,6) node [black,mynode,below] {$MC$} to [out=15,in=270] (20,14);
% ATC curve
\draw [atccolour,ultra thick,domain=225:360,name path=ATC] plot ({18+4*cos(\x)},{12+4*sin(\x)}) node [mynode,right,black] {$ATC$};
% LMC line
\draw [lmccolour,ultra thick,name path=LMC] (0,8) node [black,mynode,left] {$P_{PC}$} -- (35,8) node [black,mynode,above] {$LMC$};
% Demand line
\draw [demandcolour,ultra thick,name path=D] (0,20) -- (34,3) node [mynode,above right,black,pos=0.2] {$D$};
% MR line
\draw [dashed,mrcolour,ultra thick,name path=MR] (0,20) -- (30,0) node [mynode,below left,black,pos=0.2] {$MR$};
% axes
\draw [thick, -] (0,25) node (yaxis) [above] {\$} |- (35,0) node (xaxis) [right] {Quantity};
% intersection of MR and MC
\draw [name intersections={of=MC and MR, by=F}]
	[dotted,thick] (F) -- (xaxis -| F) node [mynode,below] {$Q_M$};
% path to create dotted line from P_m--A--F
\path [name path=Fline] (xaxis -| F) -- +(0,25);
% intersection of Fline with D
\draw [name intersections={of=D and Fline, by=A}]
	[dotted,thick] (yaxis |- A) node [mynode,left] {$P_M$} -- (A) node [mynode,above] {A} -- (F);
% intersection of LMC and D
\draw [name intersections={of=LMC and D, by=B}]
	[dotted,thick] (B) node [mynode,above] {B} -- (xaxis -| B) node [mynode,below] {$Q^{*}$};
% arrow to F
\draw [<-,thick,shorten <=1.75mm] (F) -- +(1,-2.5) node [mynode,below] {F};
\end{TikzFigure}

\newhtmlpage

The monopolist's profit-maximizing output $Q_{M}$ is where $MC$ equals $MR$.
This output is inefficient for the reason that we developed in Chapter~\ref{chap:welfare}:
If output is increased beyond $Q_{M}$ the additional benefit
exceeds the additional cost of producing it. The additional benefit is
measured by the willingness of buyers to pay -- the market demand curve. The
additional cost is the long-run $MC$ curve under the assumption of constant
returns to scale. Using the terminology from Chapter~\ref{chap:welfare},
there is a deadweight loss equal to the area ABF. This is termed %
\terminology{allocative inefficiency}.

\begin{DefBox}
	\textbf{Allocative inefficiency} arises when resources are not appropriately allocated and result in deadweight losses .
\end{DefBox}

\newhtmlpage

\subsection*{Perfect competition versus monopoly}

The area ABF can also be considered as the efficiency loss associated with
having a monopoly rather than a perfectly competitive market structure. In
perfect competition the supply curve is horizontal. This is achieved by
having firms enter and exit when more or less must be produced. Accordingly, 
\textit{if the perfectly competitive industry's supply curve approximates
	the monopolist's long-run marginal cost curve}\footnote{%
	We can think of such a transformation coming from a single supplier taking
	over a number of small suppliers, and the monopolist would thus be a
	multi-plant firm.}, we can say that if the monopoly were turned into a
competitive industry, output would increase from $Q_{M}$ to $Q^{*}$. The
deadweight loss is one measure of the superiority of the perfectly
competitive structure over the monopoly structure.

Note that this critique of monopoly is not initially focused upon profit.
While monopoly profits are what frequently irk the public, we have focused
upon resource allocation inefficiencies. But in a real sense the two are
related: Monopoly inefficiencies arise through output being restricted, and
it is this output reduction -- achieved by maintaining a higher than
competitive price -- that gives rise to those profits. Nonetheless, there is
more than just a shift in purchasing power from the buyer to the seller.
Deadweight losses arise because output is at a level lower than the point
where the $MC$ equals the value placed on the good; thus the economy is
sacrificing the possibility of creating additional surplus.

Given that monopoly has this undesirable inefficiency, what measures should
be taken, if any, to counter the inefficiency? We will see what Canada's
Competition Act has to say in Chapter~\ref{chap:government} and also examine
what other measures are available to control monopolies.