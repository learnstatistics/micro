\section{Invention, innovation and rent seeking}\label{sec:ch10sec7}

Invention and innovation are critical aspects of the modern economy. In 
some sectors of the economy, firms that cannot invent or innovate are 
liable to die. Invention is a genuine discovery, whereas innovation is 
the introduction of a new product or process. 

\begin{DefBox}
\textbf{Invention} is the discovery of a new product or process through research.

\textbf{Product innovation} refers to new or better products or services.

\textbf{Process innovation} refers to new or better production or supply.
\end{DefBox}

To this point we have said little that is good about monopolies. However, 
the economist Joseph Schumpeter argued that, while monopoly leads to 
resource misallocation in the economy, this cost might be offset by the 
greater tendency for monopoly firms to invent and innovate. This is 
because such firms have more profit and therefore more resources with 
which to fund R\&D and may therefore be more innovative than competitive 
firms. If this were true then, taking a long-run dynamic view of the 
marketplace, monopolies could have lower costs and more advanced products 
than competitive firms and thus benefit the consumer.

While this argument has some logical appeal, it falls short on several 
counts. First, even if large firms carry out more research than competitive 
firms, there is no guarantee that the ensuing benefits carry over to the 
consumer. Second, the results of such research may be used to prevent entry 
into the industry in question. Firms may register their inventions and 
gain use protection before a competitor can come up with the same or a 
similar invention. \textit{Apple} and \textit{Samsung} each own tens of thousands of patents. 
Third, the empirical evidence on the location of most R\&D is inconclusive: 
A sector with several large firms, rather than one with a single or very 
many firms, may be best. For example, if \textit{Apple} did not have \textit{Samsung} as a 
competitor, or vice versa, would the pace of innovation be as strong? 

Fourth, much research has a `public good' aspect to it. Research carried 
out at universities and government-funded laboratories is sometimes 
referred to as basic research: It explores the principles underlying 
chemistry, social relations, engineering forces, microbiology, etc., and 
has multiple applications in the commercial world. If disseminated, this 
research is like a public good -- its fruits can be used in many different 
applications, and its use in one area does not preclude its use in others. 
Consequently, rather than protecting monopolies on the promise of more R\&D, 
a superior government policy might be to invest directly in research and 
make the fruits of the research publicly available.

\newhtmlpage

Modern economies have \terminology{patent laws}, which grant inventors a 
legal monopoly on use for a fixed period of time -- perhaps fifteen years. 
By preventing imitation, patent laws raise the incentive to conduct R\&D 
but do not establish a monopoly in the long run. Over the life of a patent 
the inventor charges a higher price than would exist if his invention were 
not protected; this both yields greater profits and provides the research 
incentive. When the patent expires, competition from other producers leads 
to higher output and lower prices for the product. Generic drugs are a good 
example of this phenomenon. 

\begin{DefBox}
\textbf{Patent laws} grant inventors a legal monopoly on use for a fixed period of time.
\end{DefBox}

The power of globalization once again is very relevant in patents. Not all 
countries have patent laws that are as strong as those in North America and 
Europe. The BRIC economies (Brazil, Russia, India and China) form an emerging 
power block. But their legal systems and enforcement systems are less 
well-developed than in Europe or North America. The absence of a strong and 
transparent legal structure inhibits research and development, because their 
fruits may be appropriated by competitors.

\newhtmlpage

\subsection*{Rent seeking}

Citizens are frequently appalled when they read of lobbying activities in their 
nation's capital. Every capital city in the world has an army of lobbyists, 
seeking to influence legislators and regulators. Such individuals are in the 
business of \terminology{rent seeking}, whose goal is to direct profit to 
particular groups, and protect that profit from the forces of competition. 
For example, Canadian media owners  seek to reduce competition from their 
US-based competition by requesting the CRTC (the Canadian Radio-Television 
and Telecommunications Commission) to restrict the inflow of signals from US 
suppliers. In Virginia and Kentucky we find that state taxes on cigarettes are 
the lowest in the US -- because the tobacco leaf is grown in these states, 
and the tobacco industry makes major contributions to the campaigns of some 
political representatives. 

Rent-seeking carries a resource cost: Imagine that we could outlaw the lobbying 
business and put these lobbyists to work producing goods and services in the 
economy instead. Their purpose is to maintain as much market or quasi-monopoly 
power in the hands of their clients as possible, and to ensure that the fruits 
of this effort go to those same clients. If this practice could be curtailed then 
the time and resources involved could be redirected to other productive ends.

\begin{DefBox}
\textbf{Rent seeking} is an activity that uses productive resources to redistribute rather than create output and value.
\end{DefBox}

Industries in which rent seeking is most prevalent tend to be those in which 
the potential for economic profits is greatest -- monopolies or near-monopolies. 
These, therefore, are the industries that allocate resources to the preservation 
of their protected status. We do not observe laundromat owners or shoe-repair 
businesses lobbying in Ottawa. 

\section*{Conclusion}

We have now examined two extreme types of market structure -- perfect competition 
and monopoly.  While many sectors of the economy operate in a way that is close 
to the competitive paradigm, very few are pure monopolies in that they have no close 
substitute products. Even firms like \textit{Microsoft}, or \textit{De Beers}, that supply a huge 
percentage of the world market for their product would deny that they are monopolies 
and would argue that they are subject to strong competitive pressures from smaller 
or `fringe' producers. As a result we must look upon the monopoly paradigm as a useful 
way of analyzing markets, rather than being an exact description of the world. 
Accordingly, our next task is to examine how sectors with a few, several or multiple 
suppliers act when pursuing the objective of profit maximization. Many different 
market structures define the real economy, and we will concentrate on a limited 
number of the more important structures in the next chapter.