\newpage
	\section*{Key Terms}
\begin{keyterms}
\textbf{Monopolist}: is the sole supplier of an industry's output, and therefore the industry and the firm are one and the same.

\textbf{Natural monopoly}: one where the $ATC$ of producing any output declines with the scale of operation.

\textbf{Marginal revenue} is the change in total revenue due to selling one more unit of the good.

\textbf{Average revenue} is the price per unit sold.

\textbf{Allocative inefficiency} arises when resources are not appropriately allocated and result in deadweight losses.

\textbf{Price discrimination} involves charging different prices to different consumers in order to increase profit.

A \textbf{cartel} is a group of suppliers that colludes to operate like a monopolist.

\textbf{Rent seeking} is an activity that uses productive resources to redistribute rather than create output and value.

\textbf{Invention} is the discovery of a new product or process through research.

\textbf{Product innovation} refers to new or better products or services.

\textbf{Process innovation} refers to new or better production or supply.

\textbf{Patent laws} grant inventors a legal monopoly on use for a fixed period of time.
\end{keyterms}