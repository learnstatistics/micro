\section{Profit maximizing behaviour}\label{sec:ch10sec2}

We established in the previous chapter that, in deciding upon a
profit-maximizing output, any firm should produce up to the point where the
additional cost equals the additional revenue from a unit of output. What
distinguishes the supply decision for a monopolist from the supply decision
of the perfect competitor is that the monopolist faces a downward sloping
demand. A monopolist is the sole supplier and therefore must meet the full
market demand. This means that if more output is produced, the price must
fall. We will illustrate the choice of a profit maximizing output using
first a marginal-cost/marginal-revenue approach; then a supply/demand
approach.

\subsection*{Marginal revenue and marginal cost}

Table~\ref{table:profitmaxmonopolist} displays price and quantity values for
a demand curve in columns 1 and 2. Column three contains the sales revenue
generated at each output. It is the product of price and quantity. Since the
price denotes the revenue per unit, it is sometimes referred to as %
\terminology{average revenue}. The total revenue ($TR$) reaches a maximum at
\$32, where 4 units of output are produced. A greater output necessitates a
lower price on every unit sold, and in this case revenue falls if the fifth
unit is brought to the market. Even though the fifth unit sells for a
positive price, the price on the other 4 units is now lower and the net
effect is to reduce total revenue. This pattern reflects what we examined in
Chapter~\ref{chap:elasticities}: As price is lowered from the highest
possible value of \$14 (where 1 unit is demanded) and the corresponding
quantity increases, revenue rises, peaks, and ultimately falls as output
increases. In Chapter~\ref{chap:elasticities} we explained that this maximum
revenue point occurs where the price elasticity is unity (-1), at the
midpoint of a linear demand curve.

% Table 10.1
\begin{Table}{caption={A profit maximizing monopolist \label{table:profitmaxmonopolist}}}
\begin{tabu} to \linewidth {|X[0.7,c]X[0.7,c]X[1.15,c]X[1.15,c]X[1,c]X[1,c]X[0.7,c]|} \hline 
	\rowcolor{rowcolour}	\textbf{Quantity} & \textbf{Price} & \textbf{Total} & \textbf{Marginal} & \textbf{Marginal} & \textbf{Total} & \textbf{Profit} \\[-0.1em]
	\rowcolor{rowcolour}	\textbf{($Q$)}	&	\textbf{($P$)}	&	\textbf{revenue ($TR$)}	&	\textbf{revenue ($MR$)}	&	\textbf{cost ($MC$)}	&	\textbf{cost ($TC$)}	&	\\ \hline
	0 & 16 &  &  &  &  &  \\
	\rowcolor{rowcolour}	1 & 14 & 14 & 14 & 2 & 2 & 12 \\ 
	2 & 12 & 24 & 10 & 3 & 5 & 19 \\
	\rowcolor{rowcolour}	3 & 10 & 30 & 6 & 4 & 9 & 21 \\ 
	4 & 8 & 32 & 2 & 5 & 14 & 18 \\ 
	\rowcolor{rowcolour}	5 & 6 & 30 & -2 & 6 & 20 & 10 \\ 
	6 & 4 & 24 & -6 & 7 & 27 & -3 \\
	\rowcolor{rowcolour}	7 & 2 & 14 & -10 & 8 & 35 & -21 \\ \hline 
\end{tabu}
\end{Table}

\newhtmlpage

% Figure 10.3
\begin{TikzFigure}{xscale=0.38,yscale=0.33,descwidth=25em,caption={Total revenue and marginal revenue \label{fig:totalmarginalrevenue}},description={When the quantity sold increases total revenue/expenditure initially increases also. At a certain point, further sales require a price that not only increases quantity, but reduces revenue on units already being sold to such a degree that $TR$ declines -- where the demand elasticity equals $-1$ (the mid point of a linear demand curve). Here the midpoint occurs at $Q=4$. Where the $TR$ is a maximum the $MR=0$.}}
% Total revenue function
\draw [trcolour,ultra thick,domain=180:0,name path=TR] plot ({12.5+12.5*cos(\x)},{12.5*sin(\x)}) node [mynode,black,below] {8};
% axes
\draw [thick, -] (0,20) node (yaxis) [above] {Revenue} |- (25,0) node (xaxis) [right] {Quantity};
% paths to intersect TR curve at R=24 and R=32 (max)
\path [name path=R24] (0,10.8253) -- +(10,0);
\path [name path=R32] (0,12.5) -- +(25,0);
% intersection of TR curve and R24 and R32
\draw [name intersections={of=TR and R24, by=Q2},name intersections={of=TR and R32, by=Q4}]
	[dotted,thick] (yaxis |- Q2) node [mynode,left] {24} -| (xaxis -| Q2) node [mynode,below] {2}
	[dotted,thick] (yaxis |- Q4) node [mynode,left] {32} -| (xaxis -| Q4) node [mynode,below] {4};
% arrow pointing to maximum of TR curve
\draw [<-,thick,shorten <=1mm,shorten >=-1.5mm] (Q4) -- +(0,2) node [mynode,above] {$TR$ is a maximum, $MR=0$};
\end{TikzFigure}

\newhtmlpage

Related to the total revenue function is the \terminology{marginal revenue}
function. It is the addition to total revenue due to the sale of one more
unit of the commodity.

\begin{DefBox}
	\textbf{Marginal revenue} is the change in total revenue due to selling one more unit of the good. 
	
	\textbf{Average revenue} is the price per unit sold.
\end{DefBox}

The $MR$ in this example is defined in the fourth column of Table~\ref{table:profitmaxmonopolist}.
When the quantity sold increases from 1 unit to
2 units total revenue increases from \$14 to \$24. Therefore the marginal
revenue associated with the second unit of output is \$10. When a third unit
is sold $TR$ increases to \$30 and therefore the $MR$ of the third unit is
\$6. As output increases the $MR$ declines and eventually becomes negative
-- at the point where the $TR$ is a maximum: If $TR$ begins to decline then
the additional revenue is by definition negative.

\newhtmlpage

The $MR$ function is plotted in Figure~\ref{fig:monopolistprofitmax}. It
becomes negative when output increases from 4 to 5 units.

% Figure 10.4
\begin{TikzFigure}{xscale=1.25,yscale=0.38,descwidth=25em,caption={Monopolist's profit maximizing output \label{fig:monopolistprofitmax}},description={It is optimal for the monopolist to increase output as long as $MR$ exceeds $MC$. In this case $MR>MC$ for units 1, 2 and 3. But for the fourth unit $MC>MR$ and therefore the monopolist would reduce total profit by producing it. He should produce only 3 units of output.}}
\draw [dashed,mccolour,ultra thick]
	(0,2) -- (1,2) -- (1,3) -- (2,3) -- (2,4) -- (3,4) -- (3,5) -- (4,5) -- (4,6) -- (5,6) node [black,mynode,right] {$MC$ function};
\draw [dashed,mrcolour,ultra thick]
	(0,14) -- (1,14) -- (1,10) -- (2,10) -- (2,6) -- (3,6) -- (3,2) -- (4,2) -- (4,-2) -- (5,-2) node [black,mynode,right] {$MR$ function};
\draw [thick, -]
	(0,16) node [above] {\$} |- (8,0) node [right] {Quantity}
	(0,0) node [mynode,left] {0} -- (0,-3);
\node [mynode,left] at (0,-2) {-2};
\node [mynode,left] at (0,2) {2};
\node [mynode,left] at (0,4) {4};
\node [mynode,left] at (0,6) {6};
\node [mynode,left] at (0,8) {8};
\node [mynode,left] at (0,10) {10};
\node [mynode,left] at (0,12) {12};
\node [mynode,left] at (0,14) {14};
\node [mynode,below] at (1,0) {1};
\node [mynode,below] at (3,0) {3};
\node [mynode,below] at (5,0) {5};
\node [mynode,below] at (7,0) {7};
\end{TikzFigure}

\newhtmlpage

\subsection*{The optimal output}

This producer has a marginal cost structure given in the fifth column of the
table, and this too is plotted in Figure~\ref{fig:monopolistprofitmax}. Our
profit maximizing rule from Chapter~\ref{chap:prodcost} states that it is
optimal to produce a greater output as long as the additional revenue
exceeds the additional cost of production on the next unit of output. In
perfectly competitive markets the additional revenue is given by the fixed
price for the individual producer, whereas for the monopolist the additional
revenue is the marginal revenue. Consequently as long as $MR$ exceeds $MC$
for the next unit a greater output is profitable, but once $MC$ exceeds $MR$
the production of additional units should cease.

From Table~\ref{table:profitmaxmonopolist} and Figure~\ref{fig:monopolistprofitmax}
it is clear that the optimal output is at 3 units.
The third unit itself yields a profit of 2\$, the difference between $MR$
(\$6) and $MC$ (\$4). A fourth unit however would reduce profit by \$3,
because the $MR$ (\$2) is less than the $MC$ (\$5). What price should the
producer charge? The price, as always, is given by the demand function. At a
quantity sold of 3 units, the corresponding price is \$10, yielding total
revenue of \$30.

Profit is the difference between total revenue and total cost. In Chapter~\ref{chap:prodcost}
we computed total cost as the average cost times the
number of units produced. It can also be computed as the sum of costs
associated with each unit produced: The first unit costs \$2, the second \$3
and the third \$4. The total cost of producing 3 units is the sum of these
dollar values: $\$9=\$2+\$3+\$4$. The profit-maximizing output therefore
yields a profit of \$21 ($\$30-\$9$).

\newhtmlpage

\subsection*{Supply and demand}

When illustrating market behaviour it is convenient to describe behaviour by
simple linear supply and demand functions that are continuous, rather than
the `step' functions used in the preceding example. As explained in Chapter~\ref{chap:welfare},
in using continuous curves to represent a market we
implicitly assume that a unit of output can be broken into subunits. In the
example above we assumed that sales always involve one whole unit of
the product being sold. In fact many goods can be sold in fractional units:
Gasoline can be sold in fractions of a litre; fruits and vegetables can be
sold in fractions of a kilogram, and so forth. Table~\ref{table:discretequantities} below furnishes
the data for our analysis.

% Table 10.2
\begin{Table}{caption={Discrete quantities \label{table:discretequantities}}}
\begin{tabu} to 35em {|X[1,c]X[1,c]X[1,c]X[1,c]X[1,c]|}	\hline
	\rowcolor{rowcolour}	\textbf{Price} & \textbf{Quantity} & \textbf{Total} & \textbf{Total} & \textbf{Profit} \\[-0.1em]
	\rowcolor{rowcolour}		&	\textbf{demanded}	&	\textbf{revenue}	&	\textbf{cost}	&	\\	\hline
	12 & 0 & 0 & 0 & 0 \\ 
	\rowcolor{rowcolour}	11 & 2 & 22 & 1 & 21 \\ 
	10 & 4 & 40 & 4 & 36 \\ 
	\rowcolor{rowcolour}	9 & 6 & 54 & 9 & 45 \\ 
	8 & 8 & 64 & 16 & 48 \\ 
	\rowcolor{rowcolour}	7 & 10 & 70 & 25 & 45 \\ 
	6 & 12 & 72 & 36 & 36 \\ 
	\rowcolor{rowcolour}	5 & 14 & 70 & 49 & 21 \\ 
	4 & 16 & 64 & 64 & 0 \\ 
	\rowcolor{rowcolour}	3 & 18 & 54 & 81 & -27 \\ 
	2 & 20 & 40 & 100 & -60 \\ 
	\rowcolor{rowcolour}	1 & 22 & 22 & 121 & -99 \\ 
	0 & 24 & 0 & 144 & -144	\\	\hline
\end{tabu}
\end{Table}

\newhtmlpage

The first two columns define the demand curve. Total revenue is the product
of price and quantity and given in column 3. The cost data are given in
column 4, and profit -- the difference between total revenue and total cost
is in the final column. Profit is maximized where the difference between
revenue and cost is greatest; in this case where the output is 8 units. At
lower or higher outputs profit is less. Figure~\ref{fig:totalrevcostprofit} contains the curves
defining total revenue ($TR$), total cost ($TC$) and profit. These functions
can be obtained by mapping all of the revenue-quantity combinations, the
cost-quantity combinations, and the profit-quantity combinations as a series of
points, and joining these points to form the smooth functions displayed. The
vertical axis is measured in dollars, the horizontal axis in units of
output. Graphically, profit is maximized where the dollar difference between 
$TR$ and $TC$ is greatest; that is at the output where the vertical distance
between the two curves is greatest. This difference, which is also defined
by the profit curve, occurs at a value of 8 units, corresponding to the
outcome in Table~\ref{table:discretequantities}.

% Figure 10.5
\begin{TikzFigure}{xscale=0.48,yscale=0.07,caption={Total revenue, total cost \& profit \label{fig:totalrevcostprofit}}}
% axes
\draw [thick, -] (0,100) node (yaxis) [above] {Price} |- (20,0) node (xaxis) [right] {Quantity};
% profit
\draw [ultra thick,domain=0:16] plot (\x, {12*\x-0.75*\x*\x});
\node [mynode,right] at (14,21) {Profit};
% TR
\draw [trcolour,ultra thick,domain=0:20] plot (\x, {12*\x-0.5*\x*\x}) node [mynode,black,right] {$TR$};
% TC
\draw [tccolour,ultra thick,domain=0:20] plot (\x, {0.25*\x*\x}) node [mynode,black,right] {$TC$};
% AC
\draw [atccolour,ultra thick,domain=0:20] plot (\x, {0.25*\x}) node [mynode,black,above] {$AC$};
% y-axis nodes
\foreach \y/\ytext in {10,20,30,40,50,60,70,80,90,100} \draw [thick] (-0.1,\y) node[mynode,left]{$\ytext$} -- +(0.2,0);
% x-axis nodes
\foreach \x/\xtext in {5,10,15,20} \draw [thick] (\x,-0.8) node[mynode,below] {$\xtext$} -- +(0,1.6);
\end{TikzFigure}

\newhtmlpage

At any quantity less than this output, profit would rise with additional
output. This is because, from a less-than-optimal output, the additional
revenue from increased sales exceeds the increased cost associated with
producing those units: Stated differently, the marginal revenue would exceed
the marginal cost. Conversely, outputs greater than the optimum result in a 
$MR$ less than the associated $MC$. Accordingly, since outputs where $MR>MC$
are too low, and outputs where $MR<MC$ are too high, the optimum must be
where the $MR=MC$. Hence, the equality between $MR$ and $MC$ is implied in
this diagram at the output where the difference between $TR$ and $TC$ is
greatest.

Note finally that total revenue is maximized where the $TR$ curve reaches a
peak. In this example that occurs at a value of 12 units of output. This is
to be anticipated, as we learnt in Chapter~\ref{chap:elasticities}, because the midpoint of the
demand schedule in Table~\ref{table:discretequantities} occurs at that value.

\newhtmlpage

% Figure 10.6
\begin{TikzFigure}{xscale=0.63,yscale=0.53,caption={Market demand, the $MR$ curve, and the monopolist's $AC$ and $MC$ curves \label{fig:marketdemandmonopolistacmc}}}
% axes
\draw [thick, -] (0,13) node (yaxis) [above] {Price} |- (15,0) node (xaxis) [right] {Quantity};
% demand
\draw [ultra thick,demandcolour,name path=D] (0,12) -- node [mynode,above right,black,pos=0.25] {$D$} (14,5);
% MR
\draw [ultra thick,mrcolour,name path=MR] (0,12) -- node [mynode,above right,black,pos=0.4] {$MR$} (12,0);
% MC
\draw [ultra thick,mccolour,name path=MC] (0,0) -- (14,7) node [mynode,right,black] {$MC$};
% AC
\draw [atccolour,ultra thick,name path=AC] (0,0) -- (14,3.5) node [mynode,right,black] {$AC$};
% intersection of MR and MC
\draw [name intersections={of=MR and MC, by=promax}];
\node [mynode,right=0cm and 0.25cm] at (promax) {Profit max};
% y-axis nodes
\foreach \y/\ytext in {2,4,6,8,10,12} \draw [thick] (-0.1,\y) node[mynode,left]{$\ytext$} -- +(0.2,0);
% x-axis nodes
\foreach \x/\xtext in {2,4,6,8,10,12,14} \draw [thick] (\x,-0.1) node[mynode,below] {$\xtext$} -- +(0,0.2);
% dotted vertical line from q=8 to D line
\path [name path=dottedlinepath] (8,0) -- +(0,13);
\draw [name intersections={of=D and dottedlinepath, by=dottedlinetoD}]
	[thick,dotted] (xaxis -| dottedlinetoD) -- (dottedlinetoD);
\end{TikzFigure}

Figure~\ref{fig:marketdemandmonopolistacmc} displays the demand curve for the market, the $MR$ curve, and
the monopolist's $MC$ and $AC$ curves. Consider first the marginal revenue
curve. In contrast to the previous example, where only whole or integer
units could be sold, in this example units can be sold in fractional
amounts, and the $MR$ curve must reflect this. To determine the position of
the $MR$ curve, note that with a straight-line demand curve total revenue is
a maximum at the midpoint of the demand curve. Any increase in output
results in reduced revenue: Stated differently, the marginal revenue becomes
negative at that output. Up to that output the $MR$ is positive, as
illustrated in Figure~\ref{fig:totalmarginalrevenue}. Accordingly, the $MR$ curve must intersect the
quantity axis midway between zero and the horizontal-axis intercept of the
demand curve. Geometrically, since the $MR$ intersects the quantity axis
half way to the horizontal intercept of the demand curve, it must have a
slope that is twice the slope of the demand curve.

By observing the data in columns 1 and 2 of the table, the demand curve 
intercepts are $\{\$12,24\}$, and from above discussion the $MR$ curve 
has intercepts $\{\$12,12\}$. The $AC$ is obtained by dividing $TC$ by 
output in Table~\ref{table:discretequantities}, and the $MC$ can be also 
calculated as the change in total cost divided by the change in output 
from Table~\ref{table:discretequantities}. The result of these calculations 
is displayed in Figure~\ref{fig:marketdemandmonopolistacmc}.

The profit maximizing output is 8 units, where $MC=MR$. The price at which 
8 units can be sold is read from the demand curve\footnote{It is not difficult 
	to show that the demand curve corresponding to the data in the table is 
	given by the expression $P=12-0.5q$. Since the $MR$ curve has twice the 
	slope of the demand curve, it is given by $MR=12-q$. The data indicate 
	that the $MC$ curve can be written as $MC=0.5q$ and the average cost curve 
	by $AC=0.25q$. Setting $MR=MC$ yields $q=8$. At this output the demand 
	curve implies the price is \$8. Profit is $(P-AC)\times q=(\$8-\$2)\times 8=\$48$.}, 
or the first column in Table~\ref{table:discretequantities}. It is \$8. And, 
as expected, this price-quantity combination maximizes profit. 
Table~\ref{table:discretequantities} indicates that profit is maximized at \$48, at $q=8$.

\newhtmlpage

\subsection*{Demand elasticity and marginal revenue}

We have shown above that the $MR$ curve cuts the horizontal axis at a
quantity where the elasticity of demand is unity. We know from Chapter~\ref{chap:elasticities}
that demand is elastic at points on the demand curve
above this unit-elastic point. Furthermore, since the intersection of $MR$
and $MC$ must be at a positive dollar value ($MC$ cannot be negative), then
it must be the case that the \textit{profit maximizing price for a
monopolist always lies on the elastic segment of the demand curve}.
	
\newhtmlpage

\subsection*{A general graphical representation}

In Figure~\ref{fig:monopolyeq} we generalize the graphical representation of
the monopoly profit maximizing output by allowing the $MC$ and $ATC$ curves
to be nonlinear. The optimal output is at $Q_{E}$, where $MR=MC$, and the
price $P_{E}$ sustains that output. With the average cost known, profit per
unit is AB, and therefore total profit is this margin multiplied by the
number of units sold, $Q_{E}$.

Total profit is therefore $P_{E}ABC_{E}.$

Note that the monopolist may not always make a profit. Losses could result
in Figure~\ref{fig:monopolyeq} if average costs were to rise so that the $ATC$
were everywhere above the demand curve, or if the demand curve shifted down to
being everywhere below the $ATC$ curve. In
the longer term the monopolist would have to either reduce costs or perhaps
stimulate demand through advertising if she wanted to continue in operation.

% Figure 10.7
\begin{TikzFigure}{xscale=1.25,yscale=0.34,descwidth=25em,caption={The monopoly equilibrium \label{fig:monopolyeq}},description={The profit maximizing output is $Q_E$, where $MC=MR$. This output can be sold at a price $P_E$. The cost per unit of $Q_E$ is read from the $ATC$ curve, and equals B. Per unit profit is therefore AB and total profit is $P_E$AB$C_E$.}}
% MC curve
\draw [dashed,mccolour,ultra thick,name path=MC] (0.5,6) to [out=60,in=270] (3.5,18) node [black,mynode,above] {$MC$};
% ATC curve
\draw [atccolour,ultra thick,domain=1:4,name path=ATC] plot (\x, {16-4*sqrt(2.25-(\x-2)*(\x-3))}) node [mynode,black,above] {$ATC$};
% MR line
\draw [dashed,demandcolour,ultra thick,name path=MR] (0,16) -- (4,0) node [pos=0.9,black,mynode,above right] {$MR$};
% demand line
\draw [demandcolour,ultra thick,name path=D] (0,16) -- coordinate[midway] (MidD) (8,0) node [pos=0.9,black,mynode,above right] {$D$};
% axes
\draw [thick, -] (0,20) node (yaxis) [above] {\$} |- (8,0) node (xaxis) [right] {Quantity};
% intersection of MC and MR
\draw [name intersections={of=MC and MR, by=E}]
	[dotted,thick] (E) -- (xaxis -| E) node [mynode,below] {$Q_E$};
% path to intersect with ATC and D along same x value as point E
\path [name path=Eline] (xaxis -| E) -- +(0,20);
% intersection of Eline with ATC and D
\draw [name intersections={of=Eline and ATC, by=B},name intersections={of=Eline and D, by=A}]
	[dotted,thick] (yaxis |- A) node [mynode,left] {$P_E$} -- (A) node [mynode,above right=-0.1cm and 0cm] {A} -- (E)
	[dotted,thick] (yaxis |- B) node [mynode,left] {$C_E$} -- (B) node [mynode,above right=-0.1cm and 0cm] {B};
% arrow to midpoint of demand
\draw [<-,thick,shorten <=1mm] (MidD) -- +(1,3) node [mynode,right] {Midpoint of Demand\\Demand elasticity$=-1$};
\end{TikzFigure}