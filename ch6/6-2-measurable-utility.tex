\section{Choice with measurable utility}\label{sec:ch6sec2}

Neal loves to pump his way through the high-altitude powder at the Whistler
ski and snowboard resort. His student-rate lift-ticket cost is \$30 per
visit. He also loves to frequent the jazz bars in downtown Vancouver, and
each such visit costs him \$20. With expensive passions, Neal must allocate
his monthly entertainment budget carefully. He has evaluated how much
satisfaction, measured in utils, he obtains from each snowboard outing and
each jazz club visit. We assume that these utils are measurable, and use the
term \terminology{cardinal utility} to denote this. These measurable utility
values are listed in columns 2 and 3 of Table~\ref{table:utilsnowjazz}. They
define the \terminology{total utility} he gets from various amounts of the
two activities.

\begin{Table}{caption={Utils from snowboarding and jazz \label{table:utilsnowjazz}},description={Price of snowboard visit=\$30. Price of jazz club visit=\$20.},descwidth={22.5em}}
	\begin{tabu} to \linewidth {|X[0.5,c]X[1,c]X[1,c]X[1,c]X[1,c]X[1,c]X[1,c]|} \hline 
		\rowcolor{rowcolour}	1 & 2 & 3 & 4 & 5 & 6 & 7 \\ \hline 
		\textbf{Visit} & \textbf{Total} & \textbf{Total} & \textbf{Marginal} & \textbf{Marginal} & \textbf{Marginal} & \textbf{Marginal} \\[-0.1em]
		\textbf{\#}	&	\textbf{snowboard}	&	\textbf{jazz}	&	\textbf{snowboard}	&	\textbf{jazz utils}	&	\textbf{snowboard}	&	\textbf{jazz utils}	\\[-0.1em]
			&	\textbf{utils}	&	\textbf{utils}	&	\textbf{utils}	&	&	\textbf{utils per \$}	&	\textbf{per \$}	\\	\hline
		\rowcolor{rowcolour}	1 & 72 	& 52 	& 72 & 52 & 2.4	& 2.6 	\\
		2 & 132 & 94 	& 60 & 42 & 2.0 & 2.1 	\\
		\rowcolor{rowcolour}	3 & 182 & 128 & 50 & 34 & 1.67 	& 1.7 	\\ 
		4 & 224 & 156 & 42 & 28 & 1.4 	& 1.4	\\ 
		\rowcolor{rowcolour}	5 & 260 & 180 & 36 & 24 & 1.2 	& 1.2	\\
		6 & 292 & 201 & 32 & 21 & 1.07 	& 1.05	\\
		\rowcolor{rowcolour}	7 & 321 & 220 & 29 & 19 & 0.97 	& 0.95	\\ \hline 
	\end{tabu}
\end{Table}

\begin{DefBox}
	\textbf{Cardinal utility} is a measurable concept of satisfaction.
	
	\textbf{Total utility} is a measure of the total satisfaction derived from consuming a given amount of goods and services.
\end{DefBox}

\newhtmlpage

Neal's total utility from each activity in this example is independent of
the amount of the other activity he engages in. These total utilities are
plotted in Figures~\ref{fig:TUsnowboarding} and \ref{fig:TUjazz}.
Clearly, more of each activity yields more utility, so the additional or %
\terminology{marginal utility} ($MU$) of each activity is positive. This
positive marginal utility for any amount of the good consumed, no matter how
much, reflects the assumption of \textit{non-satiation}---more is always
better. Note, however, that the decreasing slopes of the total utility
curves show that \textit{total utility is increasing at a diminishing rate}.
While more is certainly better, each additional visit to Whistler or a jazz
club augments Neal's utility by a smaller amount. At the margin, his
additional utility declines: He has \terminology{diminishing marginal utility}. 
The marginal utilities associated with snowboarding and jazz are
entered in columns 4 and 5 of Table~\ref{table:utilsnowjazz}. They are the
differences in total utility values when consumption increases by one unit.
For example, when Neal makes a sixth visit to Whistler his total utility
increases from 260 utils to 292 utils. His marginal utility for the sixth
unit is therefore 32 utils, as defined in column 4. In light of this
example, it should be clear that we can define marginal utility as:

\begin{equation}\label{eq:marginalutility}
	\mbox{Marginal Utility} = \frac{\mbox{additional utility}}{\mbox{additional consumption}} \mbox{~~or,~~} \mbox{MU} = \frac{\Delta U}{\Delta C}\mbox{,}
\end{equation}

where $\Delta x$ denotes the change in the quantity consumed of the good or
service in question.

\begin{DefBox}
	\textbf{Marginal utility} is the addition to total utility created when one more unit of a good or service is consumed.
	
	\textbf{Diminishing marginal utility} implies that the addition to total utility from each extra unit of a good or service consumed is declining.
\end{DefBox}

% Figure 6.1
\begin{TikzFigure}{xscale=1,yscale=1,caption={TU from snowboarding \label{fig:TUsnowboarding}}}
\begin{axis}[
	axis line style=thick,
	every tick label/.append style={font=\footnotesize},
	every node near coord/.append style={font=\scriptsize},
	xticklabel style={anchor=north,/pgf/number format/1000 sep=},
	scaled y ticks=false,
	yticklabel style={/pgf/number format/fixed,/pgf/number format/1000 sep = \thinspace},
	xmin=0,xmax=9,ymin=0,ymax=350,
	x=0.9cm/1,
	y=0.9cm/50,
	xlabel={Visits to mountain},
	ylabel={Utils},
]
\addplot[datasetcolourthree,ultra thick,mark=+] coordinates {
	(1,72)
	(2,132)
	(3,182)
	(4,224)
	(5,260)
	(6,292)
	(7,321)
};
\end{axis}
\end{TikzFigure}

% Figure 6.2
\begin{TikzFigure}{xscale=1,yscale=1,caption={TU from jazz \label{fig:TUjazz}}}
\begin{axis}[
	axis line style=thick,
	every tick label/.append style={font=\footnotesize},
	every node near coord/.append style={font=\scriptsize},
	xticklabel style={anchor=north,/pgf/number format/1000 sep=},
	scaled y ticks=false,
	yticklabel style={/pgf/number format/fixed,/pgf/number format/1000 sep = \thinspace},
	xmin=0,xmax=9,ymin=0,ymax=275,
	x=0.9cm/1,
	y=1.15cm/50,
	xlabel={Jazz club visits},
	ylabel={Utils},
]
\addplot[datasetcolourthree,ultra thick,mark=+] coordinates {
	(1,52)
	(2,94)
	(3,128)
	(4,156)
	(5,180)
	(6,201)
	(7,220)
};
\addplot[black,dashed,thick,mark=none] coordinates {
	(2,94)
	(3,94)
	(3,128)
};
\node [mynode,below] at (axis cs:2.5,94) {1};
\node [mynode,right] at (axis cs:3,111) {34};
\addplot[black,dashed,thick,mark=none] coordinates {
	(4,156)
	(5,156)
	(5,180)
};
\node [mynode,below] at (axis cs:4.5,156) {1};
\node [mynode,right] at (axis cs:5,168) {24};
\end{axis}
\end{TikzFigure}

% Figure 6.3
\begin{TikzFigure}{xscale=1,yscale=1,caption={MU from snowboarding \label{fig:MUsnowboarding}}}
\begin{axis}[
	axis line style=thick,
	every tick label/.append style={font=\footnotesize},
	every node near coord/.append style={font=\scriptsize},
	xticklabel style={anchor=north,/pgf/number format/1000 sep=},
	scaled y ticks=false,
	yticklabel style={/pgf/number format/fixed,/pgf/number format/1000 sep = \thinspace},
	xmin=0,xmax=9,ymin=0,ymax=90,
	x=0.9cm/1,
	y=0.8cm/10,
	xlabel={Visits to mountain},
	ylabel={Utils},
]
\addplot[datasetcolourthree,ultra thick,mark=+] coordinates {
	(1,72)
	(2,60)
	(3,50)
	(4,42)
	(5,36)
	(6,32)
	(7,29)
};
\end{axis}
\end{TikzFigure}

% Figure 6.4
\begin{TikzFigure}{xscale=1,yscale=1,caption={MU from jazz \label{fig:MUjazz}}}
\begin{axis}[
	axis line style=thick,
	every tick label/.append style={font=\footnotesize},
	every node near coord/.append style={font=\scriptsize},
	xticklabel style={anchor=north,/pgf/number format/1000 sep=},
	scaled y ticks=false,
	yticklabel style={/pgf/number format/fixed,/pgf/number format/1000 sep = \thinspace},
	xmin=0,xmax=9,ymin=0,ymax=70,
	x=0.9cm/1,
	y=1cm/10,
	xlabel={Jazz club visits},
	ylabel={Utils},
]
\addplot[datasetcolourthree,ultra thick,mark=+] coordinates {
	(1,52)
	(2,42)
	(3,34)
	(4,28)
	(5,24)
	(6,21)
	(7,19)
};
\end{axis}
\end{TikzFigure}

\newhtmlpage

The marginal utilities associated with consuming different amounts of the
two goods are plotted in Figures~\ref{fig:MUsnowboarding} and \ref{fig:MUjazz},
using the data from columns 4 and 5 in Table~\ref{table:utilsnowjazz}.
These functions are declining, as indicated by their
negative slope. It should also be clear that the $MU$ curves can be derived
from the $TU$ curves. For example, in figure \ref{fig:TUjazz}, when going
from 2 units to 3 units of Jazz, $TU$ increases by 34 units. But $34/1$ is
the slope of the $TU$ function in this range of consumption -- the vertical
distance divided by the horizontal distance. Similarly, if jazz consumption
increases from 4 units to 5 units the corresponding change in $TU$ is 24
units, again the vertical distance divided by the horizontal distance, and
so the slope of the function. In short, the $MU$ is the slope of the $TU$
function.

Now that Neal has defined his utility schedules, he must consider the price
of each activity. Ultimately, when deciding how to allocate his monthly
entertainment budget, he must evaluate how much utility he gets from each
dollar spent on snowboarding and jazz: What ``bang for his buck'' does he
get? Let us see how he might go about allocating his budget. \textit{When he
	has fully spent his budget in the manner that will yield him greatest
	utility, we say that he has attained equilibrium}, because he will have no
incentive to change his expenditure patterns.

If he boards once, at a cost of \$30, he gets 72 utils of satisfaction,
which is 2.4 utils per dollar spent ($=72/30$). One visit to a jazz club
would yield him 2.6 utils per dollar ($=52/20$). Initially, therefore, his
dollars give him more \textit{utility per dollar} when spent on jazz. His 
$MU$ per dollar spent on each activity is given in the final two columns of
the table. These values are obtained by dividing the $MU$ associated with
each additional unit by the good's price.

We will assume that Neal has a budget of \$200. He realizes that his initial
expenditure should be on a jazz club visit, because he gets more utility per
dollar spent there. Having made one such expenditure, he sees that a second jazz
outing would yield him 2.1 utils per dollar expended, while a first visit to
Whistler would yield him 2.4 utils per dollar. Accordingly, his second
activity is a snowboard outing.

Having made one jazz and one snowboarding visit, he then decides upon a
second jazz club visit for the same reason as before---utility value for his
money. He continues to allocate his budget in this way until his budget is
exhausted. In our example, this occurs when he spends \$120 on four
snowboarding outings and \$80 on four jazz club visits. At this %
\terminology{consumer equilibrium}, he gets the same utility value per
dollar for the last unit of each activity consumed. This is a necessary
condition for him to be maximizing his utility, that is, to be in
equilibrium.

\begin{DefBox}
	\textbf{Consumer equilibrium} occurs when marginal utility per dollar spent on the last unit of each good is equal.
\end{DefBox}

\newhtmlpage

To be absolutely convinced of this, imagine that Neal had chosen instead to
board twice and to visit the jazz clubs seven times; this combination would
also exhaust his \$200 budget exactly. With such an allocation, he would get
2.0 utils per dollar spent on his marginal (second) snowboard outing, but
just 0.95 utils per dollar spent on his marginal (seventh) jazz club visit.%
\footnote{Note that, with two snowboard outings and seven jazz club visits, total
	utility is 352 $(=132+220)$, while the optimal combination of four of each
	yields a total utility of 380 $(=224+156)$.} If, instead, he were to
reallocate his budget in favour of snowboarding, he would get 1.67 utils per
dollar spent on a third visit to the hills. By reducing the number of jazz
visits by one, he would lose 0.95 utils per dollar reallocated.
Consequently, the utility gain from a reallocation of his budget towards
snowboarding would outweigh the utility loss from allocating fewer dollars
to jazz. His initial allocation, therefore, was not an optimum, or
equilibrium.

Only when the utility per dollar expended on each activity is equal at the
margin will Neal be optimizing. When that condition holds, a reallocation
would be of no benefit to him, because the gains from one more dollar on
boarding would be exactly offset by the loss from one dollar less spent on
jazz. Therefore, we can write the equilibrium condition as

\begin{equation}\label{eq:utilequilibriumcondition}
	\mbox{Equilibrium requires:~~}\frac{MU_{s}}{P_{s}}=\frac{MU_{j}}{P_{j}}\mbox{~~or~~}\frac{MU_{s}}{MU_{j}}=\frac{P_{s}}{P_{j}}. 
\end{equation}

While this example has just two goods, in the more general case of many
goods, this same \textit{condition must hold for all pairs of goods} on
which the consumer allocates his or her budget.

\newhtmlpage

\subsection*{From utility to demand}

Utility theory is a useful way of analyzing how a consumer makes choices.
But in the real world we do not observe a consumer's utility, either total
or marginal. Instead, his or her behaviour in the marketplace is observed
through the demand curve. How are utility and demand related?

Demand functions relate the quantity of a good consumed to the price of that
good, other things being equal. So let us trace out the effects of a price
change on demand, with the help of this utility framework. We will introduce
a simplification here: Goods are divisible, or that they come in small
packages relative to income. Think, for example, of kilometres driven per
year, or liters of gasoline purchased. Conceptualizing things in this way
enables us to imagine more easily experiments in which small amounts of a
budget are allocated one dollar at a time. In contrast, in the
snowboard/jazz example, we had to reallocate the budget in lumps of \$30 or
\$20 at a time because we could not ``fractionalize'' these goods.

The effects of a price change on a consumer's demand can be seen through the
condition that describes his or her equilibrium. If income is allocated to,
say, three goods $\{a, b, c\}$, such that $MU_a/P_a=MU_b/P_b=MU_c/P_c$, and
the price of, say, good $b$ falls, the consumer must reallocate the budget so
that once again the $MU$s per dollar spent are all equated. How does he do
this? Clearly, if he purchases more or less of any one good, the $MU$
changes. If the price of good $b$ falls, then the consumer initially gets more
utility from good $b$ \textit{for the last dollar he spends on it} (the
denominator in the expression $MU_b/P_b$ falls, and consequently the value
of the ratio rises to a value greater than the values for goods $a$ and $c$).

\newhtmlpage

The consumer responds to this, in the first instance, by buying more of the
cheaper good. He obtains more total utility as a consequence, and in the
process will get \textit{less} utility \textit{at the margin} from that
good. In essence, the numerator in the expression then falls, in order to
realign it with the lower price. This equality also provides an underpinning
for what is called the \terminology{law of demand}: More of a good is
demanded at a lower price. If the price of any good falls, then, in order
for the equilibrium condition to be re-established, the $MU$ of that good
must be driven down also. Since $MU$ declines when more is purchased, this
establishes that demand curves must slope downwards.

\begin{DefBox}
	The \textbf{law of demand} states that, other things being equal, more of a good is demanded the lower is its price.
\end{DefBox}

However, the effects of a price decline are normally more widespread than
this, because the quantities of other goods consumed may also change. As
explained in earlier chapters, the decline in the price of good $b$ will lead
the consumer to purchase more units of \textit{complementary goods} and
fewer units of goods that are \textit{substitutes}. So the whole budget
allocation process must be redetermined in response to any price change. But
at the end of the day, a new equilibrium must be one where the marginal
utility per dollar spent on each good is equal.

\newhtmlpage

\subsection*{Applying the theory}

The demand curves developed in Chapter~\ref{chap:classical} can be related
to the foregoing utility analysis. In our example, Neal purchased four lift
tickets at Whistler when the price was \$30. We can think of this
combination as one point on his demand curve, where the ``other things kept
constant'' are the price of jazz, his income, his tastes, etc.

Suppose now that the price of a lift ticket increased to \$40. How could we
find another point on his demand curve corresponding to this price, using
the information in Table~\ref{table:utilsnowjazz}? The marginal utility per
dollar associated with each visit to Whistler could be recomputed by
dividing the values in column 4 by 40 rather than 30, yielding a new column
6. We would then determine a new allocation of his budget between the two
goods that would maximize utility. After such a calculation we would find
that he makes three visits to Whistler and four jazz-club visits. Thus, the
combination $(P_s=\$40,Q_s=3)$ is another point on his demand curve. Note
that this allocation exactly exhausts his \$200 budget.

By setting the price equal to \$20, this exercise could be performed again,
and the outcome will be a quantity demanded of lift tickets equal to seven
(plus three jazz club visits). Thus, the combination $(P_s=\$20,Q_s=7)$ is
another point on his demand curve. Figure~\ref{fig:utiltodemand} plots a
demand curve going through these three points.

By repeating this exercise for many different prices, the demand curve is
established. We have now linked the demand curve to utility theory.


% FIGURE 6.5
\begin{TikzFigure}{xscale=0.45,yscale=0.08,descwidth=25em,caption={Utility to demand \label{fig:utiltodemand}},description={When $P=\$30$, the consumer finds the quantity such that $MU/P$ is equal for all purchases. The corresponding quantity purchased is 4 tickets. At prices of \$40 and \$20 the equilibrium condition implies quantities of 3 and 7 respectively.}}
% demand line
\draw [demandcolour,ultra thick,name path=demand] (5.3,53) to [out=-87,in=150] (18,17.25) node [black,mynode,above] {Demand};
% axes
\draw [thick, -] (0,60) node (yaxis) [above] {Price} |- (18,0) node (xaxis) [right] {Quantity};
% paths to create dotted lines
\path [name path=P20] (0,20) -- +(18,0);
\path [name path=P30] (0,30) -- +(18,0);
\path [name path=P40] (0,40) -- +(18,0);
% intersection of demand line with paths
\draw [name intersections={of=demand and P20, by=Q7},name intersections={of=demand and P30, by=Q4},name intersections={of=demand and P40, by=Q3}]
	[dotted,thick] (yaxis |- Q7) node [mynode,left] {20} -| (xaxis -| Q7) node [mynode,below] {7}
	[dotted,thick] (yaxis |- Q4) node [mynode,left] {30} -| (xaxis -| Q4) node [mynode,below] {4}
	[dotted,thick] (yaxis |- Q3) node [mynode,left] {40} -| (xaxis -| Q3) node [mynode,below] {3};
\end{TikzFigure}

\newhtmlpage

\begin{ApplicationBox}{caption={Individual and Collective Utility \label{app:indcolutilityapp}}}
The example developed in the text is not far removed from what economists do
in practice. From a philosophical standpoint, economists are supposed to be
interested in the well-being of the citizens who make up an economy or a
country. To determine how `well-off' citizens may be, social scientists
frequently carry out surveys on how `content' or `happy' people are in
their every-day lives. For example, the \textit{Earth Institute at Columbia
University} regularly produces a `World Happiness Report'. The report is
based upon responses to survey questions in numerous economies. One of the
measures it uses to compare utility levels is the \textit{Cantril ladder}. This is an
11-point scale running from 0 to 10, with the lowest value signifying the
worst possible life, and 10 the highest possible quality of life. In
reporting their findings, the researchers are essentially claiming that some
economies have, on average, more contented or happier, people than others.
Utility can be considered in exactly this way: A higher reported value on
the Cantril ladder suggests higher utility.

A slightly different measure of well-being across economies is given by the
\textit{United Nations Human Development Index}. In this case, countries score high
by having a high level of income, good health (as measured by life
expectancy), and high levels of education, as measured by the number of
years of education completed or envisaged.

In practice, social scientists are very comfortable using utility-based
concepts to describe the economic circumstances of individuals in different
economies.
\end{ApplicationBox}
