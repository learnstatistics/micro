\section{Applications of indifference analysis}\label{sec:ch6sec4}

\subsection*{Price impacts: Complements and substitutes}

The nature of complements and substitutes, defined in Chapter~\ref{chap:elasticities},
can be further understood with the help of Figure~\ref{fig:incomepriceadj}.
The new equilibrium $E_2$ has been drawn so that the
increase in the price of jazz results in more snowboarding---the quantity of 
$S$ increases to $S_2$ from $S_0$. These goods are substitutes in this
picture, because snowboarding \textit{increases} in response to an \textit{%
	increase} in the price of jazz. If the new equilibrium $E_2$ were at a point
yielding a lower level of $S$ than $S_0$, we would conclude that they were
complements.

\subsection*{Cross-price elasticities}

Continuing with the same price increase in jazz, we could compute the 
\textit{percentage} change in the quantity of snowboarding demanded as a
result of the \textit{percentage} change in the jazz price. In this example,
the result would be a positive elasticity value, because the quantity change
in snowboarding and the price change in jazz are both in the same direction,
each being positive.

\subsection*{Income impacts: Normal and inferior goods}

We know from Chapter~\ref{chap:elasticities} that the quantity demanded of a 
\textit{normal good} increases in response to an income increase, whereas
the quantity demanded of an \textit{inferior good} declines. Clearly, both
jazz and boarding are normal goods, as illustrated in Figure~\ref{fig:incomepriceadj},
because more of each one is demanded in response to
the income increase from $I_{0}$ to $I_{1}$. It would challenge the
imagination to think that either of these goods might be inferior. But if $J$
were to denote junky (inferior) goods and $S$ super goods, we could
envisage an equilibrium $E_{1}$ to the northwest of $E_{0}$ in response to
an income increase, along the constraint $I_{1}$; less $J$ and more $S$
would be consumed in response to the income increase.

\newhtmlpage

\subsection*{Policy: Income transfers and price subsidies}

Government policies that improve the purchasing power of low-income
households come in two main forms: Pure income transfers and price
subsidies. \textit{Social Assistance} payments (``welfare'') or 
\textit{Employment Insurance} benefits, for example, provide an increase in income
to the needy. Subsidies, on the other hand, enable individuals to purchase
particular goods or services at a lower price---for example, rent or daycare
subsidies.

In contrast to taxes, which \textit{reduce} the purchasing power of the
consumer, subsidies and income transfers \textit{increase} purchasing power.
The impact of an income transfer, compared with a pure price subsidy, can be
analyzed using Figures~\ref{fig:incometransfer} and \ref{fig:pricesubsidy}.

% Figure 6.11
\begin{TikzFigure}{xscale=0.38,yscale=0.34,descwidth=25em,caption={Income transfer \label{fig:incometransfer}},description={An increase in income due to a government transfer shifts the budget constraint from $I_1$ to $I_2$. This parallel shift increases the quantity consumed of the target good (daycare) \textit{and} other goods, unless one is inferior.}}
% thick black lines tangent to the indifference curves
\draw [thick,name path=I1] (0,13.55) -- (13,0) node [mynode,above right] {$I_1$};
\draw [thick,name path=I2] (0,19.55) -- (19,0) node [mynode,above right] {$I_2$};
% indifference curves
\draw [indiffcolour,ultra thick,name path=U1] (4,13) node [black,mynode,above left] {$U_1$} to [out=270,in=180] (13,4);
\draw [indiffcolour,ultra thick,name path=U2] (7,16) node [black,mynode,above left] {$U_2$} to [out=270,in=180] (16,7);
% axes
\draw [thick, -] (0,20) node (yaxis) [above] {Other goods} |- (25,0) node (xaxis) [right] {Daycare};
% intersection of black I lines with indiff curves
\draw [name intersections={of=I1 and U1, by=E1},name intersections={of=I2 and U2, by=E2}]
	[dotted,thick] (yaxis |- E1) -- (E1) node [mynode,below left] {$E_1$} -- (xaxis -| E1)
	[dotted,thick] (yaxis |- E2) -- (E2) node [mynode,above right] {$E_2$} -- (xaxis -| E2);
\end{TikzFigure}

\newhtmlpage

In Figure~\ref{fig:incometransfer}, an \textit{income transfer} increases
income from $I_1$ to $I_2$. The new equilibrium at $E_2$ reflects an
increase in utility, and an increase in the consumption of \textit{both}
daycare and other goods.

Suppose now that a government program administrator decides that, while
helping this individual to purchase more daycare accords with the intent of
the transfer, she does not intend that government money should be used to
purchase other goods. She therefore decides that a daycare \textit{subsidy}
program might better meet this objective than a pure income transfer.

A daycare subsidy reduces the price of daycare and therefore \textit{rotates
	the budget constraint outwards around the intercept on the vertical axis}.
At the equilibrium in Figure~\ref{fig:pricesubsidy}, purchases of other
goods change very little, and therefore most of the additional purchasing
power is allocated to daycare.

% Figure 6.12
\begin{TikzFigure}{xscale=0.38,yscale=0.3,descwidth=25em,caption={Price subsidy \label{fig:pricesubsidy}},description={A subsidy to the targeted good, by reducing its price, rotates the budget constraint from $I_1$ to $I_2$. This induces the consumer to direct expenditure more towards daycare and less towards other goods than an income transfer that does not change the relative prices.}}
% thick black lines tangent to the indiff curves
\draw [thick,name path=I2] (0,13.55) -- (24,2.32023) node [mynode,above right] {$I_2$};
\draw [thick,name path=I1] (0,13.55) -- (13,0) node [mynode,above right] {$I_1$};
% indifference curves
\draw [indiffcolour,ultra thick,name path=U1] (4,13) node [black,mynode,above left] {$U_1$} to [out=270,in=180] (13,4);
\draw [indiffcolour,ultra thick,name path=U2] (7,16) node [black,mynode,above left] {$U_2$} to [out=270,in=180] (16,7);
% axes
\draw [thick, -] (0,20) node (yaxis) [above] {Other goods} |- (25,0) node (xaxis) [right] {Daycare};
% intersection of I lines with indifference curves
\draw [name intersections={of=I1 and U1, by=E1},name intersections={of=I2 and U2, by=E2}]
	[dotted,thick] (yaxis |- E1) -- (E1) node [mynode,below left] {$E_1$} -- (xaxis -| E1)
	[dotted,thick] (yaxis |- E2) -- (E2) node [mynode,below left] {$E_2$} -- (xaxis -| E2);
\end{TikzFigure}

\newhtmlpage

Let us take the example one stage further. From the initial equilibrium 
$E_{1}$ in Figure~\ref{fig:pricesubsidy}, suppose that, instead of a subsidy
that took the individual to $E_{2}$, we gave an income transfer \textit{that
	enabled the consumer to purchase the combination $E_{2}$}. Such a transfer
is represented in Figure~\ref{fig:subsidytransfercomp} by a parallel outward
shift of the budget constraint from $I_{1}$ to $I'_{1}$, going
through the point $E_{2}$. We now have a
subsidy policy and an alternative income transfer policy, each permitting
the same consumption bundle ($E_{2}$). The interesting aspect of this pair
of possibilities is that the income transfer will enable the consumer to
attain a higher level of satisfaction---for example, at point $E'$%
---and will also induce her to consume more of the good on the vertical
axis. The higher level of satisfaction comes about because the consumer has
more latitude in allocating the additional real income.

\newhtmlpage

\begin{ApplicationBox}{caption={Daycare subsidies in Quebec \label{app:daycarequebec}}}
	The Quebec provincial government subsidizes daycare very heavily. In the network of daycares that are part of the government-sponsored ``Centres de la petite enfance'', lower- and middle-income households can place their children in daycare for about \$10 per day. This policy is designed to enable households to limit the share of their income expended on daycare (though higher-income households are also heavily subsidized). This policy is described accurately in Figure~\ref{fig:subsidytransfercomp}.
	
	The consequences of strong subsidization are not negligible: Excess demand, to such an extent that children are frequently placed on waiting lists for daycare places long before their parents intend to use the service. Annual subsidy costs amount to almost \$2 billion per year. At the same time, it has been estimated that the policy has enabled many more parents to enter the workforce than otherwise would have. 
\end{ApplicationBox}

% Figure 6.13
\begin{TikzFigure}{xscale=0.38,yscale=0.32,descwidth=25em,caption={Subsidy-transfer comparison \label{fig:subsidytransfercomp}},description={A price subsidy to the targeted good induces the individual to move from $E_1$ to $E_2$, facing a budget constraint $I_2$. An income transfer that permits him to consume $E_2$ is given by $I'_1$; but it also permits him to attain a higher level of satisfaction, denoted by $E'$ on the indifference curve $U_3$.}}
% black lines tangent to indiff curves
\draw [thick,name path=Iprime] (1.48661,19) -- (19.7154,0) node [mynode,above right] {$I_1'$};
\draw [thick,name path=I2] (0,13.55) -- (24,2.32023) node [mynode,above right] {$I_2$};
\draw [thick,name path=I1] (0,13.55) -- (13,0) node [mynode,above right] {$I_1$};
% indifference curves
\draw [indiffcolour,ultra thick,name path=U1] (4,13) node [black,mynode,above left=0cm and -0.25cm] {$U_1$} to [out=270,in=180] (13,4);
\draw [indiffcolour,ultra thick,name path=U2] (7,16) node [black,mynode,above left=0cm and -0.2cm] {$U_2$} to [out=270,in=180] (16,7);
\draw [indiffcolour,ultra thick,name path=U3] (7.45,16.4) node [black,mynode,above right=0cm and -0.2cm] {$U_3$} to [out=270,in=180] (16.45,7.4);
% axes
\draw [thick, -] (0,20) node (yaxis) [above] {Other goods} |- (25,0) node (xaxis) [right] {Daycare};
% intersection of black lines with indiff curves
\draw [name intersections={of=I1 and U1, by=E1},name intersections={of=I2 and U2, by=E2},name intersections={of=Iprime and U3, by=Eprime}]
	[dotted,thick] (yaxis |- E1) -- (E1) node [mynode,below left] {$E_1$} -- (xaxis -| E1)
	[dotted,thick] (yaxis |- E2) -- (E2) node [mynode,below left] {$E_2$} -- (xaxis -| E2);
\node [mynode,right=0cm and 0.25cm] at (Eprime) {$E'$};
\end{TikzFigure}

\newhtmlpage

\subsection*{The price of giving}

Imagine now that the good on the horizontal axis is charitable donations,
rather than daycare, and the government decides that for every dollar given
the individual will see a reduction in their income tax of 50 cents. This is
equivalent to cutting the `price' of donations in half, because a donation
of one dollar now costs the individual half of that amount. Graphically the
budget constraint rotates outward with the vertical intercept unchanged.
Since donations now cost less the individual has increased spending power as
a result of the price reduction for donations. The price reduction is
designed to increase the attractiveness of donations to the utility
maximizing consumer.