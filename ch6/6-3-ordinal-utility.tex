\section{Choice with ordinal utility}\label{sec:ch6sec3}

\subsection*{The budget constraint}

In the preceding section, we assumed that utility is measurable in order to
better understand how consumers allocate their budgets, and how this process
is reflected in the market demands that are observed. The belief that
utility might be measurable is not too extreme in the modern era.
Neuroscientists are mapping more and more of the human brain and
understanding how it responds to positive and negative stimuli. At the same
time, numerous sociological surveys throughout the world ask individuals to
rank their happiness on a scale of one to ten, or something similar, with a
view to making comparisons between individual-level and group-level
happiness -- see Application Box~\ref{app:indcolutilityapp}. 
Nonetheless, not every scientist may be convinced that we should
formulate behavioural rules on this basis. Accordingly we now examine the
economics of consumer behaviour without this strong assumption. We assume
instead that individuals are able to identify (a) different combinations of
goods and services that yield equal satisfaction, and (b) combinations of
goods and services that yield more satisfaction than other combinations. In
contrast to measurable (or cardinal) utility, this concept is called ordinal
utility, because it assumes only that consumers can \underbar{order} utility
bundles rather than quantify the utility.

\begin{DefBox}
	\textbf{Ordinal utility} assumes that individuals can rank commodity bundles in accordance with the level of satisfaction associated with each bundle.
\end{DefBox}

\newhtmlpage

\subsection*{The budget constraint}

Neal's monthly expenditure limit, or \terminology{budget constraint}, is
\$200. In addition, he faces a price of \$30 for lift tickets and \$20 per
visit to jazz clubs. Therefore, using $S$ to denote the number of snowboard
outings and $J$ the number of jazz club visits, if he spends his entire
budget it must be true that the sum of expenditures on each activity
exhausts his budget or income ($I$):

\begin{align*}
	\mbox{Expenditure on snowboarding}+\mbox{expenditure on Jazz} &=\mbox{Income} \\
	(\mbox{Price of }S\times\mbox{quantity of }S)+(\mbox{price of }J\times\mbox{quantity of }J)&=\mbox{Income} \\
	P_{S}S+P_{J}J = I\mbox{~~or~~}\$30S+\$20J&=\$200
\end{align*}

Since many different combinations of the two goods are affordable, it
follows that the budget constraint defines all bundles of goods that the
consumer can afford with a given budget.

\begin{DefBox}
	The \textbf{budget constraint} defines all bundles of goods that the consumer can afford with a given budget.
\end{DefBox}

The budget constraint, then, is just what it claims to be---a limit on
behaviour. Neal's budget constraint is illustrated in Figure~\ref{fig:budgetline},
where the amount of each good consumed is given on the
axes. If he spends all of his \$200 income on jazz, he can make exactly ten
jazz club visits $(\$200/\$20 = 10)$. The calculation also applies to visits
to Whistler. The intercept value is always obtained by dividing income by
the price of the good or activity in question.

\newhtmlpage

% Figure 6.6
\begin{TikzFigure}{xscale=0.7,yscale=0.7,descwidth=25em,caption={The budget line \label{fig:budgetline}},description={FC is the budget constraint and defines the affordable combinations of snowboarding and jazz. F represents all income spent on snowboarding. Thus F$=I/P_s$. Similarly C$=I/P_j$. Points above FC are not attainable. The slope $=$ OF/OC $=(I/P_s)/(I/P_j)=P_j/P_s=20/30=2/3$. The affordable set is 0FC.}}
% axes
\draw [thick, -] (0,8) node (yaxis) [above] {Snowboarding} -- (0,0) node [mynode,below left] {0} -- (12,0) node (xaxis) [right] {Jazz};
% budget line
\draw [budgetcolour,ultra thick,name path=budget] (0,6.666) node [mynode,black,above right] {F$=I/P_s$} -- (10,0) node [black,mynode,above right] {C$=I/P_j$};
% paths to intersect with budget line
\path [name path=S2] (0,2) -- +(12,0);
\path [name path=S4] (0,4) -- +(12,0);
% intersection with budget line
\draw [name intersections={of=budget and S2, by=J7},name intersections={of=budget and S4, by=J4}]
	[dotted,thick]	(yaxis |- J7) node [mynode,left] {2} -- node [mynode,below,pos=0.3,yshift=-0.5em] {Affordable\\region $=$ 0FC} (J7) node (K) [mynode,above right] {K} -- (xaxis -| J7) node [mynode,below] {7}
	[dotted,thick] (yaxis |- J4) node [mynode,left] {4} -- (J4) node (A) [mynode,above right] {A} -- (xaxis -| J4) node [mynode,below] {4};
% vertical path from J4
\path [name path=J4line] (xaxis -| J4) -- +(0,8);
% intersection of J4line and S2
\draw [name intersections={of=J4line and S2, by=B}];
\node [mynode,above right] at (B) {B};
% Non-affordable region
\path [name path=NAffregion] (A) -- node [mynode,above right=1em and 1em] {Non-affordable\\region} (K);
\end{TikzFigure}

In addition to these affordable extremes, Neal can also afford many other
bundles, e.g., $(S=2,J=7)$, or $(S=4,J=4)$, or $(S=6,J=1)$. The set of
feasible, or \terminology{affordable}, combinations is bounded by the budget
line, and this is illustrated in Figure~\ref{fig:budgetline}.

\begin{DefBox}
	The \textbf{affordable set} of goods and services for the consumer is bounded by the budget line from above; the \textbf{non-affordable set} lies strictly above the budget line.
\end{DefBox}

\newhtmlpage

The slope of the budget line is informative. As illustrated in Chapter~\ref{chap:intro}, it
indicates how many snowboard visits must be sacrificed for one additional
jazz visit; it defines the consumer's \textit{trade-offs}. To illustrate:
Suppose Neal is initially at point A $(J=4,S=4)$, and moves to point K 
$(J=7,S=2)$. Clearly, both points are affordable. In making the move, he
trades two snowboard outings in order to get three additional jazz club
visits, a trade-off of 2/3. This trade-off is the slope of the budget line,
which, in Figure~\ref{fig:budgetline}, is AB/BK$=-2/3$, where the negative
sign reflects the downward slope.

Could it be that this ratio reflects the two prices (\$20/\$30)? The answer
is yes: The slope of the budget line is given by the vertical
distance divided by the horizontal distance, OF/OC. The points F and C were
obtained by dividing income by the respective price---remember that the jazz
intercept is $\$200/\$20 = 10$. Formally, that is $I/P_J$. The intercept on
the snowboard axis is likewise $I/P_S$. Accordingly, the slope of the budget 
constraint is:

\[
\text{Slope}=\text{OF/OC}=\frac{(I/P_s)}{(I/P_j)}=\frac{I}{P_s}\times\frac{P_j}{I}=\frac{P_j}{P_s}.
\]

Since the budget line has a negative slope, it is technically correct to
define it with a negative sign. But, as with elasticities, the sign is
frequently omitted.

\newhtmlpage

\subsection*{Tastes and indifference}

We now consider how to represent a consumer's tastes in two dimensions,
given that he can order, or rank, different consumption bundles, and that he
can define a series of different bundles that all yield the same
satisfaction. We limit ourselves initially to considering just ``goods,''
and not ``bads'' such as pollution.

% Figure 6.7
\begin{TikzFigure}{xscale=0.38,yscale=0.32,descwidth=25em,caption={Ranking consumption bundles \label{fig:rankconbundle}},description={L is preferred to R since more of each good is consumed at L, while points such as V are less preferred than R. Points W and T contain more of one good and less of the other than R. Consequently, we cannot say if they are preferred to R without knowing how the consumer trades the goods off -- that is, his preferences.}}
% thick lines separating bundles
\draw [budgetcolour,ultra thick,name path=BV] (7,2) -- (7,19);
\draw [budgetcolour,ultra thick,name path=BH] (3,6) -- (23,6);
% axes
\draw [thick, -] (0,20) node (yaxis) [above] {Snowboarding} |- (25,0) node (xaxis) [right] {Jazz};
% intersection of thick lines
\draw [name intersections={of=BV and BH, by=R}];
\node [mynode,above right] at (R) {R};
\node [mynode,above left=2em and 1em] at (R) {W};
\node [mynode,below left=1em and 1em] at (R) {V};
\node [mynode,below right=1em and 2em] at (R) {T};
\node [mynode,above right=3em and 3em] at (R) {L};
\node [mynode,above right=4em and 4em] at (R) {Region preferred to R};
\end{TikzFigure}

Figure~\ref{fig:rankconbundle} examines the implications of these
assumptions about tastes. Each point shows a consumption bundle of
snowboarding and jazz. Let us begin at bundle R. Since more of a good is
preferred to less, any point such as L, which lies to the northeast of R, is
preferred to R, since L offers more of both goods than R. Conversely, points
to the southwest of R offer \textit{less of each good} than R, and therefore
R is preferred to a point such as V.

\newhtmlpage

Without knowing the consumer's tastes, we cannot be sure at this stage how
points in the northwest and southeast regions compare with R. At W or T, the
consumer has more of one good and less of the other than at R. Someone who
really likes snowboarding might prefer W to R, but a jazz buff might prefer
T to R.

Let us now ask Neal to disclose his tastes, by asking him to define several
combinations of snowboarding and jazz that \textit{yield him exactly the
	same degree of satisfaction as the combination at R}. Suppose further, for
reasons we shall understand shortly, that his answers define a series of
points that lie on the beautifully smooth contour $U_R$ in Figure~\ref{fig:indiffcurves}.
Since he is indifferent between all points on $U_R$ by
construction, this contour is an \terminology{indifference curve}.

% Figure 6.8
\begin{TikzFigure}{xscale=0.38,yscale=0.32,descwidth=25em,caption={Indifference curves \label{fig:indiffcurves}},description={An indifference curve defines a series of consumption bundles, all of which yield the same satisfaction. The slope of an indifference curve is the marginal rate of substitution ($MRS$) and defines the number of units of the good on the vertical axis that the individual will trade for one unit of the good on the horizontal axis. The $MRS$ declines as we move south-easterly, because the consumer values the good more highly when he has less of it.}}
% indifference curves
\draw [indiffcolour,ultra thick,name path=UV] (1,10) to [out=270,in=180] node [mynode,below left,pos=0.5,black] {V} (10,1) node [black,mynode,right] {$U_V$};
\draw [indiffcolour,ultra thick,name path=UR] (4,13) to [out=270,in=180] (13,4) node [black,mynode,right] {$U_R$};
\draw [indiffcolour,ultra thick,name path=UL] (7,16) to [out=270,in=180] node [mynode,above right,pos=0.5,black] {L} (16,7) node [black,mynode,right] {$U_L$};
% axes
\draw [thick, -] (0,20) node (yaxis) [above] {Snowboarding} |- (25,0) node (xaxis) [right] {Jazz};
% paths to create solid black lines between points on indifference curve U_R
\path [name path=CMline] (4.5,0) -- +(0,20);
\path [name path=Rline] (6.5,0) -- +(0,20);
\path [name path=NPline] (9.5,0) -- +(0,20);
\path [name path=Hline] (11.5,0) -- +(0,20);
% intersection of paths with U_R line
\draw [name intersections={of=UR and CMline, by=M},name intersections={of=UR and Rline, by=R},name intersections={of=UR and NPline, by=N},name intersections={of=UR and Hline, by=H}]
	[thick,black] (M) node [mynode,left] {M} -- (R -| M) node [mynode,below] {C} -- (R) node [mynode,below] {R}
	[thick,black] (N) node [mynode,above] {N} -- (H -| N) node [mynode,below] {F} -- (H) node [mynode,below] {H};
\end{TikzFigure}

\newhtmlpage

\begin{DefBox}
	An \textbf{indifference curve} defines combinations of goods and services that yield the same level of satisfaction to the consumer.
\end{DefBox}

Pursuing this experiment, we could take other points in Figure~\ref{fig:indiffcurves},
such as L and V, and ask the consumer to define bundles
that would yield the same level of satisfaction, or indifference. These
combinations would yield additional contours, such as $U_L$ and $U_V$ in
Figure~\ref{fig:indiffcurves}. This process yields a series of indifference
curves that together form an \terminology{indifference map}.

\begin{DefBox}
	An \textbf{indifference map} is a set of indifference curves, where curves further from the origin denote a higher level of satisfaction.
\end{DefBox}

Let us now explore the properties of this map, and thereby understand why
the contours have their smooth convex shape. They have four properties. The
first three follow from our preceding discussion, and the fourth requires
investigation.

\begin{enumerate}
	\item Indifference curves \textit{further from the origin reflect higher
		levels of satisfaction}.
	
	\item Indifference curves are \textit{negatively sloped}. This reflects the
	fact that if a consumer gets more of one good she should have less of the
	other in order to remain indifferent between the two combinations.
	
	\item Indifference curves \textit{cannot intersect}. If two curves were to
	intersect at a given point, then we would have two different levels of
	satisfaction being associated with the same commodity bundle---an
	impossibility.
	
	\item Indifference curves are convex when viewed from the origin, reflecting
	a \textit{diminishing marginal rate of substitution}.
\end{enumerate}

The convex shape reflects an important characteristic of preferences: When
consumers have a lot of some good, they value a marginal unit of it less
than when they have a small amount of that good. More formally, they have a 
\textit{higher marginal valuation at low consumption levels}---that first
cup of coffee in the morning provides greater satisfaction than the second
or third cup.

\newhtmlpage

Consider the various points on $U_R$, starting at M in Figure~\ref{fig:indiffcurves}.
At M Neal snowboards a lot; at N he boards much less.
The convex shape of his indifference map shows that he values a marginal
snowboard trip more at N than at M. To see this, consider what happens as he
moves along his indifference curve, starting at M. We have chosen the
coordinates on $U_R$ so that, in moving from M to R, and again from N to H,
the additional amount of jazz is the same: CR$=$FH. From M, if Neal moves to
R, he consumes an additional amount of jazz, CR. By definition of the
indifference curve, he is willing to give up MC snowboard outings. The ratio
MC/CR defines his willingness to substitute one good for the other. This
ratio, being a vertical distance divided by a horizontal distance, is the
slope of the indifference curve and is called the \terminology{marginal rate
	of substitution}, $MRS$.

\begin{DefBox}
	The \textbf{marginal rate of substitution} is the slope of the indifference curve. It defines the amount of one good the consumer is willing to sacrifice in order to obtain a given increment of the other, while maintaining utility unchanged.
\end{DefBox}

At N, the consumer is willing to sacrifice the amount NF of boarding to get
the same additional amount of jazz. Note that, when he boards \textit{less},
as at N, he is willing to give up less boarding than when he has a lot of
it, as at M, in order to get the same additional amount of jazz. His
willingness to substitute \textit{diminishes} as he moves from M to N: The
quantity NF is less than the quantity MC. In order to reflect this taste
characteristic, the indifference curve has a \terminology{diminishing
	marginal rate of substitution}: A flatter slope as we move down along its
surface.

\begin{DefBox}
	A \textbf{diminishing marginal rate of substitution} reflects a higher marginal value being associated with smaller quantities of any good consumed.
\end{DefBox}

\newhtmlpage

\subsection*{Optimization}

We are now in a position to examine how the consumer optimizes---how he gets
to the highest level of satisfaction possible. The constraint on his
behaviour is the affordable set defined in Figure~\ref{fig:budgetline}, the
budget line.

Figure~\ref{fig:consumeroptimum} displays several of Neal's indifference
curves in conjunction with his budget constraint. We propose that he
maximizes his utility, or satisfaction, at the point E, on the indifference
curve denoted by $U_3$. While points such as F and G are also on the
boundary of the affordable set, they do not yield as much satisfaction as E,
because E lies on a higher indifference curve. \textit{The highest possible
	level of satisfaction is attained, therefore, when the budget line touches
	an indifference curve at just a single point---that is, where the constraint
	is tangent to the indifference curve}. E is such a point.

% Figure 6.9
\begin{TikzFigure}{xscale=0.38,yscale=0.32,descwidth=25em,caption={The consumer optimum \label{fig:consumeroptimum}},description={The budget constraint constrains the individual to points on or below HK. The highest level of satisfaction attainable is $U_3$, where the budget constraint just touches, or is just tangent to, it. At this optimum the slope of the budget constraint ($-P_j/P_s$) equals the $MRS$.}}
\draw [thick,-] (0,13.55) node [mynode,left] {H} -- (2.32044,11.1314) node [mynode,right] {F} -- (6.50558,6.76923) node [mynode,above right] {E} -- (11.6087,1.45014) node [mynode,above] {G} -- (13,0) node [mynode,below] {K};
\draw [indiffcolour,ultra thick,-]
	(0.5,10) to [out=270,in=180] (9.5,1) node [black,mynode,right] {$U_1$}
	(2,14) to [out=270,in=180] (15,1) node [black,mynode,right] {$U_2$}
	(4,13) to [out=270,in=180] (13,4) node [black,mynode,right] {$U_3$}
	(7,16) to [out=270,in=180] (16,7) node [black,mynode,right] {$U_4$};
\draw [thick, -] (0,20) node [above] {Snowboarding} |- (25,0) node [right] {Jazz};
\end{TikzFigure}

\newhtmlpage

This tangency between the budget constraint and an indifference curve
requires that the slopes of each be the same at the point of tangency. We
have already established that the slope of the budget constraint is the
negative of the price ratio ($=-P_{x}/P_{y}$). The slope of the indifference
curve is the marginal rate of substitution $MRS$. It follows, therefore,
that the \terminology{consumer optimizes} where the marginal rate of
substitution equals the slope of the price line.

Optimization requires:

\begin{equation}\label{eq:MRS}
	\text{Slope of Indifference curve}=\text{marginal rate of substitution}=-\frac{P_{j}}{P_{s}}.
\end{equation}

\begin{DefBox}
	A \textbf{consumer optimum} occurs where the chosen consumption bundle is a point such that the price ratio equals the marginal rate of substitution.
\end{DefBox}

Notice the resemblance between this condition and the one derived in the
first section as Equation~\ref{eq:utilequilibriumcondition}. There we argued
that equilibrium requires the ratio of the marginal utilities be same as the
ratio of prices. Here we show that the $MRS$ must equal the ratio of prices.
In fact, with a little mathematics it can be shown that the $MRS$ is indeed
the same as the (negative of the) ratio of the marginal utilities: 
$MRS=-MU_{j}/MU_{s}$. Therefore the two conditions are in essence the same!
However, it was not necessary to assume that an individual can actually
measure his utility in obtaining the result that the $MRS$ should equal the
price ratio in equilibrium. The concept of ordinal utility is sufficient.

\newhtmlpage

\subsection*{Adjusting to income changes}

Suppose now that Neal's income changes from \$200 to \$300. How will this
affect his consumption decisions? In Figure~\ref{fig:incomepriceadj}, this
change is reflected in a \textit{parallel} outward shift of the budget
constraint. Since no price change occurs, the slope remains constant. By
recomputing the ratio of income to price for each activity, we find that the
new snowboard and jazz intercepts are 10 $(=\$300/\$30)$ and 15 $(=\$300/\$20)$,
respectively. Clearly, the consumer can attain a higher level of
satisfaction---at a new tangency to a higher indifference curve---as a
result of the size of the affordable set being expanded. In Figure~\ref{fig:incomepriceadj},
the new equilibrium is at $E_1$.

% Figure 6.10
\begin{TikzFigure}{xscale=0.38,yscale=0.32,descwidth=25em,caption={Income and price adjustments \label{fig:incomepriceadj}},description={An income increase shifts the budget constraint from $I_0$ to $I_1$. This enables the consumer to attain a higher indifference curve. A price rise in jazz tickets rotates the budget line $I_0$ inwards around the snowboard intercept to $I_2$. The price rise reflects a lower real value of income and results in a lower equilibrium level of satisfaction.}}
% black lines tangent to Indiff curves
\draw [thick,name path=L0] (0,13.55) node [mynode,left] {H} -- (13,0) node [mynode,above right] {$I_0$};
\draw [thick,name path=L2] (0,13.55) -- (2.99032,0) node [mynode,above right] {$I_2$};
\draw [thick,name path=L1] (0,19.55) -- (19,0) node [mynode,above right] {$I_1$};
% indifference curves
\draw [indiffcolour,ultra thick,name path=I2] (1,10) to [out=270,in=180] (10,1);
\draw [indiffcolour,ultra thick,name path=I0] (4,13) to [out=270,in=180] (13,4);
\draw [indiffcolour,ultra thick,name path=I1] (7,16) to [out=270,in=180] (16,7);
% axes
\draw [thick, -] (0,20) node (yaxis) [above] {Snowboarding} -- (0,0) -- (25,0) node (xaxis) [right] {Jazz};
% intersection of black lines with Indiff curves
\draw [name intersections={of=L0 and I0, by=E0},name intersections={of=L1 and I1, by=E1},name intersections={of=L2 and I2, by=E2}]
	[dotted,thick] (yaxis |- E0) node [mynode,left] {$S_0$} -- (E0) node [mynode,above right] {$E_0$} -- (xaxis -| E0) node [mynode,below] {$J_0$}
	[dotted,thick] (yaxis |- E1) node [mynode,left] {$S_1$} -- (E1) node [mynode,above right] {$E_1$} -- (xaxis -| E1) node [mynode,below] {$J_1$}
	[dotted,thick] (yaxis |- E2) node [mynode,left] {$S_2$} -- (E2) node [mynode,above right] {$E_2$} -- (xaxis -| E2) node [mynode,below] {$J_2$};
\end{TikzFigure}

\newhtmlpage

\subsection*{Adjusting to price changes}

Next, consider the impact of a price change from the initial equilibrium $E_0$ 
in Figure~\ref{fig:incomepriceadj}. Suppose that jazz now costs more. This 
reduces the purchasing power of the given budget of \$200. The new jazz
intercept is therefore reduced. The budget constraint
becomes steeper and rotates around the snowboard intercept H, which is
unchanged because its price is constant. The new equilibrium is at $E_2$,
which reflects a lower level of satisfaction because the affordable set has
been reduced by the price increase. As explained in Section~\ref{sec:ch6sec2},
$E_0$ and $E_2$ define points on the demand
curve for jazz ($J_0$ and $J_2$): They reflect the consumer response to a
change in the price of jazz with all other things held constant. In
contrast, the price increase for jazz \textit{shifts} the demand curve for
snowboarding: As far as the demand curve for snowboarding is concerned, a
change in the price of jazz is one of those things other than own-price that
determine its position.

\subsection*{Philanthropy}

Individuals in the foregoing analysis aim to maximize their utility, given
that they have a fixed budget. Note that this behavioural assumption does
not rule out the possibility that these same individuals may be
philanthropic -- that is, they get utility from the act of giving to their
favourite charity or the United Way or Centre-aide. To see this suppose that
donations give utility to the individual in question -- she gets a `warm
glow' feeling as a result of giving, which is to say she gets utility from
the activity. There is no reason why we cannot put charitable donations on
one axis and some other good or combination of goods on the remaining axis.
At equilibrium, the marginal utility per dollar of contributions to charity
should equal the marginal utility per dollar of expenditure on other goods;
or, stated in terms of ordinal utility, the marginal rate of substitution
between philanthropy and any other good should equal the ratio of their
prices. Evidently the price of a dollar of charitable donations is one
dollar.
