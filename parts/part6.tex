\continuouspart{Government and Trade}

{\large\color{parttextcolour}
\begin{description}
\item[\textmd{\ref{chap:government}.}] Government and government activity in the Canadian economy

\item[\textmd{\ref{chap:internationaltrade}.}] International trade
\end{description}
}

\vspace{1cm}

Governments play a major role in virtually every economy. They account for one third or more of national product. In addition to providing a legal and constitutional framework and implementing the law, governments can moderate and influence the operation or markets, provide public goods and furnish missing information. Governments distribute and redistribute widely and supply major services such as health and education. Modern economies could not function without a substantial role for their governments. These roles are explored and developed in Chapter~\ref{chap:government}.

The final chapter looks outwards. Canada is an open economy, with a high percentage of its production imported and exported. We explore the theory of absolute and comparative advantage and illustrate the potential for consumer gains that trade brings. We analyze barriers to trade such as tariffs and quotas and end with an overview of the World's major trading groups and institutions.

