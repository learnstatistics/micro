\continuouspart{Responsiveness and the Value of Markets}

{\large\color{parttextcolour}
\begin{description}
\item[\textmd{\ref{chap:elasticities}.}] Elasticities

\item[\textmd{\ref{chap:welfare}.}] Welfare economics, externalities and non-classical markets
\end{description}
}

\vspace{1cm}

The degree to which individuals or firms, or any economic agent, respond to incentives is important to ascertain for pricing and policy purposes: If prices change, to what degree will suppliers and buyers respond? How will markets respond to taxes? Chapter~\ref{chap:elasticities} explores and develops the concept of elasticity, which is the word economists use to define responsiveness. A meaningful metric, one formulated in percentage terms, that is applicable to virtually any market or incentive means that behaviours can be compared in different environments. 

In Chapter~\ref{chap:welfare} we explore how markets allocate resources and how the well-being of society's members is impacted by uncontrolled and controlled markets. A central theme of this chapter is that markets are very useful environments, but, if they are to serve the social interest, need to be controlled in many circumstances.

