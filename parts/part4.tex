\continuouspart{Market Structures}

{\large\color{parttextcolour}
\begin{description}
\item[\textmd{\ref{chap:perfectcompetition}.}] Perfect competition

\item[\textmd{\ref{chap:monopoly}.}] Monopoly

\item[\textmd{\ref{chap:imperfectcompetition}.}] Imperfect competition
\end{description}
}

\vspace{1cm}

Markets are all around us and they come in many forms. Some are on-line, others are physical. Some involve goods such as food and vehicles; others involve health provision or financial advice. Markets differ also by the degree of competition associated with each. For example the wholesale egg market is very homogeneous in that the product has minimal variation. The restaurant market offers food, but product variation is high. Each market has many suppliers.

In contrast to the egg and restaurant market, many other markets are characterized by just a few suppliers or in some cases just one. For example, passenger train services may have just a single supplier, and this supplier is therefore a monopolist. Pharmaceuticals tend to be supplied by a limited number of large international corporations plus a group of generic drug manufacturers. The internet and communications services market usually has only a handful of providers.

In this part we examine the reasons why markets take on different forms and display a variety of patterns of behaviour. We delve into the working of each market structure to understand why these markets retain their structure.
