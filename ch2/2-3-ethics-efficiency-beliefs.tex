\section{Ethics, efficiency and beliefs}\label{sec:ch2sec3}

\terminology{Positive economics} studies objective or scientific
explanations of how the economy functions. Its aim is to understand and
generate predictions about how the economy may respond to changes and policy
initiatives. In this effort economists strive to act as detached scientists,
regardless of political sympathies or ethical code. Personal judgments and
preferences are (ideally) kept apart. In this particular sense, economics is
similar to the natural sciences such as physics or biology. To date in this
chapter we have been exploring economics primarily from a positive
standpoint.

In contrast, \terminology{normative economics} offers recommendations based
partly on value judgments. While economists of different political
persuasions can agree that raising the income tax rate would lead to some
reduction in the number of hours worked, they may yet differ in their views
on the advisability of such a rise. One economist may believe that the
additional revenue that may come in to government coffers is not worth the
disincentives to work; another may think that, if such monies can be
redistributed to benefit the needy, or provide valuable infrastructure, the
negative impact on the workers paying the income tax is worth it.

\begin{DefBox}
\textbf{Positive economics} studies objective or scientific explanations of how the economy functions.

\textbf{Normative economics} offers recommendations that incorporate value judgments.
\end{DefBox}

\newhtmlpage

Scientific research can frequently resolve differences that arise in
positive economics---not so in normative economics. For example, if we claim
that ``the elderly have high medical bills, and the
government should cover all of the bills'', we are making
both a positive and a normative statement. The first part is positive, and
its truth is easily established. The latter part is normative, and
individuals of different beliefs may reasonably differ. Some people may
believe that the money would be better spent on the environment and have the
aged cover at least part of their own medical costs. Positive economics does
not attempt to show that one of these views is correct and the other false.
The views are based on value judgments, and are motivated by a concern for %
\terminology{equity}. Equity is a vital guiding principle in the formation
of policy and is frequently, though not always, seen as being in competition
with the drive for economic growth. Equity is driven primarily by normative
considerations. Few economists would disagree with the assertion that a
government should implement policies that improve the lot of the poor---but
to what degree?

\begin{DefBox}
\textbf{Economic equity} is concerned with the distribution of well-being among members of the economy.
\end{DefBox}

Most economists hold normative views, sometimes very strongly. They
frequently see themselves, not just as cold hearted scientists, but as
champions for their (normative) cause in addition. Conservative economists
see a smaller role for government than left-leaning economists. 

Many economists see a conflict between equity and the efficiency
considerations that we developed in Chapter~\ref{chap:intro}. For example,
high taxes may provide disincentives to work in the marketplace and
therefore reduce the efficiency of the economy: Plumbers and gardeners may
decide to do their own gardening and their own plumbing because, by staying
out of the marketplace where monetary transactions are taxed, they can avoid
the taxes. And avoiding the taxes may turn out to be as valuable as the
efficiency gains they forgo.

In other areas the equity-efficiency trade-off is not so obvious: If taxes
(that may have disincentive effects) are used to educate individuals who
otherwise would not develop the skills that follow education, then economic
growth may be higher as a result of the intervention.

\newhtmlpage

\subsection*{Revisiting the definition of economics -- core beliefs}

This is an appropriate point at which to return to the definition of
economics in Chapter~\ref{chap:intro} that we borrowed from Nobel Laureate
Christopher Sims: Economics is a set of ideas and methods for the betterment
of society.

If economics is concerned about the betterment of society, clearly there are
ethical as well as efficiency considerations at play. And given the
philosophical differences among scientists (including economists), can we
define an approach to economics that is shared by the economics profession
at large? Most economists would answer that the profession shares a set of
beliefs, and that differences refer to the extent to which one consideration
may collide with another.

\begin{itemize}
\item First of all we believe that \textit{markets are critical} because
they facilitate exchange and therefore encourage efficiency. Specialization and 
trade creates benefits for the trading parties. For example,
Canada has not the appropriate climate for growing coffee beans, and
Colombia has not the terrain for wheat. If Canada had to be self-sufficient,
we might have to grow coffee beans in green-houses---a costly proposition.
But with trade we can specialize, and then exchange some of our wheat for Colombian
coffee. Similar benefits arise for the Colombians.

\medskip
A frequent complaint against trade is that its modern-day form
(globalization) does not benefit the poor. For example, workers in the
Philippines may earn only a few dollars per day manufacturing clothing for
Western markets. From this perspective, most of the gains from trade go to
the Western consumers and capitalists, come at the expense of jobs to
western workers, and provide Asian workers with meagre rewards.  

\item A corollary of the centrality of markets is \textit{that incentives
matter}. If the price of business class seats on your favourite airline is
reduced, you may consider upgrading. Economists believe strongly that the
price mechanism influences behaviour, and therefore favour the use of price
incentives in the marketplace and public policy more generally.
Environmental economists, for example, advocate the use of pollution permits
that can be traded at a price between users, or carbon taxes on the emission
of greenhouse gases. We will develop such ideas in \textit{Microeconomics}
Chapter~\ref{chap:welfare} more fully.

\item In saying that economists believe in incentives, we are not proposing
that human beings are purely mercenary. People have many motivations:
Self-interest, a sense of public duty, kindness, etc. Acting out of a sense
of self-interest does not imply that people are morally empty or have no
altruistic sense. 

\item Economists believe universally in the \textit{importance of the rule
of law}, no matter where they sit on the political spectrum. Legal
institutions that govern contracts are critical to the functioning of an
economy. If goods and services are to be supplied in a market economy, the
suppliers must be guaranteed that they will be remunerated. And this
requires a developed legal structure with penalties imposed on individuals
or groups who violate contracts. Markets alone will not function efficiently.

\medskip
Modern development economics sees the implementation of the rule of law as
perhaps the central challenge facing poorer economies. There is a strong 
correlation between economic growth and national wealth on the one hand, and
an effective judical and policing system on the other. The consequence on
the world stage is that numerous `economic' development projects now focus
upon training jurists, police officers and bureaucrats in the rule of law!

\item Finally, economists believe in the centrality of government.
Governments can solve a number of problems that arise in market economies
that cannot be addressed by the private market place. For example,
governments can best address the potential abuses of monopoly power.
Monopoly power, as we shall see in \textit{Microeconomics} Chapter~\ref{chap:monopoly}, not
only has equity impacts it may also reduce economic efficiency. Governments
are also best positioned to deal with environmental or other types of
externalities -- the impact of economic activity on sectors of the economy
that are not directly involved in the activity under consideration. 
\end{itemize}

In summary, governments have a variety of roles to play in the economy.
These roles involve making the economy more equitable and more efficient by
using their many powers.