\newpage
\section*{Exercises for Chapter~\ref{chap:tmd}}

\begin{Filesave}{solutions}
\subsubsection*{Chapter~\ref{chap:tmd} Solutions}
\end{Filesave}

\begin{enumialphparenastyle}

\begin{econex}\label{ex:ch2ex1}
An examination of a country's recent international trade flows yields the data in the table below.
\begin{Table}{}
	\begin{tabu} to 30em {|X[1,c]X[1,c]X[1,c]|}\hline
		\rowcolor{rowcolour}	\textbf{Year}	&	\textbf{National Income (\$b)}	&	\textbf{Imports (\$b)}	\\	\hline
								2011			&	1,500							&	550						\\
		\rowcolor{rowcolour}	2012			&	1,575							&	573						\\
								2013			&	1,701							&	610						\\
		\rowcolor{rowcolour}	2014			&	1,531							&	560						\\
								2015			&	1,638							&	591						\\	\hline
	\end{tabu}
\end{Table}
\begin{enumerate}
	\item	Based on an examination of these data do you think the national income and imports are not related, positively related, or negatively related?
	\item	Plot each pair of observations in a two-dimensional line diagram to illustrate your view of the import/income relationship. Measure income on the horizontal axis and imports on the vertical axis. This can be done using graph paper or a spreadsheet-cum-graphics software.
\end{enumerate}
\begin{econsolution}
	These variables are positively related.
	
	\begin{center*}
	\begin{tikzpicture}[background color=figurebkgdcolour,use background]
	\begin{axis}[
		axis line style=thick,
		every tick label/.append style={font=\footnotesize},
		ymajorgrids,
		grid style={dotted},
		every node near coord/.append style={font=\scriptsize},
		xticklabel style={rotate=90,anchor=east,/pgf/number format/1000 sep=},
		scaled y ticks=false,
		yticklabel style={/pgf/number format/fixed,/pgf/number format/1000 sep = \thinspace},
		xmin=1450,xmax=1750,ymin=540,ymax=620,
		y=1.15cm/20,
		x=1.65cm/50,
		x label style={at={(axis description cs:0.5,-0.05)},anchor=north},
		xlabel={Imports},
		ylabel={National Income},
	]
	\addplot[datasetcolourone,ultra thick,mark=none] table {
		X		Y
		1500	550
		1531	560
		1575	573
		1638	591
		1701	610
	};
	\end{axis}
	\end{tikzpicture}
	\end{center*}
\end{econsolution}
\end{econex}

\begin{econex}\label{ex:ch2ex2}
The average price of a medium coffee at \textit{Wakeup Coffee Shop} in each of the past ten years is given in the table below.
\begin{Table}{}
\begin{tabu} to \linewidth {|X[1,c]X[1,c]X[1,c]X[1,c]X[1,c]X[1,c]X[1,c]X[1,c]X[1,c]X[1,c]|}	\hline
\rowcolor{rowcolour}	2005	&	2006	&	2007	&	2008	&	2009	&	2010	&	2011	&	2012	&	2013	&	2014		\\
						\$1.05	&	\$1.10	&	\$1.14	&	\$1.20	&	\$1.25	&	\$1.25	&	\$1.33	&	\$1.35	&	\$1.45	&	\$1.49		\\	\hline
\end{tabu}
\end{Table}
\begin{enumerate}
	\item	Construct an annual `coffee price index' for this time period using 2005 as the base year. [\textit{Hint}: follow the procedure detailed in the chapter -- divide each yearly price by the base year price.]
	\item	Based on your price index, what was the percentage change in the price of a medium coffee from 2005 to 2012?
	\item	Based on your index, what was the average annual percentage change in the price of coffee from 2005 to 2010?
	\item	Assuming the inflation rate in this economy was 2\% every year, what was the real change in the price of coffee between 2007 and 2008; and between 2009 and 2010?
\end{enumerate}
\begin{econsolution}
	For (b) the answer is 30\%, for (c) the answer is 4.0\%, and for (d) the answers are 3.5\% and -2.0\%.
	\begin{Table}{}
	\begin{tabu} to \linewidth {|X[1,c]X[1,c]X[1,c]X[1,c]X[1,c]X[1,c]X[1,c]X[1,c]X[1,c]X[1,c]X[1,c]|}\hline
	\rowcolor{rowcolour}\textbf{Year} & 2005	& 2006	& 2007	& 2008	& 2009	& 2010	& 2011	& 2012	& 2013	& 2014 	\\
						\textbf{Index}& 100.0	& 105.0 & 109.0 & 115.0 & 120.0 & 120.0 & 127.0 & 130.0 & 139.0 & 142.0 \\ \hline
	\end{tabu}
	\end{Table}
\end{econsolution}
\end{econex}

\begin{econex}\label{ex:ch2ex3}
The following table shows hypothetical consumption spending by households and income of households in billions of dollars.
\begin{Table}{}
\begin{tabu} to 27em {|X[1,c]X[1,c]X[1,c]|}	\hline
\rowcolor{rowcolour}	\textbf{Year}	&	\textbf{Income}	&	\textbf{Consumption}	\\	\hline
						2006			&	476				&	434						\\
\rowcolor{rowcolour}	2007			&	482				&	447						\\
						2008			&	495				&	454						\\
\rowcolor{rowcolour}	2009			&	505				&	471						\\
						2010			&	525				&	489						\\
\rowcolor{rowcolour}	2011			&	539				&	509						\\
						2012			&	550				&	530						\\
\rowcolor{rowcolour}	2013			&	567				&	548						\\	\hline
\end{tabu}
\end{Table}
\begin{enumerate}
	\item	Plot the scatter diagram with consumption on the vertical axis and income on the horizontal axis.
	\item	Fit a line through these points.
	\item	Does the line indicate that these two variables are related to each other?
	\item	How would you describe the \textit{causal relationship} between income and consumption?
\end{enumerate}
\begin{econsolution}
	The scatter diagram plots observed combinations of income and consumption as follows. For parts (c) and (d): the variables are positively related and the causation runs from income to consumption.
	
	\begin{center*}
		\begin{tikzpicture}[background color=figurebkgdcolour,use background]
		\begin{axis}[
		axis line style=thick,
		every tick label/.append style={font=\footnotesize},
		extra y ticks={100,300,500},
		ymajorgrids,
		grid style={dotted},
		every node near coord/.append style={font=\scriptsize},
		xticklabel style={rotate=90,anchor=east,/pgf/number format/1000 sep=},
		scaled y ticks=false,
		yticklabel style={/pgf/number format/fixed,/pgf/number format/1000 sep = \thinspace},
		xmin=460,xmax=580,ymin=0,ymax=600,
		y=0.8cm/100,
		x=1.6cm/20,
		x label style={at={(axis description cs:0.5,-0.05)},anchor=north},
		xlabel={Income},
		ylabel={Consumption},
		]
		\addplot[datasetcolourone,ultra thick,mark=none] table {
			X		Y
			476		434
			482		447
			495		454
			505		471
			525		489
			539		509
			550		530
			567		548
		};
		\end{axis}
		\end{tikzpicture}
	\end{center*}
\end{econsolution}
\end{econex}

\begin{econex}\label{ex:ch2ex4}
Using the data from Exercise~\ref{ex:ch2ex3}, compute the percentage change in consumption and the percentage change in income for each pair of adjoining years between 2006 and 2013.
\begin{econsolution}
	The percentage changes in income are:
	\begin{Table}{}
		\begin{tabu} to \linewidth {|X[1,c]X[1,c]X[1,c]X[1,c]X[1,c]X[1,c]X[1,c]X[1,c]|}	\hline
			\rowcolor{rowcolour} \textbf{Pct Inc} & 1.3 & 2.7 & 2.0 & 4.0 & 2.7 & 2.0 & 3.1 \\
			\textbf{Pct Con} & 3.0 & 1.6 & 3.7 & 3.8 & 4.1 & 4.1 & 3.4 \\ \hline
		\end{tabu}
	\end{Table}
\end{econsolution}
\end{econex}

\begin{econex}\label{ex:ch2ex5}
You are told that the relationship between two variables, $X$ and $Y$, has the form $Y=10+2X$. By trying different values for $X$ you can obtain the corresponding predicted value for $Y$ (e.g., if $X=3$, then $Y=10+2\times 3=16$). For values of $X$ between 0 and 12, compute the matching value of $Y$ and plot the scatter diagram.
\begin{econsolution}
	The relationship given by the equation $Y=10+2X$ when plotted has an intercept on the vertical ($Y$) axis of 10 and the slope of the line is 2. The maximum value of $Y$ (where $X$ is 12) is 34.
	\begin{center*}
		\begin{tikzpicture}[background color=figurebkgdcolour,use background,xscale=0.5,yscale=0.15]
		\draw [thick] (0,40) node (yaxis) [mynode1,above] {$Y$} |- (15,0) node (xaxis) [mynode1,right] {$X$};
		\draw [ultra thick,supplycolour,name path=Y102X] (0,10) node [mynode,left,black] {10} -- (13,36) node [mynode,above,black] {$Y=10+2X$};
%		\draw [ultra thick,dashed,demandcolour,name path=Y1005X] (0,10) -- (15,2.5) node [mynode,right,black] {$Y=10-0.5X$};
%		\draw [ultra thick,dashed,demandcolour,name path=Y4] (0,4) node [mynode,left,black] {4} -- (15,4);
		\path [name path=Y22] (0,22) -- +(15,0);
		\path [name path=Y34] (0,34) -- +(15,0);
		\draw [name intersections={of=Y22 and Y102X, by=i1},name intersections={of=Y34 and Y102X, by=i2}]
		[dotted,thick] (yaxis |- i1) node [mynode,left] {22} -| (xaxis -| i1) node [mynode,below] {6}
		[dotted,thick] (yaxis |- i2) node [mynode,left] {34} -| (xaxis -| i2) node [mynode,below] {12};	
		\end{tikzpicture}
	\end{center*}
	\begin{Table}{}
		\begin{tabu} to \linewidth {|X[1,c]X[1,c]X[1,c]X[1,c]X[1,c]X[1,c]X[1,c]X[1,c]X[1,c]X[1,c]X[1,c]X[1,c]X[1,c]X[1,c]|}	\hline
			\rowcolor{rowcolour} \textbf{X} & 0 & 1 & 2 & 3 & 4 & 5 & 6 & 7 & 8 & 9 & 10 & 11 & 12 \\
			\textbf{Y} & 10 & 12 & 14 & 16 & 18 & 20 & 22 & 24 & 26 & 28 & 30 & 32 & 34 \\ \hline
		\end{tabu}
	\end{Table}
\end{econsolution}
\end{econex}

\begin{econex}\label{ex:ch2ex6}
For the data below, plot a scatter diagram with variable $Y$ on the vertical axis and variable $X$ on the horizontal axis.
\begin{Table}{}
	\begin{tabu} to 27em {|X[1,c]X[1,c]X[1,c]X[1,c]X[1,c]X[1,c]X[1,c]X[1,c]|}	\hline
		\rowcolor{rowcolour}	\textbf{Y}	&	40	&	33	&	29	&	56	&	81	&	19	&	20	\\
		\textbf{X}	&	5	&	7	&	9	&	3	&	1	&	11	&	10	\\	\hline
	\end{tabu}
\end{Table}
\begin{enumerate}
	\item	Is the relationship between the variables positive or negative?
	\item	Do you think that a linear or non-linear line better describes the relationship?
\end{enumerate}
\begin{econsolution}
	\begin{center*}
		\begin{tikzpicture}[background color=figurebkgdcolour,use background]
		\begin{axis}[
		axis line style=thick,
		every tick label/.append style={font=\footnotesize},
		ymajorgrids,
		grid style={dotted},
		every node near coord/.append style={font=\scriptsize},
		xticklabel style={rotate=90,anchor=east,/pgf/number format/1000 sep=},
		scaled y ticks=false,
		yticklabel style={/pgf/number format/fixed,/pgf/number format/1000 sep = \thinspace},
		xmin=0,xmax=12,ymin=0,ymax=100,
		y=0.6cm/10,
		x=0.8cm/1,
		x label style={at={(axis description cs:0.5,-0.05)},anchor=north},
		xlabel={$X$},
		ylabel={$Y$},
		]
		\addplot[datasetcolourtwo,ultra thick,only marks] table {
			X	Y
			1	81
			3	56
			5	40
			7	33
			9	29
			10	20
			11	19
		};
		\addplot [reglinecolour,ultra thick,no marks] expression[domain=1:11,samples=100] {0.5357*x^2 - 12.2197*x + 90.4753};
		\end{axis}
		\end{tikzpicture}
	\end{center*}
	\begin{enumerate}
		\item	The relationship is negative.
		\item	The relationship is non-linear.
	\end{enumerate}
\end{econsolution}
\end{econex}

\end{enumialphparenastyle}
