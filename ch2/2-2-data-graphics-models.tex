\section{Data, theory and economic models}\label{sec:ch2sec2}

Let us now investigate the interplay between economic theories on the one
hand and data on the other. We will develop two examples. The first will be
based upon the data on house prices, the second upon a new data set.

\subsection*{House prices -- theory}

Remember from Chapter~\ref{chap:intro} that a theory is a logical argument regarding economic
relationships. A theory of house prices would propose that the price of
housing depends upon a number of elements in the economy. In particular, if
borrowing costs are low then buyers are able to afford the interest costs on
larger borrowings. This in turn might mean they are willing to pay higher
prices. Conversely, if borrowing rates are higher. Consequently, the
borrowing rate, or mortgage rate, is a variable for an economic model of
house prices. A second variable might be available space for development: If
space in a given metropolitan area is tight then the land value will reflect
this, and consequently the higher land price should be reflected in higher
house prices. A third variable would be the business climate: If there is a
high volume of high-value business transacted in a given area then buildings
will be more in demand, and that in turn should be reflected in higher
prices. For example, both business and residential properties are more
highly priced in San Francisco and New York than in Moncton, New Brunswick. A
fourth variable might be environmental attractiveness: Vancouver may be more
enticing than other towns in Canada. A fifth variable might be
the climate.

\newhtmlpage

\subsection*{House prices -- evidence}

These and other variables could form the basis of a \textit{theory} of house
prices. A \textit{model} of house prices, as explained in Chapter~\ref{chap:intro}, focuses upon
what we would consider to be the most important subset of these variables.
In the limit, we could have an extremely simple model that specified a
dependence between the price of housing and the mortgage rate alone. To test
such a simple model we need data on house prices and mortgage rates. The
final column of Table~\ref{table:housepriceindex} contains data on the 5-year fixed-rate mortgage
for the period in question. Since our simple model proposes that prices
depend (primarily) upon mortgage rates, in Figure~\ref{fig:housemortgageprice} we plot the 
house price series on the vertical axis, and the mortgage rate on the
horizontal axis, for each year from 2001 to 2011. As before, each point (shown as a `$+$')
represents a pair of price and mortgage rate values.

% Figure 2.3
\begin{TikzFigure}{xscale=1,yscale=1,caption={Price of housing \label{fig:housemortgageprice}}}
\begin{axis}[
	%axis lines=left,
	axis line style=thick,
	every tick label/.append style={font=\footnotesize},
	every node near coord/.append style={font=\scriptsize},
	xticklabel style={anchor=north,/pgf/number format/1000 sep=},
	scaled y ticks=false,
	yticklabel style={/pgf/number format/fixed,/pgf/number format/1000 sep = \thinspace},
	xmin=4,xmax=9,ymin=0,ymax=900,
	y=0.75cm/100,
	x=2cm/1,
	x label style={at={(axis description cs:0.5,-0.05)},anchor=north},
	xlabel={Mortgage rate},
	ylabel={House price},
]
\addplot[datasetcolourtwo,only marks,mark=+] table {
	X		Y
	7.75	350
	6.85	360
	6.6		395
	5.8		434
	6.05	477
	6.3		580
	6.65	630
	7.3		710
	5.8		605
	5.4		740
	5.2		800
};
\addplot[reglinecolour,ultra thick,mark=none] table[
	y={create col/linear regression={y=Y}}] % compute a linear regression from the coordinates above
{
	X			Y
	7.75	350
	6.85	360
	6.6		395
	5.8		434
	6.05	477
	6.3		580
	6.65	630
	7.3		710
	5.8		605
	5.4		740
	5.2		800
};
\addplot[black,dotted] table {
	X		Y
	6.5		0
	6.5		535.588
	0		535.588
};
\addplot[black,dotted] table {
	X		Y
	7.5		0
	7.5		430.28
	0		430.28
};
\end{axis}
\end{TikzFigure}


The resulting plot (called a scatter diagram) suggests that there is a
negative relationship between these two variables. That is, higher prices
are correlated with lower mortgage rates. Such a correlation is consistent
with our theory of house prices, and so we might conclude that changes in
mortgage rates \textit{cause} changes in house prices. Or at least the data
suggest that we should not reject the idea that such causation is in the
data.

\newhtmlpage

\subsection*{House prices -- inference}

To summarize the relationship between these variables, the pattern suggests
that a straight line through the scatter plot would provide a reasonably
good description of the relationship between these variables. Obviously it
is important to define the most appropriate line -- one that `fits' the data
well.\footnote{This task is the job of econometricians, who practice econometrics.
Econometrics is the science of examining and quantifying relationships
between economic variables. It attempts to determine the separate influences
of each variable, in an environment where many things move simultaneously.
Computer algorithms that do this are plentiful. Computers can also work in
many dimensions in order to capture the influences of \textit{several
variables simultaneously} if the model requires that.} The line we have
drawn through the data points is informative, because it relates the two
variables in a \textit{quantitative manner}. It is called a \terminology{%
regression line}. It predicts that, on average, if the mortgage rate
increases, the price of housing will respond in the downward direction. This
particular line states that a one point change in the mortgage rate will
move prices in the opposing direction by \$105,000. This is easily verified
by considering the dollar value corresponding to say a mortgage value of
6.5, and then the value corresponding to a mortgage value of 7.5. Projecting
vertically to the regression line from each of these points on the
horizontal axis, and from there across to the vertical axis will produce a
change in price of \$105,000. 

Note that the line is not at all a `perfect' fit. For example, the mortgage
rate declined between 2008 and 2009, but the price declined also -- contrary
to our theory. The model is not a perfect predictor; it states that \textit{%
on average} a change in the magnitude of the x-axis variable leads to a
change of a specific amount in the magnitude of the y-axis variable.

In this instance the slope of the line is given by -105,000/1, which is the
vertical distance divided by the corresponding horizontal distance. Since
the line is straight, this slope is unchanging.

\begin{DefBox}
\textbf{Regression line}: representation of the average relationship between two variables in a scatter diagram.
\end{DefBox}

\newhtmlpage

\subsection*{Road fatalities -- theory, evidence and inference}

Table~\ref{table:driverfatalage} contains data on annual road fatalities per 100,000 drivers for
various age groups. In the background, we have a \textit{theory}, proposing
that driver fatalities depend upon the age of the driver, the quality of
roads and signage, speed limits, the age of the automobile stock and perhaps
some other variables. Our model focuses upon a subset of these variables,
and in order to present the example in graphical terms we specify fatalities
as being dependent upon a single variable -- age of driver.

% Table 2.3
\begin{Table}{caption={Non-linearity: Driver fatality rates Canada, 2009 \label{table:driverfatalage}},description={\textit{Source}: Transport Canada, Canadian motor vehicle traffic collision statistics, 2009.},descwidth={28em}}
	\begin{tabu} to 25em {|X[1,c]X[1,c]|}	\hline
		\rowcolor{rowcolour}	\textbf{Age of driver} & \textbf{Fatality rate} \\
		\rowcolor{rowcolour}	&	\textbf{per 100,000 drivers}	\\
		20-24 & 9.8 \\ 
		\rowcolor{rowcolour}	25-34 & 4.4 \\ 
		35-44 & 2.7 \\ 
		\rowcolor{rowcolour}	45-54 & 2.4 \\ 
		55-64 & 1.9 \\ 
		\rowcolor{rowcolour}	65+ & 2.9 \\ \hline
	\end{tabu}
\end{Table}

The scatter diagram is presented in Figure~\ref{fig:drivermortality}. Two aspects of this plot
stand out. First, there is an exceedingly steep decline in the fatality rate
when we go from the youngest age group to the next two age groups. The
decline in fatalities between the youngest and second youngest groups is
about 20 points, whereas the decline between the third and fourth age groups
is less than 2 points. This suggests that behaviour is not the same
throughout the age distribution. Second, we notice that fatalities increase
for the oldest age group, perhaps indicating that the oldest drivers are not
as good as middle-aged drivers.

\newhtmlpage

These two features suggest that the relationship between fatalities and age
differs across the age spectrum. Accordingly, a straightline would not be an
accurate way of representing the behaviours in these data. A straight line
through the plot implies that a given change in age should have a similar
impact on fatalities, no matter the age group. Accordingly we have an
example of a \textit{non-linear relationship}. Such a non-linear
relationship might be represented by the curve going through the plot.
Clearly the slope of this line varies as we move from one age category to
another. 

% Figure 2.4
\begin{TikzFigure}{xscale=1,yscale=1,descwidth=25em,caption={Non-linearity: Driver fatality rates Canada, 2009 \label{fig:drivermortality}},description={Fatality rates vary non-linearly with age: At first they decline, then increase again, relative to the youngest age group.}}
\begin{axis}[
	%axis lines=left,
	axis line style=thick,
	every tick label/.append style={font=\footnotesize},
	every node near coord/.append style={font=\scriptsize},
	xticklabel style={rotate=90,anchor=east,/pgf/number format/1000 sep=},
	scaled y ticks=false,
	yticklabel style={/pgf/number format/fixed,/pgf/number format/1000 sep = \thinspace},
	xmin=10,xmax=80,ymin=0,ymax=12,
	y=1.1cm/2,
	x=1.1cm/10,
	x label style={at={(axis description cs:0.5,-0.05)},anchor=north},
	xlabel={Average age},
	ylabel={Fatality rate},
]
\addplot[datasetcolourtwo,only marks,mark=+] table {
	X	Y
	22	9.8
	30	4.4
	40	2.7
	50	2.4
	60	1.9
	70	2.9
};
\addplot [reglinecolour,ultra thick,no marks] expression[domain=20:70,samples=100] {0.00721259*x^2 - 0.78274*x + 22.6908};
\end{axis}
\end{TikzFigure}