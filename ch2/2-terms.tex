\newpage
	\section*{Key Terms}
\begin{keyterms}
\textbf{Variables}: measures that can take on different sizes.

\textbf{Data}: recorded values of variables.

\textbf{Time series data}: a set of measurements made sequentially at different points in time.

\textbf{High (low) frequency data} series have short (long) intervals between observations.

\textbf{Cross-section data}: values for different variables recorded at a point in time.

\textbf{Repeated cross-section data}: cross-section data recorded at regular or irregular intervals.

\textbf{Longitudinal data} follow the same units of observation through time.

\textbf{Percentage change}$=(\text{change in values})/\text{original value}\times 100$.

\textbf{Consumer price index}: the average price level for consumer goods and services.

\textbf{Inflation (deflation) rate}: the annual percentage increase (decrease) in the level of consumer prices.

\textbf{Real price}: the actual price adjusted by the general (consumer) price level in the economy.

\textbf{Index number}: value for a variable, or an average of a set of variables, expressed relative to a given base value.

\textbf{Regression line}: representation of the average relationship between two variables in a scatter diagram.

\textbf{Positive economics} studies objective or scientific explanations of how the economy functions.

\textbf{Normative economics} offers recommendations that incorporate value judgments.

\textbf{Economic equity} is concerned with the distribution of well-being among members of the economy.
\end{keyterms}