\chapter{Theories, data and beliefs}\label{chap:tmd}

\begin{topics}
	\textbf{In this chapter we will explore:}
	\begin{description}
		\item[\ref{sec:ch2sec1}] Data analysis
		\item[\ref{sec:ch2sec2}] Data, theory and economic models
		\item[\ref{sec:ch2sec3}] Ethics, efficiency and beliefs
	\end{description}
\end{topics}

Economists, like other scientists and social scientists, observe and analyze
behaviour and events. Economists are concerned primarily with the economic
causes and consequences of what they observe. They want to understand an
extensive range of human experience, including: money, government finances,
industrial production, household consumption, inequality in income
distribution, war, monopoly power, professional and amateur sports,
pollution, marriage, music, art, and much more.

Economists approach these issues using theories and models. To present,
explain, illustrate and evaluate their theories and models they have
developed a set of techniques or tools. These involve verbal descriptions
and explanations, diagrams, algebraic equations, data tables and charts and
statistical tests of economic relationships.

This chapter covers some of these basic techniques of analysis.
