\section{Market failure}\label{sec:ch14sec1}

Markets are fine institutions when all of the conditions for their efficient
operation are in place. In Chapter~\ref{chap:welfare} we explored the
meaning of efficient resource allocation, by developing the concepts of
consumer and producer surpluses. But, while we have emphasized the benefits
of efficient resource allocation in a market economy, there are many
situations where markets deliver inefficient outcomes. Several problems
beset the operation of markets. The principal sources of \textit{market
	failure} are: \textit{Externalities, public goods, asymmetric information,}
and the \textit{concentration of power}. In addition markets may produce
outcomes that are \textit{unfavourable} to certain groups -- perhaps those
on low incomes. The circumstances described here lead to what is termed 
\terminology{market failure}.

\begin{DefBox}
	\textbf{Market failure} defines outcomes in which the allocation of resources is not efficient.  
\end{DefBox}

\subsection*{Externalities}

A negative externality is one resulting, perhaps, from the polluting
activity of a producer, or the emission of greenhouse gases into the
atmosphere. A positive externality is one where the activity of one
individual confers a benefit on others. An example here is where individuals
choose to get immunized against a particular illness. As more people become
immune, the lower is the probability that the illness can propagate itself
through hosts, and therefore the greater the benefits to those not immunized.

Solutions to these market failures come in several forms: Government taxes
and subsidies, or quota systems that place limits on the production of
products generating externalities. Such solutions were explored in Chapter~%
\ref{chap:welfare}. Taxes on gasoline discourage its use and therefore
reduce the emission of poisons into the atmosphere. Taxes on cigarettes and
alcohol lower the consumption of goods that may place an additional demand
on our publicly-funded health system. The provision of free, or low-cost,
immunization against specific diseases to children benefits the whole
population.

These measures attempt to \textit{compensate for the absence of a market} in
certain activities. Producers may not wish to pay for the right to emit
pollutants, and consequently if the government steps in to counter such an
externality, the government is effectively implementing a solution to the
missing market.

\newhtmlpage

\subsection*{Public goods}

Public goods are sometimes called collective consumption goods, on account
of their non-rivalrous and non-excludability characteristics. For example,
if the government meteorological office provides daily forecasts over the
nation's airwaves, it is no more expensive to supply that information to one
million than to one hundred individuals in the same region. Its provision to
one is not rivalrous with its provision to others -- in contrast to private
goods that cannot be `consumed' simultaneously by more than one individual.
In addition, it may be difficult to exclude certain individuals from
receiving the information.

\begin{DefBox}
	\textbf{Public goods} are non-rivalrous, in that they can be consumed simultaneously by more than one individual; additionally they may have a non-excludability characteristic.
\end{DefBox}

\newhtmlpage

Examples of such goods and services abound: Highways (up to their congestion
point), street lighting, information on trans-fats and tobacco, or public
defence provision. Such goods pose a problem for private markets: If it is
difficult to exclude individuals from their consumption, then potential
private suppliers will likely be deterred from supplying them because the
suppliers cannot generate revenue from \textit{free-riders}. Governments
therefore normally supply such goods and services. But how much should
governments supply? An answer is provided with the help of Figure~\ref{fig:optimalpublicgood}.

% Figure 14.1
\begin{TikzFigure}{xscale=0.34,yscale=0.21,descwidth=27em,caption={Optimal provision of a public good \label{fig:optimalpublicgood}},description={The total demand for the public good $D$ is the vertical sum of the individual demands $D_A$ and $D_B$. The optimal provision is where the $MC$ equals the aggregate marginal valuation, as defined by the demand curve $D$. At the optimum $Q^*$, each individual is supplied the same amount of the public good.}}
% demand lines
\draw [demandcolour,ultra thick,name path=D] (0,17) -- (24,8) node [black,mynode,right] {$D$};
\draw [demandcolour!50,ultra thick]
	(0,7) -- (24,3) node [black,mynode,right] {$D_A$}
	(0,10) -- (24,5) node [black,mynode,right] {$D_B$};
% MC curve
\draw [mccolour,ultra thick,name path=MC]
	(5,5) to [out=0,in=270] (20,20) node [black,mynode,above] {$MC$};
% axes
\draw [thick] (0,25) node (yaxis) [mynode1,above] {\$} -- (0,0) node [mynode,below left] {0} -- (25,0) node (xaxis) [mynode1,right] {Quantity};
% intersection of MC and D
\draw [name intersections={of=MC and D, by=Qstar}]
	[dotted,thick] (Qstar) -- (xaxis -| Qstar) node [mynode,below] {$Q^*$};
\end{TikzFigure}

\newhtmlpage

This is a supply-demand diagram with a difference. The supply side is
conventional, with the $MC$ of production representing the supply curve. An
efficient use of the economy's resources, we already know, dictates that an
amount should be produced so that the cost at the margin equals the benefit
to consumers at the margin. In contrast to the total market demand for
private goods, which is obtained by summing individual demands horizontally,
the demand for public goods is obtained by summing individual demands 
\textit{vertically}.

Figure~\ref{fig:optimalpublicgood} depicts an economy with just two
individuals whose demands for street lighting are given by $D_A$ and $D_B$.
These demands reveal the value each individual places on the various output
levels of the public good, measured on the $x$-axis. However, since each
individual can consume the public good \textit{simultaneously}, the
aggregate value of any output produced is the \textit{sum of each individual
	valuation}. The valuation in the market of any quantity produced is
therefore the vertical sum of the individual demands. $D$ is the vertical
sum of $D_A$ and $D_B$, and the optimal output is $Q^*$. At this equilibrium
each individual consumes the same quantity of street lighting, and the $MC$
of the last unit supplied equals the value placed upon it by society -- both
individuals. Note that this `optimal' supply depends upon the income
distribution, as we have stated several times to date. A different
distribution of income may give rise to different demands $D_A$ and $D_B$,
and therefore a different `optimal' output.

\begin{DefBox}
	\textbf{Efficient supply of public goods} is where the marginal cost equals the sum of individual marginal valuations, and each individual consumes the same quantity.
\end{DefBox}

\begin{ApplicationBox}{caption={Are Wikipedia, Google and MOOCs public goods? \label{app:wikigoogle}}}
	Wikipedia is one of the largest on-line sources of free information in the world. It is an encyclopedia that functions in multiple languages and that furnishes information on millions of topics. It is freely accessible, and is maintained and expanded by its users. Google is the most frequently used search engine on the World Wide Web. It provides information to millions of users simultaneously on every subject imaginable, free of charge. MOOCs are `monster open online courses' offered by numerous universities, frequently for no charge to the student. Are these services public goods in the sense we have described?
	
	Very few goods and services are pure public goods, some have the major characteristics of public goods nonetheless. In this general sense, Google, Wikipedia and MOOCs have public good characteristics. Wikipedia is funded by philanthropic contributions, and its users expand its range by posting information on its servers. Google is funded from advertising revenue. MOOCs are funded by university budgets.
	
	A pure public good is available to additional users at zero marginal cost. This condition is essentially met by these services since their server capacity rarely reaches its limit. Nonetheless, they are constantly adding server capacity, and in that sense cannot furnish their services to an unlimited number of additional users at no additional cost.
	
	Knowledge is perhaps the ultimate public good; Wikipedia, Google and MOOCs all disseminate knowledge, knowledge which has been developed through the millennia by philosophers, scientists, artists, teachers, research laboratories and universities.
\end{ApplicationBox}

A challenge in providing the optimal amount of government-supplied public
goods is to know the value that users may place upon them -- how can the
demand curves $D_A$ and $D_B$, be ascertained, for example, in Figure~\ref{fig:optimalpublicgood}?
In contrast to markets for private goods, where
consumer demands are essentially revealed through the process of purchase,
the demands for public goods may have to be uncovered by means of surveys
that are designed so as to elicit the true valuations that users place upon
different amounts of a public good. A second challenge relates to the
pricing and funding of public goods: For example, should highway lighting be
funded from general tax revenue, or should drivers pay for it? These are
complexities that are beyond our scope of our current inquiry.

\newhtmlpage

\subsection*{Asymmetric information}

Markets for information abound in the modern economy. Governments frequently
supply information on account of its public good characteristics. But the
problem of \textit{asymmetric information} poses additional challenges. %
\terminology{Asymmetric information} is where at least one party in an
economic relationship has less than full information. This situation
characterizes many interactions: Bosses do not always know how hard their
subordinates work; life-insurance companies do not have perfect information
on the lifestyle and health of their clients.

\begin{DefBox}
	\textbf{Asymmetric information} is where at least one party in an economic relationship has less than full information and has a different amount of information from another party.
\end{DefBox}

Asymmetric information can lead to two kinds of problems. The first is %
\terminology{adverse selection}. For example, can the life-insurance company
be sure that it is not insuring only the lives of people who are high risk
and likely to die young? If primarily high-risk people buy such insurance
then the insurance company must set its premiums accordingly: The company is
getting an adverse selection rather than a random selection of clients.
Frequently governments decide to run universal compulsory-membership insurance plans
(auto or health are examples in Canada) precisely because they may not wish
to charge higher rates to higher-risk individuals.

\begin{DefBox}
	\textbf{Adverse selection} occurs when incomplete or asymmetric information describes an economic relationship.
\end{DefBox}

\newhtmlpage

A\ related problem is \terminology{moral hazard}. If an individual does not
face the full consequences of his actions, his behaviour may be influenced:
If the boss cannot observe the worker's effort level, the worker may shirk.
Or, if a homeowner has a fully insured home he may be less security
conscious than an owner who does not.

In Chapter~\ref{chap:firminvestorcapital} we described how US mortgage
providers lent large sums to borrowers with uncertain incomes in the early
years of the new millennium. The lenders were being rewarded on the basis of
the amount lent, not the safety of the loan. Nor were the lenders
responsible for loans that were not repaid. This `sub-prime mortgage crisis'
was certainly a case of moral hazard.

\begin{DefBox}
	\textbf{Moral hazard} may characterize behaviour where the costs of certain activities are not incurred by those undertaking them.
\end{DefBox}

Solutions to these problems do not always involve the government, but in
critical situations do. For example, the government requires most
professional societies and orders to ensure that their members are trained,
accredited and capable. Whether for a medical doctor, a plumber or an
engineer, a license or certificate of competence is a signal that the work
and advice of these professionals is \textit{bona fide}. Equally, the
government sets \textit{standards} so that individuals do not have to incur
the cost of ascertaining the quality of their purchases -- bicycle helmets
must satisfy specific crash norms; so too must air-bags in automobiles.

These situations differ from those where solutions to the information
problem can be dealt with reasonably well in the market place. For example,
life insurance companies can frequently establish the past medical history
of its clients, and thus form an estimate of what the client's future health
will be.

\newhtmlpage

\subsection*{Concentration of power}

Monopolistic and imperfectly-competitive market structures can give rise to
inefficient outcomes, in the sense that the value placed on the last unit of
output does not equal the cost at the margin. In monopoly structures this
arises because the supplier uses his market power in order to maximize
profits.

What can governments do about such power concentrations? Every developed
economy has a body similar to Canada's \textit{Competition Bureau}. Such
regulatory bodies are charged with seeing that the interests of the
consumer, and the economy more broadly, are represented in the market place.
Interventions, regulatory procedures and efforts to prevent the abuse of
market power come in a variety of forms. These measures are examined in
Section~\ref{sec:ch14sec5}.

\subsection*{Unfavourable market outcomes}

Even if governments successfully address the problems posed by the market
failures described above, there is nothing to guarantee that market-driven
outcomes will be `fair', or accord with the prevailing notions of justice or
equity. The marketplace generates many low-paying jobs, unemployment and
poverty. Governments address these outcomes through a variety of social
programmes and transfers that are discussed in Section~\ref{sec:ch14sec4}.
