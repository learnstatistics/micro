\section{Land}\label{sec:ch12sec6}

Land is an input used in production, though is not a capital good in the way
we defined capital goods earlier -- production inputs that are themselves
produced in the economy. Land is relatively fixed in supply \textit{to the
economy}, even in the long run. While this may not be literally true -- the
Netherlands reclaimed from the sea a great quantity of low-lying farmland,
and fertilizers can turn marginal land into fertile land -- it is a good
approximation to reality. Figure~\ref{fig:marketlandservices} shows the
derived demand $D_0$ for land services. With a fixed supply $S$, the
equilibrium rental is $R_0$.

% Figure 12.6
\begin{TikzFigure}{xscale=0.3,yscale=0.23,descwidth=25em,caption={The market for land services \label{fig:marketlandservices}},description={The supply of land is relatively fixed, and therefore the return to land is primarily demand determined. Shifts in demand give rise to differences in returns.}}
% Demand lines
\draw [demandcolour,ultra thick,name path=D0] (0,15) -- (15,0) node [black,mynode,above right,pos=0.95] {$D_0$};
\draw [demandcolour,ultra thick,name path=D1] (0,20) -- (27,0) node [black,mynode,above right,pos=0.95] {$D_1$};
% Supply lines
\draw [supplycolour,ultra thick,name path=S] (12,0) -- (12,24) node [black,mynode,right] {$S$};
% axes
\draw [thick] (0,25) node (yaxis) [mynode1,above] {Rental\\Rate} |- (30,0) node (xaxis) [mynode1,right] {Services\\of Land};
% intersection of supply and demand lines
\draw [name intersections={of=S and D0, by=E0},name intersections={of=S and D1, by=E1}]
	[dotted,thick] (yaxis |- E0) node [mynode,left] {$R_0$} -- (E0)
	[dotted,thick] (yaxis |- E1) node [mynode,left] {$R_1$} -- (E1);
\end{TikzFigure}

\newhtmlpage

In contrast to this economy-wide perspective, consider now a retailer who
rents space in a commercial mall. The area around the mall experiences a
surge in development and more people are shopping and doing business there.
The retailer finds that she sells more, but also finds that her rent
increases on account of the additional demand for space by commercial
enterprises in the area. Her landlord is able to charge a higher rent
because so many potential clients wish to rent space in the area.
Consequently, despite the additional commerce in the area, the retailer's
profit increase will be moderated by the higher rents she must pay: The
demand for retail space is a \textit{derived demand}. The situation can be
explained with reference to Figure~\ref{fig:marketlandservices} again. On
account of growth in this area, the demand for retail space shifts from 
$D_{0}$ to $D_{1}$. Space in the area is restricted, and thus the vertical
supply curve describes the supply side well. So with little or no
possibility of higher prices bringing forth additional supply, the
additional demand makes for a steep price (rent) increase. 

Land has many uses and the returns to land must reflect this. Land in
downtown Vancouver is priced higher than land in rural Saskatchewan. Land
cannot be moved from the latter to the former location however, and
therefore the rent differences represent an equilibrium. In contrast, land
in downtown Winnipeg that is used for a parking lot may not be able to
compete with the use of that land for office development. Therefore, for it
to remain as a parking lot, the rental must reflect its high opportunity
cost. This explains why parking fees in big US cities such as Boston or New
York may run to \$40 per day. If the parking owners could not obtain this
fee, they could profitably sell the land to a developer. Ultimately it is
the \textit{value in its most productive use} that determines the price of
land.