\section{The capital market}\label{sec:ch12sec5}

\subsection*{Demand}

The analysis of the demand for the services of capital parallels closely
that of labour demand: The rental rate for capital replaces the wage rate
and capital services replace the hours of labour. It is important to keep in
mind the distinction we drew above between capital services on the one hand
and the amount of capital on the other. Capital services are produced by
capital assets, just as work is produced by humans. Terms that are analogous
to the marginal product of labour emerge naturally: The %
\terminology{marginal product of capital} ($MP_K$) is the output produced by
one additional unit of capital services, with other inputs held constant.
The \terminology{value of this marginal product} ($VMP_K$) is its value in
the market place. It is the $MP_K$ multiplied by the price of output.

The $MP_K$ must eventually decline with a fixed amount of other factors of
production. So, if the price of output is fixed for the firm, it follows
that the $VMP_K$ must also decline. We could pursue an analysis of the 
short-run demand for capital services, assuming labour was fixed, that would
completely mirror the short-run demand for labour that we have already
developed. But this would not add any new insights, so we move on to the
supply side.

\begin{DefBox}
The \textbf{marginal product of capital} is the output produced by one additional unit of capital services, with all other inputs being held constant.

The \textbf{value of the marginal product of capital} is the marginal product of capital multiplied by the price of the output it produces.
\end{DefBox}

\newhtmlpage

\subsection*{Supply}

We can grasp the key features of the market for capital by recognizing that
the \textit{flow} of capital services is determined by the capital \textit{%
stock}: More capital means more services. The analysis of supply is complex
because we must distinguish between the long run and the short run, and also
between the supply to an industry and the supply in the whole economy.

In the \textit{short run} the total supply of capital assets, and therefore
services, is fixed to the \textit{economy}, since new production capacity
cannot come on stream overnight: The short-run supply of services is
therefore vertical. In contrast, \textit{a particular industry} in the short
run faces a positively sloped supply: By offering a higher rental rate for
trucks, one industry can bid them away from others.

The \textit{long run} is a period of sufficient length to permit an addition
to the capital stock. A supplier of capital, or capital services, must
estimate the likely return he will get on the equipment he is contemplating
having built. To illustrate: He is analyzing the purchase or construction of
an earthmover that will cost \$100,000. Assuming that the annual maintenance
and depreciation costs are \$10,000, and that the interest rate is 5\%
(implying that annual interest cost is \$5,000), it follows that the annual
cost of owning such a machine is \$15,000. If the entrepreneur is to
undertake the investment she must therefore earn at least this amount
annually (by renting it to others, or using it herself), and this is what is
termed the \terminology{required rental}. We can think of it as the
opportunity cost of ownership.

\begin{DefBox}
The \textbf{required rental} covers the sum of \textit{maintenance}, \textit{depreciation} and \textit{interest costs}.
\end{DefBox}

\newhtmlpage

\subsection*{Prices and returns}

In the long run, capital services in any sector of the economy must earn the
required rental. If they earn more, entrepreneurs will be induced to build
or purchase additional capital goods; if they earn less, owners of capital
will allow machines to depreciate, or move the machines to other sectors of
the economy. 

As an example, the price of oil on world markets fell by half during 2015;
from about \$100US per barrel to \$50US. At this price, many oil wells were
no longer profitable, and oil drilling equipment was decommissioned.
Technically, the value of the marginal product of capital declined, because
the price of the good it was producing declined. In the near and medium
term, no new investment in capital goods will take place in the oil drilling
sector of the economy. If the price of oil should increase in the future,
some of the decommissioned capital will be brought back into service. But
some of this capital will deteriorate or depreciate and simply `die', and be
sold for scrap metal -- particularly the older vintage capital. Only when the
stock of oil drilling equipment is reduced by depreciation and decay to the
required level will any new investment in this form of capital take place. 

Note that the capital in this example is sector-specific. Drillng
equipment cannot be easily redirected for use in other sectors. In contrast,
earth movers can move from one sector of the economy to another with
greater ease. An earth mover can be used to dig foundations for
housing or commercial buildings; it can be used for strip mining; to build
roads and bridges; to build tennis courts, golf courses and public parks.
Such equipment may thus be moved to other sectors of the economy if in one
particular sector the capital no longer can earn the required rental.

The prices of capital goods in the long run will be determined by the supply
and demand for the services they provide. If the value of the services, as
determined by supply and demand is high, then the price of assets will
reflect this.