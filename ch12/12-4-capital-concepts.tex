\section{Capital -- concepts}\label{sec:ch12sec4}

The share of national income accruing to capital is more substantial than
commonly recognized. National income in Canada is divided 60-40, favoring
labour. This leaves a very large component going to the owners of capital.
The stock of \terminology{physical capital} includes assembly-line
machinery, rail lines, dwellings, consumer durables, school buildings and so
forth. It is the stock of produced goods used as inputs to the production of
other goods and services.

\begin{DefBox}
\textbf{Physical capital} is the stock of produced goods that are inputs to the production of other goods and services.
\end{DefBox}

Physical capital is distinct from land in that the former is produced,
whereas land is not. These in turn differ from \textit{financial wealth},
which is not an input to production. We add to the capital stock by
undertaking investment. But, because capital depreciates, investment in new
capital goods is required merely to stand still. \terminology{Depreciation}
accounts for the difference between \terminology{gross and net investment}.

\begin{DefBox}
\textbf{Gross investment} is the production of new capital goods and the improvement of existing capital goods.

\textbf{Net investment} is gross investment minus depreciation of the existing capital stock.

\textbf{Depreciation} is the annual change in the value of a physical asset.
\end{DefBox}

\newhtmlpage

Since capital is a \terminology{stock} of productive assets we must
distinguish between the value of services that \terminology{flow} from
capital and the value of capital assets themselves.

\begin{DefBox}
A \textbf{stock} is the quantity of an asset at a point in time.

A \textbf{flow} is the stream of services an asset provides during a period of time.
\end{DefBox}

When a car is rented it provides the driver with a service; the car is the
asset, or stock of capital, and the driving, or ability to move from place
to place, is the service that flows from the use of the asset. When a
photocopier is leased it provides a stream of services to the user. The
copier is the asset; it represents a stock of physical capital. The printed
products result from the service the copier provides per unit of time.

The \terminology{price of an asset} is what a purchaser pays for the asset.
The owner then obtains the future stream of capital services it provides.
Buying a car for \$30,000 entitles the owner to a stream of future transport
services. The term \terminology{rental rate} defines the cost of
the services from capital.

\begin{DefBox}
\textbf{Capital services} are the production inputs generated by capital assets.

The \textbf{rental rate} is the cost of using capital services.

The \textbf{price of an asset} is the financial sum for which the asset can be purchased.
\end{DefBox}

But what determines the \textit{price} of a productive asset? The price must
reflect the value of future services that the capital provides. But we
cannot simply add up these future values, because a dollar today is more
valuable than a dollar several years from now. The key to valuing an asset
lies in understanding how to compute the \textit{present value} of a future
income stream.

\newhtmlpage

\subsection*{Present values and discounting}

When capital is purchased it generates a stream of dollar values (returns)
in the future. A critical question is: How is the price that should be paid
for capital today related to the benefits that capital will bring in the
future? Consider the simplest of examples: A business is contemplating
buying a computer. This business has a two-year horizon. It believes that
the purchase of the computer will yield a return of \$500 in the first year
(today), \$500 in the second year (one period into the future), and have a
scrap value of \$200. What is the maximum price the entrepreneur should pay
for the computer? The answer is obtained by \textit{discounting} the future
returns to the present. Since a dollar today is worth more than a dollar
tomorrow, we cannot simply add the dollar values from different time periods.

The value today of \$500 received a year from now is less than \$500,
because if you had this amount today you could invest it at the going rate
of interest and end up with more than \$500 tomorrow. For example, if the
rate of interest is 10\% $(= 0.1)$, then \$500 today is worth \$550 next
period. By the same reasoning, \$500 tomorrow is worth less than \$500
today.  Formally, the value next period of any amount is that amount plus
the interest earned; in this case the value next period of \$500 today is 
$\$500\times (1+r)=\$500\times 1.1=\$550$, where $r$ is the interest rate. It
follows that if we multiply a given sum by $(1+r)$ to obtain its value next
period, then we must divide a sum received next period to obtain its value
today. Hence the value today of \$500 next period is simply 
$\$500/(1+r)=\$500/1.1=\$454.54$. To see that this must be true, note that if
you have \$454.54 today you can invest it and obtain \$500 next period if
the interest rate is 10\%. In general:
\begin{align*}
\text{Value next period} &=\text{value this period}\times (1+\text{interest rate}) \\
\text{Value this period } &= \frac{\text{Value next period}}{(1+\text{interest rate)}}.
\end{align*}

This rule carries over to any number of future periods. The value of a sum
of money today two periods into the future is obtained by multiplying the
today value by $(1+\text{interest rate})$ twice. Or the value of a sum of money
today that will be received two periods from now is that sum divided by 
$(1+\text{interest rate})$ twice. And so on, for any number of time periods. So if the
amount is received twenty years into the future, its value today would be
obtained by dividing that sum by $(1+\text{interest rate})$ twenty times; if
received `$n$' periods into the future it must be divided by 
$(1+\text{interest rate})$ `$n$' times.

\newhtmlpage

Two features of this discounting are to be noted: First, if the interest
rate is high, the value today of future sums is smaller than if the interest
rate is low. Second, sums received far in the future are worth much less
than sums received in the near future.

Let us return to our initial example, assuming the interest rate is 0.1 (or
10\%). The value of the year 1 return is \$500. The value of the year 2
return today is \$454.54, and the scrap value in today's terms is \$181.81.
The value of all returns discounted to today is thus \$1,136.35. 

\begin{Table}{caption={Present value of an asset ($i=10\%$) \label{table:presentvalueasset}}}
\begin{tabu} to \linewidth {|X[1.5,c]X[1,c]X[1,c]X[1.5,c]|} \hline 
	\rowcolor{rowcolour}	\textbf{Year} & \textbf{Annual return} & \textbf{Scrap value} & \textbf{Discounted values} \\ \hline
							Year 1 & 500 &  & 500 \\ 
	\rowcolor{rowcolour}	Year 2 & 500 & 200 & 454.54 + 181.81 \\ 
							Asset value today &  &  & 1,136.35 \\  \hline
\end{tabu}
\end{Table}

\begin{DefBox}
The \textbf{present value of a stream of future earnings} is the sum of each year's earnings divided by one plus the interest rate `$n$' times, where `$n$' is the number of years in the future when the amount will be received.
\end{DefBox}

We are now in a position to determine how much the buyer should be willing
to pay for the computer. Clearly if the value of the computer today,
measured in terms of future returns to the entrepreneur's business, is
\$1,136.35, then the potential buyer should be willing to pay any sum less
than that amount. Paying more makes no economic sense.

Discounting is a technique used in countless applications. It underlies the
prices we are willing to pay for corporate stocks: Analysts make estimates
of future earnings of corporations; they then \textit{discount those
earnings back to the present}, and suggest that we not pay more for a unit
of stock than suggested by the present value of future earnings.