\section{The supply of labour}\label{sec:ch12sec2}

Most prime-age individuals work, but some do not. The decision to join the %
\terminology{labour force} is called the \terminology{participation decision}. 
Of those who do participate in the labour force, some individuals work
full time, others work part time, and yet others cannot find a job. The %
\terminology{unemployment rate} is the fraction of the labour force actively
seeking employment that is not employed.

\begin{DefBox}
The \textbf{participation rate} for the economy is the fraction of the population in the working age group that joins the labour force.

The \textbf{labour force} is that part of the population either employed or seeking employment.

The \textbf{unemployment rate} is the fraction of the labour force actively seeking employment that is not employed.
\end{DefBox}

\newhtmlpage

Data on participation rates in Canada are given in Table~\ref{table:labourforcepartrate} below for
specific years in the modern era. The overall participation for men and
women combined has increased since 1976 from 61.5\% to 65.7\% This
aggreagated rate camouflages different patterns for men and women. The rates
for women have been rising while the rates for men have fallen. The former
trend reflects changes in social customs and changes in household
productivity. Women today are more highly educated, and their role in
society and the economy is viewed very differently than in the earlier
period. Female participation has increased both because of changing social
norms, a rise in household productivity, the development of service
industries designed to support home life, and the development of the
institution of daycare for young children. 

In contrast, male participation rates declined until the mid-1990s on
account of an increasing fraction of the male labour force retiring before
the traditional age of 65. This trend has reversed itself in the most recent
decade. 

\begin{Table}{caption={Labour force participation rate, Canada 1976-2015 \label{table:labourforcepartrate}},description={\textit{Source}: Statistics Canada, CANSIM 282-0087.},descwidth={17em}}
\begin{tabu} to \linewidth {|X[1,c]X[1,c]X[1,c]X[1,c]X[1,c]|} \hline 
\rowcolor{rowcolour}	\textbf{Year} & \textbf{Total} & \textbf{Men} & \textbf{Women} & \textbf{Men $>$ 55} \\ 	\hline
						1976 & 61.5 & 78.0 & 45.4 & 48.2 \\ 
\rowcolor{rowcolour}	1990 & 67.5 & 76.9 & 58.5 & 37.5 \\ 
						2001 & 65.9 & 72.4 & 59.7 & 33.0 \\ 
\rowcolor{rowcolour}	2008 & 67.6 & 72.9 & 62.5 & 40.8 \\ 
						2015 & 65.7 & 70.4 & 61.2 & 42.7 \\ \hline
\end{tabu}
\end{Table}

At the micro level, the participation rate of individuals depends upon
several factors. First, the wage rate that an individual can earn in the
market is crucial. If that wage is low, then the individual may be more
efficient in producing home services directly, rather than going into the
labour market, earning a modest income and having to pay for home services.
Second, there are fixed costs associated with working. A decision to work
means that the individual must have work clothing, must undertake the costs
of travel to work, and pay for daycare if there are children in the family.
Third, the participation decision depends upon non-labour income. If the
individual in question has a partner who earns a substantial amount, or if
she has investment income, she will have less incentive to participate.
Fourth, it depends inversely upon the tax rate.

\newhtmlpage

The supply curve relates the supply decision to the price of labour -- the
wage rate. Economists who have studied the labour market tell us that the
individual supply curve is upward sloping: As the wage increases, the
individual wishes to supply more labour. From the point $e_{0}$ on the
supply function in Figure~\ref{fig:indlaboursupply}, let the wage increase
from $W_{0}$ to $W_{1}$.

% Figure 12.2
\begin{TikzFigure}{xscale=0.4,yscale=0.3,descwidth=25em,caption={Individual labour supply \label{fig:indlaboursupply}},description={A wage increase from $W_0$ to $W_1$ induces the individual to \emph{substitute} away from leisure, which is now more expensive, and work \emph{more}. But the higher wage also means the individual can work fewer hours for a given standard of living; therefore the income effect induces \emph{fewer hours}. On balance the substitution effect tends to dominate and the supply curve therefore slopes upward.}}
% Supply curve
\draw [supplycolour,ultra thick,domain=270:360,name path=S] plot ({5+10*cos(\x)},{15+10*sin(\x)}) node [mynode,above,black] {$S$};
% axes
\draw [thick, -] (0,20) node (yaxis) [mynode1,above] {Wage\\Rate} |- (20,0) node (xaxis) [right] {Hours};
% paths for e0 and e1
\path [name path=e0line] (10,0) -- +(0,20);
\path [name path=e1line] (14.5,0) -- +(0,20);
% intersection of S with paths
\draw [name intersections={of=S and e0line, by=e0},name intersections={of=S and e1line, by=e1}]
	[dotted,thick] (yaxis |- e0) node [mynode,left] {$W_0$} -- (e0) node [mynode,above left] {$e_0$} -- (xaxis -| e0) node [mynode,below] {$H_0$}
	[dotted,thick] (yaxis |- e1) node [mynode,left] {$W_1$} -- (e1) node [mynode,above left] {$e_1$} -- (xaxis -| e1) node [mynode,below] {$H_1$};
\end{TikzFigure}

The individual offers more labour, $H_{1}$, at the higher wage. What is the
economic intuition behind the higher amount of labour supplied? Like much of choice
theory there are two impacts associated with a higher price. First, the
higher wage makes leisure more expensive relative to working. That midweek
game of golf has become more expensive in terms of what the individual could
earn. So the individual should \textit{substitute} away from the more
expensive `good', leisure, towards labour. But at the same time, in order to
generate a given income target the individual can work fewer hours at the
higher wage. This is a type of \textit{income effect}, indicating that
income is greater at a higher wage regardless of the amount worked, and this
induces the individual to work less. The fact that we draw the labour supply
curve with a positive slope means that the substitution effect is
the more important of the two. That is what statistical research has revealed.

\newhtmlpage

\subsection*{Elasticity of the supply of labour}

The value of the supply elasticity depends upon how the market in question
is defined. In particular, it depends upon how large or small a given sector
of the economy is, and whether we are considering the short run or the long
run.

Suppose an industry is small relative to the whole economy and employs
workers with common skills. These industries tend to pay the `going wage'.
For example, very many students are willing to work at the going rate for
telemarketing firms, which compose a small sector of the economy. This means
that the supply curve of such labour, \textit{as far as that sector is
concerned}, is in effect horizontal -- infinitely elastic. 

But some industries may not be small relative to the total labour supply.
And in order to get more labour to work in such large sectors it may be
necessary to provide the inducement of a higher wage: Additional workers may
have to be attracted from another sector by means of higher wages. To
illustrate: Consider the behaviour of two related sectors in housing -- new
construction and home restoration. In order to employ more plumbers and
carpenters, new home builders may have to offer higher wages to induce them
to move from the renovation sector. In this case the new housing industry's
labour supply curve slopes upwards.

In the time dimension, a longer period is always associated with more
flexibility. In this context, the supply of labour to any sector is more
elastic, because it may take time for workers to move from one sector to
another. Or, in cases where skills must be built up: When a sectoral
expansion bids up the wages of information technology (IT) workers, more
school leavers are likely to develop IT skills. Time will be required before
additional graduates are produced, but in the long run, such additional
supply will moderate the short-run wage increases.

Taxes can be viewed as being part of the wage. That is, wages can be defined
as being before-tax or after-tax. The after-tax, or take-home, wage is more 
important than the gross wage in determining the quantity of labour to be 
supplied. If taxes on additional hours of work are very high, workers are 
more likely to supply less hours than if tax rates are lower.