\section{Labour market equilibrium and mobility}\label{sec:ch12sec3}

The fact that labour is a derived demand differentiates the labour market's
equilibrium from the goods-market equilibrium. Let us investigate this with
the help of Figure~\ref{fig:eqindustrylabour}; it contains supply and demand
functions for one particular industry -- the cement industry, let us assume.

In Figure~\ref{fig:demandforlabour} we illustrated the impact on the demand
for labour of a decline in the price of the output produced -- a decline in
the output price reduced the value of the marginal product of labour. In the
current example, suppose that a slowdown in construction results in a
decline in the price of cement. The impact of this price fall is to reduce
the output value of each worker in the cement producing industry, because
their output now yields a lower price. This decline in the $VMP_{L}$ is
represented in Figure~\ref{fig:eqindustrylabour} as a shift from $D_{0}$ to 
$D_{1}$. The new $VMP_{L}$ curve ($D_{1}$) results in the new equilibrium 
$E_{1}$.

% Figure 12.3
\begin{TikzFigure}{xscale=0.33,yscale=0.25,descwidth=25em,caption={Equilibrium in an industry labour market \label{fig:eqindustrylabour}},description={A fall in the \emph{price of the good} produced in a particular industry reduces the value of the $MP_L$. Demand for labour thus falls from $D_0$ to $D_1$ and a new equilibrium $E_1$ results. Alternatively, from $E_0$, an increase in wages in another sector of the economy induces some labour to move to that sector. This is represented by the shift of $S_0$ to $S_1$ and the new equilibrium $E_2$.}}
% demand lines
\draw [demandcolour,ultra thick,name path=D1] (0,15) -- (15,0) node [black,mynode,above right] {$D_1$};
\draw [demandcolour,ultra thick,name path=D0] (0,20) -- (27,0) node [black,mynode,above right] {$D_0$};
% supply lines
\draw [supplycolour,ultra thick,name path=S0] (0,1) -- (30,22) node [black,mynode,above] {$S_0$};
\draw [supplycolour,ultra thick,name path=S1] (0,3) -- (25,25) node [black,mynode,above] {$S_1$};
\draw [thick, -] (0,25) node (yaxis) [mynode1,above] {Wage\\Rate} |- (30,0) node (xaxis) [right] {Labour};
% intersection of supply and demand lines
\draw [name intersections={of=S0 and D0, by=E0},name intersections={of=S0 and D1, by=E1},name intersections={of=S1 and D0, by=E2}]
	[dotted,thick] (yaxis |- E0) node [mynode,left] {$W_0$} -- (E0) node [mynode,above] {$E_0$} -- (xaxis -| E0) node [mynode,below] {$L_0$}
	[dotted,thick] (yaxis |- E1) node [mynode,left] {$W_1$} -- (E1) node [mynode,above] {$E_1$} -- (xaxis -| E1) node [mynode,below] {$L_1$}
	[dotted,thick] (yaxis |- E2) node [mynode,left] {$W_2$} -- (E2) node [mynode,above] {$E_2$} -- (xaxis -| E2) node [mynode,below] {$L_2$};
% paths to create arrows between supply and demand lines
\path [name path=supplyarrowline] (0,20) -- +(30,0);
\path [name path=demandarrowline] (2,0) -- +(0,25);
% arrows between supply and demand lines
\draw [name intersections={of=S0 and supplyarrowline, by=s0},name intersections={of=S1 and supplyarrowline, by=s1}]
	[->,thick,shorten >=1mm,shorten <=1mm] (s0) -- (s1);
\draw [name intersections={of=D0 and demandarrowline, by=d0},name intersections={of=D1 and demandarrowline, by=d1}]
	[->,thick,shorten >=1mm,shorten <=1mm] (d0) -- (d1);
\end{TikzFigure}


\newhtmlpage



As a second example: Suppose that wages in some other sectors of the economy
increase. The impact of this on the cement sector is that the supply of
labour to the cement sector is reduced. In Chapter~\ref{chap:classical} we showed that a change
in other prices may \textit{shift} the demand or supply curve of interest.
In Figure~\ref{fig:eqindustrylabour} supply shifts from $S_{0}$ to $S_{1}$
and the equilibrium goes from $E_{0}$ to $E_{2}$.

How large are these impacts likely to be? That will depend upon how mobile
labour is between sectors: Spillover effects will be smaller if labour is
less mobile. This brings us naturally to the concepts of %
\terminology{transfer earnings} and \terminology{rent}.

\newhtmlpage

\subsection*{Transfer earnings and rent}

Consider the case of a performing violinist whose wage is \$80,000. If, as a
best alternative, she can earn \$60,000 as a music teacher then her rent is
\$20,000 and her transfer earnings \$60,000: Her rent is the excess she
currently earns above the best alternative. Another violinist in the same
orchestra, earning the same amount, who could earn \$55,000 as a teacher has
rent of \$25,000. The alternative is called the \terminology{reservation
wage}. The violinists should not work in the orchestra unless they earn at
least what they can earn in the next best alternative.

\begin{DefBox}
\textbf{Transfer earnings} are the amount that an individual can earn in the next highest paying alternative job.

\textbf{Rent} is the excess remuneration an individual currently receives above the next best alternative. This alternative is the \textbf{reservation wage}.
\end{DefBox}

These concepts are illustrated in Figure~\ref{fig:transferearningrent}. In
this illustration, different individuals are willing to work for different
amounts, but all are paid the same wage $W_{0}$. The market labour supply
curve by definition defines the wage for which each individual is willing to
work. Thus the rent earned by labour in this market is the sum of the excess
of the wage over each individual's transfer earnings -- the area $W_{0}E_{0}A$. 
This area is also what we called producer or supplier surplus in Chapter~\ref{chap:welfare}.

% Figure 12.4
\begin{TikzFigure}{xscale=0.3,yscale=0.25,descwidth=25em,caption={Transfer earnings and rent \label{fig:transferearningrent}},description={Rent is the excess of earnings over reservation wages. Each individual earns $W_0$ and is willing to work for the amount defined by the labour supply curve. Hence rent is $W_0E_0$A and transfer earnings OA$E_0L_0$. Rent is thus the term for supplier surplus in this market.}}
% demand line
\draw [demandcolour,ultra thick,name path=D] (0,20) -- (27,0) node [black,mynode,above right,pos=0.95] {$D$};
% supply line
\draw [supplycolour,ultra thick,name path=S] (0,1) node [black,mynode,left] {A} -- (30,22) node [black,mynode,above] {$S$};
% axes
\draw [thick] (0,25) node (yaxis) [mynode1,above] {Wage\\Rate} -- (0,0) node [mynode,below left] {O} -- (30,0) node (xaxis) [right] {Labour};
% intersection of supply and demand line
\draw [name intersections={of=S and D, by=E0}]
	[dotted,thick] (yaxis |- E0) node [mynode,left] {$W_0$} -- node [mynode,below=0.25cm and 0cm,pos=0.4] {Rent} (E0) node [mynode,above] {$E_0$} -- node [mynode,left=0cm and 0.5cm,pos=0.6] {Transfer\\earnings} (xaxis -| E0) node [mynode,below] {$L_0$};
\end{TikzFigure}

\newhtmlpage

\subsection*{Free labour markets?}

Real-world labour markets are characterized by trade unions, minimum wage
laws, benefit regulations, severance packages, parental leave, sick-day
allowances and so forth. So can we really claim that markets work in the way
we have described them -- essentially as involving individual agents
demanding and supplying labour? While labour markets are not completely
'free' in the conventional sense, the important issue is whether these
interventions, that are largely designed to protect workers, have a large or
small impact on the market. One reason why
unemployment rates are generally higher in European economies than in Canada
and the US is that labour markets are less subject to controls, and workers
have a less supportive social safety net in North America.

\begin{ApplicationBox}{caption={Are high salaries killing professional sports? \label{app:highsalarysport}}}
It is often said that the agents of professional players are killing their sport by demanding unreasonable salaries. Frequently the major leagues are threatened with strikes, even though players are paid millions each year. In fact, wages are high because the derived demand is high. Fans are willing to pay high ticket prices, and television rights generate huge revenues. Combined, these revenues not only make ownership profitable, but increase the demand for the top players.

The lay person may be horrified at ten-million dollar salaries. But in reality, many players receiving such salaries may be earning less than their marginal product! If Tom Brady did not play for the New England Patriots the team would have a lower winning record, attract fewer fans and make less profit. If Brady is paid \$10m per season, he is being paid less than his marginal product if the team were to lose \$40m in revenue as a result of his absence.

Given this, why do some teams incur financial losses? In fact very few teams make losses: Cries of poverty on the part of owners are more frequently part of the bargaining process. Occasionally teams are located in the wrong city and they should therefore either exit the industry or move the franchise to another market. 
\end{ApplicationBox}

\newhtmlpage

The impact of `frictions', such as unionization and minimum wages, in the
labour market can be understood with the help of Figure~\ref{fig:marketinterventions}.
The initial `free market' equilibrium is at $E_{0}$, 
assuming that the workers are not unionized. In contrast, if the workers
in this industry form a union, and negotiate a higher wage, for example 
$W_{1}$ rather than $W_{0}$, then fewer workers will be employed. But how big
will this reduction be? Clearly it depends on the elasticities of demand and 
supply. With the demand curve $D$, the excess supply at the wage $W_{1}$ is the
difference $E_{1}F$. However, if the demand curve is less elastic, as
illustrated by the curve $D^{\prime}$, the excess supply is $E^{\prime}F$.
The excess supply will depend also upon the supply elasticity. It is
straightforward to see that a less elastic (more vertical) supply curve
through $E_{0}$ would result is less excess supply.

% Figure 12.5
\begin{TikzFigure}{xscale=0.3,yscale=0.25,descwidth=25em,caption={Market interventions \label{fig:marketinterventions}},description={$E_0$ is the equilibrium in the absence of a union. If the presence of a union forces the wage to $W_1$ fewer workers are employed. The magnitude of the decline from $L_0$ to $L_1$ depends on the elasticity of demand for labour. The excess supply at the wage $W_1$ is (F-$E_1$). With a less elastic demand curve ($D^{\prime}$) the excess supply is reduced to (F-$E^{\prime}$).}}
% Demand line
\draw [demandcolour,ultra thick,name path=D] (0,20) -- (27,0) node [black,mynode,above right,pos=0.95] {$D$};
% Supply line
\draw [supplycolour,ultra thick,name path=S] (0,1) -- (30,22) node [black,mynode,above] {$S$};
% axes
\draw [thick, -] (0,25) node (yaxis) [mynode1,above] {Wage\\Rate} |- (30,0) node (xaxis) [right] {Labour};
% intersection of demand and supply
\draw [name intersections={of=S and D, by=E0}]
	[dotted,thick] (yaxis |- E0) node [mynode,left] {$W_0$} -- (E0) node [mynode,right=0.25cm and 0.25cm] {$E_0$} -- (xaxis -| E0) node [mynode,below] {$L_0$};
% Demand line D'
\draw [demandcolour,ultra thick,name path=Dprime,shorten >=-3.15cm] (6,22) -- node [mynode,black,right,pos=0.05] {$D^{\prime}$} (E0);
% path for W1
\path [name path=W1line] (0,15) -- +(30,0);
% intersection of supply and demand lines with W1line
\draw [name intersections={of=W1line and S, by=F},name intersections={of=W1line and D, by=E1}]
	[dotted,thick] (yaxis |- E1) node [mynode,left] {$W_1$} -| (xaxis -| E1) node [mynode,below] {$L_1$};
\draw [dotted,thick,name path=E1Fline] (E1) node [mynode,above] {$E_1$} -- (F) node [mynode,above] {F};
% intersection of D' and E1Fline
\draw [name intersections={of=Dprime and E1Fline, by=Eprime}];
\node [mynode,above right] at (Eprime) {$E^{\prime}$};
\end{TikzFigure}

\newhtmlpage

Beyond elasticity, the magnitude of the excess supply will also depend upon
the degree to which the minimum wage, or the union-negotiated wage, lies
above the equilibrium. That is, a larger value of the difference ($W_{1}-W_{0})$ 
results in more excess supply than a smaller difference. 

While the above discussion pertains to unionization, it could equally well
be interpreted in a minimum-wage context. If this figure describes the
market for low-skill labour, and the government intervenes by setting a
legal minimum at $W_{1}$, then this will induce some degree of excess
supply, depending upon the actual value of $W_{1}$ and the elasticities of
supply and demand. 

Despite the fact that a higher wage may induce some excess supply, it may
increase total earnings. In Chapter~\ref{chap:elasticities} we saw that the
dollar value of expenditure on a good increases when the price rises if the
demand is inelastic. In the current example the `good' is labour. Hence, a
union-negotiated wage increase, or a higher minimum wage will each increase
total remuneration if the demand for labour is inelastic. A case which has
stirred great interest is described in Application Box~\ref{app:davidcardminwage}.

\newpage

\begin{ApplicationBox}{caption={David Card on minimum wage \label{app:davidcardminwage}}}
David Card is a famous Canadian-born labour economist who has worked at Princeton University and University of California, Berkeley. He is a winner of the prestigious Clark medal, an award made annually to an outstanding economist under the age of forty. Among his many contributions to the discipline, is a study of the impact of minimum wage laws on the employment of fast-food workers. With Alan Krueger as his co-researcher, Card examined the impact of the 1992 increase in the minimum wage in New Jersey and contrasted the impact on employment changes with neighbouring Pennsylvania, which did not experience an increase. They found virtually no difference in employment patterns between the two states. This research generated so much interest that it led to a special conference. Most economists now believe that modest changes in the level of the minimum wage have a small impact on employment levels.

In the year 2015, numerous movements favoring higher wages for low-paid workers have proposed a \$15 minimum. Some political parties have supported this movement, as have specific cities and municipalities and governments. While any increase in the minimum wage must by definition help those working, care must be exercised in implementing particularly large increases. This is because large increases in particular areas or spheres may induce production units to move outside of the area covered, and thereby shift jobs to lower-wage areas.
\end{ApplicationBox}
