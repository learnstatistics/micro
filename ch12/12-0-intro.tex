\chapter{Labour and capital}\label{chap:marketlabourcapital}

\begin{topics}
	\textbf{In this chapter we will explore:}
	\begin{description}
		\item[\ref{sec:ch12sec1}] The demand for labour
		\item[\ref{sec:ch12sec2}] Labour supply
		\item[\ref{sec:ch12sec3}] Market equilibrium and labour mobility
		\item[\ref{sec:ch12sec4}] The concepts of capital
		\item[\ref{sec:ch12sec5}] The capital market
		\item[\ref{sec:ch12sec6}] Land
	\end{description}
\end{topics}

The chapter deals with the markets for the factors of production---labour
and capital. The analysis will be presented in terms of the demand, supply
and market equilibrium for each. While this is a standard analytical
approach in microeconomics, the markets for labour and capital differ from
goods and services markets. 

In the first instance, goods and services are purchased and consumed by
the buyers. In contrast, labour and capital are used as inputs in producing
those `final' goods and services. So the value of labour and capital to a
producer depends in part upon the value of the products that the labour and
capital are used to produce. Economists say that the value of the factors of
production \textit{derives} from the value of the products they ultimately
produce.

Secondly, labour and capital offer services. When an employer hires a worker,
that worker supplies her time and skills and energy to the employer. When a
piece of equipment is rented to a producer, or purchased by a producer, that
equipment provides a stream of productive services also. The employer does
not purchase the worker, given that we do not live in a society where
slavery is legal. In contrast, she may decide to purchase the capital, or
else rent it.

The third characteristic of these markets is the time dimension associated
with labour and capital. Specifically, once built, a machine will
customarily have a lifetime of several years, during which it depreciates in
value. Furthermore, it may become obsolete on account of technological
change before the end of its anticipated life. Labour too may become
obsolete, or at least lose some of its value with the passage of time, if
the skills embodied in the labour cease to be required in the economy.