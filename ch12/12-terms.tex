\newpage
	\section*{Key Terms}
\begin{keyterms}
\textbf{Demand for labour}: a derived demand, reflecting the demand for the output of final goods and services.

\textbf{Value of the marginal product} is the marginal product multiplied by the price of the good produced.

\textbf{Marginal revenue product of labour} is the additional revenue generated by hiring one more unit of labour where the marginal revenue declines.

\textbf{Monopsonist} is the sole buyer of a good or service and faces an upward-sloping supply curve.

\textbf{Participation rate}: the fraction of the population in the working age group that joins the labour force.

The \textbf{labour force} is that part of the population either employed or seeking employment.

\textbf{Unemployment rate}: the fraction of the labour force actively seeking employment that is not employed.

\textbf{Transfer earnings} are the amount that an individual can earn in the next highest paying alternative job.

\textbf{Rent} is the excess remuneration an individual currently receives above the next best alternative. This alternative is the reservation wage.

\textbf{Physical capital} is the stock of produced goods that are inputs to the production of other goods and services.

\textbf{Gross investment} is the production of new capital goods and the improvement of existing capital goods.

\textbf{Net investment} is gross investment minus depreciation of the existing capital stock.

\textbf{Depreciation} is the annual change in the value of a physical asset.

\textbf{Stock} is the quantity of an asset at a point in time.

\textbf{Flow} is the stream of services an asset provides during a period of time.

\textbf{Capital services} are the production inputs generated by capital assets.

\textbf{Rental rate}: the cost of using capital services.

\textbf{Asset price}: the financial sum for which the asset can be purchased.

\textbf{Present value of a stream of future earnings}: the sum of each year's earnings divided by one plus the interest rate raised to the appropriate power.

\textbf{Marginal product of capital} is the output produced by one additional unit of capital services, with all other inputs being held constant.

\textbf{Value of the marginal product} of capital is the marginal product of capital multiplied by the price of the output it produces.

\textbf{Required rental} covers the sum of maintenance, depreciation and interest costs.
\end{keyterms}