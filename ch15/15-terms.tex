\newpage
	\section*{Key Terms}
\begin{keyterms}
\textbf{Autarky} denotes the no-trade situation.

\textbf{Principle of comparative advantage} states that even if one country has an absolute advantage in producing both goods, gains to specialization and trade still materialize, provided the opportunity cost of producing the goods differs between economies.

\textbf{Terms of trade} define the rate at which goods trade internationally.

\textbf{Consumption possibility frontier} defines what an economy can consume after production specialization and trade.

\textbf{Intra-industry trade} is two-way international trade in products produced within the same industry.

\textbf{Intra-firm trade} is two-way trade in international products produced within the same firm.

\textbf{Dynamic gains}: the potential for domestic producers to increase productivity by competing with, and learning from, foreign producers.

\textbf{Total factor productivity}: how efficiently the factors of production are combined.

\textbf{Tariff} is a tax on an imported product that is designed to limit trade in addition to generating tax revenue. It is a barrier to trade.

\textbf{Quota} is a quantitative limit on an imported product.

\textbf{Trade subsidy} to a domestic manufacturer reduces the domestic cost and limits imports.

\textbf{Non-tariff barriers}, such as product content requirements, limits the gains from trade.

\textbf{Dumping} is a predatory practice, based on artificial costs aimed at driving out domestic producers.
\end{keyterms}