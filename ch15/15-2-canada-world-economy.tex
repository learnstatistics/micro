\section{Canada in the world economy}\label{sec:ch15sec2}

World trade has grown rapidly since the end of World War II, indicating that
trade has become ever more important to national economies. Canada has been
no exception. Canada signed the Free Trade Agreement with the US in 1989,
and this agreement was expanded in 1994 when Mexico was included under the
North America Free Trade Agreement (NAFTA). Imports and exports rose
dramatically, from approximately one quarter to forty percent of GDP. Canada
is now what is termed a very `open' economy -- one where trade forms a large
fraction of total production. As of 2015 Canada was involved in the
Trans-Pacific Partnership negotiations, which are aimed at strengthening
trade relations among nations around the Pacific Rim.

Smaller economies are typically more open than large economies---Belgium and
the Netherlands depend upon trade more than the United States. This is
because large economies have a sufficient variety of resources to supply
much of an individual country's needs. The European Union is similar, in
population terms, to the United States, but it is composed of many distinct
economies. Some European economies are equal in size to individual American
states. But trade between California and New York is not international,
whereas trade between Italy and the Spain is.

Because our economy is increasingly open to international trade, events in
the world economy affect our daily lives much more than in the past. The
conditions in international markets for basic commodities and energy affect
all nations, both importers and exporters. For example, the prices of
primary commodities on world markets increased dramatically in the latter
part of the 2000s. Higher prices for grains, oil, and fertilizers on world
markets brought enormous benefits to Canada, particularly the Western
provinces, which produce these commodities. In contrast, by early 2015, many
of these prices dropped dramatically and Canadian producers suffered as a
consequence. 

The service sector accounts for more of our GDP than the manufacturing
sector. As incomes grow, the demand for health, education, leisure,
financial services, tourism, etc., dominates the demand for physical products.
Technically the income elasticity demand for the former group exceeds the
income elasticity of demand for the latter. Internationally, while trade in
services is growing rapidly, it still forms a relatively small part of total
world trade. Trade in goods---merchandise trade---remains dominant, partly
because many countries import goods, add some value, and re-export them.
Even though the value added from such import-export activity may make just
a small contribution to GDP, the gross flows of imports and exports can
still be large relative to GDP. The transition from agriculture to
manufacturing and then to services has been underway in developed economies
for over a century. This transition has been facilitated in recent decades
by the communications revolution and globalization. Globalization has seen a
rapid shift in production from the developed to the developing world.

\newhtmlpage

\begin{Table}{caption={Canada's merchandise trade patterns 2014 \label{table:cdnmerchtradepatterns}},description={\textit{Source}: \url{http://www.statcan.gc.ca/cgi-bin/sum-som/fl/saveas-eng.cgi}.},descwidth={29em}}
	\begin{tabu} to \linewidth {|X[1.75,l]X[1,c]X[1,c]|}	\hline
		\rowcolor{rowcolour}	\textbf{Country}	& \textbf{Exports to}	& \textbf{Imports from}	\\ \hline
		United States 			& 73.5 			& 63.1 			\\ 
		\rowcolor{rowcolour}	European Union 			& 8.6 			& 10.4 			\\ 
		United Kingdom 			& 3.5 			& 2.3 			\\ 
		\rowcolor{rowcolour}	China 					& 3.2 			& 6.3 			\\ 
		Mexico 					& 1.8 			& 2.7 			\\ 
		\rowcolor{rowcolour}	Japan 					& 2.4 			& 2.5 			\\ 
		Others 					& 7.0 			& 12.7 			\\ 
		\rowcolor{rowcolour}	Total 					& 100 			& 100			\\	\hline
	\end{tabu}
\end{Table}

\begin{Table}{caption={Canadian goods exports 2012 \label{table:cdngoodsexports}},description={\textit{Source:} Adapted from Statistics Canada CANSIM Database, \url{http://cansim2.statcan.gc.ca}, Table 376-0007.},descwidth={30em}}
	\begin{tabu} to 30em {|X[2.5,l]X[1,c]|}	\hline
		\rowcolor{rowcolour}	\textbf{Sector}								&	\textbf{Percentage of total}	\\	\hline
		Agriculture and fishing				&	9.0					\\
		\rowcolor{rowcolour}	Energy								&	24.5				\\
		Forestry							&	4.9					\\
		\rowcolor{rowcolour}	Industrial goods and materials		&	25.5				\\
		Machinery and equipment				&	17.6				\\
		\rowcolor{rowcolour}	Automotive products					&	12.9				\\
		Other consumer goods				&	3.6					\\
		\rowcolor{rowcolour}	Total								&	100.0				\\	\hline
	\end{tabu}
\end{Table}

Table~\ref{table:cdnmerchtradepatterns} shows the patterns of Canadian
merchandise trade in 2008. The United States was and still is Canada's major
trading partner, buying almost three quarters of our exports and supplying almost 
two thirds of Canadian imports. Table~\ref{table:cdngoodsexports} details
exports by type. Although exports of resource-based products account for
only about 40 percent of total exports, Canada is now viewed as a
resource-based economy. This is in part because manufactured products
account for almost 80 percent of US and European exports but only about 60
percent of Canadian exports. Nevertheless, Canada has important export
strength in machinery, equipment, and automotive products.
