\section{Returns to scale and dynamic gains from trade}\label{sec:ch15sec4}

The theory of comparative advantage explains why economies should wish to
trade. The theory is based upon the view that economies are `inherently'
different in their production capabilities. But trade is influenced by more
than these differences. We will explore how returns to scale may be
exploited to generate benefits from trade and also how economies might gain
from one-another by learning as a result of trading. This learning can
increase domestic productivity.

\subsection*{Returns to scale}

One of the reasons Canada signed the North America Free trade Agreement
(NAFTA) was that economists convinced the Canadian government that a larger
market would enable Canadian producers to be even \textit{more efficient}
than in the presence of trade barriers. Rather than opening up trade in
order to take advantage of existing comparative advantage, it was proposed
that efficiencies would actually increase with market size. This argument is
easily understood in terms of increasing returns to scale concepts that we
developed in Chapter~\ref{chap:prodcost}. Essentially, economists suggested
that there were several sectors of the Canadian economy that were operating
on the downward sloping section of their long-run average cost curve.

Increasing returns are evident in the world market place as well as the
domestic marketplace. Witness the small number of aircraft
manufacturers---\textit{Airbus} and \textit{Boeing} are the world's two major manufacturers of
large aircraft. Enormous fixed costs---in the form of research, design, and
development---or capital outlays frequently result in decreasing unit costs,
and the world marketplace can be supplied at a lower cost if some
specialization can take place. Consider the specific example of automotive
trade. In North America, Canadian auto plants produce
different vehicle models than their counterparts in the US. Canada exports
some \textit{models} of a given manufacturer to the United States and
imports other \textit{models}. This is the phenomenon of %
\terminology{intra-industry trade} and \terminology{intra-firm trade}. How
can we explain these patterns?

\begin{DefBox}
	\textbf{Intra-industry trade} is two-way international trade in products produced within the same industry.
	
	\textbf{Intra-firm trade} is two-way trade in international products produced within the same firm.
\end{DefBox}

\newhtmlpage

In the first instance, intra-industry trade reflects the preference of
consumers for a choice of brands; consumers do not all want the same car, or
the same software, or the same furnishings. The second element to
intra-industry trade is that increasing returns to scale characterize many
production processes. Let us see if we can transform the returns to scale
ideas developed in earlier chapters into a production possibility framework.

% Figure 15.3
\begin{TikzFigure}{xscale=0.35,yscale=0.3,descwidth=25em,caption={Intra industry trade \label{fig:intraindustry}},description={Hunda can produce either 100,000 of each vehicle or 40,000 of both in each plant. Hence production possibilities are given by the points A, Z, and B. Pre-trade it produces at Z in each economy due to trade barriers. Post-trade it produces at A in one economy and B in the other, and ships the vehicles internationally. Total production increases from 160,000 to 200,000 using the same resources.}}
\draw [dotted,thick]
	(0,8) node [mynode,above right] {40,000} -- (8,8) node [mynode,below left] {Z} -- (8,0) node [mynode,above right] {40,000};
\draw [ppfcolourthree,ultra thick,-]
	(0,20) node [black,mynode,above right] {A=100,000} -- (20,0) node [black,mynode,above right] {B=100,000} node [black,mynode,above right,pos=0.5] {Post-trade\\outputs};
\draw [thick, -] (0,25) node [mynode1,above] {Sedans} |- (27,0) node [mynode1,right] {SUVs};
\end{TikzFigure}

\newhtmlpage

Consider the example presented in Figure~\ref{fig:intraindustry}. Consider the
hypothetical company \textit{Hunda Motor Corporation} currently 
has a large assembly plant in \textit{each of}
Canada and the US. Restrictions on trade in automobiles between the two
countries make it too costly to ship models across the border. Hence Hunda
produces both sedans and SUVs in each plant. But for several reasons, 
\textit{switching between models is costly and results in reduced output}.
Hunda can produce 40,000 vehicles of each type per annum in its plants, but
could produce 100,000 of a single model in each plant, using the same amount
of capital and labour. This is a situation of increasing returns to scale,
and in this instance these scale economies are what determine the trade
outcome rather than any innate comparative advantage between the economies.
If trade barriers against the shipment of autos across national boundaries
can be eliminated, then Hunda can take advantage of scale economies and
increase its total production without using more capital and labour. 

As this example implies, an opening up of trade increases the potential
market size (in addition to more competition), and producers who experience
increasing returns to scale stand to benefit from an enlarged market because
their potential unit costs fall.

\newhtmlpage

\subsection*{Dynamic gains from trade}

The term \terminology{dynamic gains} denotes the potential for domestic producers
to increase productivity as a result of competing with, and learning from,
foreign producers.

\begin{DefBox}
	\textbf{Dynamic gains}: the potential for domestic producers to increase productivity by competing with, and learning from, foreign producers.
\end{DefBox}

Production processes in reality are seldom static. Innovation is constant in
the modern world, and innovation is manifested in the form of productivity
improvements. An economy's production possibility frontier is determined by
its endowments of capital and labour and also the efficiency with which it
uses those productive factors. \terminology{Total factor productivity} defines
how efficiently the factors of production are combined. Research suggests
that in developed economies this productivity increases by about 1\% per
annum. This means that more output can be produced using the same amounts of
capital and labour because production is being carried out more efficiently.
In graphical terms, such productivity improvements effectively push out an
economy's production possibility frontier by 1\% per annum. For economies
in the process of development, this productivity growth may be as high as 
3\% or 4\% per annum -- for the reason that these economies can observe
and learn from economies that are ahead of it technologically.

Freer trade forces domestic firms to compete with foreign firms that may be
more productive. Domestic firms that can learn and adapt to competition by
becoming more efficient will survive, firms that cannot adapt will not.
Inevitably, there will be winners and losers in the production sector of the
economy, whereas in the consumption sector most consumers should be winners.

\begin{DefBox}
	\textbf{Total factor productivity}: how efficiently the factors of production are combined.
\end{DefBox}