\section{The politics of protection}\label{sec:ch15sec6}

Objections to imports are frequent and come from many different sectors of
the economy. In the face of the gains from trade which we have illustrated
in this chapter, why do we observe such strong opposition to imported goods
and services?

\subsection*{Structural change and technology}

In a nutshell the answer is that, while consumers in the aggregate gain from
the reduction of trade barriers, and there is a net gain to the economy at
large, some \textit{individual sectors of the economy lose out}. Not
surprisingly the sectors that will be adversely affected are vociferous in
lodging their objections. Sectors of the economy that cannot compete with
overseas suppliers generally see a reduction in jobs. This has been the case
in the manufacturing sector of the Canadian and US economies in recent
decades, as manufacturing and assembly has flown off-shore to Asia and
Mexico where labour costs are lower. Domestic job losses are painful, and
frequently workers who have spent decades in a particular job find
reemployment difficult, and rarely get as high a wage as in their displaced
job.

Such job losses are reflected in calls for tariffs on imports from China,
for example, in order to `level the playing field' -- that is, to counter
the impact of lower wages in China. Of course it is precisely because of
lower labour costs in China that the Canadian consumer benefits.

In Canada we deal with such dislocation first by providing unemployment
payments to workers, and by furnishing retraining allowances, both coming
from Canada's Employment Insurance program. While such support does not
guarantee an equally good alternative job, structural changes in the
economy, due to both internal and external developments, must be confronted.
For example, the information technology revolution made tens of thousands of
`data entry' workers redundant. Should producers have shunned the
technological developments which increased their productivity dramatically?
If they did, would they be able to compete in world markets?

While job losses feature heavily in protests against technological
development and freer trade, most modern economies continue to grow and
create more jobs in the service sector than are lost in the manufacturing
sector. Developed economies now have many more workers in service than
manufacture. Service jobs are not just composed of low-wage jobs in fast
food establishments -- `Mcjobs', they are high paying jobs in the health,
education, legal, financial and communications sectors of the economy.

\newhtmlpage

\subsection*{Successful lobbying and concentration}

While efforts to protect manufacture have not resulted in significant
barriers to imports of manufactures, objections in some specific sectors of
the economy seem to be effective worldwide. One sector that stands out is
agriculture. Not only does Canada have very steep barriers to imports of
dairy products such as milk and cheese, the US and the European Union have
policies that limit imports and subsidize their domestic producers. Europe
has had its `wine lakes' and `butter mountains' due to excessive government
intervention in these sectors; the US has subsidies to sugar producers and
Canada has a system of quotas on the domestic supply of dairy products and
corresponding limits on imports.

Evidently, in the case of agriculture, political conditions are conducive to
the continuance of protection and what is called `supply management' --
domestic production quotas. The reason for `successful' supply limitation
appears to rest in the geographic concentration of potential beneficiaries
of such protection and the scattered beneficiaries of freer trade on the one
hand, and the costs and benefits of political organization on the other:
Farmers tend to be concentrated in a limited number of rural electoral
ridings and hence they can collectively have a major impact on electoral
outcomes. Second, the benefits that accrue to trade restriction are heavily
concentrated in the economy -- keep in mind that about three percent of the
population lives on farms, or relies on farming for its income. By contrast
the costs on a per person scale are small, and are spread over the whole
population. Thus, in terms of the costs of political organization, the
incentives for consumers are small, but the incentives for producers are
high.

In addition to the differing patterns of costs and benefits, rural
communities tend to be more successful in pushing trade restrictions based
on a `way-of-life' argument. By permitting imports that might displace local
supply, lobbyists are frequently successful in convincing politicians that
long-standing way-of-life traditions would be endangered, even if such
`traditions' are accompanied by monopolies and exceptionally high tariffs.

\newhtmlpage

\subsection*{Valid trade barriers: Infant industries and dumping?}

An argument that carries both intellectual and emotional appeal to voters is
the `infant industry' argument. The argument goes as follows: New ventures
and sectors of the economy may require time before that can compete
internationally. Scale economies may be involved, for example, and time may
be required for producers to expand their scale of operation, at which time
costs will have fallen to international (i.e. competitive) levels. In
addition, learning-by-doing may be critical in more high-tech sectors and,
once again, with the passage of time costs should decline for this reason
also.

The problem with this stance is that these `infants' have insufficient
incentive to `grow up' and become competitive. A protection measure that is
initially intended to be temporary can become permanent because of the
potential job losses associated with a cessation of the protection to an
industry that fails to become internationally competitive. Furthermore,
employees and managers in protected sectors have insufficient incentive to
make their production competitive if they realize that their government will
always be there to protect them.

In contrast to the infant industry argument, economists are more favourable
to restrictions that are aimed at preventing `dumping'. \terminology{Dumping}
is a predatory practice, based on artificial costs aimed at driving out
domestic producers.

\begin{DefBox}
	\textbf{Dumping} is a predatory practice, based on artificial costs aimed at driving out domestic producers.
\end{DefBox}

Dumping may occur either because foreign suppliers choose to sell at
artificially low prices (prices below their marginal cost for example), or
because of surpluses in foreign markets resulting from oversupply. For
example, if, as a result of price support in its own market, a foreign
government induced oversupply in butter and it chose to sell such butter on
world markets at a price well below the going (`competitive') world supply
price, such a sale would constitute dumping. Alternatively, an established
foreign supplier might choose to enter our domestic market by selling its
products at artificially low prices, with a view to driving domestic
competition out of the domestic market. Having driven out the domestic
competition it would then be in a position to raise prices. This is
predatory pricing as explored in the last chapter. Such behaviour differs
from a permanently lower price on the part of foreign suppliers. This latter
may be welcomed as a gain from trade, whereas the former may generate no
gains and serve only to displace domestic labour and capital.