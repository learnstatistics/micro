\section{Performance-based measures of structure -- market power}\label{sec:ch11sec2}

Sectors of the economy do not fit neatly into the limited number of categories
described above. The best we can say in most cases is that they resemble
more closely one type of market than another. Consider the example of
Canada's brewing sector: It has two large brewers in \textit{Molson-Coors}
and \textit{Labatt}, a couple of intermediate sized firms such as 
\textit{Sleeman}, and an uncountable number of small boutique brew pubs. While such
a large number of brewers satisfy one requirement for perfect competition,
it would not be true to say that the biggest brewers wield no market power;
and this is the most critical element in defining market structure.

By the same token, we could not define this market as a duopoly: Even though
there are just two major participants, there are countless others who,
together, are important.

One way of defining what a particular structure most closely resembles is to
examine the percentage of sales in the market that is attributable to a small number
of firms. For example: What share is attributable to the largest three or
four firms? The larger the share, the more concentrated the market
power. Such a statistic is called a concentration ratio. The %
\terminology{$N$-firm concentration ratio} is the sales share of the largest 
$N$ firms in that sector of the economy.

\begin{DefBox}
	The \textbf{$N$-firm concentration ratio} is the sales share of the largest $N$ firms in that sector of the economy.
\end{DefBox}

\begin{Table}{caption={Concentration in Canadian food processing 2011 \label{table:canfoodpro}},description={\textit{Source}: ``Four Firm Concentration Ratios (CR4s) for selected food processing sectors,'' adapted from Statistics Canada publication Measuring industry concentration in Canada's food processing sectors, Agriculture and Rural Working Paper series no. 70, Catalogue 21-601, \url{http://www.statcan.gc.ca/pub/21-601-m/21-601-m2004070-eng.pdf}.},descwidth={30em}}
	\begin{tabu} to 30em {|X[1,c]X[1,c]|} \hline 
		\rowcolor{rowcolour}	\textbf{Sector}		& \textbf{\% of shipments}	\\	\hline
		Sugar				& 98						\\
		\rowcolor{rowcolour}	Breakfast cereal	& 96						\\ 
		Canning				& 60						\\
		\rowcolor{rowcolour}	Meat processing		& 23						\\ \hline 
	\end{tabu}
\end{Table}

Table~\ref{table:canfoodpro} contains information on the 4-firm
concentration ratio for several sectors of the Canadian economy. It
indicates that, at one extreme, sectors such as breakfast cereals and sugars
have a high degree of concentration, whereas meat processing has much less.
A high degree of concentration suggests market power, and possibly economies
of scale.