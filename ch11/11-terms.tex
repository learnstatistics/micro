\newpage
	\section*{Key Terms}
\begin{keyterms}
\textbf{Imperfectly competitive firms} face a downward-sloping demand curve, and their output price reflects the quantity sold.

\textbf{Oligopoly} defines an industry with a small number of suppliers.

\textbf{Monopolistic competition} defines a market with many sellers of products that have similar characteristics. Monopolistically competitive firms can exert only a small influence on the whole market.

\textbf{Duopoly} defines a market or sector with just two firms.

\textbf{Concentration ratio}: $N$-firm concentration ratio is the sales share of the largest $N$ firms in that sector of the economy.

\textbf{Differentiated product} is one that differs slightly from other products in the same market.

The \textbf{monopolistically competitive equilibrium} in the long run requires the firm's demand curve to be tangent to the $ATC$ curve at the output where $MR=MC$.

\textbf{Collusion} is an explicit or implicit agreement to avoid competition with a view to increasing profit.

\textbf{Conjecture}: a belief that one firm forms about the strategic reaction of another competing firm.

\textbf{Game}: a situation in which contestants plan strategically to maximize their profits, taking account of rivals' behaviour.

\textbf{Strategy}: a game plan describing how a player acts, or moves, in each possible situation.

\textbf{Nash equilibrium}: one in which each player chooses the best strategy, given the strategies chosen by the other player, and there is no incentive for any player to move.

\textbf{Dominant strategy}: a player's best strategy, whatever the strategies adopted by rivals.

\textbf{Payoff matrix}: defines the rewards to each player resulting from particular choices.

\textbf{Credible threat}: one that, after the fact, is still optimal to implement.

\textbf{Cournot behaviour} involves each firm reacting optimally in their choice of output to their competitors' decisions.

\textbf{Reaction functions} define the optimal choice of output conditional upon a rival's output choice.
\end{keyterms}