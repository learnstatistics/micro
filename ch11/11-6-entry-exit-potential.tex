\section{Entry, exit \& potential competition}\label{sec:ch11sec6}

At this point we inquire about the potential entry and impact of new firms
-- firms who might enter the industry if conditions were sufficiently
enticing, meaning the presence of economic profits. One way of examining
entry in this oligopolistic world is to envisage potential entry barriers as
being either intended or unintended, though the difference between the two
can be blurred. Broadly, an unintended or `natural' barrier is one related
to cost conditions and the size of the market. An intended barrier involves
a strategic decision on the part of the firm to prevent entry.

\subsection*{Unintended entry barriers}

Oligopolists tend to have substantial fixed costs, accompanied by declining
average costs up to very high output levels. Such a cost structure
`naturally' gives rise to a supply side with a small number of suppliers.
For examples, given demand and cost structures, could Vancouver support two
professional soccer teams; could Calgary support two professional hockey
teams; could Montreal sustain two professional football teams? The answer to
each of these questions is likely `no'. Because given the cost structure of
these markets, it would not be possible to induce twice as many spectators
without reducing the price per game ticket to such a degree that revenue
would be insufficient to cover costs. (We will neglect for the moment that
the governing bodies of these sports also have the power to limit entry.)
Fixed costs include stadium costs, staff payrolls and player payrolls. In
fact most costs in these markets are relatively fixed. Market size relative
to fixed and variable costs is not large enough to sustain two teams in most
cities. Exceptions in reality are huge urban areas such as New York and Los
Angeles.

Accordingly, it is possible that the existing team, or teams, may earn
economic profit from their present operation; but such profit does not
entice further entry, because the market structure is such that the entry of
an additional team could lead to all teams making losses.

\newhtmlpage

\subsection*{Intended entry barriers}

\textit{Patent law} is one form of protection for incumbent firms. Research
and development is required for the development of many products in the
modern era. Pharmaceuticals are an example. If innovations were not
protected, firms and individuals would not be incentivized to devote their
energies and resources to developing new drugs. Society would be poorer as a
result. Patent protection is obviously a legal form of protection.

\textit{Advertizing} is a second form of entry deterrence. In this instance
firms attempt to market their product as being distinctive 
and even enviable. For example, \textit{Coca-Cola}
and \textit{PepsiCo} invest hundreds of millions annually to project their products
in this light. They sponsor sports, artistic and cultural events. Entry
into the cola business is not impossible, but brand image is so strong for
these firms that potential competitors would have a very low probability of
entering this sector profitably. Likewise, in the `energy-drinks' market,
\textit{Red Bull} spends hundreds of millions of dollars per annum to project its
brand as being just as unique and desirable as Pepsi or Coca-Cola.

\textit{Predatory pricing} is an illegal form of entry deterrence. It
involves an incumbent charging an artificially low price for its product in
the event of entry of a new competitor. This is done with a view to making
it impossible for the entrant to earn a profit. Given that incumbents have
generally greater resources than entrants, they can survive a battle of
losses for a more prolonged period, thus ultimately driving out the entrant.

\textit{Network externalities} arise when the existing number of buyers
itself influences the total demand for a product. \textit{Facebook} is now a
classic example. It has many more members than \textit{MySpace} or \textit{Google+}, and hence
finds it easier to attract new users. An individual contemplating joining a
social network has an incentive to join one where she has many existing
`friends'.

\textit{Transition costs} can be erected by firms who do not wish to lose
their customer base. Cell-phone plans are a good example.
Contract-termination costs are one obstacle to moving to a new supplier.
Some carriers grant special low rates to users communicating with other
users within the same network, or offer special rates for a block of users
(perhaps within a family).

\newhtmlpage

An \textit{over-investment} strategy means that an existing supplier generates
additional production capacity through investment in new plant or capital.
This is costly to the incumbent and is intended as a signal to any potential
entrant that this capacity could be brought on-line immediately should a potential
competitor contemplate entry. For example, a ski-resort owner may invest in
a new chair-lift, even if she does not use it. The existence of the
additional capacity may scare potential entrants. A key component of this
strategy is that the incumbent firm  invests ahead of time -- and inflicts a
cost on itself. The incumbent does not simply say ``I will
build another chair-lift if you decide to develop a nearby mountain into a
ski hill.'' That policy does not carry the same degree of
credibility as actually incurring the cost of construction ahead of time.
However, such a strategy may not always be feasible: It might be just too costly to
pre-empt entry by putting spare capacity in place. Spare capacity is not
so different from brand development through advertising; both are types of
sunk cost. The threats associated with the incumbent's behaviour become a 
\terminology{credible threat} because the incumbent incurs costs up front.

\begin{DefBox}
\textbf{A credible threat} is one
that is effective in deterring specific behaviours; a competitor must
believe that the threat will be implemented if the competitor behaves in a
certain way.
\end{DefBox}

\section*{Conclusion}

Monopoly and perfect competition are interesting paradigms; but few markets
resemble them in the real world. In this chapter we addressed some of the
complexities that define the economy we inhabit: It is characterized by
strategic planning, entry deterrence, differentiated products and so forth.

Entry and exit are critical to competitive markets. Frequently entry is
blocked because of scale economies -- an example of a natural or unintended
entry barrier. In addition, incumbents can formulate strategies to limit
entry, some involving credible threats.

Firms act strategically -- particularly when there are just a few
participants in the market. Before acting, firms make conjectures about how
their competitors will react, and incorporate such reactions into their own
planning. Competition between suppliers can frequently be analyzed in terms
of a game, and such games usually have an equilibrium outcome. The Cournot
duopoly model that we developed is a game between two competitors in which
an equilibrium market output is determined from a pair of reaction functions.

Scale economies are critical. Large development costs or setup costs mean
that the market can generally support just a limited number of producers.
In turn this implies that potential new (small-scale) firms cannot benefit
from the scale economies and will not survive competition from large-scale
suppliers.

Finally, product differentiation is critical. If small differences exist
between products produced in markets where there is free entry we get a
monopolistically competitive structure. In these markets long-run profits
are `normal' and firms operate with some excess capacity. It is not possible
to act strategically in this kind of market.