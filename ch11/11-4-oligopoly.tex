\section{Oligopoly and games}\label{sec:ch11sec4}

Under perfect competition or monopolistic competition, there are so many
firms in the industry that each one can ignore the immediate effect of its
own actions on particular rivals. However, in an oligopolistic industry 
\textit{each firm must consider how its actions affect the decisions of its
	relatively few competitors}. Each firm must guess how its rivals will react.
Before discussing what constitutes an intelligent guess, we investigate
whether they are likely to collude or compete. \terminology{Collusion} is a
means of reducing competition with a view to increasing profit.

\begin{DefBox}
	\textbf{Collusion} is an explicit or implicit agreement to avoid competition with a view to increasing profit.
\end{DefBox}

A particular form of collusion occurs when firms co-operate to form a
cartel, as we saw in the last chapter. Collusion is more difficult if there
are many firms in the industry, if the product is not standardized, or if
demand and cost conditions are changing rapidly. In the absence of
collusion, each firm's demand curve depends upon how competitors react: If
Air Canada contemplates offering customers a seat sale on a particular
route, how will West Jet react? Will it, too, make the same offer to buyers?
If Air Canada thinks about West Jet's likely reaction, will it go ahead with
the contemplated promotion? A \terminology{conjecture} is a belief that one
firm forms about the strategic reaction of another competing firm.

\begin{DefBox}
	A \textbf{conjecture} is a belief that one firm forms about the strategic reaction of another competing firm.
\end{DefBox}

Good poker players will attempt to anticipate their opponents' moves or
reactions. Oligopolists are like poker players, in that they try to
anticipate their rivals' moves. To study interdependent decision making, we
use game theory. A \terminology{game} is a situation in which contestants
plan strategically to maximize their profits, taking account of rivals'
behaviour.

\begin{DefBox}
	A \textbf{game} is a situation in which contestants plan strategically to maximize their profits, taking account of rivals' behaviour.
\end{DefBox}

\newhtmlpage

The \textit{players} in the game try to maximize their own \textit{payoffs}.
In an oligopoly, the firms are the players and their payoffs are their
profits. Each player must choose a \terminology{strategy}, which is a plan
describing how a player moves or acts in different situations.

\begin{DefBox}
	A \textbf{strategy} is a game plan describing how a player acts, or moves, in each possible situation.
\end{DefBox}

How do we arrive at an equilibrium in these games? Let us begin by defining
a commonly used concept of equilibrium. A \terminology{Nash equilibrium} is
one in which each player chooses the best strategy, given the strategies
chosen by the other players and there is no incentive to move or change
choice.

\begin{DefBox}
	A \textbf{Nash equilibrium} is one in which each player chooses the best strategy, given the strategies chosen by the other player, and there is no incentive for any player to move.
\end{DefBox}

In such an equilibrium, no player wants to change strategy, since the other
players' strategies were already figured into determining each player's own
best strategy. This concept and theory are attributable to the Princeton
mathematician John Nash, who was popularized by the Hollywood movie version
of his life \textit{A Beautiful Mind}.

In most games, each player's best strategy depends on the strategies chosen
by their opponents. Sometimes, though not always, a player's best strategy
is independent of those chosen by rivals. Such a strategy is called a %
\terminology{dominant strategy}.

\begin{DefBox}
	A \textbf{dominant strategy} is a player's best strategy, independent of the strategies adopted by rivals.
\end{DefBox}

\newhtmlpage

We now illustrate these concepts with the help of two different games. These
games differ in their outcomes and strategies. Table~\ref{table:gamedomstrat}
contains the domestic happiness game\footnote{%
	This presentation is inspired by a similar problem in Ted Bergstrom and Hal
	Varian's book \textit{``Workouts''}.}. Will
and Kate are attempting to live in harmony, and their happiness depends upon
each of them carrying out domestic chores such as shopping, cleaning and
cooking. The first element in each pair defines Will's outcome, the second
Kate's outcome. If both contribute to domestic life they each receive a
happiness or utility level of 5 units. If one contributes and the other does
not the happiness levels are 2 for the contributor and 6 for the
non-contributor, or `free-rider'. If neither contributes happiness levels
are 3 each. When each follows the same strategy the payoffs are on the
diagonal, when they follow different strategies the payoffs are on the
off-diagonal. Since the elements of the table define the payoffs resulting
from various choices, this type of matrix is called a \terminology{payoff
	matrix}.

\begin{DefBox}
	\textbf{A payoff matrix} defines the rewards to each player resulting from particular choices.
\end{DefBox}

So how is the game likely to unfold? In response to Will's choice of a
contribute strategy, Kate's utility maximizing choice involves lazing: She
gets 6 units by not contributing as opposed to 5 by contributing. Instead,
if Will decides to be lazy what is in Kate's best interest? Clearly it is to
be lazy also because that strategy yields 3 units of happiness compared to 2
units if she contributes. In sum, Kate's best strategy is to be lazy,
regardless of Will's behaviour. So the strategy of not contributing is a 
\textit{dominant strategy}.

Will also has a dominant strategy -- identical to Kate's. This is not
surprising since the payoffs are symmetric in the table. Hence, since each
has a dominant strategy of not contributing the Nash equilibrium is in the
bottom right cell, where each receives a payoff of 3 units. Interestingly
this equilibrium is not the one that yields maximum combined happiness.

\begin{Table}{caption={A game with dominant strategies \label{table:gamedomstrat}},description={The first element in each cell denotes the payoff or utility to Will; the second element the utility to Kate.},descwidth={30em}}
\begin{tabu} to 35em {X[1,c]X[1,c]|X[1,c]X[1,c]|}	\hhline{~~--}
	&	& \multicolumn{2}{c|}{\cellcolor{rowcolour}\textbf{Kate's choice}} \\ 
	&	& Contribute & Laze \\ \hline 
	\multicolumn{1}{|c}{\cellcolor{rowcolour}} & Contribute & \cellcolor{rowcolour}5,5 & \cellcolor{rowcolour}2,6 \\[-0.1em]
	\multicolumn{1}{|c}{\cellcolor{rowcolour}\multirow{-2}{7em}{\textbf{Will's choice}}} & Laze & \cellcolor{rowcolour}6,2 & \cellcolor{rowcolour}3,3 \\ \hline 
\end{tabu}
\end{Table}

\newhtmlpage

The reason that the equilibrium yields less utility for each player in this
game is that the game is competitive: Each player tends to their own
interest and seeks the best outcome conditional on the choice of the other
player. This is evident from the $(5,5)$ combination. From this position
Kate would do better to defect to the Laze strategy, because her utility
would increase\footnote{This game is sometimes called the ``prisoners'
	dilemma'' game because it can be constructed to reflect a
	scenario in which two prisoners, under isolated questioning, each confess to
	a crime and each ends up worse off than if neither confessed.}.

To summarize: This game has a unique equilibrium and each player has a
dominant strategy. But let us change the payoffs just slightly to the values
in Table~\ref{table:gamenodomstrat}. The off-diagonal elements have changed.
The contributor now gets no utility as a result of his or her contributions:
Even though the household is a better place, he or she may be so annoyed
with the other person that no utility flows to the contributor.

\begin{Table}{caption={A game without dominant strategies \label{table:gamenodomstrat}},description={The first element in each cell denotes the payoff or utility to Will; the second element the utility to Kate.},descwidth={30em}}
	\begin{tabu} to 35em {X[1,c]X[1,c]|X[1,c]X[1,c]|}	\hhline{~~--}
		&	& \multicolumn{2}{c|}{\cellcolor{rowcolour}\textbf{Kate's choice}} \\ 
		&	& Contribute & Laze \\ \hline 
		\multicolumn{1}{|c}{\cellcolor{rowcolour}} & Contribute & \cellcolor{rowcolour}5,5 & \cellcolor{rowcolour}0,4 \\[-0.1em]
		\multicolumn{1}{|c}{\cellcolor{rowcolour}\multirow{-2}{7em}{\textbf{Will's choice}}} & Laze & \cellcolor{rowcolour}4,0 & \cellcolor{rowcolour}3,3 \\ \hline 
	\end{tabu}
\end{Table}

What are the optimal choices here? Starting again from Will choosing to
contribute, what is Kate's best strategy? It is to contribute: She gets 5
units from contributing and 4 from lazing, hence she is better contributing.
But what is her best strategy if Will decides to laze? It is to laze,
because that yields her 3 units as opposed to 0 by contributing. This set of
payoffs therefore \textit{contains no dominant strategy} for either player.

As a result of there being no dominant strategy, there arises the
possibility of more than one equilibrium outcome. In fact there are two
equilibria in this game now: If the players find themselves both
contributing and obtaining a utility level of $(5,5)$ it would not be
sensible for either one to defect to a laze option. For example, if Kate
decided to laze she would obtain a payoff of 4 utils rather than the 5 she
enjoys at the $(5,5)$ equilibrium. By the same reasoning, if they find
themselves at the (laze, laze) combination there is no incentive to move to
a contribute strategy.

Once again, it is to be emphasized that the twin equilibria emerge in a
competitive environment. If this game involved cooperation or collusion the
players should be able to reach the $(5,5)$ equilibrium rather than the 
$(3,3)$ equilibrium. But in the competitive environment we cannot say 
\textit{ex ante} which equilibrium will be attained.

\newhtmlpage

This game illustrates the tension between collusion and competition. While
we have developed the game in the context of the household, it can equally
be interpreted in the context of a profit maximizing game between two market
competitors. Suppose the numbers define profit levels rather than utility as
in Table~\ref{table:collusionposs}. The `contribute' option can be
interpreted as `cooperate' or `collude', as we described for a cartel in the
previous chapter. They collude by agreeing to restrict output, sell that
restricted output at a higher price, and in turn make a greater total profit
which they split between themselves. The combined best profit outcome 
$(5,5)$ arises when each firm restricts its output.

\begin{Table}{caption={Collusion possibilities \label{table:collusionposs}},description={The first element in each cell denotes the profit to Firm W; the second element the profit to Firm K.},descwidth={30em}}
	\begin{tabu} to 35em {X[1.5,c]X[1,c]|X[1,c]X[1,c]|}	\hhline{~~--}
		&	& \multicolumn{2}{c|}{\cellcolor{rowcolour}\textbf{Firm K's profit}} \\ 
		&	& Low output & High output \\ \hline 
		\multicolumn{1}{|c}{\cellcolor{rowcolour}} & Low output & \cellcolor{rowcolour}5,5 & \cellcolor{rowcolour}2,6 \\[-0.1em]
		\multicolumn{1}{|c}{\cellcolor{rowcolour}\multirow{-2}{10em}{\textbf{Firm W's profit}}} & High output & \cellcolor{rowcolour}6,2 & \cellcolor{rowcolour}3,3 \\ \hline 
	\end{tabu}
\end{Table}

But again there arises an incentive to defect: If Firm W agrees to maintain
a high price and restrict output, then Firm K has an incentive to renege and
increase output, hoping to improve its profit through the willingness of
Firm W to restrict output. Since the game is symmetric, each firm has an
incentive to renege. Each firm has a dominant strategy -- high output, and
there is a unique equilibrium $(3,3)$.

Obviously there arises the question of whether these firms can find an
operating mechanism that would ensure they each generate a profit of 5 units
rather than 3 units, while remaining purely self-interested. This question
brings us to the realm of games that are repeated many times. For example,
suppose that firms make strategic choices each quarter of the year. If firm
K had `cheated' on the collusive strategy it had agreed with firm W in the
previous quarter, what would happen in the following quarter? Would firms
devise a strategy so that cheating would not be in the interest of either
one, or would the competitive game just disintegrate into an unpredictable
pattern? These are interesting questions and have provoked a great deal of
thought among game theorists. But they are beyond our scope at the present
time. 

Instead of pursuing games that are repeated many times between the
competitors, we examine what might happen in one-shot games of the type we
have been examining, but in the context of many possible choices. In
particular, instead of assuming that each firm can choose a high or low
output, how would the outcome of the game be determined if each firm can
choose an output that can lie anywhere between a high and low output? In
terms of the demand curve for the market, this means that the firms can
choose some output and price that is consistent with demand conditions:
There may be an infinite number of choices.