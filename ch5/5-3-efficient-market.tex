\section{Efficient market outcomes}\label{sec:ch5sec3}

The definition and measurement of the surplus is straightforward
provided the supply and demand functions are known. An important
characteristic of the marketplace is that in certain circumstances it
produces what we call an efficient outcome, or an \terminology{efficient
market}. Such an outcome yields the highest possible sum of surpluses. 

\begin{DefBox}
An \textbf{efficient market} maximizes the sum of producer and consumer surpluses.
\end{DefBox}

To see that this outcome achieves the goal of maximizing the total surplus,
consider what would happen if the quantity $Q=48$ in the taxi example were not supplied. Suppose
that the city's taxi czar decreed that 50 units should be supplied, and
the czar forced additional drivers on the road. If 2 additional units are
to be traded in the market, consider the value of this at the margin.
Suppliers value the supply more highly than the buyers are willing to pay.
So on these additional 2 units negative surplus would accrue, thus
reducing the total.

Second, potential buyers who would like a cheaper ride and drivers who would
like a higher hourly payment do not get to participate in the market. On the
demand side those individuals can take public transit, and on the supply
side the those drivers can allocate their time to alternative activities.
Obviously, only those who participate in the market benefit from a surplus.

\newhtmlpage

One final characteristic of surplus measurement should be emphasized. That
is, the surplus number is not unique, it depends upon the economic
environment. We can illustrate this easily using the taxi example. A well
recognized feature of \textit{Uber} taxi rides is that the price varies with road and
weather conditions. Poor weather conditions mean that there is an increased
demand, and poor road or weather conditions mean that drivers are less
willing to supply their services -- their reservation payment increases. This
situation is illustrated in Figure~\ref{fig:taximarketB}. 

% Figure 5.3	fig:taximarketB
\begin{TikzFigure}{xscale=0.45,yscale=0.45,descwidth=25em,caption={The taxi market \label{fig:taximarketB}},description={The curves represented by $D'$ and $S'$ represent the curves for bad weather: Taxi rides are more highly valued on the demand side, and drivers must be paid more to supply in less favourable work conditions.}}
% axes
\draw [thick] (0,15) node (yaxis) [above,mynode1] {Price (fee\\per hour)} |- (15,0) node (xaxis) [right,mynode1] {Quantity (thousands\\of hours of rides)};
% demand line
\draw [demandcolour,ultra thick,name path=D] (0,9) node [mynode,left,black] {90} -- node [mynode,above right,black,pos=0.25] {$D$} (9,0);
\draw [demandcolour,ultra thick,name path=Dprime] (0,12) -- node [mynode,above right,black,pos=0.25] {$D'$} (12,0);
% supply line
\draw [supplycolour,ultra thick,name path=S] (0,3) node [mynode,left,black] {30} -- (12,6) node [mynode,right,black] {$S$};
\draw [supplycolour,ultra thick,name path=Sprime] (0,3) -- (12,9) node [mynode,right,black] {$S'$};
% intersection of S and D, and Sprime and Dprime
\draw [name intersections={of=D and S, by=E}, name intersections={of=Dprime and Sprime, by=Eprime}]
	[thick,dotted] (yaxis |- E) node [mynode,left] {42} -- (E) node [mynode,above] {E} -- (xaxis -| E) node [mynode,below] {48}
	[thick,dotted] (yaxis |- Eprime) node [mynode,left] {60} -- (Eprime) node [mynode,above] {$\text{E}'$} -- (xaxis -| Eprime) node [mynode,below] {60};
\end{TikzFigure}

\newpage
The demand curve has shifted
upwards and the supply curve has also changed in such a way that any
quantity will now be supplied at a higher price. The new equilibrium is
given by $E^{\prime}$ rather than $E$.\footnote{%
For example, if the demand curve shifts upwards, parallel, to become $%
P=120-1Q$ and the supply curve changes in slope to $P=30+(1/2)Q$, the new
equilibrium solution is $\{P=\$60,Q=60\}$.} There is a new equilibrium
price-quantity combination that \textit{is efficient in the new market
conditions}. This illustrates that there is no such thing as a unique
unchanging efficient outcome. When economic factors that influence the
buyers' valuations (demand) or the suppliers' reservation prices (supply)
change, then the efficient market outcome must be recomputed.
