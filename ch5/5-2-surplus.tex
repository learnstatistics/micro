\section{Consumer and producer surplus}\label{sec:ch5sec2}

An understanding of economic efficiency is greatly facilitated as a result
of understanding two related measures: Consumer surplus and producer
surplus. Consumer surplus relates to the demand side of the market, producer
surplus to the supply side. Producer surplus is also termed supplier
surplus. These measures can be understood with the help of a standard
example, the market for city apartments.

\subsection*{The market for apartments}

Table~\ref{table:consupsurplus} and Figure~\ref{fig:apartmentmarket}
describe the hypothetical data. We imagine first a series of city-based
students who are in the market for a standardized downtown apartment. These
individuals are not identical; they value the apartment differently. For
example, Alex enjoys comfort and therefore places a higher value on a unit
than Brian. Brian, in turn, values it more highly than Cathy or Don. Evan
and Frank would prefer to spend their money on entertainment, and so on.
These valuations are represented in the middle column of the demand panel in
Table~\ref{table:consupsurplus}, and also in Figure~\ref{fig:apartmentmarket}
with the highest valuations closest to the origin. The valuations reflect
the willingness to pay of each consumer.

% Table 5.1
\begin{Table}{caption={Consumer and supplier surpluses \label{table:consupsurplus}}}
\begin{tabu} to \linewidth {|X[1,c]X[1,c]X[1,c]|} \hline 
\multicolumn{3}{|c|}{\cellcolor{rowcolour}\textbf{Demand}} \\
\textbf{Individual} & \textbf{Demand valuation} & \textbf{Surplus} \\ \hline 
\rowcolor{rowcolour}	Alex & 900 & 400 \\
						Brian & 800 & 300 \\ 
\rowcolor{rowcolour}	Cathy & 700 & 200 \\ 
						Don & 600 & 100 \\
\rowcolor{rowcolour}	Evan & 500 & 0 \\ 
						Frank & 400 & 0 \\ \hline 
\multicolumn{3}{c}{}	\\	\hline
\multicolumn{3}{|c|}{\cellcolor{rowcolour}\textbf{Supply}} \\
\textbf{Individual} & \textbf{Reservation value} & \textbf{Surplus} \\ \hline 
\rowcolor{rowcolour}	Gladys & 300 & 200 \\
						Heward & 350 & 150 \\
\rowcolor{rowcolour}	Ian & 400 & 100 \\ 
						Jeff & 450 & 50 \\
\rowcolor{rowcolour}	Kirin & 500 & 0 \\
						Lynn & 550 & 0 \\ \hline 
\end{tabu}
\end{Table}


On the supply side we imagine the market as being made up of different
individuals or owners, who are willing to put their apartments on the market
for different prices. Gladys will accept less rent than Heward, who in turn
will accept less than Ian. The minimum prices that the suppliers are willing
to accept are called \textit{reservation} prices or values, and these are
given in the lower part of Table~\ref{table:consupsurplus}. Unless the
market price is greater than their reservation price, suppliers will hold
back.

By definition, as stated in Chapter~\ref{chap:classical}, the demand curve
is made up of the valuations placed on the good by the various demanders.
Likewise, the reservation values of the suppliers form the supply curve. If
Alex is willing to pay \$900, then that is his demand price; if Heward is
willing to put his apartment on the market for \$350, he is by definition
willing to supply it for that price. Figure~\ref{fig:apartmentmarket}
therefore describes the demand and supply curves in this market. The steps
reflect the willingness to pay of the buyers and the reservation valuations
or prices of the suppliers.

\newhtmlpage

% Figure 5.1	fig:apartmentmarket
\begin{TikzFigure}{xscale=0.6,yscale=0.6,descwidth=25em,caption={The apartment market \label{fig:apartmentmarket}},description={Demanders and suppliers are ranked in order of the value they place on an apartment. The market equilibrium is where the marginal demand value of Evan equals the marginal supply value of Kirin at \$500. Five apartments are rented in equilibrium.}}
\draw [demandcolour,ultra thick,-]
	(0,9) node [black,mynode,left] {\$900} -- (2,9) node [black,mynode,midway,above] {Alex}
	(2,8) -- (4,8) node [black,mynode,midway,above] {Brian}
	(4,7) -- (6,7) node [black,mynode,midway,above] {Cathy}
	(6,6) -- (8,6) node [black,mynode,midway,above] {Don}
	(10,4) -- (12,4) node [black,mynode,midway,above] {Frank};
\draw [supplycolour,ultra thick,-]
	(0,3) node [black,mynode,left] {\$300} -- (2,3) node [black,mynode,midway,below] {Gladys}
	(2,3.5) -- (4,3.5) node [black,mynode,midway,below] {Heward}
	(4,4) -- (6,4) node [black,mynode,midway,below] {Ian}
	(6,4.5) -- (8,4.5) node [black,mynode,midway,below] {Jeff}
	(10,5.5) -- (12,5.5) node [black,mynode,midway,above] {Lynn};
\draw [ultra thick] (8,5) coordinate (Eq) -- (10,5) node [mynode,midway,above] {Evan} node [mynode,midway,below] {Kirin};
% axes
\draw [thick, -] (0,10) node (yaxis) [above] {Rent} |- (15,0) node (xaxis) [right] {Quantity};
% dotted line to equilibrium point
\draw [dotted,thick] (yaxis |- Eq) -- +(15,0) node [mynode,right] {Equilibrium\\price=\$500.};
\end{TikzFigure}

%\newhtmlpage

In this example, the equilibrium price for apartments will be \$500. Let us
see why. At that price the value placed on the marginal unit supplied by
Kirin equals Evan's willingness to pay. Five apartments will be rented. A
sixth apartment will not be rented because Lynne will let her apartment only
if the price reaches \$550. But the sixth potential demander is willing to
pay only \$400. Note that, as usual, there is just a single price in the
market. Each renter pays \$500, and therefore each supplier also receives
\$500.

The consumer and supplier surpluses can now be computed. Note that, while
Don is willing to pay \$600, he actually pays \$500. His consumer surplus is
therefore \$100. In Figure~\ref{fig:apartmentmarket}, we can see that each %
\terminology{consumer's surplus} is the distance between the market price
and the individual's valuation. These values are given in the final column
of the top half of Table~\ref{table:consupsurplus}.

\begin{DefBox}
\textbf{Consumer surplus} is the excess of consumer willingness to pay over the market price.
\end{DefBox}

Using the same reasoning, we can compute each \terminology{supplier's
surplus}, which is the excess of the amount obtained for the rented
apartment over the reservation price. For example, Heward obtains a surplus
on the supply side of \$150, while Jeff gets \$50. Heward is willing to put
his apartment on the market for \$350, but gets the equilibrium price/rent
of \$500 for it. Hence his surplus is \$150.

\begin{DefBox}
\textbf{Supplier or producer surplus} is the excess of market price over the reservation price of the supplier.
\end{DefBox}

It should now be clear why these measures are called surpluses. \textit{The
suppliers and demanders are all willing to participate in this market
because they earn this surplus}. It is a measure of their gain from being
involved in the trading. The sum of each participant's surplus in the final
column of Table~\ref{table:consupsurplus} defines the total surplus in the
market. Hence, on the demand side a total surplus arises of \$1,000 and on
the supply side a value of \$500.

\newhtmlpage

\subsection*{The taxi market}

We do not normally think of demand and supply functions in terms of the
steps illustrated in Figure~\ref{fig:apartmentmarket}. Usually there are so
many participants in the market that the differences in reservation prices
on the supply side and willingness to pay on the demand side are exceedingly
small, and so the demand and supply curves are drawn as continuous lines. So
our second example reflects this, and comes from the market for taxi rides.
We might think of this as an \textit{Uber}-type taxi operation. 

Let us suppose that the demand and supply curves for taxi rides in a given
city are given by the functions in Figure~\ref{fig:taximarketA}. 


\newhtmlpage

% Figure 5.2	fig:taximarketA
\begin{TikzFigure}{xscale=0.5,yscale=0.5,descwidth=24em,caption={The taxi market \label{fig:taximarketA}},description={Consumer surplus is the area ABE, supplier surplus is the area BCE.}}
% axes
\draw [thick] (0,11) node (yaxis) [above,mynode1] {Price (fee\\per hour)} |- (12,0) node (xaxis) [right,mynode1] {Quantity (thousands\\of hours of rides)};
% demand line
\draw [demandcolour,thick,name path=D] (0,9) node [mynode,left,black] {90} node [mynode,above right,black] {A} -- node [mynode,above right,black,pos=0.25] {$D$} (9,0);
% supply line
\draw [supplycolour,thick,name path=S] (0,3) node [mynode,left,black] {30} node [mynode,below right,black] {C} -- (12,6) node [mynode,right,black] {$S$};
% intersection of S and D
\draw [name intersections={of=D and S, by=E}]
	[thick,dotted] (yaxis |- E) node [mynode,left] {42} node [mynode,above right] {B} -- (E) node [mynode,above] {E} -- (xaxis -| E) node [mynode,below] {48};
\end{TikzFigure}

The demand curve represents
the willingness to pay on the part of riders. The supply curve represents
the willingness to supply on the part of drivers. The price per hour of
rides defines the vertical axis; hours of rides (in thousands) are measured on the
horizontal axis. The demand intercept of \$90 says that the person who
values the ride most highly is willing to pay \$90 per hour. The downward
slope of the demand curve states that other buyers are willing to pay less.
On the supply side no driver is willing to supply his time and vehicle
unless he obtains at least \$30 per hour. To induce additional suppliers a
higher price must be paid, and this is represented by the upward sloping
supply curve. 

The intersection occurs at a price of \$42 per hour and the equilibrium
number of ride-hours supplied is 48 thousand\footnote{%
The demand and supply functions behind these curves are $P=90-1Q$ and $%
P=30+(1/4)Q$. Equating supply and demand yields the solutions in the text.}.
Computing the surpluses is very straightforward. By definition the consumer
surplus is the excess of the willingness to pay by each buyer above the
uniform price. Buyers who value the ride most highly obtain the biggest
surplus -- the highest valuation rider gets a surplus of \$48 per hour, the
difference between his willingness to pay of \$90 and the actual price of 
\$42. Each successive rider gets a slightly lower surplus until the final
rider, who obtains zero. She pays \$42 and values the ride hours at \$42
also. On the supply side, the drivers who are willing to supply rides at the
lowest reservation price (\$30 and above) obtain the biggest surplus. The
`marginal' supplier gets no surplus, because the price equals her
reservation price.


From this discussion it follows that the consumer surplus is given by the
area ABE and the supplier surplus by the area CBE. These are two
triangular areas, and measured as half of the base by the perpendicular
height. Therefore, in thousands of units:

\begin{align*}
\text{Consumer Surplus}& =\text{(demand value - price) = area ABE} \\
& =(1/2)\times 48\times \$48=\$1,152 \\
\text{Producer Surplus }& \text{= (price - reservation supply value) = area
BEC} \\
& =(1/2)\times 48\times \$12=\$288
\end{align*}

The total surplus that arises in the market is the sum of producer and
consumer surpluses, and since the units are in thousands of hours the total
surplus here is  $(\$1,152+\$288)\times 1,000=\$1,440,000$.
