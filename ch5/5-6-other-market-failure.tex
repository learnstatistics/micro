\section{Other market failures}\label{sec:ch5sec6}

There are other ways in which markets can fail to reflect accurately the
social value or social cost of economic activity. Profit-seeking monopolies,
which restrict output in order to increase profits, create inefficient
markets, and we will see why in the chapter on monopoly. Or the market may
not deal very well with what are called public goods. These are goods, like
radio and television service, national defence, or health information: With
such goods and services many individuals can be supplied with the same good
at the same total cost as one individual. We will address this problem in
our chapter on government. And, of course, there are international
externalities that cannot be corrected by national governments because the
interests of adjoining states may differ: One economy may wish to see cheap
coal-based electricity being supplied to its consumers, even if this means
acid rain or reduced air quality in a neighbouring state. Markets may fail
to supply an ``efficient'' amount of a good
or service in all of these situations. Global warming is perhaps the best,
and most extreme, example of international externalities and market failure.