\section{Taxation, surplus and efficiency}\label{sec:ch5sec4}

Despite enormous public interest in taxation and its impact on the economy,
it is one of the least understood areas of public policy. In this section we
will show how an understanding of two fundamental tools of
analysis -- elasticities and economic surplus -- provides powerful insights
into the field of taxation.

We begin with the simplest of cases: The federal government's goods and
services tax (GST) or the provincial governments' sales taxes (PST). These
taxes combined vary by province, but we suppose that a typical rate is 13
percent. In some provinces these two taxes are harmonized. Note that
this is a \textit{percentage}, or \textit{ad valorem},
tax, not a \textit{specific} tax of so many dollars per unit traded. 

% Figure 5.4	fig:efficiencycosttax
\begin{TikzFigure}{xscale=1.5,yscale=2.1,descwidth=25em,caption={The efficiency cost of taxation \label{fig:efficiencycosttax}},description={The tax shifts $S$ to $S_t$ and reduces the quantity traded from $Q_0$ to $Q_t$. At $Q_t$ the demand value placed on an additional unit exceeds the supply valuation by $E_t$A. Since the tax keeps output at this lower level, the economy cannot take advantage of the additional potential surplus between $Q_t$ and $Q_0$. Excess burden = deadweight loss = A$E_tE_0$.}}
% supply lines
\draw [supplycolour,ultra thick,name path=S] (0,0.5) node [mynode,below right,black] {F} -- (4,1.5) node [mynode,right,black] {$S$};
\draw [supplycolour,ultra thick,name path=St] (0,0.75)  -- (4,2.25) node [mynode,right,black] {$S_t$};
% demand line
\draw [demandcolour,ultra thick,domain=0:4,name path=D] (0,2) node [mynode,above right,black] {B} -- (4,0) node [black,mynode,above right,black] {$D$};
% axes
\draw [thick, -] (0,3) node (yaxis) [above] {Price} |- (5,0) node (xaxis) [right] {Quantity};
% intersection of demand and supply lines
\draw [name intersections={of=S and D, by=E0},name intersections={of=St and D, by=Et}]
	[dotted,thick] (yaxis |- E0) node [mynode,left] {$P_0$} -- (E0) node [mynode,above] {$E_0$} -- (xaxis -| E0) node [mynode,below] {$Q_0$}
	[dotted,thick] (yaxis |- Et) node [mynode,left] {$P_t$} -- (Et) node [mynode,above] {$E_t$} -- (xaxis -| Et) node [mynode,below] {$Q_t$};
% path to create dotted line from P_ts to A
\path [name path=PtsA] (xaxis -| Et) -- +(0,3);
% intersection of S with PtsA
\draw [name intersections={of=S and PtsA, by=A}]
	[dotted,thick] (yaxis |- A) node [mynode,left] {$P_{ts}$} -- (A) node [mynode,below right] {A};
% path to create tax wedge
\path [name path=taxwedge] (3.5,0) -- (3.5,3);
% intersection of taxwedge with supply lines
\draw [name intersections={of=S and taxwedge, by=notax},name intersections={of=St and taxwedge, by=withtax}]
	[<->,thick,shorten >=1mm,shorten <=1mm] (notax) -- node [mynode,right,midway] {Tax\\wedge} (withtax);
\end{TikzFigure}

\newhtmlpage

Figure~
\ref{fig:efficiencycosttax} illustrates the supply and demand curves for
some commodity. In the absence of taxes, the equilibrium $E_0$ is defined by
the combination $(P_0, Q_0)$. 

A 13-percent tax is now imposed, and the new supply curve $S_t$ lies 13
percent above the no-tax supply $S$. A \terminology{tax wedge} is therefore
imposed between the price the consumer must pay and the price that the
supplier receives. The new equilibrium is $E_t$, and the new market price is
at $P_t$. The price received by the supplier is lower than that paid by the
buyer by the amount of the tax wedge. The post-tax supply price is denoted
by $P_{ts}$.

There are two \textit{burdens} associated with this tax. The first is the %
\terminology{revenue burden}, the amount of tax revenue paid by the market
participants and received by the government. On each of the $Q_t$ units
sold, the government receives the amount $(P_t-P_{ts})$. Therefore, tax
revenue is the amount $P_{t}E_{t}$A$P_{ts}$. As illustrated in Chapter~\ref{chap:elasticities},
the degree to which the market price $P_t$ rises
above the no-tax price $P_0$ depends on the supply and demand elasticities.

\begin{DefBox}
A \textbf{tax wedge} is the difference between the consumer and producer prices.

The \textbf{revenue burden} is the amount of tax revenue raised by a tax.
\end{DefBox}

The second burden of the tax is called the \textit{excess burden}. The
concepts of consumer and producer surpluses help us comprehend this. The
effect of the tax has been to reduce consumer surplus by $%
P_{t}E_{t}E_{0}P_{0}$. This is the reduction in the pre-tax surplus given by
the triangle $P_{0}$B$E_{0}$. By the same reasoning, supplier surplus is
reduced by the amount $P_{0}E_{0}$A$P_{ts}$; prior to the tax it was $%
P_{0}E_{0}F$. Consumers and suppliers have therefore seen a reduction in
their well-being that is measured by these dollar amounts. Nonetheless, the
government has additional revenues amounting to $P_{t}E_{t}$A$P_{ts}$, and
this tax imposition therefore represents a \textit{transfer} from the
consumers and suppliers in the marketplace to the government. Ultimately,
the citizens should benefit from this revenue when it is used by the
government, and it is therefore not considered to be a net loss of surplus.

\newhtmlpage

However, there remains a part of the surplus loss that is not transferred,
the triangular area $E_{t}E_{0}$A. This component is called the %
\terminology{excess burden}, for the reason that it represents the component
of the economic surplus that is not transferred to the government in the
form of tax revenue. It is also called the \terminology{deadweight loss},
DWL.

\begin{DefBox}
The \textbf{excess burden}, or \textbf{deadweight loss}, of a tax is the component of consumer and producer surpluses forming a net loss to the whole economy.
\end{DefBox}

The intuition behind this concept is not difficult. At the output $Q_{t}$,
the value placed by consumers on the last unit supplied is $P_{t}$ ($=E_{t}$),
while the production cost of that last unit is $P_{ts}$ ($=$A). But the
potential surplus ($P_{t}-P_{ts}$) associated with producing an additional
unit cannot be realized, because the tax dictates that the production
equilibrium is at $Q_{t}$ rather than any higher output. Thus, if output
could be increased from $Q_{t}$ to $Q_{0}$, a surplus of value over cost
would be realized on every additional unit equal to the vertical distance
between the demand and supply functions $D$ and $S$. Therefore, the loss
associated with the tax is the area $E_{t}E_{0}$A.

In public policy debates, this excess burden is rarely discussed. The reason
is that notions of consumer and producer surpluses are not well understood
by non-economists, despite the fact that the value of lost surpluses is
frequently large. Numerous studies have estimated the excess burden
associated with raising an additional dollar from the tax system. They
rarely find that the excess burden is less than 25 percent of total
expenditure. This is a sobering finding. It tells us that if the government
wished to implement a new program by raising additional tax revenue, the
benefits of the new program should be 25 percent greater than the amount
expended on it!

The impact of taxes and other influences that result in an inefficient use
of the economy's resources are frequently called \terminology{distortions}
because they necessarily lead the economy away from the efficient output. 
The magnitude of the excess burden is determined by the elasticities of
supply and demand in the markets where taxes are levied. To see this, return
to Figure~\ref{fig:efficiencycosttax}, and suppose that the demand curve
through $E_0$ were more elastic (with the same supply curve, for
simplicity). The post-tax equilibrium $E_t$ would now yield a lower $Q_t$
value and a price between $P_t$ and $P_0$. The resulting tax revenue
raised and the magnitude of the excess burden would differ because of the
new elasticity.

\begin{DefBox}
A \textbf{distortion} in resource allocation means that production is not at an efficient output.
\end{DefBox}
