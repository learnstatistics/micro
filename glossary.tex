\chapter*{Glossary}

\textbf{Accounting profit}: is the difference between revenues and explicit costs. (\ref{sec:ch7sec2})

\textbf{Adverse selection} occurs when incomplete or asymmetric information describes an economic relationship. (\ref{sec:ch14sec1})

\textbf{Affordable set} of goods and services for the consumer is bounded by the budget line from above; the \textbf{non-affordable set} lies strictly above the budget line. (\ref{sec:ch6sec3})

\textbf{Age-earnings profiles} define the pattern of earnings over time for individuals with different characteristics. (\ref{sec:ch13sec2})

\textbf{Agent}: usually a manager who works in a corporation and is directed to follow the corporation's interests. (\ref{sec:ch7sec2})

\textbf{Allocative inefficiency} arises when resources are not appropriately allocated and result in deadweight losses. (\ref{sec:ch10sec4})

\textbf{Asset price}: the financial sum for which the asset can be purchased. (\ref{sec:ch12sec4})

\textbf{Asymmetric information} is where at least one party in an economic relationship has less than full information and has a different amount of information from another party. (\ref{sec:ch14sec1})

\textbf{Autarky} denotes the no-trade situation. (\ref{sec:ch15sec3})

\textbf{Average fixed cost} is the total fixed cost per unit of output. (\ref{sec:ch8sec4})

\textbf{Average product of labour} is the number of units of output produced per unit of labour at different levels of employment. (\ref{sec:ch8sec3})

\textbf{Average revenue} is the price per unit sold. (\ref{sec:ch10sec2})

\textbf{Average total cost} is the sum of all costs per unit of output. (\ref{sec:ch8sec4})

\textbf{Average variable cost} is the total variable cost per unit of output. (\ref{sec:ch8sec4})

\textbf{Bid rigging} is an illegal practice in which bidders (buyers) conspire to set prices in their own interest. (\ref{sec:ch14sec5})

\textbf{Boom}: a period of high growth that raises output above normal capacity output. (\ref{sec:ch1sec6})

\textbf{Break-even price} corresponds to the minimum of the $ATC$ curve. (\ref{sec:ch9sec3})

\textbf{Budget constraint} defines all bundles of goods that the consumer can afford with a given budget. (\ref{sec:ch6sec3})

\textbf{Capital gains (losses)} arise from the ownership of a corporation when an individual sells a share at a price higher (lower) than when the share was purchased. (\ref{sec:ch7sec1})

\textbf{Capital market}: a set of financial institutions that funnels financing from investors into bonds and stocks. (\ref{sec:ch7sec4})

\textbf{Capital services} are the production inputs generated by capital assets. (\ref{sec:ch12sec4})

\textbf{Capital stock}: the buildings, machinery, equipment and software used in producing goods and services. (\ref{sec:ch1sec6})

\textbf{Carbon taxes} are a market-based system aimed at reducing GHGs. (\ref{sec:ch5sec7})

\textbf{Cardinal utility} is a measurable concept of satisfaction. (\ref{sec:ch6sec2})

\textbf{Cartel}: is a group of suppliers that colludes to operate like a monopolist. (\ref{sec:ch10sec6})

\textbf{Cluster}: a group of firms producing similar products, or engaged in similar research. (\ref{sec:ch8sec8})

\textbf{Collusion} is an explicit or implicit agreement to avoid competition with a view to increasing profit. (\ref{sec:ch11sec4})

\textbf{Comparative static analysis} compares an initial equilibrium with a new equilibrium, where the difference is due to a change in one of the other things that lie behind the demand curve or the supply curve. (\ref{sec:ch3sec4})

\textbf{Complementary goods}: when a price reduction (rise) for a related product increases (reduces) the demand for a primary product, it is a complement for the primary product. (\ref{sec:ch3sec4})

\textbf{Concentration ratio}: $N$-firm concentration ratio is the sales share of the largest $N$ firms in that sector of the economy. (\ref{sec:ch11sec2})

\textbf{Conjecture}: a belief that one firm forms about the strategic reaction of another competing firm. (\ref{sec:ch11sec4})

\textbf{Constant returns to scale} implies that output increases in direct proportion to an equal proportionate increase in all inputs. (\ref{sec:ch8sec6})

\textbf{Consumer equilibrium} occurs when marginal utility per dollar spent on the last unit of each good is equal. (\ref{sec:ch6sec2})

\textbf{Consumer optimum} occurs where the chosen consumption bundle is a point such that the price ratio equals the marginal rate of substitution. (\ref{sec:ch6sec3})

\textbf{Consumer price index}: the average price level for consumer goods and services. (\ref{sec:ch2sec1})

\textbf{Consumer surplus} is the excess of consumer willingness to pay over the market price. (\ref{sec:ch5sec2})

\textbf{Consumption possibility frontier (CPF)}: the combination of goods that can be consumed as a result of a given production choice. (\ref{sec:ch1sec4})

\textbf{Consumption possibility frontier} defines what an economy can consume after production specialization and trade. (\ref{sec:ch15sec3})

\textbf{Corporation or company} is an organization with a legal identity separate from its owners that produces and trades. (\ref{sec:ch7sec1})

\textbf{Corrective tax} seeks to direct the market towards a more efficient output. (\ref{sec:ch5sec5})

\textbf{Cournot behaviour} involves each firm reacting optimally in their choice of output to their competitors' decisions. (\ref{sec:ch11sec5})

\textbf{Credible threat}: one that, after the fact, is still optimal to implement. (\ref{sec:ch11sec6})

\textbf{Cross-price elasticity of demand} is the percentage change in the quantity demanded of a product divided by the percentage change in the price of another. (\ref{sec:ch4sec4})

\textbf{Cross-section data}: values for different variables recorded at a point in time. (\ref{sec:ch2sec1})

\textbf{Data}: recorded values of variables. (\ref{sec:ch2sec1})

\textbf{Deadweight loss} of a tax is the component of consumer and producer surpluses forming a net loss to the whole economy. (\ref{sec:ch5sec4})

\textbf{Decreasing returns to scale} implies that an equal proportionate increase in all inputs leads to a less than proportionate increase in output. (\ref{sec:ch8sec6})

\textbf{Demand} is the quantity of a good or service that buyers wish to purchase at each possible price, with all other influences on demand remaining unchanged. (\ref{sec:ch3sec2})

\textbf{Demand curve} is a graphical expression of the relationship between price and quantity demanded, with other influences remaining unchanged. (\ref{sec:ch3sec3})

\textbf{Demand for labour}: a derived demand, reflecting the demand for the output of final goods and services. (\ref{sec:ch12sec1})

\textbf{Demand is elastic} if the price elasticity is greater than unity. It is \textbf{inelastic} if the value lies between unity and 0. It is \textbf{unit elastic} if the value is exactly one. (\ref{sec:ch4sec1})

\textbf{Depreciation} is the annual change in the value of a physical asset. (\ref{sec:ch12sec4})

\textbf{Differentiated product} is one that differs slightly from other products in the same market. (\ref{sec:ch11sec3})

\textbf{Diminishing marginal rate of substitution} reflects a higher marginal value being associated with smaller quantities of any good consumed. (\ref{sec:ch6sec3})

\textbf{Diminishing marginal utility} implies that the addition to total utility from each extra unit of a good or service consumed is declining. (\ref{sec:ch6sec2})

\textbf{Discrimination} implies an earnings differential that is attributable to a trait other than human capital. (\ref{sec:ch13sec6})

\textbf{Distortion} in resource allocation means that production is not at an efficient output. (\ref{sec:ch5sec4})

\textbf{Diversification} reduces the total risk of a portfolio by pooling risks across several different assets whose individual returns behave independently. (\ref{sec:ch7sec4})

\textbf{Dividends} are payments made from after-tax profits to company shareholders. (\ref{sec:ch7sec1})

\textbf{Dominant strategy}: a player's best strategy, whatever the strategies adopted by rivals. (\ref{sec:ch11sec4})

\textbf{Dumping} is a predatory practice, based on artificial costs aimed at driving out domestic producers. (\ref{sec:ch15sec6})

\textbf{Duopoly} defines a market or sector with just two firms. (\ref{sec:ch11sec1})

\textbf{Dynamic gains}: the potential for domestic producers to increase productivity by competing with, and learning from, foreign producers. (\ref{sec:ch15sec4})

\textbf{Economic (supernormal) profits} are those profits above normal profits that induce firms to enter an industry. Economic profits are based on the opportunity cost of the resources used in production. (\ref{sec:ch9sec4})

\textbf{Economic efficiency} defines a production structure that produces output at least cost. (\ref{sec:ch8sec1})

\textbf{Economic equity} is concerned with the distribution of well-being among members of the economy. (\ref{sec:ch2sec3})

\textbf{Economic profit}: is the difference between revenue and the sum of explicit and implicit costs. (\ref{sec:ch7sec2})

\textbf{Economies of scope} occur if the unit cost of producing particular products is less when combined with the production of other products than when produced alone. (\ref{sec:ch8sec8})

\textbf{Economy-wide PPF} is the set of goods combinations that can be produced in the economy when all available productive resources are in use. (\ref{sec:ch1sec5})

\textbf{Education premium}: the difference in earnings between the more and less highly educated. (\ref{sec:ch13sec2})

\textbf{Efficiency} addresses the question of how well the economy's resources are used and allocated. (\ref{sec:ch5sec1})

\textbf{Efficient market}: maximizes the sum of producer and consumer surpluses. (\ref{sec:ch5sec3})

\textbf{Efficient supply of public goods} is where the marginal cost equals the sum of individual marginal valuations, and each individual consumes the same quantity. (\ref{sec:ch14sec1})

\textbf{Elasticity of supply} is defined as the percentage change in quantity supplied divided by the percentage change in price. (\ref{sec:ch4sec6})

\textbf{Equilibrium price}: equilibrates the market. It is the price at which quantity demanded equals the quantity supplied. (\ref{sec:ch3sec2})

\textbf{Equity} deals with how society's goods and rewards are, and should be, distributed among its different members, and how the associated costs should be apportioned. (\ref{sec:ch5sec1})

\textbf{Excess burden} of a tax is the component of consumer and producer surpluses forming a net loss to the whole economy. (\ref{sec:ch5sec4})

\textbf{Excess demand} exists when the quantity demanded exceeds quantity supplied at the going price. (\ref{sec:ch3sec2})

\textbf{Excess supply} exists when the quantity supplied exceeds the quantity demanded at the going price. (\ref{sec:ch3sec2})

\textbf{Exclusive sale}: where a retailer is obliged (perhaps illegally) to purchase all wholesale products from a single supplier only. (\ref{sec:ch14sec5})

\textbf{Explicit costs}: are the measured financial costs. (\ref{sec:ch7sec2})

\textbf{Externality} is a benefit or cost falling on people other than those involved in the activity's market. It can create a difference between private costs or values and social costs or values. (\ref{sec:ch5sec5})

\textbf{Firm-specific skills} raise a worker's productivity in a particular firm. (\ref{sec:ch13sec3})

\textbf{Fixed costs} are costs that are independent of the level of output. (\ref{sec:ch8sec4})

\textbf{Flow} is the stream of services an asset provides during a period of time. (\ref{sec:ch12sec4})

\textbf{Full employment output} $Y_c=(\text{number of workers at full employment})\times(\text{output per worker})$. (\ref{sec:ch1sec6})

\textbf{Game}: a situation in which contestants plan strategically to maximize their profits, taking account of rivals' behaviour. (\ref{sec:ch11sec4})

\textbf{General skills} enhance productivity in many jobs or firms. (\ref{sec:ch13sec3})

\textbf{Gini index}: a measure of how far the Lorenz curve lies from the line of equality. Its maximum value is one; its minimum value is zero. (\ref{sec:ch13sec7})

\textbf{Globalization} is the tendency for international markets to be ever more integrated. (\ref{sec:ch8sec7})

\textbf{Greenhouse gases} that accumulate excessively in the earth's atmosphere prevent heat from escaping and lead to global warming. (\ref{sec:ch5sec7})

\textbf{Gross investment} is the production of new capital goods and the improvement of existing capital goods. (\ref{sec:ch12sec4})

\textbf{High (low) frequency data} series have short (long) intervals between observations. (\ref{sec:ch2sec1})

\textbf{Human capital} is the stock of expertise accumulated by a worker that determines future productivity and earnings. (\ref{sec:ch13sec1})

\textbf{Imperfectly competitive firms} face a downward-sloping demand curve, and their output price reflects the quantity sold. (\ref{sec:ch11sec1})

\textbf{Implicit costs}: represent the opportunity cost of the resources used in production. (\ref{sec:ch7sec2})

\textbf{Income elasticity of demand} is the percentage change in quantity demanded divided by a percentage change in income. (\ref{sec:ch4sec5})

\textbf{Increasing (decreasing) cost} industry is one where costs rise (fall) for each firm because of the scale of industry operation. (\ref{sec:ch9sec5})

\textbf{Increasing returns to scale} implies that, when all inputs are increased by a given proportion, output increases more than proportionately. (\ref{sec:ch8sec6})

\textbf{Index number}: value for a variable, or an average of a set of variables, expressed relative to a given base value. (\ref{sec:ch2sec1})

\textbf{Indifference curve} defines combinations of goods and services that yield the same level of satisfaction to the consumer. (\ref{sec:ch6sec3})

\textbf{Indifference map} is a set of indifference curves, where curves further from the origin denote a higher level of satisfaction. (\ref{sec:ch6sec3})

\textbf{Industry supply (short run)} in perfect competition is the horizontal sum of all firms' supply curves. (\ref{sec:ch9sec3})

\textbf{Industry supply in the long run in perfect competition} is horizontal at a price corresponding to the minimum of the representative firm's long-run $ATC$ curve. (\ref{sec:ch9sec5})

\textbf{Inferior good} is one whose demand falls in response to higher incomes. (\ref{sec:ch3sec4})

\textbf{Inferior goods} have a negative income elasticity. (\ref{sec:ch4sec5})

\textbf{Inflation (deflation) rate}: the annual percentage increase (decrease) in the level of consumer prices. (\ref{sec:ch2sec1})

\textbf{Intra-firm trade} is two-way trade in international products produced within the same firm. (\ref{sec:ch15sec4})

\textbf{Intra-industry trade} is two-way international trade in products produced within the same industry. (\ref{sec:ch15sec4})

\textbf{Invention} is the discovery of a new product or process through research. (\ref{sec:ch10sec7})

\textbf{Labour force}: is that part of the population either employed or seeking employment. (\ref{sec:ch12sec2})

\textbf{Law of demand} states that, other things being equal, more of a good is demanded the lower is its price. (\ref{sec:ch6sec2})

\textbf{Law of diminishing returns}: when increments of a variable factor (labour) are added to a fixed amount of another factor (capital), the marginal product of the variable factor must eventually decline. (\ref{sec:ch8sec3})

\textbf{Learning by doing} can reduce costs. A longer history of production enables firms to accumulate knowledge and thereby implement more efficient production processes. (\ref{sec:ch8sec8})

\textbf{Limited liability} means that the liability of the company is limited to the value of the company's assets. (\ref{sec:ch7sec1})

\textbf{Long-run}: a period of time that is sufficient to enable all factors of production to be adjusted. (\ref{sec:ch8sec2})

\textbf{Long-run average total cost} is the lower envelope of all the short-run ATC curves. (\ref{sec:ch8sec6})

\textbf{Long-run equilibrium} in a competitive industry requires a price equal to the minimum point of a firm's $ATC$. At this point, only normal profits exist, and there is no incentive for firms to enter or exit. (\ref{sec:ch9sec4})

\textbf{Long-run marginal cost} is the increment in cost associated with producing one more unit of output when all inputs are adjusted in a cost minimizing manner. (\ref{sec:ch8sec6})

\textbf{Longitudinal data} follow the same units of observation through time. (\ref{sec:ch2sec1})

\textbf{Lorenz curve} describes the cumulative percentage of the income distribution going to different quantiles of the population. (\ref{sec:ch13sec7})

\textbf{Luxury good} or service is one whose income elasticity equals or exceeds unity. (\ref{sec:ch4sec5})

\textbf{Macroeconomics} studies the economy as a system in which feedback among sectors determine national output, employment and prices. (\ref{sec:ch1sec1})

\textbf{Marginal abatement curve} reflects the cost to society of reducing the quantity of pollution by one unit. (\ref{sec:ch5sec7})

\textbf{Marginal cost} of production is the cost of producing each additional unit of output. (\ref{sec:ch8sec4})

\textbf{Marginal damage curve} reflects the cost to society of an additional unit of pollution. (\ref{sec:ch5sec7})

\textbf{Marginal product of capital} is the output produced by one additional unit of capital services, with all other inputs being held constant. (\ref{sec:ch12sec5})

\textbf{Marginal product of labour} is the addition to output produced by each additional worker. It is also the slope of the total product curve. (\ref{sec:ch8sec3})

\textbf{Marginal rate of substitution} is the slope of the indifference curve. It defines the amount of one good the consumer is willing to sacrifice in order to obtain a given increment of the other, while maintaining utility unchanged. (\ref{sec:ch6sec3})

\textbf{Marginal revenue} is the additional revenue accruing to the firm resulting from the sale of one more unit of output. (\ref{sec:ch9sec3})

\textbf{Marginal revenue} is the change in total revenue due to selling one more unit of the good. (\ref{sec:ch10sec2})

\textbf{Marginal revenue product of labour} is the additional revenue generated by hiring one more unit of labour where the marginal revenue declines. (\ref{sec:ch12sec1})

\textbf{Marginal utility} is the addition to total utility created when one more unit of a good or service is consumed. (\ref{sec:ch6sec2})

\textbf{Market demand}: the horizontal sum of individual demands. (\ref{sec:ch3sec8})

\textbf{Market failure} defines outcomes in which the allocation of resources is not efficient. (\ref{sec:ch14sec1})

\textbf{Microeconomics} is the study of individual behaviour in the context of scarcity. (\ref{sec:ch1sec1})

\textbf{Minimum efficient scale} defines a threshold size of operation such that scale economies are almost exhausted. (\ref{sec:ch8sec6})

\textbf{Mixed economy}: goods and services are supplied both by private suppliers and government. (\ref{sec:ch1sec1})

\textbf{Model} is a formalization of theory that facilitates scientific inquiry. (\ref{sec:ch1sec2})

\textbf{Monopolist}: is the sole supplier of an industry's output, and therefore the industry and the firm are one and the same. (\ref{sec:ch10sec1})

\textbf{Monopolistic competition} defines a market with many sellers of products that have similar characteristics. Monopolistically competitive firms can exert only a small influence on the whole market. (\ref{sec:ch11sec1})

\textbf{Monopolistically competitive equilibrium}: in the long run requires the firm's demand curve to be tangent to the $ATC$ curve at the output where $MR=MC$. (\ref{sec:ch11sec3})

\textbf{Monopsonist} is the sole buyer of a good or service and faces an upward-sloping supply curve. (\ref{sec:ch12sec1})

\textbf{Moral hazard} may characterize behaviour where the costs of certain activities are not incurred by those undertaking them. (\ref{sec:ch14sec1})

\textbf{Nash equilibrium}: one in which each player chooses the best strategy, given the strategies chosen by the other player, and there is no incentive for any player to move. (\ref{sec:ch11sec4})

\textbf{Natural monopoly}: one where the $ATC$ of producing any output declines with the scale of operation. (\ref{sec:ch10sec1})

\textbf{Necessity} is one whose income elasticity is greater than zero and is less than unity. (\ref{sec:ch4sec5})

\textbf{Net investment} is gross investment minus depreciation of the existing capital stock. (\ref{sec:ch12sec4})

\textbf{Non-tariff barriers}, such as product content requirements, limits the gains from trade. (\ref{sec:ch15sec5})

\textbf{Normal good} is one whose demand increases in response to higher incomes. (\ref{sec:ch3sec4})

\textbf{Normative economics} offers recommendations that incorporate value judgments. (\ref{sec:ch2sec3})

\textbf{Oligopoly} defines an industry with a small number of suppliers. (\ref{sec:ch11sec1})

\textbf{On-the-job training} improves human capital through work experience. (\ref{sec:ch13sec3})

\textbf{Opportunity cost} of a choice is what must be sacrificed when a choice is made. (\ref{sec:ch1sec3})

\textbf{Ordinal utility} assumes that individuals can rank commodity bundles in accordance with the level of satisfaction associated with each bundle. (\ref{sec:ch6sec3})

\textbf{Participation rate}: the fraction of the population in the working age group that joins the labour force. (\ref{sec:ch12sec2})

\textbf{Partnership}: a business owned jointly by two or more individuals, who share in the profits and are jointly responsible for losses. (\ref{sec:ch7sec1})

\textbf{Patent laws} grant inventors a legal monopoly on use for a fixed period of time. (\ref{sec:ch10sec7})

\textbf{Payoff matrix}: defines the rewards to each player resulting from particular choices. (\ref{sec:ch11sec4})

\textbf{Percentage change}$=[(\text{change in values})/(\text{original value})]\times 100$. (\ref{sec:ch2sec1})

\textbf{Perfect competition}: an industry in which many suppliers, producing an identical product, face many buyers, and no one participant can influence the market. (\ref{sec:ch9sec1})

\textbf{Physical capital} is the stock of produced goods that are inputs to the production of other goods and services. (\ref{sec:ch12sec4})

\textbf{Portfolio}: a combination of assets that is designed to secure an income from investing and to reduce risk. (\ref{sec:ch7sec4})

\textbf{Positive economics} studies objective or scientific explanations of how the economy functions. (\ref{sec:ch2sec3})

\textbf{Predatory pricing} is a practice that is aimed at driving out competition by artificially reducing the price of one product sold by a supplier. (\ref{sec:ch14sec5})

\textbf{Present value of a stream of future earnings}: the sum of each year's earnings divided by one plus the interest rate raised to the appropriate power. (\ref{sec:ch12sec4})

\textbf{Price controls} are government rules or laws that inhibit the formation of market-determined prices. (\ref{sec:ch3sec7})

\textbf{Price discrimination} involves charging different prices to different consumers in order to increase profit. (\ref{sec:ch10sec5})

\textbf{Price elasticity of demand} is measured as the percentage change in quantity demanded, divided by the percentage change in price. (\ref{sec:ch4sec1})

\textbf{Principal or owner}: delegates decisions to an agent, or manager. (\ref{sec:ch7sec2})

\textbf{Principal-agent problem}: arises when the principal cannot easily monitor the actions of the agent, who therefore may not act in the best interests of the principal. (\ref{sec:ch7sec2})

\textbf{Principle of comparative advantage} states that even if one country has an absolute advantage in producing both goods, gains to specialization and trade still materialize, provided the opportunity cost of producing the goods differs between economies. (\ref{sec:ch15sec3})

\textbf{Process innovation} refers to new or better production or supply. (\ref{sec:ch10sec7})

\textbf{Product innovation} refers to new or better products or services. (\ref{sec:ch10sec7})

\textbf{Production function}: a technological relationship that specifies how much output can be produced with specific amounts of inputs. (\ref{sec:ch8sec1})

\textbf{Production possibility frontier (PPF)} defines the combination of goods that can be produced using all of the resources available. (\ref{sec:ch1sec4})

\textbf{Productivity of labour} is the output of goods and services per worker. (\ref{sec:ch1sec6})

\textbf{Profit maximization} is the goal of competitive suppliers -- they seek to maximize the difference between revenues and costs. (\ref{sec:ch9sec1})

\textbf{Public goods} are non-rivalrous, in that they can be consumed simultaneously by more than one individual; additionally they may have a non-excludability characteristic. (\ref{sec:ch14sec1})

\textbf{Quantity demanded} defines the amount purchased at a particular price. (\ref{sec:ch3sec2})

\textbf{Quantity supplied} refers to the amount supplied at a particular price. (\ref{sec:ch3sec2})

\textbf{Quota} is a quantitative limit on an imported product. (\ref{sec:ch15sec5})

\textbf{Quotas} are physical restrictions on output. (\ref{sec:ch3sec7})

\textbf{Reaction functions} define the optimal choice of output conditional upon a rival's output choice. (\ref{sec:ch11sec5})

\textbf{Real price}: the actual price adjusted by the general (consumer) price level in the economy. (\ref{sec:ch2sec1})

\textbf{Real return on corporate stock}: the sum of dividend plus capital gain, adjusted for inflation. (\ref{sec:ch7sec1})

\textbf{Real return}: the nominal return minus the rate of inflation. (\ref{sec:ch7sec1})

\textbf{Recession}: when output falls below the economy's capacity output. (\ref{sec:ch1sec6})

\textbf{Refusal to deal}: an illegal practice where a supplier refuses to sell to a purchaser. (\ref{sec:ch14sec5})

\textbf{Regression line}: representation of the average relationship between two variables in a scatter diagram. (\ref{sec:ch2sec2})

\textbf{Rent} is the excess remuneration an individual currently receives above the next best alternative. This alternative is the reservation wage. (\ref{sec:ch12sec3})

\textbf{Rent seeking} is an activity that uses productive resources to redistribute rather than create output and value. (\ref{sec:ch10sec7})

\textbf{Rental rate}: the cost of using capital services. (\ref{sec:ch12sec4})

\textbf{Repeated cross-section data}: cross-section data recorded at regular or irregular intervals. (\ref{sec:ch2sec1})

\textbf{Required rental} covers the sum of maintenance, depreciation and interest costs. (\ref{sec:ch12sec5})

\textbf{Resale price maintenance} is an illegal practice wherein a producer requires sellers to maintain a specified price. (\ref{sec:ch14sec5})

\textbf{Retained earnings} are the profits retained by a company for reinvestment and not distributed as dividends. (\ref{sec:ch7sec1})

\textbf{Revenue burden} is the amount of tax revenue raised by a tax. (\ref{sec:ch5sec4})

\textbf{Risk}: the risk associated with an investment can be measured by the dispersion in possible outcomes. A greater dispersion in outcomes implies more risk. (\ref{sec:ch7sec4})

\textbf{Risk pooling}: a means of reducing risk and increasing utility by aggregating or pooling multiple independent risks. (\ref{sec:ch7sec4})

\textbf{Screening} is the process of obtaining information by observing differences in behaviour. (\ref{sec:ch13sec4})

\textbf{Shareholders} invest in corporations and therefore are the owners. They have limited liability personally if the firm incurs losses. (\ref{sec:ch7sec1})

\textbf{Sharing economy}: involves enterprises that are internet based, and that use production resources that have use outside of the marketplace. (\ref{sec:ch14sec5})

\textbf{Short run}: a period during which at least one factor of production is fixed. If capital is fixed, then more output is produced by using additional labour. (\ref{sec:ch8sec2})

\textbf{Short-run equilibrium} in perfect competition occurs when each firm maximizes profit by producing a quantity where $P=MC$. (\ref{sec:ch9sec3})

\textbf{Short-run supply curve for perfect competitor}: the portion of the $MC$ curve above the minimum of the $AVC$. (\ref{sec:ch9sec3})

\textbf{Short side of the market} determines outcomes at prices other than the equilibrium. (\ref{sec:ch3sec2})

\textbf{Shut-down price} corresponds to the minimum value of the $AVC$ curve. (\ref{sec:ch9sec3})

\textbf{Signalling} is the decision to undertake an action in order to reveal information. (\ref{sec:ch13sec4})

\textbf{Sole proprietor} is the single owner of a business and is responsible for all profits and losses. (\ref{sec:ch7sec1})

\textbf{Spending power} of a federal government arises when the federal government can influence lower level governments due to its financial rather than constitutional power. (\ref{sec:ch14sec2})

\textbf{Stock} is the quantity of an asset at a point in time. (\ref{sec:ch12sec4})

\textbf{Stock option}: an option to buy the stock of the company at a future date for a fixed, predetermined price. (\ref{sec:ch7sec2})

\textbf{Strategy}: a game plan describing how a player acts, or moves, in each possible situation. (\ref{sec:ch11sec4})

\textbf{Substitute goods}: when a price reduction (rise) for a related product reduces (increases) the demand for a primary product, it is a substitute for the primary product. (\ref{sec:ch3sec4})

\textbf{Sunk cost} is a fixed cost that has already been incurred and cannot be recovered, even by producing a zero output. (\ref{sec:ch8sec5})

\textbf{Supplier or producer surplus} is the excess of market price over the reservation price of the supplier. (\ref{sec:ch5sec2})

\textbf{Supply} is the quantity of a good or service that sellers are willing to sell at each possible price, with all other influences on supply remaining unchanged. (\ref{sec:ch3sec2})

\textbf{Supply curve} is a graphical expression of the relationship between price and quantity supplied, with other influences remaining unchanged. (\ref{sec:ch3sec3})

\textbf{Tariff} is a tax on an imported product that is designed to limit trade in addition to generating tax revenue. It is a barrier to trade. (\ref{sec:ch15sec5})

\textbf{Tax incidence} describes how the burden of a tax is shared between buyer and seller. (\ref{sec:ch4sec7})

\textbf{Tax wedge} is the difference between the consumer and producer prices. (\ref{sec:ch5sec4})

\textbf{Technological change} represents innovation that can reduce the cost of production or bring new products on line. (\ref{sec:ch8sec7})

\textbf{Technological efficiency} means that the maximum output is produced with the given set of inputs. (\ref{sec:ch8sec1})

\textbf{Terms of trade} define the rate at which goods trade internationally. (\ref{sec:ch15sec3})

\textbf{Theory} is a logical view of how things work, and is frequently formulated on the basis of observation. (\ref{sec:ch1sec2})

\textbf{Tied sale}: one where the purchaser must agree to purchase a bundle of goods from the one supplier. (\ref{sec:ch14sec5})

\textbf{Time series data}: a set of measurements made sequentially at different points in time. (\ref{sec:ch2sec1})

\textbf{Total cost} is the sum of fixed cost and variable cost. (\ref{sec:ch8sec4})

\textbf{Total factor productivity}: how efficiently the factors of production are combined. (\ref{sec:ch15sec4})

\textbf{Total product} is the relationship between total output produced and the number of workers employed, for a given amount of capital. (\ref{sec:ch8sec3})

\textbf{Total utility} is a measure of the total satisfaction derived from consuming a given amount of goods and services. (\ref{sec:ch6sec2})

\textbf{Tradable permits} are a market-based system aimed at reducing GHGs. (\ref{sec:ch5sec7})

\textbf{Trade subsidy} to a domestic manufacturer reduces the domestic cost and limits imports. (\ref{sec:ch15sec5})

\textbf{Transfer earnings} are the amount that an individual can earn in the next highest paying alternative job. (\ref{sec:ch12sec3})

\textbf{Two-part tariff}: involves an access fee and a per unit of quantity fee. (\ref{sec:ch14sec5})

\textbf{Unemployment rate}: the fraction of the labour force actively seeking employment that is not employed. (\ref{sec:ch12sec2})

\textbf{Value of the marginal product} is the marginal product multiplied by the price of the good produced. (\ref{sec:ch12sec1})

\textbf{Value of the marginal product of capital} is the marginal product of capital multiplied by the price of the output it produces. (\ref{sec:ch12sec5})

\textbf{Variable costs} are related to the output produced. (\ref{sec:ch8sec4})

\textbf{Variables}: measures that can take on different sizes. (\ref{sec:ch2sec1})

\textbf{Very long run}: a period sufficiently long for new technology to develop. (\ref{sec:ch8sec2})

\textbf{Welfare economics} assesses how well the economy allocates its scarce resources in accordance with the goals of efficiency and equity. (\ref{sec:ch5sec1})