\section{Economy-wide production possibilities}\label{sec:ch1sec5}

The $PPF$s in Figures~\ref{fig:absadvprod} and \ref{fig:absadvcon} define the amounts of the goods
that each \textit{individual} can produce while using all of their
productive capacity---time in this instance. The national, or economy-wide, 
$PPF$ for this two-person economy reflects these individual possibilities
combined. Such a frontier can be constructed using the individual frontiers
as the component blocks.

First let us define this economy-wide frontier precisely. The %
\terminology{economy-wide $PPF$} is the set of goods and services
combinations that can be produced in the economy when all available
productive resources are in use. Figure~\ref{fig:econwideppf} contains both
of the individual frontiers plus the aggregate of these, represented by the
kinked line \textit{ace}. The point on the $V$ axis, $a=27$, represents the
total amount of $V$ that could be produced if both individuals devoted all
of their time to it. The point $e=30$ on the horizontal axis is the
corresponding total for fish.


\newhtmlpage

% Figure 1.3
\begin{TikzFigure}{xscale=0.25,yscale=0.2,descwidth=25em,caption={Economy-wide PPF \label{fig:econwideppf}},description={From a, to produce Fish it is more efficient to use Zoe because her opportunity cost is less (segment ac). When Zoe is completely specialized, Amanda produces (ce). With complete specialization this economy can produce 27$V$ or 30$F$.}}
\draw [ppfcolourone,ultra thick, -] (0,9) node [black,mynode,left] {9} -- node[mynode,black,above right,pos=0.75] {Zoe's $PPF$} (18,0) node [black,mynode,below] {18};
\draw [ppfcolourtwo,ultra thick, -] (0,18) node [black,mynode,left] {18} -- node[mynode,black,above right,pos=0.25] {Amanda's $PPF$} (12,0) node [black,mynode,below] {12};
\draw [ppfcolourthree,ultra thick, -] (0,27) node [black,mynode,left] {27} node [mynode,black,above right] {a} -- node [mynode,black,above right,midway] {$PPF$ for whole economy} (18,18) node [black,mynode,above right] {c (18,18)} -- (30,0) node [black,mynode,below] {30} node [mynode,black,above right] {e};
\draw [thick, -] (0,30) node [above] {Vegetable} |- (33,0) node [right] {Fish};
\end{TikzFigure}


\begin{DefBox}
	\textbf{Economy-wide PPF}: the set of goods and services combinations that can be produced in the economy when all available productive resources are in use.
\end{DefBox}

To understand the point $c$, imagine initially that all resources are
devoted to $V$. From such a point, $a$, consider a reduction in $V$ and an
increase in $F$. The most efficient way of increasing $F$ production at the
point $a$ is to use the individual whose opportunity cost is lower. Zoe can produce
one unit of $F$ by sacrificing just 0.5 units of $V$, whereas Amanda must
sacrifice 1.5 units of $V$ to produce 1 unit of $F$. Hence, at this stage
Amanda should stick to $V$ and Zoe should devote some time to fish. In fact
as long as we want to produce more fish Zoe should be the one to do it,
until she has exhausted her time resource. This occurs after she has
produced 18$F$ and has ceased producing $V$. At this point the economy will
be producing 18$V$ and 18$F$ -- the point $c$.

From this combination, if the economy wishes to produce more fish Amanda
must become involved. Since her opportunity cost is 1.5 units of $V$ for
each unit of $F$, the next segment of the economy-wide $PPF$ must see a
reduction of 1.5 units of $V$ for each additional unit of $F$. This is
reflected in the segment $ce$. When both producers allocate all of their
time to $F$ the economy can produce 30 units. Hence the economy's $PPF$ is
the two-segment line $ace$. Since this has an outward kink, we call it
concave (rather than convex).

\newhtmlpage

As a final step consider what this $PPF$ would resemble if the economy were
composed of many persons with differing efficiencies. A
little imagination suggests (correctly) that it will have a segment for each
individual and continue to have its outward concave form. Hence, a
four-person economy in which each person had a different opportunity cost
could be represented by the segmented line $abcde$, in Figure~\ref{fig:multipersonppf}.
Furthermore, we could represent the $PPF$ of an
economy with a very large number of such individuals by a somewhat smooth $%
PPF$ that accompanies the 4-person $PPF$. The logic for its shape continues
to be the same: As we produce less $V$ and more $F$ we progressively bring
into play resources, or individuals, whose opportunity cost, in terms of
reduced $V$ is higher.

% Figure 1.4
\begin{TikzFigure}{xscale=0.3,yscale=0.3,descwidth=25em,caption={A multi-person PPF \label{fig:multipersonppf}},description={The PPF for the whole economy, abcde, is obtained by allocating productive resources most efficiently. With many individuals we can think of the PPF as the \emph{concave envelope} of the individual capabilities.}}
\draw [ppfcolourfour,ultra thick] (0,16) to [out=0,in=90] (16,0);
\draw [thick, -] (0,16) node [mynode,above right] {a} -- (6.123,14.78207) node [mynode,above right] {b} -- (11.3137,11.3137) node [mynode,above right] {c}-- (14.78207,6.123) node [mynode,above right] {d} -- (16,0) node [mynode,above right] {e};
\draw [thick, -] (0,20) node [above] {Vegetable}-- (0,0) -- (25,0) node [right] {Fish};
\end{TikzFigure}

The outputs $V$ and $F$ in our economic model require just one input --
time, but if other productive resources were required the result would be
still a concave $PPF$. Furthermore, we generally interpret the $PPF$ to
define the output possibilities \textit{when the economy is running at its
normal capacity}. In this example, we consider a work week of 36 hours to be
the `norm'. Yet it is still possible that the economy's producers might work
some additional time in exceptional circumstances, and this would increase
total production possibilities. This event would be represented by an
outward movement of the $PPF$.
