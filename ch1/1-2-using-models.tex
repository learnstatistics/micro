\section{Understanding through the use of models}\label{sec:ch1sec2}

Most students have seen an image of Ptolemy's concept of our Universe.
Planet Earth forms the centre, with the other planets and our sun revolving
around it. The ancients' anthropocentric view of the universe necessarily
placed their planet at the centre. Despite being false, this view of our
world worked reasonably well -- in the sense that the ancients could predict
celestial motions, lunar patterns and the seasons quite accurately.

More than one Greek astronomer believed that it was more natural for smaller
objects such as the earth to revolve around larger objects such as the sun,
and they knew that the sun had to be larger as a result of having studied
eclipses of the moon and sun. Nonetheless, the Ptolemaic description of the
universe persisted until Copernicus wrote his treatise ``On the Revolutions
of the Celestial Spheres'' in the early sixteenth century. And it was
another hundred years before the Church accepted that our corner of the
universe is heliocentric. During this time evidence accumulated as a result
of the work of Brahe, Kepler and Galileo. The time had come for the
Ptolemaic \textit{model} of the universe to be supplanted with a better 
\textit{model}.

All disciplines progress and develop and explain themselves using models of
reality. A \terminology{model} is a formalization of theory that facilitates
scientific inquiry. Any history or philosophy of science book will describe
the essential features of a model. First, it is a stripped down, or reduced,
version of the phenomenon that is under study. It incorporates the key
elements while disregarding what are considered to be secondary elements.
Second, it should accord with reality. Third, it should be able to make
meaningful predictions. Ptolemy's model of the known universe met these
criteria: It was not excessively complicated (for example distant stars were
considered as secondary elements in the universe and were excluded); it
corresponded to the known reality of the day, and made pretty good
predictions. Evidently not all models are correct and this was the case here.

\begin{DefBox}
	\textbf{Model}: a formalization of theory that facilitates scientific inquiry.
\end{DefBox}

\newhtmlpage

In short, models are frameworks we use to organize how we think about a
problem. Economists sometimes interchange the terms theories and models,
though they are conceptually distinct. A \terminology{theory} is a logical
view of how things work, and is frequently formulated on the basis of
observation. A model is a formalization of the essential elements of a
theory, and has the characteristics we described above. As an example of an
economic model, suppose we theorize that a household's expenditure depends
on its key characteristics: A corresponding model might specify that wealth,
income, and household size determine its expenditures, while it might ignore
other, less important, traits such as the household's neighbourhood or its
religious beliefs. The model reduces and simplifies the theory to manageable
dimensions. From such a reduced picture of reality we develop an analysis of
how an economy and its components work.

\begin{DefBox}
	\textbf{Theory}: a logical view of how things work, and is frequently formulated on the basis of observation.  
\end{DefBox}

An economist uses a model as a tourist uses a map. Any city map misses out
some detail---traffic lights and speed bumps, for example. But with careful
study you can get a good idea of the best route to take. Economists are not
alone in this approach; astronomers, meteorologists, physicists, and genetic
scientists operate similarly. Meteorologists disregard weather conditions in
South Africa when predicting tomorrow's conditions in Winnipeg. Genetic
scientists concentrate on the interactions of limited subsets of genes that
they believe are the most important for their purpose. Even with huge
computers, all of these scientists build \textit{models} that concentrate on
the essentials.