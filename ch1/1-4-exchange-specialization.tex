\newpage
\section{A model of exchange and specialization}\label{sec:ch1sec4}

\subsection*{Production and specialization}

We have two producers and two goods: Amanda and Zoe produce vegetables ($V$)
and or fish ($F$). Their production capabilities are defined in Table~\ref{table:pptwoperson} 
and in Figure~\ref{fig:absadvprod}, where the quantity
of $V$ appears on the vertical axis and the quantity of $F$ on the
horizontal axis. Zoe and Amanda each have 36-hou1r weeks and they devote that
time to producing the two goods. But their efficiencies differ: Amanda
requires two hours to produce a unit of $V$ and three hours for a unit of $F$. 
As a consequence, if she devotes all of her time to $V$ she can produce 18
units, or if she devotes all of her time to $F$ she can produce 12 units.
Or, she could share her time between the two. This environment can also be
illustrated and analyzed graphically, as in Figure~\ref{fig:absadvprod}.

\begin{Table}{caption={Production possibilities in a two-person economy \label{table:pptwoperson}},description={Each producer has a time allocation of 36 hours. By allocating total time to one activity, Amanda can produce 12$F$ or 18$V$, Zoe can produce 18$F$ or 9$V$. By splitting their time each person can also produce a combination of the two.},descwidth={30em}}
	\begin{tabu} to \linewidth {|X[1,c]X[1,c]X[1,c]X[1,c]X[1,c]|}	\hline
		\rowcolor{rowcolour}				&	Hours/	&	Hours/		&	Fish		&	Vegetable	\\[-0.1em]
		\rowcolor{rowcolour}				&	fish	&	vegetable	&	specialization	&	specialization	\\
		\textbf{Amanda}						&	3		&	2			&	12			&	18			\\
		\rowcolor{rowcolour}\textbf{Zoe}	&	2		&	4			&	18			&	9			\\	\hline
	\end{tabu}
\end{Table}

Two-dimensional graphics are a means of portraying the operation of a model,
as defined above. We will use these graphical representations throughout the
text. In this case, Amanda's production capability is represented by the
line that meets the vertical axis at 18 and the horizontal axis at 12. The
vertical point indicates that she can produce 18 units of $V$ if she
produces zero units of $F$ -- keep in mind that where $V$ has a value of 18,
Amanda has no time left for fish production. Likewise, if she devotes all of
her time to fish she can produce 12 units, since each unit requires 3 of her
36 hours. The point $F=12$ is thus another possibility for her. In addition
to these two possibilities, which we can term `specialization', she could
allocate her time to producing some of each good. For example, by dividing
her 36 hours equally she could produce 6 units of $F$ and 9 units of $V$. A
little computation will quickly convince us that different allocations of
her time will lead to combinations of the two goods that lie along a
straight line joining the specialization points.

\newhtmlpage

% Figure 1.1
\begin{TikzFigure}{xscale=0.32,yscale=0.28,descwidth=30em,caption={Absolute advantage -- production \label{fig:absadvprod}},description={Amanda's $PPF$ indicates that she can produce either 18$V$ (and zero $F$), or 12$F$ (and zero $V$), or some combination. Zoe's $PPF$ indicates she can produce either 9$V$ (and zero $F$), or 18$F$ (and zero $V$), or some combination. Amanda is more efficient in producing $V$ and Zoe is more efficient at producing $F$.}}
\draw [ppfcolourone,ultra thick,name path=zoeppf] (0,9) node [black,mynode,left] {9} -- node[mynode,black,above right,pos=0.75] {Zoe's $PPF$} (18,0) node [black,mynode,below] {18};
\draw [ppfcolourtwo,ultra thick,name path=amandappf] (0,18) node [black,mynode,left] {18} -- node[mynode,black,above right,pos=0.25] {Amanda's $PPF$} (12,0) node [black,mynode,below] {12};
\draw [thick, -] (0,20) node [above] {Vegetable} |- (25,0) node [right] {Fish};
\end{TikzFigure}


 We will call this straight
line Amanda's \terminology{production possibility frontier ($PPF$)}: It is
the combination of goods she can produce while using all of her resources --
time. She could not produce combinations of goods represented by points
beyond this line (to the top right). She could indeed produce combinations
below it (lower left) -- for example, a combination of 4 units of $V$ and 4
units of $F$; but such points would not require all of her time. The $(4,4)$
combination would require just 20 hours. In sum, points beyond this line are
not feasible, and points within it do not require all of her time resources.

\begin{DefBox}
	\textbf{Production possibility frontier (PPF)}: the combination of goods that can be produced using all of the resources available.
\end{DefBox}

\newhtmlpage

Having developed Amanda's $PPF$, it is straightforward to develop a
corresponding set of possibilities for Zoe. If she requires 4 hours to
produce a unit of $V$ and 2 hours to produce a unit of $F$, then her 36
hours will enable her to specialize in 9 units of $V$ or 18 units of $F$; or
she could produce a combination represented by the straight line that joins
these two specialty extremes.

Consider now the opportunity costs for each person. Suppose Amanda is
currently producing 18 $V$ and zero $F$, and considers producing some $F$
and less $V$. For each unit of $F$ she wishes to produce, it is evident from
her PPF that she must sacrifice 1.5 units of $V$. This is because $F$
requires 50\% more hours than $V$. Her trade-off is $1.5:1.0$. The
additional time requirement is also expressed in the intercept values: She
could give up 18 units of $V$ and produce 12 units of $F$ instead; this
again is a ratio of $1.5:1.0$. This ratio defines her opportunity cost:
The cost of an additional unit of $F$ is that 1.5 units of $V$ must be
`sacrificed'.

Applying the same reasoning to Zoe's $PPF$, her opportunity cost is $0.5:1$;
she must sacrifice one half of a unit of $V$ to free up enough time to
produce one unit of $F$. 

So we have established two things about Amanda and Zoe's production
possibilities. First, if Amanda specializes in $V$ she can produce more than
Zoe, just as Zoe can produce more than Amanda if Zoe specializes in $F$.
Second, their opportunity costs are different: Amanda must sacrifice more $V$
than Zoe in producing one more unit of $F$. The different opportunity costs
translate into potential gains for each individual.

\newhtmlpage

\subsection*{The gains from exchange}

We shall illustrate the gains that arise from specialization and exchange
graphically. Note first that if these individuals are self-sufficient, in
the sense that they consume their own production, each individual's
consumption combination will lie on their own $PPF$. For example, Amanda
could allocate half of her time to each good, and produce (and consume) 6$F$
and 9$V$. Such a point necessarily lies on her $PPF$. Likewise for Zoe. So, 
\textit{in the absence of exchange}, each individual's $PPF$ is also her %
\terminology{consumption possibility frontier ($CPF$)}. In Figure~\ref{fig:absadvprod} 
the $PPF$ for each individual is thus also her $CPF$.

\begin{DefBox}
	\textbf{Consumption possibility frontier (CPF)}: the combination of goods that can be consumed as a result of a given production choice.
\end{DefBox}

% Figure 1.2
\begin{TikzFigure}{xscale=0.32,yscale=0.27,descwidth=25em,caption={Absolute advantage -- consumption \label{fig:absadvcon}},description={With specialization and trade at a rate of $1:1$ they consume along the line joining the specialization points. If Amanda trades 8$V$ to Zoe in return for 8$F$, Amanda moves to the point A(8,10) and Zoe to Z(10,8). Each can consume more after specialization than before specialization.}}
\draw [ppfcolourthree,ultra thick,name path=totalppf] (18,0) node [black,mynode,below] {18} -- (8,10) node [black,mynode,right] {A (8,10)}-- (10,8) node [black,mynode,right] {Z (10,8)} -- node[mynode,black,above right,pos=0.4] {Consumption possibilities\\for Amanda and Zoe} (0,18) node [black,mynode,left] {18};
\draw [ppfcolourone,ultra thick,name path=zoeppf] (0,9) node [black,mynode,left] {9} -- coordinate[pos=0.7] (Z) (18,0) node [black,mynode,below] {18};
\draw [ppfcolourtwo,ultra thick,name path=amandappf] (0,18) node [black,mynode,left] {18} -- coordinate[pos=0.3] (A) (12,0) node [black,mynode,below] {12};
\draw [thick,<-,shorten <=0.5mm] (A) -- +(2,3) node [mynode,above] {Amanda's $PPF$};
\draw [thick,<-,shorten <=0.75mm] (Z) -- +(4,2) node [mynode,right] {Zoe's $PPF$};
\draw [thick, -] (0,20) node [above] {Vegetable} |- (25,0) node [right] {Fish};
\draw [fill] (10,8) circle [radius=0.15];
\draw [fill] (8,10) circle [radius=0.15];
\end{TikzFigure}

\newhtmlpage

Upon realizing that they are not equally efficient in producing the two
goods, they decide to specialize completely in producing just the single
good where they are most efficient. Amanda specializes in $V$ and Zoe in $F$. 
Next they must agree to a rate at which to exchange $V$ for $F$. Since
Amanda's opportunity cost is $1.5:1$ and Zoe's is $0.5:1$, suppose they
agree to exchange $V$ for $F$ at an intermediate rate of $1:1$. There are
many trading, or exchange, rates possible; our purpose is to illustrate that
gains are possible for \textit{both} individuals at some exchange rate. The
choice of this rate also makes the graphic as simple as possible. At this
exchange rate, 18$V$ must exchange for 18$F$. In Figure~\ref{fig:absadvcon},
this means that each individual is now able to consume along the line
joining the coordinates $(0,18)$ and $(18,0)$.\footnote{When two values, separated by 
a comma, appear in parentheses, the first value refers to the horizontal-axis
variable, and the second to the vertical-axis variable.} This is because Amanda produces 18$V$
and she can trade at a rate of $1:1$, while Zoe produces 18$F$ and trades at
the same rate of $1:1$.

The fundamental result illustrated in Figure~\ref{fig:absadvcon} is that,
as a result of specialization and trade, each individual can consume
combinations of goods that lie on a line beyond her initial consumption
possibilities.  Their consumption well-being has thus improved. For example,
suppose Amanda trades away 8$V$ to Zoe and obtains 8$F$ in return. The
points `$A$' and `$Z$' with coordinates $(8,10)$ and $(10,8)$
respectively define their final consumption. Pre-specialization, if Amanda
wished to consume 8$F$ she would have been constrained to consume 6$V$
rather than the 10$V$ now possible. Zoe benefits correspondingly.\footnote{
	In the situation we describe above one individual is absolutely more
	efficient in producing one of the goods and absolutely less efficient in the
	other. We will return to this model in Chapter~\ref{chap:internationaltrade}
	and illustrate that consumption gains of the type that arise here can also
	result if one of the individuals is absolutely more efficient in producing
	both goods, but that the degree of such advantage differs across goods.}
