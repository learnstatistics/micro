\newpage
\section*{Exercises for Chapter~\ref{chap:intro}}

\begin{Filesave}{solutions}
\subsubsection*{Chapter~\ref{chap:intro} Solutions}
\end{Filesave}

\begin{enumialphparenastyle}

\begin{econex}\label{ex:ch1ex1}
An economy has 100 identical workers. Each one can produce four cakes or three shirts, regardless of the number of other individuals producing each good.
\begin{enumerate}
	\item	How many cakes can be produced in this economy when all the workers are cooking?
	\item	How many shirts can be produced in this economy when all the workers are sewing? 
	\item	On a diagram with cakes on the vertical axis, and shirts on the horizontal axis, join these points with a straight line to form the $PPF$.
	\item	Label the inefficient and unattainable regions on the diagram.
\end{enumerate}
\begin{econsolution}
\begin{enumerate}
	\item	If all 100 workers make cakes their output is $100\times 4=400$.
	\item	If all workers make shirts their output is $100\times 3=300$.
	\item	The diagram shows the $PPF$ for this economy.
	\item	As illustrated in the diagram.
\end{enumerate}
\begin{center*}
	\begin{tikzpicture}[background color=figurebkgdcolour,use background,xscale=0.27,yscale=0.27]
	\draw [thick] (0,20) node (yaxis) [mynode1,above] {Cakes} |- (20,0) node (xaxis) [mynode1,right] {Shirts};
	\draw [ultra thick,ppfcolourone] (0,16) node [mynode,left,black] {400} -- node [mynode,above right,pos=0.8,black] {$PPF$} node [mynode,midway,below left=1em and 0em,black] {inefficient} node [mynode,midway,above right=1em and 1.5em,black] {unattainable} (12,0) node [mynode,below,black] {300};
	\end{tikzpicture}
\end{center*}
\end{econsolution}
\end{econex}

\begin{econex}\label{ex:ch1ex2}
In the table below are listed a series of points that define an economy's production possibility frontier for goods $Y$ and $X$.
\begin{Table}{}
\begin{tabu} to \linewidth {|X[1,c]X[1,c]X[1,c]X[1,c]X[1,c]X[1,c]X[1,c]X[1,c]X[1,c]X[1,c]X[1,c]X[1,c]|}\hline
	\rowcolor{rowcolour}	$Y$	&	1000	&	900		&	800		&	700		&	600		&	500		&	400		&	300		&	200		&	100		&	0			\\
	$X$	&	0			&	1600	&	2500	&	3300	&	4000	&	4600	&	5100	&	5500	&	5750	&	5900	&	6000	\\	\hline
\end{tabu}
\end{Table}
\begin{enumerate}
	\item	Plot these pairs of points to scale, on graph paper, or with the help of a spreadsheet.
	\item	Given the shape of this $PPF$ is the economy made up of individuals who are similar or different in their production capabilities?
	\item	What is the opportunity cost of producing 100 more $Y$ at the combination $(X=5500,Y=300)$.
	\item	Suppose next there is technological change so that at every output level of good $Y$ the economy can produce 20 percent more $X$. Enter a new row in the table containing the new values, and plot the new $PPF$.
\end{enumerate}
\begin{econsolution}
\begin{enumerate}
	\item	The $PPF$ is curved outwards with intercepts of 1000 on the Thinkpod axis and 6000 on the iPad axis. Each point on the $PPF$ shows one combination of outputs.
	\item	Different.
	\item	400 $X$.
	\item	The new $PPF$ in the diagram has the same Thinkpod intercept, 1000, but a new iPad intercept of 7200.
\end{enumerate}
\begin{center*}
	\begin{tikzpicture}[background color=figurebkgdcolour,use background,xscale=0.23,yscale=0.23]
	\draw [thick] (0,20) node (yaxis) [mynode1,above] {Thinkpods} |- (30,0) node (xaxis) [mynode1,right] {iPads};
	\draw [ultra thick,dashed,ppfcolourtwo,name path=ppf2] (0,18) to[out=0,in=105] (28,0) node [mynode,below,black] {7200};
	\draw [ultra thick,ppfcolourone,name path=ppf1] (0,18) node [mynode,left,black] {1000} to[out=0,in=105] (24,0) node [mynode,below,black] {6000};
	\path [name path=combo] (0,10.8) -- +(30,0);
	\path [name path=arrow] (0,3) -- +(30,0);
	\draw [name intersections={of=combo and ppf1, by=E}]
	[dotted,thick] (yaxis |- E) node [mynode,left] {600} -| (xaxis -| E) node [mynode,below] {4000};
	\draw [name intersections={of=arrow and ppf1, by=i1},name intersections={of=arrow and ppf2, by=i2}]
	[->,thick,shorten >=1mm,shorten <=1mm] (i1) -- (i2);
	\end{tikzpicture}
\end{center*}
\end{econsolution}
\end{econex}

\begin{econex}\label{ex:ch1ex3}
Using the $PPF$ that you have graphed using the data in Exercise~\ref{ex:ch1ex2}, determine if the following combinations are attainable or not: $(X=3000,Y=720)$, $(X=4800,Y=480)$.
\begin{econsolution}
	By examining the opportunity cost in the region where the combinations are defined, and by assuming a linear trade-off between each set of combinations, it can be seen that the first combination in the table is feasible, but not the second combination.
	
\end{econsolution}
\end{econex}

\begin{econex}\label{ex:ch1ex4}
You and your partner are highly efficient people. You can earn \$20 per hour in the workplace; your partner can earn \$30 per hour.
\begin{enumerate}
	\item	What is the opportunity cost of one hour of leisure for you?
	\item	What is the opportunity cost of one hour of leisure for your partner?
	\item	Now consider what a $PPF$ would look like: You can produce/consume two things, leisure and income. Since income buys things you can think of the $PPF$ as having these two 'products' -- leisure and consumption goods/services. So, with leisure on the horizontal axis and income in dollars is on the vertical axis, plot your $PPF$. You can assume that you have 12 hours per day to allocate to either leisure or income. [\textit{Hint}: the leisure axis will have an intercept of 12 hours. The income intercept will have a dollar value corresponding to where all hours are devoted to work.]
	\item	Draw the $PPF$ for your partner.
\end{enumerate}
\begin{econsolution}
\begin{enumerate}
	\item	\$20.
	\item	\$30.
	\item	See lower curve on diagram.
	\item	See upper curve on diagram.
\end{enumerate}
\begin{center*}
	\begin{tikzpicture}[background color=figurebkgdcolour,use background,xscale=0.27,yscale=0.27]
	\draw [thick] (0,20) node (yaxis) [mynode1,above] {\$ Inc} |- (20,0) node (xaxis) [mynode1,right] {Leisure};
	\draw [ultra thick,ppfcolourone] (0,10) node [mynode,left,black] {\$240} -- (15,0);
	\draw [ultra thick,ppfcolourtwo] (0,15) node [mynode,left,black] {\$360} -- (15,0) node [mynode,below,black] {12};
	\end{tikzpicture}
\end{center*}
\end{econsolution}
\end{econex}

\begin{econex}\label{ex:ch1ex5}
Louis and Carrie Anne are students who have set up a summer business in their neighbourhood. They cut lawns and clean cars. Louis is particularly efficient at cutting the grass -- he requires one hour to cut a typical lawn, while Carrie Anne needs one and one half hours. In contrast, Carrie Anne can wash a car in a half hour, while Louis requires three quarters of an hour.
\begin{enumerate}
	\item If they decide to specialize in the tasks, who should cut the grass and who should wash cars?
	\item If they each work a twelve hour day, how many lawns can they cut and how many cars can they wash if they each specialize in performing the task where they are most efficient?
	\item Illustrate the $PPF$ for each individual where lawns are on the horizontal axis and car washes on the vertical axis, if each individual has twelve hours in a day.
\end{enumerate}
\begin{econsolution}
\begin{enumerate}
	\item	Louis has an advantage in cutting the grass while Carrie Anne should wash cars.
	\item	If they each work a twelve-hour day, between them they can cut 12 lawns and wash 24 cars.
\end{enumerate}
\end{econsolution}
\end{econex}

\begin{econex}\label{ex:ch1ex6}
Continuing with the same data set, suppose Carrie Anne's productivity improves so that she can now cut grass as efficiently as Louis; that is, she can cut grass in one hour, and can still wash a car in one half of an hour.
\begin{enumerate}
	\item	In a new diagram draw the $PPF$ for each individual.
	\item	In this case does specialization matter if they are to be as productive as possible as a team?
	\item	Draw the PPF for the whole economy, labelling the intercepts and the `kink' point coordinates.
\end{enumerate}
\begin{econsolution}
\begin{enumerate}
	\item	Carrie Anne's lawn intercept is now 12 rather than 8.
	\item	Yes, specialization still matters because C.A. is more efficient at cars.
	\item	The new coordinates will be 39 on the vertical axis, 24 on the horizontal axis and the kink point is the same.
\end{enumerate}
\end{econsolution}
\end{econex}

\begin{econex}\label{ex:ch1ex7}
Going back to the simple $PPF$ plotted for Exercise~\ref{ex:ch1ex1} where each of 100 workers can produce either four cakes or three shirts, suppose a recession reduces demand for the outputs to 220 cakes and 129 shirts.
\begin{enumerate}
	\item	Plot this combination of outputs in the diagram that also shows the $PPF$.
	\item	How many workers are needed to produce this output of cakes and shirts?
	\item	What percentage of the 100 worker labour force is unemployed?
\end{enumerate}
\begin{econsolution}
\begin{enumerate}
	\item	220 cakes requires 55 workers, the remaining 45 workers can produce 135 shirts. Hence this combination lies inside the $PPF$ described in Exercise~\ref{ex:ch1ex1}.
	\item	98 workers.
	\item	2\%.
\end{enumerate}
\end{econsolution}
\end{econex}

\end{enumialphparenastyle}
