\newpage
	\section*{Key Terms}
\begin{keyterms}
\textbf{Macroeconomics} studies the economy as system in which feedback among sectors determine national output, employment and prices.

\textbf{Microeconomics} is the study of individual behaviour in the context of scarcity.

\textbf{Mixed economy}: goods and services are supplied both by private suppliers and government.

\textbf{Model} is a formalization of theory that facilitates scientific inquiry.

\textbf{Theory} is a logical view of how things work, and is frequently formulated on the basis of observation.

\textbf{Opportunity cost} of a choice is what must be sacrificed when a choice is made.

\textbf{Production possibility frontier (PPF)} defines the combination of goods that can be produced using all of the resources available.

\textbf{Consumption possibility frontier (CPF)}: the combination of goods that can be consumed as a result of a given production choice.

\textbf{Economy-wide PPF} is the set of goods combinations that can be produced in the economy when all available productive resources are in use.

\textbf{Productivity of labour} is the output of goods and services per worker.

\textbf{Capital stock}: the buildings, machinery, equipment and software used in producing goods and services.

\textbf{Full employment output} $Y_c=(\text{number of workers at full employment})\times(\text{output per worker})$.

\textbf{Recession}: when output falls below the economy's capacity output.

\textbf{Boom}: a period of high growth that raises output above normal capacity output.
\end{keyterms}