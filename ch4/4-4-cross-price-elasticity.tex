\section{Cross-price elasticities -- cable or satellite}\label{sec:ch4sec4}

The price elasticity of demand tells us about consumer responses to price
changes in different regions of the demand curve, holding constant all other
influences. One of those influences is the price of other goods and
services. A \terminology{cross-price elasticity} indicates how demand is
influenced by changes in the prices of other products.

\begin{DefBox}
The \textbf{cross-price elasticity of demand} is the percentage change in the quantity demanded of a product divided by the percentage change in the price of another.
\end{DefBox}

We write the cross price elasticity of the demand for $x$ due to a change in
the price of $y$ as

\begin{equation*}
\varepsilon_{d(x,y)}=\frac{\text{percentage change in quantity demanded of x%
}}{\text{percentage change in price of good y}}=\frac{\%\Delta Q_{x}}{%
\%\Delta P_{y}} 
\end{equation*}

For example, if the price of cable-supply internet services declines, by how
much will the demand for satellite-supply services change? The cross-price
elasticity may be positive or negative. These particular goods are clearly 
\textit{substitutable}, and this is reflected in a \textit{positive} value
of this cross-price elasticity: The percentage change in satellite
subscribers will be negative in response to a decline in the price of cable;
a negative divided by a negative is positive. In contrast, a change in the
price of tablets or electronic readers should induce an opposing change in
the quantity of e-books purchased: Lower tablet prices will induce greater
e-book purchases. In this case the price and quantity movements are in
opposite directions and the elasticity is therefore negative -- the goods are
complements.