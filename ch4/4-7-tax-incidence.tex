\section{Elasticities and tax incidence}\label{sec:ch4sec7}

Elasticity values are critical in determining the impact of a government's
taxation policies. The spending and taxing activities of the government
influence the use of the economy's resources. By taxing cigarettes, alcohol
and fuel, the government can restrict their use; by taxing income, the
government influences the amount of time people choose to work. Taxes have a
major impact on almost every sector of the Canadian economy.

To illustrate the role played by demand and supply elasticities in tax
analysis, we take the example of a sales tax. These can be of the \textit{%
specific} or \textit{ad valorem} type. A \textit{specific} tax involves a
fixed dollar levy per unit of a good sold (e.g., \$10 per airport
departure). An \textit{ad valorem} tax is a percentage levy, such as
Canada's Goods and Services tax (e.g., 5 percent on top of the retail price
of goods and services). The impact of each type of tax is similar, and we
will use the specific tax in our example below.

A layperson's view of a sales tax is that the tax is borne by the consumer.
That is to say, if no sales tax were imposed on the good or service in
question, the price paid by the consumer would be the same net of tax price
as exists when the tax is in place. Interestingly, this is not always the
case. The study of the \terminology{incidence of taxes} is the study of who
really bears the tax burden, and this in turn depends upon supply and demand
elasticities.

\begin{DefBox}
\textbf{Tax Incidence} describes how the burden of a tax is shared between buyer and seller.
\end{DefBox}

\newhtmlpage

Consider Figures~\ref{fig:taxelasticsupply} and \ref{fig:taxinelasticsupply},
which define an imaginary market for inexpensive wine. Let us suppose
that, without a tax, the equilibrium price of a bottle of wine is \$5, and 
$Q_0$ is the equilibrium quantity traded. The pre-tax equilibrium is at
the point A. The government now imposes a specific tax of \$4 per
bottle. The impact of the tax is represented by an upward shift in supply of 
\$4: Regardless of the price that the consumer pays, \$4 of that price
must be remitted to the government. As a consequence, the price paid to the
supplier must be \$4 less than the consumer price, and this is represented
by twin supply curves: One defines the price at which the supplier is
willing to supply, and the other is the tax-inclusive supply curve that the
consumer faces.

% Figure 4.7	fig:taxelasticsupply
\begin{TikzFigure}{xscale=0.5,yscale=0.4,descwidth=25em,caption={Tax incidence with elastic supply \label{fig:taxelasticsupply}},description={The imposition of a specific tax of \$4 shifts the supply curve vertically by \$4. The final price at B ($P_t$) increases by \$3 over the equilibrium price at A. At the new quantity traded, $Q_t$, the supplier gets \$4 per unit ($P_{ts}$), the government gets \$4 also and the consumer pays \$8. The greater part of the incidence is upon the buyer, on account of the relatively elastic supply curve: His price increases by \$3 of the \$4 tax.}}
% demand function
\draw [demandcolour,ultra thick,name path=demand] (0,14) -- (9.333,0) node [mynode,above right,pos=0.1,black] {$D$};
% supply functions
\draw [supplycolour,ultra thick,name path=St] (0,6) -- (14,13) node [black,mynode,right] {$S_t$};
\draw [supplycolour,ultra thick,name path=S] (0,2) -- (14,9) node [black,mynode,right] {$S$};
% axes
\draw [thick, -] (0,15) node (yaxis) [above] {Price} |- (15,0) node (xaxis) [right] {Quantity};
% intersection of demand and supply lines
\draw [name intersections={of=demand and St, by=B},name intersections={of=demand and S, by=A}]
	[dotted,thick] (yaxis |- B) node [mynode,left] {$P_t=8$} -- (B) node [mynode,above] {B} -- (xaxis -| B) node [mynode,below] {$Q_t$}
	[dotted,thick] (yaxis |- A) node [mynode,left] {$P_0=5$} -- (A) node [mynode,above] {A} -- (xaxis -| A) node [mynode,below] {$Q_0$};
% path for P_ts=4 line
\path [name path=Pts4] (0,4) -- (15,4);
% intersection of S with Pts4 line
\draw [name intersections={of=S and Pts4, by=C}]
	[dotted,thick] (yaxis |- C) node [mynode,left] {$P_{ts}=4$} -- (C) node [mynode,below right] {C};
% path for $4=tax arrow
\path [name path=taxline] (12,0) -- (12,15);
% intersection of taxline with supply lines and resulting arrow
\draw [name intersections={of=S and taxline, by=notax},name intersections={of=St and taxline, by=withtax}]
	[<->,thick,shorten <=1mm,shorten >=1mm] (notax) -- node [mynode,right,midway] {\$4$=$tax} (withtax);
\end{TikzFigure}

\newhtmlpage

The introduction of the tax in Figure~\ref{fig:taxelasticsupply} means that
consumers now face the supply curve $S_t$. The new equilibrium is at point
B. Note that the price has increased by less than the full amount of the
tax---in this example it has increased by \$3. This is because the reduced
quantity at B is provided at a lower supply price: The supplier is willing
to supply the quantity $Q_t$ at a price defined by C (\$4), which is
lower than A (\$5).

So what is the incidence of the \$4 tax? Since the market price has
increased from \$5 to \$8, and the price obtained by the supplier has
fallen by \$1, we say that the incidence of the tax falls mainly on the
consumer: The price to the consumer has risen by three dollars and the price
received by the supplier has fallen by just one dollar.

Consider now Figure~\ref{fig:taxinelasticsupply}, where the supply curve is
less elastic, and the demand curve is unchanged. Again the supply curve must
shift upward with the imposition of the $\$4$ specific tax. But here the
price received by the supplier is lower than in Figure~\ref{fig:taxelasticsupply},
and the price paid by the consumer does not rise as
much -- the incidence is different. The consumer faces a price increase that
is one-quarter, rather than three-quarters, of the tax value. The supplier
faces a lower supply price, and bears a higher share of the tax.

% Figure 4.8	fig:taxinelasticsupply
\begin{TikzFigure}{xscale=0.5,yscale=0.4,descwidth=25em,caption={Tax incidence with inelastic supply \label{fig:taxinelasticsupply}},description={The imposition of a specific tax of \$4 shifts the supply curve vertically by \$4. The final price at B ($P_t$) increases by \$2 over the no-tax price at A. At the new quantity traded, $Q_t$, the supplier gets \$3 per unit ($P_{ts}$), the government gets \$4 also and the consumer pays \$7. The incidence is shared equally by suppliers and demanders.}}
% demand function
\draw [demandcolour,ultra thick,name path=demand] (0,14) -- (9.333,0) node [mynode,above right,pos=0.1,black] {$D$};
% supply functions
\draw [supplycolour,ultra thick,name path=St] (2.333,0) -- (9.333,14) node [black,mynode,above] {$S_t$};
\draw [supplycolour,ultra thick,name path=S] (4.333,0) -- (11.333,14) node [black,mynode,above] {$S$};
% axes
\draw [thick, -] (0,15) node (yaxis) [above] {Price} |- (15,0) node (xaxis) [right] {Quantity};
% intersection of demand and supply
\draw [name intersections={of=demand and S, by=A},name intersections={of=demand and St, by=B}]
	[dotted,thick] (yaxis |- A) node [mynode,left] {$P_0=5$} -- (A) node [mynode,above] {A} -- (xaxis -| A) node [mynode,below] {$Q_0$}
	[dotted,thick] (yaxis |- B) node [mynode,left] {$P_t=7$} -- (B) node [mynode,above] {B} -- (xaxis -| B) node [mynode,below] {$Q_t$};
% path for P_ts=2 line
\path [name path=Pts2] (0,2) -- (15,2);
% intersection of Pts2 line and S line
\draw [name intersections={of=Pts2 and S, by=C}]
	[dotted,thick] (yaxis |- C) node [mynode,left] {$P_{ts}=3$} -- (C) node [mynode,below right] {C};
% path to create $4=tax arrow
\path [name path=taxline] (8,0) -- (8,15);
% intersection of taxline with supply lines
\draw [name intersections={of=taxline and S, by=notax},name intersections={of=taxline and St, by=withtax}]
	[<->,thick,shorten >=1mm,shorten <=1mm] (notax) -- node [mynode,right=0cm and 0.4cm,midway] {\$4$=$tax} (withtax);
\end{TikzFigure}

\newhtmlpage

We can conclude from this example that, for any given demand, \textit{the
more elastic is supply, the greater is the price increase} in response to a
given tax. Furthermore, a \textit{more elastic supply curve} means that the 
\textit{incidence falls more on the consumer}; while a \textit{less elastic
supply} curve means the \textit{incidence falls more on the supplier}. This
conclusion can be verified by drawing a third version of Figure~\ref{fig:taxelasticsupply}
and \ref{fig:taxinelasticsupply}, in which the supply
curve is horizontal -- perfectly elastic. When the tax is imposed the price
to the consumer increases by the full value of the tax, and the full
incidence falls on the buyer. While this case corresponds to the layperson's
intuition of the incidence of a tax, economists recognize it as a special
case of the more general outcome, where the incidence falls on both the
supply side and the demand side.

These are key results in the theory of taxation. It is equally the case that 
\textit{the incidence of the tax depends upon the demand elasticity}. In
Figure~\ref{fig:taxelasticsupply} and \ref{fig:taxinelasticsupply} we used
the same demand curve. However, it is not difficult to see that, if we were
to redo the exercise with a demand curve of a different elasticity, the
incidence would not be identical. At the same time, the general result on
supply elasticities still holds. We will return to this material in Chapter~%
\ref{chap:welfare}.

\newhtmlpage

\subsection*{Statutory incidence}

In the above example the tax is analyzed by means of shifting the supply
curve. This implies that the supplier is obliged to charge the consumer a
tax and then return this tax revenue to the government. But suppose the
supplier did not bear the obligation to collect the revenue; instead the
buyer is required to send the tax revenue to the government, as in the case
of employers who are required to deduct income tax from their employees' pay
packages (the employers here are the demanders). If this were
the case we could analyze the impact of the tax by reducing the market 
\textit{demand} curve by the \$4. This is because the demand curve
reflects what buyers are willing to pay, and when suppliers are paid in
the presence of the tax they will be paid the buyers' demand price minus the
tax that the buyers must pay. It is not difficult to show that whether we
move the supply curve upward (to reflect the responsibility of the supplier
to pay the government) or move the demand curve downward, the outcome is the
same -- in the sense that the same price and quantity will be traded in each
case. Furthermore the incidence of the tax, measured by how the price change
is apportioned between the buyers and sellers is also unchanged.

\subsection*{Tax revenues and tax rates}

It is useful to relate elasticity values to the policy question of the
impact of higher or lower taxes on government tax revenue. Consider a
situation in which a tax is already in place and the government considers
increasing the rate of tax. Can an understanding of elasticities inform us
on the likely outcome? The answer is yes. Suppose that at the initial 
tax-inclusive price demand is inelastic. We know immediately that a tax rate
increase that increases the price must increase total expenditure. Hence the
outcome is that the government will get a higher share of an increased total
expenditure. In contrast, if demand is elastic at the initial tax-inclusive
price a tax rate increase that leads to a higher price will \textit{decrease}
total expenditure. In this case the government will get a larger share of a
smaller pie -- not as valuable from a tax-revenue standpoint as a larger
share of a larger pie.