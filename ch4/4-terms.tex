\newpage
	\section*{Key Terms}
\begin{keyterms}
\textbf{Price elasticity of demand} is measured as the percentage change in quantity demanded, divided by the percentage change in price.

\textbf{Demand is elastic} if the price elasticity is greater than unity. It is \textbf{inelastic} if the value lies between unity and 0. It is \textbf{unit elastic} if the value is exactly one.

\textbf{Cross-price elasticity of demand} is the percentage change in the quantity demanded of a product divided by the percentage change in the price of another.

\textbf{Income elasticity of demand} is the percentage change in quantity demanded divided by a percentage change in income.

\textbf{Luxury good} or service is one whose income elasticity equals or exceeds unity.

\textbf{Necessity} is one whose income elasticity is greater than zero and is less than unity.

\textbf{Inferior goods} have a negative income elasticity.

\textbf{Elasticity of supply} is defined as the percentage change in quantity supplied divided by the percentage change in price.

\textbf{Tax Incidence} describes how the burden of a tax is shared between buyer and seller.
\end{keyterms}