\section{Price responsiveness of demand}\label{sec:ch4sec1}

Put yourself in the position of an entrepreneur. One of your many challenges
is to price your product appropriately. You may be Michael Dell choosing a
price for your latest computer, or the local restaurant owner pricing your
table d'h\^{o}te, or you may be pricing your part-time snow-shoveling
service. A key component of the pricing decision is to know how \textit{%
responsive} your market is to variations in your pricing. How we measure
responsiveness is the subject matter of this chapter.

We begin by analyzing the responsiveness of consumers to price changes. For
example, consumers tend not to buy much more or much less food in response
to changes in the general price level of food. This is because food is a
pretty basic item for our existence. In contrast, if the price of textbooks
becomes higher, students may decide to search for a second-hand copy, or
make do with lecture notes from their friends or downloads from the course
web site. In the latter case students have ready alternatives to the new
text book, and so their expenditure patterns can be expected to reflect
these options, whereas it is hard to find alternatives to food. In the case
of food consumers are not very responsive to price changes; in the case of
textbooks they are. The word `elasticity' that appears in this chapter title
is just another term for this concept of responsiveness. Elasticity has many
different uses and interpretations, and indeed more than one way of being
measured in any given situation. Let us start by developing a suitable
numerical measure.

The slope of the demand curve suggests itself as one measure of
responsiveness: If we lowered the price of a good by \$1, for example, how
many more units would we sell? The difficulty with this measure is that it
does not serve us well when comparing different products. One dollar may be
a substantial part of the price of your morning coffee and croissant, but
not very important if buying a computer or tablet. Accordingly, when goods
and services are measured in different units (croissants versus tablets), or
when their prices are very different, it is often best to use a \textit{%
percentage} change measure, which is \textit{unit-free}.

The \terminology{price elasticity of demand} is measured as the percentage
change in quantity demanded, divided by the percentage change in price.
Although we introduce several other elasticity measures later, when
economists speak of the demand elasticity they invariably mean the price
elasticity of demand defined in this way.

\begin{DefBox}
The \textbf{price elasticity of demand} is measured as the percentage change in quantity demanded, divided by the percentage change in price.
\end{DefBox}

\newhtmlpage

The price elasticity of demand can be written in different forms. We will
use the Greek letter epsilon, $\varepsilon$, as a shorthand symbol, with a
subscript $d$ to denote demand, and the capital delta, $\Delta$, to denote a
change. Therefore, we can write
\begin{equation*}
\text{Price elasticity of demand}=\varepsilon_{d}=\frac{\text{Percentage
change in quantity demanded}}{\text{Percentage change in price}} 
\end{equation*}%
or, using a shortened expression, 
\begin{equation}\label{eq:priceelastdemand}
\varepsilon_{d}=\frac{\%\Delta Q}{\%\Delta P}
\end{equation}

Calculating the value of the elasticity is not difficult. If we are told
that a 10 percent price increase reduces the quantity demanded by 20
percent, then the elasticity value is $-20\%/10\%=-2.$ The negative sign
denotes that price and quantity move in opposite directions, but for brevity
the negative sign is often omitted.

Consider now the data in Table~\ref{table:gaselastrev} and the accompanying
Figure~\ref{fig:elasticitylineardemand}. These data reflect the demand
relation for natural gas that we introduced in Chapter~\ref{chap:classical}.
Note first that, when the price and quantity change, we must decide what 
\textit{reference price and quantity} to use in the percentage change
calculation in the definition above. We could use the initial or final
price-quantity combination, or an average of the two. Each choice will yield
a slightly different numerical value for the elasticity. The best convention
is to \textit{use the midpoint of the price values and the corresponding
midpoint of the quantity values}. This ensures that the elasticity value is
the same regardless of whether we start at the higher price or the lower
price. Using the subscript 1 to denote the initial value and 2 the final
value:
\begin{equation*}
\text{Average quantity }\overline{Q}=(Q_{1}+Q_{2})/2 
\end{equation*}%
\begin{equation*}
\text{Average price }\overline{P}=(P_{1}+P_{2})/2 
\end{equation*}

\newhtmlpage

% Table 4.1
\begin{Table}{caption={The demand for natural gas: Elasticities and revenue \label{table:gaselastrev}},description={Elasticity calculations are based upon \$2 price changes.},descwidth={19em}}
\begin{tabu} to \linewidth {|X[1,c]X[1,c]X[1,c]X[1,c]|} \hline 
\rowcolor{rowcolour}\textbf{Price (\$)} & \textbf{Quantity} & \textbf{Elasticity} & \textbf{Total} \\[-0.1em]
\rowcolor{rowcolour}	&	\textbf{demanded}	&	\textbf{value}	&	\textbf{revenue (\$)}	\\	\hline
						10 & 0 &  & 0 \\ 
\rowcolor{rowcolour}	9 & 1 & -9.0 & 9 \\ 
						8 & 2 &  & 16 \\ 
\rowcolor{rowcolour}	7 & 3 & -2.33 & 21 \\ 
						6 & 4 &  & 24 \\ 
\rowcolor{rowcolour}	5 & 5 & -1.0 & 25 \\ 
						4 & 6 &  & 24 \\ 
\rowcolor{rowcolour}	3 & 7 & -0.43 & 21 \\ 
						2 & 8 &  & 16 \\ 
\rowcolor{rowcolour}	1 & 9 & -0.11 & 9 \\ 
						0 & 10 &  & 0 \\ \hline 
\end{tabu}
\end{Table}

% Figure 4.1	fig:elasticitylineardemand
\begin{TikzFigure}{xscale=0.52,yscale=0.42,descwidth=25em,caption={Elasticity variation with linear demand \label{fig:elasticitylineardemand}},description={In the high-price region of the demand curve the elasticity takes on a high value. At the midpoint of a linear demand curve the elasticity takes on a value of one, and at lower prices the elasticity value continues to fall.}}
% demand line
\draw [demandcolour,ultra thick,name path=demand] (0,10) node [black,mynode,left] {$P=10$} -- node [mynode,above right, pos=0.2,black] {High elasticity range (elastic)} node [mynode,above right,pos=0.8,black] {Low elasticity range (inelastic)} (10,0) node [black,mynode,below] {$Q=10$};
% paths to intersect with demand and create dotted lines
\path [name path=price8] (0,8) -- (10,8);
\path [name path=price5] (0,5) -- (10,5);
\path [name path=price2] (0,2) -- (10,2);
% axes
\draw [thick, -] (0,15) node (yaxis) [above] {Price} |- (15,0) node (xaxis) [right] {Quantity};
% intersection of paths and demand
\draw [name intersections={of=price8 and demand, by=HighE},name intersections={of=price5 and demand, by=MidE},name intersections={of=price2 and demand, by=LowE}]
	[dotted,thick] (yaxis |- HighE) node [mynode,left] {8} -| (xaxis -| HighE) node [mynode,below] {2}
	[dotted,thick] (yaxis |- MidE) node [mynode,left] {5} -| (xaxis -| MidE) node [mynode,below] {5}
	[dotted,thick] (yaxis |- LowE) node [mynode,left] {2} -| (xaxis -| LowE) node [mynode,below] {8};
% arrow pointing to MidE
\draw [<-,thick,shorten <=1mm] (MidE) -- +(2,2) node [mynode,right] {Midpoint of $D$: $\varepsilon=-1$};
\end{TikzFigure}


\newhtmlpage

Using this rule, consider now the value of $\varepsilon_{d}$ when price
drops from \$10.00 to \$8.00. The change in price is \$2.00 and the
average price is therefore $\$9.00$ $[=(\$10.00+\$8.00)/2]$. On the quantity
side, demand goes from zero to 2 units (measured in thousands of cubic
feet), and the average quantity demanded is therefore $(0+2)/2=1$. Putting
these numbers into the formula yields:
\begin{equation*}
\varepsilon_{d}=\frac{(Q_{2}-Q_{1})/\overline{Q}}{(P_{2}-P_{1})/\overline{P}
}=\frac{(2/1)}{-(2/9)}=-\left(\frac{2}{1}\right)\times\left(\frac{9}{2}\right)=-9. 
\end{equation*}

Note that the price has declined in this instance and thus the change in
price is negative. Continuing down the table in this fashion yields the full
set of elasticity values in the third column.

The demand elasticity is said to be \textit{high} if it is a large negative
number; the large number denotes a high degree of sensitivity. Conversely,
the elasticity is \textit{low} if it is a small negative number. High and
low refer to the size of the number, ignoring the negative sign. The term
\textit{arc elasticity} is also used to define what we have just
measured, indicating that it defines consumer responsiveness over a segment
or \textit{arc} of the demand curve.

It is helpful to analyze this numerical example by means of the
corresponding demand curve that is plotted in Figure~\ref{fig:elasticitylineardemand},
and which we used in Chapter~\ref{chap:classical}. It is a
straight-line demand curve; but, despite this, the elasticity is not
constant. At high prices the elasticity is high; at low prices it is low.
The intuition behind this pattern is as follows. When the price is high, a
given price change represents a small \textit{percentage} change, because
the average price in the price-term denominator is large. At high prices the
quantity demanded is small and therefore the percentage quantity change
tends to be large due to the small quantity value in its denominator. In
sum, at high prices the elasticity is large. By the same reasoning, at low
prices the elasticity is small.

We can carry this reasoning one step further to see what happens when the
demand curve intersects the axes. At the horizontal axis the average price
is tending towards zero. Since this extremely small value appears in the
denominator of the price term it means that the price term as a whole is
extremely large. Accordingly, with an extremely large value in the
denominator of the elasticity expression, that whole expression is tending
towards a zero value. By the same reasoning the elasticity value at the
vertical intercept is tending towards an infinitely large value.

\newhtmlpage

\subsection*{Extreme cases}

The elasticity decreases in going from high prices to low prices. This is
true for most non-linear demand curves also. Two exceptions are when the
demand curve is horizontal and when it is vertical.

When the demand curve is vertical, no quantity change results from a change
in price from $P_1$ to $P_2$, as illustrated in Figure~\ref{fig:limitingcasepriceelasticity}
using the demand curve $D_v$.
Therefore, the numerator in Equation~\ref{eq:priceelastdemand} is zero,
and the elasticity has a zero value.

% Figure 4.2	fig:limitingcasepriceelasticity
\begin{TikzFigure}{xscale=0.5,yscale=0.4,descwidth=25em,caption={Limiting cases of price elasticity \label{fig:limitingcasepriceelasticity}},description={When the demand curve is vertical ($D_v$), the elasticity is zero: A change in price from $P_1$ to $P_2$ has no impact on the quantity demanded because the numerator in the elasticity formula has a zero value. When $D$ becomes more horizontal the elasticity becomes larger and larger at $P_1$, eventually becoming infinite.}}
% axes
\draw [thick, -] (0,15) node (yaxis) [above] {Price} |- (15,0) node (xaxis) [right] {Quantity};
% demand lines
\draw [demandcolour,ultra thick,name path=demandv] (7,0) node [black,mynode,below] {$Q_0$} -- (7,14) node [black,mynode,above] {$D_v$};
\draw [demandcolour,ultra thick,name path=demandh] (0,5) node [black,mynode,left] {$P_1$} -- (14,5) node [black,mynode,right] {$D_h$};
\draw [demandcolour,ultra thick,dashed,name path=demandprime] (0,6) -- (14,4) node [black,mynode,right] {$D'$};
% path for P_2 and intersection with D_v for dotted line
\path [name path=p2line] (0,10) -- (15,10);
\draw [name intersections={of=p2line and demandv, by=Q0}]
	[dotted,thick] (yaxis |- Q0) node [mynode,left] {$P_2$} -- (Q0);
% path to draw arrows to D_h and D'
\path [name path=Eline] (11,0) -- (11,15);
% arrows from Infinite elasticity and Large elasticity to D_h and D', respectively
\draw [name intersections={of=Eline and demandh, by=arrowh},name intersections={of=Eline and demandprime, by=arrowprime}]
	[<-,thick,shorten <=1mm,shorten >=-1mm] (arrowh) -- +(0,1) node [mynode,above] {Infinite\\elasticity};
\draw [<-,thick,shorten <=1mm,shorten >=-1mm] (arrowprime) -- +(0,-1) node [mynode,below] {Large\\elasticity};
% arrow from Zero elasticity to D_v line
\draw [<-,thick,shorten <=1mm] ([yshift=2cm]Q0) -- +(3,0) node [mynode,right] {Zero\\elasticity};
\end{TikzFigure}

\newhtmlpage

In the horizontal case, we say that the elasticity is \textit{infinite},
which means that any percentage price change brings forth an infinite
quantity change! This case is also illustrated in Figure~\ref{fig:limitingcasepriceelasticity}
using the demand curve $D_h$. As with the
vertical demand curve, this is not immediately obvious. So consider a demand
curve that is almost horizontal, such as $D^{\prime}$ instead of $D_h$. In
this instance, we can achieve large changes in quantity demanded by
implementing very small price changes. In terms of Equation~\ref{eq:priceelastdemand},
the numerator is large and the denominator small,
giving rise to a large elasticity. Now imagine that this demand curve
becomes ever more elastic (horizontal). The same quantity response can be
obtained with a smaller price change, and hence the elasticity is larger.
Pursuing this idea, we can say that, \textit{as the demand curve becomes
ever more elastic, the elasticity value tends towards infinity.}

A \textit{non-linear demand curve} is illustrated in Figure~\ref{fig:nonlineardemand}.
If price increases from $P_0$ to $P_1$, the
corresponding quantity change is given by $(Q_0-Q_1)$. When the price
declines to $P_2$ the quantity increases from $Q_0$ to $Q_2$. When
statisticians study data to determine how responsive purchases are to price
changes they do not always find a linear relationship between price and
quantity. But a linear relationship is frequently a good approximation or
representation of actual data and we will continue to analyze responsiveness
in a linear framework in this chapter.

% Figure 4.3	fig:nonlineardemand
\begin{TikzFigure}{xscale=0.5,yscale=0.4,descwidth=25em,caption={Non-linear demand curves \label{fig:nonlineardemand}},description={When the demand curve is non-linear the slope changes with the price. Hence, equal price changes do not lead to equal quantity changes: The quantity change associated with a change in price from $P_0$ to $P_1$ is smaller than the change in quantity associated with the same change in price from $P_0$ to $P_2$.}}
% axes
\draw [thick] (0,15) node (yaxis) [above] {Price} |- (15,0) node (xaxis) [right] {Quantity};
% demand curve
\draw [demandcolour,ultra thick,domain=1:15,name path=demand] plot (\x, {14/\x});
% paths to intersect with demand curve
\path [name path=p2line] (0,2) -- (15,2);
\path [name path=p0line] (0,4.5) -- (15,4.5);
\path [name path=p1line] (0,7) -- (15,7);
% intersection of paths with demand curve
\draw [name intersections={of=p2line and demand, by=C},name intersections={of=p0line and demand, by=A},name intersections={of=p1line and demand, by=B}]
	[dotted,thick] (yaxis |- C) node [mynode,left] {$P_2$} -- (C) node [mynode,above right] {C} -- (xaxis -| C) node [mynode,below] {$Q_2$}
	[dotted,thick] (yaxis |- A) node [mynode,left] {$P_0$} -- (A) node [mynode,above right] {A} -- (xaxis -| A) node [mynode,below] {$Q_0$}
	[dotted,thick] (yaxis |- B) node [mynode,left] {$P_1$} -- (B) node [mynode,above right] {B} -- (xaxis -| B) node [mynode,below] {$Q_1$};
\end{TikzFigure}

\newhtmlpage

\subsection*{Elastic and inelastic demands}

While the elasticity value falls as we move down the demand curve, an
important dividing line occurs at the value of $-1$. This is illustrated in
Table~\ref{table:gaselastrev}, and is a property of all straight-line demand
curves. Disregarding the negative sign, demand is said to be %
\terminology{elastic} if the price elasticity is greater than unity, and %
\terminology{inelastic} if the value lies between unity and 0. It is %
\terminology{unit elastic} if the value is exactly one.

\begin{DefBox}
Demand is \textbf{elastic} if the price elasticity is greater than unity. It is \textbf{inelastic} if the value lies between unity and 0. It is \textbf{unit elastic} if the value is exactly one.
\end{DefBox}

Economists frequently talk of goods as having a ``high''
or ``low'' demand elasticity. What does this mean, given that the elasticity varies throughout
the length of a demand curve? It signifies that, \textit{at the price
usually charged}, the elasticity has a high or low value. For example, your
weekly demand for regular coffee at Starbucks might be unresponsive to
variations in price around the value of \$2.00, but if the price were \$4,
you might be more responsive to price variations. Likewise, when we stated
at the beginning of this chapter that the demand for food tends to be
inelastic, we really meant that \textit{at the price we customarily face for
food}, demand is inelastic.

\subsection*{Determinants of price elasticity}

Why is it that the price elasticities for some goods and services are high
and for others low? 

\begin{itemize}
\item One answer lies in \textit{tastes}: If a good or service is a basic
necessity in one's life, then price variations have a minimal effect on the
quantity demanded, and these products thus have a relatively inelastic
demand.

\item A second answer lies in the \textit{ease with which we can substitute}
alternative goods or services for the product in question. If \textit{Apple}
Corporation had no serious competition in the smart-phone market, it could
price its products higher than in the presence of \textit{Samsung} and \textit{Google},
who also supply smart phones. A supplier who increases her price will lose
more sales if there are ready substitutes to which buyers can switch, than
if no such substitutes exist. It follows that a critical role for the
marketing department in a firm is to convince buyers of the uniqueness of
the firm's product.

\item Where \textit{product groups} are concerned, the price elasticity of
demand for one product is necessarily higher than for the group as a whole:
Suppose the price of one computer tablet brand alone falls. Buyers would be
expected to substitute towards this product in large numbers -- its
manufacturer would find demand to be highly responsive. But if \textit{all}
brands are reduced in price, the increase in demand for any one will be more
muted. In essence, the one tablet whose price falls has several close
substitutes, but tablets in the aggregate do not.

\item Finally, there is a \textit{time dimension} to responsiveness, and
this is explored in Section~\ref{sec:ch4sec3}.
\end{itemize}