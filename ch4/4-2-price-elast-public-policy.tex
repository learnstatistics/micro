\section{Price elasticities and public policy}\label{sec:ch4sec2}

In Chapter~\ref{chap:classical} we explored the implications of putting price floors and supply
quotas in place. We saw that price floors can lead to excess supply. An
important public policy question therefore is why these policies actually
exist. It turns out that we can understand why with the help of elasticity
concepts.

\subsection*{Price elasticity and expenditure}

Let us return to Table~\ref{table:gaselastrev} and explore what happens to total
expenditure/revenue as the price varies. Since total revenue is simply the
product of price times quantity it can be computed from the first two
columns. The result is given in the final column. We see immediately that
total expenditure on the good is highest at the midpoint of the demand
curve corresponding to these data. At a price of \$5 expenditure is \$25.
No other price yields more expenditure or revenue. Obviously the value 
\$5 is midway between the zero value and the price or quantity intercept of the
demand curve in Figure~\ref{fig:elasticitylineardemand}. This is a general result for linear demand
curves: Expenditure is greatest at the midpoint, and the mid-price
corresponds to the mid-quantity on the horizontal axis.

Geometrically this can be seen from Figure~\ref{fig:elasticitylineardemand}. Since expenditure is the
product of price and quantity, in geometric terms it is the area of the
rectangle mapped out by any price-quantity combination. For example, at $\{%
P=\$8,Q=2\}$ total expenditure is \$16 -- the area of the
rectangle bounded by these price and quantity values. Following this line of
reasoning, if we were to compute the area bounded by a price of \$7 and a
corresponding quantity of 3 units we get a larger rectangle -- a value of 
\$21. This example indicates that the largest rectangle occurs at the
midpoint of the demand curve. As a general geometric rule this is always
the case. Hence we can conclude that the price that generates the greatest
expenditure is the midpoint of a linear demand curve.

Let us now apply this rule to pricing in the market place. If our existing
price is high and our goal is to generate more revenue, then we should
reduce the price. Conversely, if our price is low and our goal is again to
increase revenue we should raise the price. Starting from a high price let
us see why this is so. By lowering the price we induce an increase in
quantity demanded. Of course the lower price reduces the revenue obtained on
the units already being sold at the initial high price. But since total
expenditure increases at the new lower price, it must be the case that 
\textit{the additional sales caused by the lower price more than compensate
for this loss on the units being sold at the initial high price}. But
there comes a point when this process ceases. Eventually the loss in revenue
on the units being sold at the higher price is not offset by the revenue
from additional quantity. We \textit{lose a margin on so many existing units
that the additional sales cannot compensate}. Accordingly revenue falls.

\newhtmlpage

Note next that the top part of the demand curve is elastic and the lower
part is inelastic. So, as a general rule we can state that:

\begin{quote}
\textit{A price decline (quantity increase) on an elastic segment of a
demand curve necessarily increases revenue, and a price increase (quantity
decline) on an inelastic segment also increases revenue.}
\end{quote}

The result is mapped in Figure~\ref{fig:totalrevenueelasticity}, which plots
total revenue as a function of the quantity demanded -- columns 2 and 4 from
Table~\ref{table:gaselastrev}. At low quantity values the price is high and the demand is
elastic; at high quantity values the price is low and the demand is
inelastic. The revenue maximizing point is the midpoint of the demand curve.

% Figure 4.4	fig:totalrevenueelasticity
\begin{TikzFigure}{xscale=0.48,yscale=0.48,descwidth=25em,caption={Total revenue and elasticity \label{fig:totalrevenueelasticity}},description={Based upon the data in Table~\ref{table:gaselastrev}, revenue increases with quantity sold up to sales of 5 units. Beyond this output, the decline in price that must accompany additional sales causes revenue to decline.}}
% Total revenue function
\draw [trcolour,ultra thick,name path=TR] (0,0) to [out=90,in=180] (7.5,7.5) to [out=0,in=90] (15,0);
% axes
\draw [thick, -] (0,12) node (yaxis) [above] {Revenue} |- (15,0) node (xaxis) [right] {Quantity};
% paths for "rev=$16" and "rev=$25" dotted lines
\path [name path=rev16] (5,5.3033) -- (15,5.3033);
\path [name path=rev25] (0,7.5) -- (15,7.5);
% intersection of paths with TR
\draw [name intersections={of=rev16 and TR, by=Q8},name intersections={of=rev25 and TR, by=Q5}]
	[dotted,thick] (yaxis |- Q8) node [mynode,left] {Rev$=\$16$} -| (xaxis -| Q8) node [mynode,below] {8}
	[dotted,thick] (yaxis |- Q5) node [mynode,left] {Rev$=\$25$} -| (xaxis -| Q5) node [mynode,below] {5};
% arrow to maximum point
\draw [<-,thick,shorten <=1mm,shorten >=-1.5mm] (Q5) -- +(0,2) node [mynode,above] {Revenue a maximum\\where elasticity is unity};
\end{TikzFigure}

\newhtmlpage

We now have a general conclusion: In order to maximize the possible revenue
from the sale of a good or service, it should be priced where the demand
elasticity is unity.

Does this conclusion mean that every entrepreneur tries to find this magic
region of the demand curve in pricing her product? Not necessarily: Most
businesses seek to maximize their \textit{profit} rather than their revenue,
and so they have to focus on cost in addition to sales. We will examine this
interaction in later chapters. Secondly, not every firm has control over the
price they charge; the price corresponding to the unit elasticity may be too
high relative to their competitors' price choices. Nonetheless, many firms,
especially in the early phase of their life-cycle, focus on revenue growth
rather than profit, and so, if they have any power over their price, the
choice of the unit-elastic price may be appropriate.

\newhtmlpage

\subsection*{The agriculture problem}

We are now in a position to address the question we posed above: Why are
price floors frequently found in agricultural markets? The answer is that
governments believe that the pressures of competition would force farm/food
prices so low that many farmers would not be able to earn a reasonable
income from farming. Accordingly, governments impose price floors.
Keep in mind that price floors are prices above the market equilibrium and
therefore lead to excess supply.

Since the demand for foodstuffs is inelastic we know that a higher
price will induce more revenue, even with a lower quantity being sold. The
government can force this outcome on the market by a policy of supply
management. It can force farmers in the aggregate to bring only a specific
amount of product to the market, and thus ensure that the price floor does
not lead to excess supply. This is the system of supply management we
observe in dairy markets in Canada, for example, and that we examined in
the case of maple syrup in Chapter~\ref{chap:classical}. Its supporters praise it
because it helps farmers, its critics point out that higher food prices hurt
lower-income households more than high-income households, and therefore it
is not a good policy.

Elasticity values are frequently more informative than diagrams and figures.
Our natural inclination is to view demand curves with a somewhat vertical
profile as being inelastic, and demand curves with a flatter profile as
elastic. But we must keep in mind that, as explained in Chapter~\ref{chap:tmd}, the
vertical and horizontal axis of any diagram can be scaled in such a way as
to change the visual impact of the data underlying the curves. But a
numerical elasticity value will never deceive in this way. If its value is
less than unity it is inelastic, regardless of the visual aspect of the
demand curve.

\newhtmlpage

At the same time, if we have two demand curves intersecting at a particular
price-quantity combination, we can say that the curve with the more vertical
profile is \textit{relatively} more elastic, or less inelastic. This is
illustrated in Figure~\ref{fig:elastquantfluctuations}. It is clear that, 
\textit{at the price-quantity combination where they intersect}, the demand
curve $D^{\prime}$ will yield a greater (percentage) quantity change than
the demand curve $D$, for a given (percentage) price change. Hence, on the
basis of diagrams, we can compare demand elasticities \textit{in relative
terms at a point where the two intersect}. 

% Figure 4.5	fig:elastquantfluctuations
\begin{TikzFigure}{xscale=0.5,yscale=0.4,descwidth=25em,caption={The impact of elasticity on quantity fluctuations \label{fig:elastquantfluctuations}},description={In the lower part of the demand curve $D$, demand is inelastic: At the point A, a shift in supply from $S_1$ to $S_2$ induces a large percentage increase in price, and a small percentage decrease in quantity demanded. In contrast, for the demand curve $D'$ that goes through the original equilibrium, the region A is now an \emph{elastic} region, and the impact of the supply shift is contrary: The \%$\Delta P$ is smaller and the \%$\Delta Q$ is larger.}}
% demand lines
\draw [demandcolour,ultra thick,name path=demand] (4,15) node [black,mynode,left] {$D$} -- (10,0);
\draw [demandcolour,ultra thick,name path=demandprime] (0,7) -- (15,3.25) node [black,mynode,above] {$D'$};
% supply lines
\draw [supplycolour,ultra thick,name path=supplytwo] (1,0) -- (8.5,15) node [black,mynode,right] {$S_2$};
\draw [supplycolour,ultra thick,name path=supplyone] (5,0) -- (14,15) node [black,mynode,right] {$S_1$};
\draw [thick] (0,15) node (yaxis) [above] {Price} |- (15,0) node (xaxis) [right] {Quantity};
% intersection of demand and supplyone
\draw [name intersections={of=demand and supplyone, by=P1Q1}]
	[dotted,thick] (yaxis |- P1Q1) node [mynode,left] {$P_1$} -- (P1Q1) node [mynode,above=0.25em and 0em] {A} -- (xaxis -| P1Q1) node [mynode,below] {$Q_1$};
% intersection of demand and supplytwo
\draw [name intersections={of=demand and supplytwo, by=P2Q2}]
	[dotted,thick] (yaxis |- P2Q2) node [mynode,left] {$P_2$} -| (xaxis -| P2Q2) node [mynode,below] {$Q_2$};
% intersection of demandprime and supplytwo
\draw [name intersections={of=demandprime and supplytwo, by=P3Q3}]
	[dotted,thick] (yaxis |- P3Q3) node [mynode,left] {$P_3$} -| (xaxis -| P3Q3) node [mynode,below] {$Q_3$};
\end{TikzFigure}
