\section{Technical tricks with elasticities}\label{sec:ch4sec8}

We can easily compute elasticities at any point on a demand curve, rather
than over a range or arc, by using the explicit formula for the demand
curve. To see this note that we can rewrite Equation~\ref{eq:priceelastdemand} as: 
\begin{equation*}
\varepsilon_{d}=\frac{\%\Delta Q}{\%\Delta P}=\frac{\Delta Q/Q}{\Delta P/P}=%
\frac{\Delta Q}{\Delta P}\times \frac{P}{Q}.
\end{equation*}%
Now the first term is obtained from the slope of the demand curve, and the
second term is defined by the point on the curve that interests us. For
example if our demand curve is $P=10-1\times Q$, then $\Delta P/\Delta Q=-1$.
Inverting this to get $\Delta Q/\Delta P$ yields $-1$ also. So, the
elasticity value at $\{P=\$6,Q=4\}$ is $-1\times 6/4=-1.5$. This formula has
the benefit that we can obtain the value of the elasticity at a particular
point on the demand curve, rather than over a range of values or an arc.
Consequently it is called the point elasticity of demand. And obviously we
could apply it to a demand curve that is not linear provided we know the
mathematical form and are able to establish the slope.