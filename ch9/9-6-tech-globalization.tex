\section{Globalization and technological change}\label{sec:ch9sec6}

Globalization and technological change have had a profound impact on the way
goods and services are produced and brought to market in the modern world.
The cost structure of many firms has been reduced by \textit{outsourcing to
lower-wage economies}. Furthermore, the advent of the communications
revolution has effectively \textit{increased the minimum efficient scale for
many industries}, as illustrated in Chapter~\ref{chap:prodcost} 
(Figure~\ref{fig:techchange}). Larger firms are less difficult to manage nowadays,
and the $LAC$ curve may not slope upwards until very high output levels are
attained. The consequence is that some industries may not have sufficient
``production space'' to sustain a large number of firms. In order to reap
the advantages of scale economies, firms become so large that they can
supply a significant part of the market. They are no longer so small as to
have no impact on the price.

Outsourcing and easier communications have in many cases simply eliminated
many industries in the developed world. Garment making is an example. Some
decades ago Quebec was Canada's main garment maker: Brokers dealt with
`cottage-type' garment assemblers outside Montreal and Quebec City. But
ultimately the availability of cheaper labour in the developing world
combined with efficient communications undercut the local manufacture. Most
of Canada's garments are now imported. Other North American and European
industries have been impacted in similar ways. Displaced labour has had to
reskill, retool, reeducate itself, and either seek alternative employment in
the manufacturing sector, or move to the service sector of the economy, or
retire.

Globalization has had a third impact on the domestic economy, in so far as
it \textit{reduces the cost of components}. Even industries that continue to
operate within national boundaries see a reduction in their cost structure
on account of globalization's impact on input costs. This is particularly in
evidence in the computing industry, where components are produced in
numerous low-wage economies, imported to North America and assembled into
computers domestically. Such components are termed \textit{intermediate goods}.