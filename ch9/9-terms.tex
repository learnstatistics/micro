\newpage
	\section*{Key Terms}
\begin{keyterms}
\textbf{Perfect competition}: an industry in which many suppliers, producing an identical product, face many buyers, and no one participant can influence the market.

\textbf{Profit maximization} is the goal of competitive suppliers -- they seek to maximize the difference between revenues and costs.

\textbf{Marginal revenue} is the additional revenue accruing to the firm resulting from the sale of one more unit of output.

\textbf{Shut-down price} corresponds to the minimum value of the $AVC$ curve.

\textbf{Break-even price} corresponds to the minimum of the $ATC$ curve.

\textbf{Short-run supply curve for perfect competitor}: the portion of the $MC$ curve above the minimum of the $AVC$.

\textbf{Industry supply (short run)} in perfect competition is the horizontal sum of all firms' supply curves.

\textbf{Short-run equilibrium} in perfect competition occurs when each firm maximizes profit by producing a quantity where $P=MC$.

\textbf{Economic (supernormal) profits} are those profits above normal profits that induce firms to enter an industry. Economic profits are based on the opportunity cost of the resources used in production.

\textbf{Long-run equilibrium} in a competitive industry requires a price equal to the minimum point of a firm's $ATC$. At this point, only normal profits exist, and there is no incentive for firms to enter or exit.

\textbf{Industry supply in the long run in perfect competition} is horizontal at a price corresponding to the minimum of the representative firm's long-run $ATC$ curve.

\textbf{Increasing (decreasing) cost} industry is one where costs rise (fall) for each firm because of the scale of industry operation.
\end{keyterms}