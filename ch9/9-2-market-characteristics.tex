\section{Market characteristics}\label{sec:ch9sec2}

The key attributes of a perfectly competitive market are the following:

\begin{enumerate}
	\item There must be \textit{many firms}, each one small and powerless
	relative to the entire industry.
	
	\item The \textit{product must be standardized}. Barber shops offer a
	standard product, but a Lexus differs from a Ford. Barbers tend to be price
	takers, but Lexus does not charge the same price as Ford, and is a price
	setter.
	
	\item Buyers are assumed to have \textit{full information} about the product
	and its pricing. For example, buyers know that the products of different
	suppliers really are the same in quality.
	
	\item There are \textit{many buyers}.
	
	\item There is \textit{free entry and exit} of firms.
\end{enumerate}

In terms of the demand curve that suppliers face, these market
characteristics imply that the demand curve facing the perfectly competitive
firm is horizontal, or infinitely elastic, as we defined in Chapter~\ref{chap:elasticities}.
In contrast, the demand curve facing the whole industry
is downward sloping. The demand curve facing a firm is represented in 
Figure~\ref{fig:optoutput}. It implies that the supplier can sell any output he
chooses at the going price $P_{0}$. But what quantity should he choose, or
what quantity will maximize his profit? The profit-maximizing choice is his
target, and the $MC$ curve plays a key role in this decision.