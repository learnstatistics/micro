\section{Efficient resource allocation}\label{sec:ch9sec7}

Economists have a particular liking for competitive markets. The reason is
not, as is frequently thought, that we love competitive battles; it really
concerns resource allocation in the economy at large. In Chapter~\ref{chap:welfare} 
we explained why markets are frequently an excellent vehicle
for transporting the economy's resources to where they are most valued: A
perfectly competitive marketplace in which there are no externalities
results in resources being used up to the point where the demand and supply
prices are equal. If demand is a measure of marginal benefit and supply is a
measure of marginal cost, then a perfectly competitive market ensures that
this condition will hold in equilibrium. \textit{Perfect competition,
therefore, results in resources being used efficiently}.

Our initial reaction to this perspective may be: If market equilibrium is
such that the quantity supplied always equals the quantity demanded, is not
every market efficient? The answer is no. As we shall see in the next
chapter on monopoly, the monopolist's supply decision does not reflect the
marginal cost of resources used in production, and therefore does not result
in an efficient allocation in the economy.