\section{Dynamics: Entry and exit}\label{sec:ch9sec4}

We have now described the market and firm-level equilibrium in the short
run. However, this equilibrium may be only temporary; whether it can be
sustained or not depends upon whether profits (or losses) are being
incurred, or whether all participant firms are making what are termed %
normal profits. Such profits are considered an essential part
of a firm's operation. They reflect the opportunity cost of the resources
used in production. Firms do not operate if they cannot make a minimal, or
normal, profit level. Above such profits are \terminology{economic profits}
(also called \terminology{supernormal profits}), and these are what entice
entry into the industry.

Recall from Chapter~\ref{chap:firminvestorcapital} that accounting and economic profits are different.
The economist includes opportunity costs in determining profit, whereas the
accountant considers actual revenues and costs. In the example developed in
Section~\ref{sec:ch7sec2} the entrepreneur recorded accounting profit, but not economic
profit. Suppose now that the numbers were slightly different, and are as
defined in Table~\ref{table:economicprofitstable}: Felicity
invests \$250,000 in her business in the form of capital, as before. But she
now has gross revenues of \$165,000 and incurs a cost of \$90,000 to buy the
clothing wholesale that she then sells retail. She pays herself a salary of
\$35,000. If these numbers represent her balance sheet, then she records an
accounting profit of \$40,000.

\begin{Table}{caption={Economic profits \label{table:economicprofitstable}}}
\begin{tabu} to 30em {|X[2,l]X[1,c]X[1,c]|}	\hline
	\rowcolor{rowcolour}	Sales &  & \$165,000 \\ 
	Materials costs &  & \$90,000 \\ 
	\rowcolor{rowcolour}	Wage costs &  & \$35,000 \\ 
	\textbf{Accounting profit} &  & \textbf{\$40,000} \\ 
	\rowcolor{rowcolour}	Capital invested & \$250,000 &  \\ 
	Implicit return on capital at 4\% &  & \$10,000 \\ 
	\rowcolor{rowcolour}	Additional implicit wage costs &  & \$20,000 \\ 
	\textbf{Total implicit costs} &  & \textbf{\$30,000} \\ 
	\rowcolor{rowcolour}	\textbf{Economic profit} &  & \textbf{\$10,000}	\\	\hline
\end{tabu}
\end{Table}

\newhtmlpage

Her economic profit calculation must include opportunity costs. The
opportunity cost of tying up \$250,000 of capital, if the interest rate is
4\%, amounts to \$10,000. In addition, if Felicity could earn \$55,000 in
her best alternative job then an additional implicit cost of \$20,000 must
be considered. When these two opportunity (or implicit) costs are added to
the balance sheet, her profit is reduced to \$10,000. This is her economic
profit. If Felicity's economic profit is representative of the retail
clothing sector of the economy, then that profitability should attract new
entrepreneurs. Our conclusion is that this sector of the economy 
\textit{should experience new entrants and hence an outward shift of the supply curve}. 
In contrast, in the numerical example considered in Section~\ref{sec:ch7sec2}, Felicity was
experiencing losses (negative economic profits), and in the longer
term she would have to consider leaving the business. If she and other
suppliers exited, then the market supply curve would shift back to the left
-- representing a reduction in supply.

The critical point in this distinction between accounting and economic cost
is that the decision to enter or leave a market in the longer term is based
on what the entrepreneur can earn in the wider market place. That is,
economic profits rather than accounting profits will determine the
equilibrium number of firms in the long term. In terms of our cost curves,
we will assume that the full economic costs are included in the various
curves that we use. Consequently any profits (or losses) that arise are
based upon the full economic costs of the firm's operation.

\begin{DefBox}
	\textbf{Economic (supernormal) profits} are those profits above normal profits that induce firms to enter an industry.
\end{DefBox}

\newhtmlpage

Let us return to our graphical analysis, and begin by supposing that the
market equilibrium described in Figure~\ref{fig:marketeq} results in profits
being made by some firms. Such an outcome is described in 
Figure~\ref{fig:shortrunprofits}, where the price exceeds the $ATC$. At the price 
$P_{E}$, a profit-making firm supplies the quantity $q_{E}$, as determined
by its $MC$ curve. On average, the cost of producing each unit of output, 
$q_{E}$, is defined by the point on the $ATC$ at that output level, point $k$. 
Profit per unit is thus given by the value $(m-k)$ -- the difference
between revenue per unit and cost per unit. Total (economic) profit is
therefore the area $P_{E}mkh$, which is quantity times profit per unit.

% Figure 9.5
\begin{TikzFigure}{xscale=0.36,yscale=0.3,descwidth=25em,caption={Short-run profits for the firm \label{fig:shortrunprofits}},description={At the price $P_E$, determined by the intersection of market demand and market supply, an individual firm produces the amount $Q_E$. The $ATC$ of this output is k and therefore profit per unit is mk. Total profit is therefore $P_E$mkh$=0q_E\times$mk$=TR-TC$.}}
% MC curve
\draw [dashed,mccolour,ultra thick,domain=3:18,name path=MC] plot (\x, {0.25*pow(1.25,\x)+4}) node [mynode,black,right] {$MC$};
% AVC curve
\draw [avccolour,ultra thick,domain=3:17,name path=AVC] plot (\x, {-1*sqrt(16-(\x-10)*(\x-10)/4)+10.32831}) node [black,mynode,right] {$AVC$};
% ATC curve
\draw [atccolour,ultra thick,domain=11.5:18.5,name path=ATC] plot (\x, {-2*sqrt(16-(\x-15)*(\x-15))+19.1054}) node [black,mynode,right] {$ATC$};
% axes
\draw [thick, -] (0,20) node (yaxis) [above] {Price} -- (0,0) node [mynode,left] {0} -- (25,0) node (xaxis) [right] {Quantity};
% path to intersect with MC
\path [name path=PEmpath] (0,14) -- +(25,0);
% intersection of PEmpath with MC
\draw [name intersections={of=MC and PEmpath, by=m}]
	[dotted,thick] (yaxis |- m) node [mynode,left] {$P_E$} -- (m) node [mynode,above left] {m} -- (xaxis -| m) node [mynode,below] {$q_E$};
% path to create dotted line from h to k
\path [name path=mkline] (xaxis -| m) -- (m);
% intersection of mkline with ATC
\draw [name intersections={of=ATC and mkline, by=k}]
	[dotted,thick] (yaxis |- k) node [mynode,left] {h} -- (k) node [mynode,below right] {k};
\end{TikzFigure}

\newhtmlpage

% Figure 9.6
\begin{TikzFigure}{xscale=0.36,yscale=0.3,descwidth=25em,caption={Entry of firms due to economic profits \label{fig:entryprofit}},description={If economic profits result from the price $P_E$ new firms enter the industry. This entry increases the market supply to $S'$ and the equilibrium price falls to $P'$. Entry continues as long as economic profits are present. Eventually the price is driven to a level where only normal profits are made, and entry ceases.}}
% supply curves
\draw [supplycolour,ultra thick,domain=1:14,name path=S] plot (\x, {0.5*pow(1.25,\x)+8}) node [black,mynode,right] {$S$=Sum of existing\\firms' $MC$ curves};
\draw [supplycolour,ultra thick,domain=3:17,name path=Sprime] plot (\x, {0.25*pow(1.25,\x)+4}) node [black,mynode,right] {$S'$=Sum of new and existing\\firms' $MC$ curves};
% demand curve
\draw [demandcolour,ultra thick,domain=5:22,name path=D] plot (\x, {0.5*pow(1.25,-1*\x+20)+4}) node [black,mynode,right] {$D$};
% axes
\draw [thick, -] (0,20) node (yaxis) [above] {Price} |- (25,0) node (xaxis) [right] {Quantity};
% intersection of supply and demand
\draw [name intersections={of=S and D, by=E},name intersections={of=Sprime and D, by=prime}]
	[dotted,thick] (yaxis |- E) node [mynode,left] {$P_E$} -| (xaxis -| E) node [mynode,below] {$q_E$}
	[dotted,thick] (yaxis |- prime) node [mynode,left] {$P'$} -| (xaxis -| prime) node [mynode,below] {$q'$};	
\end{TikzFigure}

While $q_{E}$ represents an equilibrium for the firm, it is only a
short-run, or temporary, equilibrium for the industry. The assumption of
free entry and exit implies that the presence of economic profits will
induce new entrepreneurs to enter and start producing. The impact of this
dynamic is illustrated in Figure~\ref{fig:entryprofit}. An increased number
of firms shifts supply rightwards to become $S^{\prime}$, thereby
increasing the amount supplied at any price. The impact on price of this
supply shift is evident: With an unchanged demand, the equilibrium price
must fall.

\newhtmlpage

How far will the price fall, and how many new firms will enter this
profitable industry? As long as economic profits exist new firms will enter
and the resulting increase in supply will continue to drive the price
downwards. But, once the price has been driven down to the minimum of the 
$ATC$ of a representative firm, there is no longer an incentive for new
entrepreneurs to enter. Therefore, the \terminology{long-run industry equilibrium} 
is where the market price equals the minimum point of a firm's 
$ATC$ curve. This generates normal profits, and there is no incentive for
firms to enter or exit.

\begin{DefBox}
	A \textbf{long-run equilibrium} in a competitive industry requires a price equal to the minimum point of a firm's $ATC$. At this point, only normal profits exist, and there is no incentive for firms to enter or exit.
\end{DefBox}

In developing this dynamic, we began with a situation in which economic
profits were present. However, we could have equally started from a position
of losses. With a market price between the minimum of the $AVC$ and the
minimum of the $ATC$ in Figure~\ref{fig:shortrunprofits}, revenues per unit
would exceed variable costs but not total costs per unit. When firms cannot
cover their $ATC$ in the long run, they will cease production. Such closures
must reduce aggregate supply; consequently the market supply curve
contracts, rather than expands as it did in Figure~\ref{fig:entryprofit}.
The reduced supply drives up the price of the good. This process continues
as long as firms are making losses. A final industry equilibrium is attained
only when the price reaches a level where firms can make a normal profit.
Again, this will be at the minimum of the typical firm's $ATC$.

Accordingly, the long-run equilibrium is the same, regardless of whether we
begin from a position in which firms are incurring losses, or where they are
making profits.