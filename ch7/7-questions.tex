\newpage
\section*{Exercises for Chapter~\ref{chap:firminvestorcapital}}

\begin{Filesave}{solutions}
\subsubsection*{Chapter~\ref{chap:firminvestorcapital} Solutions}
\end{Filesave}

\begin{enumialphparenastyle}

\begin{econex}\label{ex:ch7ex1}
Henry is contemplating opening a microbrewery and investing his savings of \$100,000 in it. He will quit his current job as a quality controller at Megaweiser where he is paid an annual salary of \$50,000. He plans on paying himself a salary of \$40,000 at the microbrewery. He also anticipates that his beer sales minus all costs other than his salary will yield him a surplus of \$55,000 per annum. The rate of return on savings is 7\%.
\begin{enumerate}
	\item Calculate the accounting profits envisaged by Henry.
	\item Calculate the economic profits.
	\item Should Henry open the microbrewery?
	\item If all values except the return on savings remain the same, what rate of return would leave him indifferent between opening the brewery and not?
\end{enumerate}
\begin{econsolution}
\begin{enumerate}
	\item	Accounting profit is \$15,000 ($=\$55,000-\$40,000$).
	\item	Economic costs include opportunity cost of the investment (\$7,000) plus the additional earnings of \$10,000 ($=\$50,000-\$40,000$). These additional costs of \$17,000 mean that economic profits are negative.
	\item	No.
	\item	5\%
\end{enumerate}
\end{econsolution}
\end{econex}

\begin{econex}\label{ex:ch7ex2}
You see an advertisement for life insurance for everyone 55 years of age and older. The advertisement says that no medical examination is required prior to purchasing insurance. If you are a very healthy 57-year old, do you think you will get a good deal from purchasing this insurance?
\begin{econsolution}
	If there is no medical exam then it is probable that less healthy individuals will avail of it. Knowing this, the firm should choose its benefit/payout structure to reflect a high-cost clientele. It will have lower payouts and/or higher premiums. Therefore a healthy individual would likely not obtain favourable insurance terms.
	
\end{econsolution}
\end{econex}

\begin{econex}\label{ex:ch7ex3}
In which of the following are risks being pooled, and in which would risks likely be spread by insurance companies?
\begin{enumerate}
\item	Insurance against Alberta's Bow River Valley flooding.
\item	Life insurance.
\item	Insurance for the voice of Avril Lavigne or Celine Dion.
\item	Insuring the voices of the lead vocalists in Metallica, Black Eyed Peas, Incubus, Evanescence, Green Day, and Jurassic Five.
\end{enumerate}
\begin{econsolution}
\begin{enumerate}
\item	Spread.
\item	Pooled.
\item	Spread.
\item	Spread and pooling.
\end{enumerate}
\end{econsolution}
\end{econex}

\begin{econex}\label{ex:ch7ex4}
Your house has a one in five hundred probability chance of burning down in any given year. It is valued at \$350,000.
\begin{enumerate}
	\item What insurance premium would be actuarially fair for this situation?
	\item If the owner is willing to pay a premium of \$900, does she dislike risk or is she indifferent to risk?
\end{enumerate}
\begin{econsolution}
\begin{enumerate}
	\item	\$700 ($=\$350,000/500$)
	\item	She is averse to risk.
\end{enumerate}
\end{econsolution}
\end{econex}

\begin{econex}\label{ex:ch7ex5}
If individuals experience diminishing marginal utility from income it means that their utility function will resemble the total utility functions developed graphically in Section~\ref{sec:ch6sec2}. Let us imagine specifically that if $Y$ is income and $U$ is utility, the individual gets utility from income according to the relation $U=\sqrt{Y}$.
\begin{enumerate}
	\item In a spreadsheet or using a calculator, calculate the amount of utility the individual gets for all income values running from \$1 to \$25. 
	\item Graph the result with utility on the vertical axis and income on the horizontal axis, and verify from its shape that the marginal utility of income is declining.
	\item Using your calculations, how much utility will the individual get from \$4, \$9 and \$16?
	\item Suppose now that income results from a lottery and half of the time the individual gets \$4 and half of the time he gets \$16. How much utility will he get on average?
	\item Now suppose he gets \$10 each time with certainty. How much utility will he get from this?
	\item Since \$10 is exactly an average of \$4 and \$16, can you explain why \$10 with certainty gives him more utility than getting \$4 and \$16 each half of the time?
\end{enumerate}
\begin{econsolution}
\begin{enumerate}
	\item	See the diagram below.
	\item	See the diagram below.
	\item	2, 3, 4.
	\item	3 units ($=\text{average of }\sqrt{4}+\sqrt{16}$).
	\item	($=\sqrt{10}$)
	\item	This is because of diminishing marginal utility: the loss in utility in going from \$10 to \$4 exceeds the gain in utility from \$10 to \$16.
\end{enumerate}
\begin{center*}
	\begin{tikzpicture}[background color=figurebkgdcolour,use background]
	\begin{axis}[
	axis line style=thick,
	every tick label/.append style={font=\footnotesize},
	ymajorgrids,
	grid style={dotted},
	every node near coord/.append style={font=\scriptsize},
	xticklabel style={rotate=90,anchor=east,/pgf/number format/1000 sep=},
	scaled y ticks=false,
	yticklabel style={/pgf/number format/fixed,/pgf/number format/1000 sep = \thinspace},
	xmin=0,xmax=26,ymin=0,ymax=6,
	y=1.1cm/1.25,
	x=1.4cm/5,
	x label style={at={(axis description cs:0.5,-0.02)},anchor=north},
	y label style={at={(axis description cs:0.01,0.5)},anchor=north},
	xlabel={$Y$},
	ylabel={$U$},
	]
	\addplot[tucolour,ultra thick,mark=x] table {
		X	Y
		0	0	
		1	1.00	
		2	1.41	
		3	1.73	
		4	2.00	
		5	2.24	
		6	2.45	
		7	2.65	
		8	2.83	
		9	3.00	
		10	3.16	
		11	3.32		
		12	3.46	
		13	3.61	
		14	3.74	
		15	3.87	
		16	4.00	
		17	4.12	
		18	4.24	
		19	4.36	
		20	4.47	
		21	4.58	
		22	4.69	
		23	4.80	
		24	4.90	
		25	5.00	
		26	5.10
	};
	\end{axis}
	\end{tikzpicture}
\end{center*}
\end{econsolution}
\end{econex}

\begin{econex}\label{ex:ch7ex6}
In Question~\ref{ex:ch7ex5}, suppose that the individual gets utility according to the relation $U=\frac{1}{2}Y.$ Repeat the calculations for each part of the question and see if you can understand why the answers are different.
\begin{econsolution}
\begin{enumerate}
	\item	See the diagram below.
	\item	See the diagram below.
	\item	2, 4.5, 8.
	\item	5 ($=\text{average of }2+8$).
	\item	5
	\item	In this case the $MU$ of money is constant. Each increment in income yields the same additional utility. Hence the utility loss of -\$6 equals the utility gain of +\$6.
\end{enumerate}
\begin{center*}
	\begin{tikzpicture}[background color=figurebkgdcolour,use background]
	\begin{axis}[
	axis line style=thick,
	every tick label/.append style={font=\footnotesize},
	ymajorgrids,
	grid style={dotted},
	every node near coord/.append style={font=\scriptsize},
	xticklabel style={rotate=90,anchor=east,/pgf/number format/1000 sep=},
	scaled y ticks=false,
	yticklabel style={/pgf/number format/fixed,/pgf/number format/1000 sep = \thinspace},
	xmin=0,xmax=26,ymin=0,ymax=14,
	y=1.2cm/3,
	x=1.4cm/5,
	x label style={at={(axis description cs:0.5,-0.02)},anchor=north},
	y label style={at={(axis description cs:0.01,0.5)},anchor=north},
	xlabel={$Y$},
	ylabel={$U$},
	]
	\addplot[tucolour,ultra thick,mark=x] table {
		X	Y
		0	0	
		1	0.5	
		2	1	
		3	1.5	
		4	2	
		5	2.5	
		6	3	
		7	3.5	
		8	4	
		9	4.5	
		10	5	
		11	5.5		
		12	6	
		13	6.5	
		14	7	
		15	7.5	
		16	8	
		17	8.5	
		18	9	
		19	9.5	
		20	10	
		21	10.5	
		22	11	
		23	11.5	
		24	12	
		25	12.5	
		26	13
	};
	\end{axis}
	\end{tikzpicture}
\end{center*}
\end{econsolution}
\end{econex}

\end{enumialphparenastyle}
