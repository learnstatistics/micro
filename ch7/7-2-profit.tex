\section{Profit}\label{sec:ch7sec2}

\subsection*{Ownership and corporate goals}

As economists, we believe that profit maximization accurately describes a
typical firm's objective. However, since large firms are not run by their
owners but by their executives or agents, it is frequently hard for the
shareholders to know exactly what happens within a company. Even the board
of directors---the guiding managerial group---may not be fully aware of the
decisions, strategies, and practices of their executives and managers.
Occasionally things go wrong, and managers decide to follow their own
interests rather than the interests of the company. In technical terms, the
interests of the corporation and its shareholders might not be aligned with
the interests of its managers. For example, managers might have a short
horizon and take steps to increase their own income in the short term,
knowing that they will move to another job before the long-term effects of their
decisions impact the firm.

At the same time, the marketplace for the ownership of corporations exerts a
certain discipline: If firms are not as productive or profitable as
possible, they may become \textit{subject to takeover} by other firms. Fear
of such takeover can induce executives and boards to maximize profits.

The shareholder-manager relationship is sometimes called a
\terminology{principal-agent relationship}, and it can give rise to a
principal-agent problem. If it is costly or difficult to monitor the
behaviour of an agent because the agent has additional information about his
own performance, the principal may not know if the agent is working to
achieve the firm's goals. This is the \terminology{principal-agent problem}.

\begin{DefBox}
\textbf{Principal or owner}: delegates decisions to an agent, or manager.

\textbf{Agent}: usually a manager who works in a corporation and is directed to follow the corporation's interests.

\textbf{Principal-agent problem}: arises when the principal cannot easily monitor the actions of the agent, who therefore may not act in the best interests of the principal.
\end{DefBox}

\newhtmlpage

In an effort to deal with such a challenge, corporate executives frequently
get bonuses or \terminology{stock options} that are related to the overall
profitability of their firm. Stock options usually take the form of an
executive being allowed to purchase the company's stock in the future -- but
at a price that is predetermined. If the company's profits do increase, then
the price of the company's stock will reflect this and increase likewise.
Hence the executive has an incentive to work with the objective of
increasing profits because that will enable him to buy the company stock in
the future at a lower price than it will be worth.

\begin{DefBox}
\textbf{Stock option}: an option to buy the stock of the company at a future date for a fixed, predetermined price.
\end{DefBox}

The threat of takeover and the structure of rewards, together, imply that
the assumption of profit maximization is a reasonable one.

\newhtmlpage

\begin{ApplicationBox}{caption={The `Sub-Prime' mortgage crisis: A principal-agent problem \label{app:subprime}}}
With a decline in interest and mortgage rates in the early part of the twenty first century, many individuals believed they could afford to buy a house because the borrowing costs were lower than before. Employees and managers of lending companies believed likewise, and they structured loans in such a way as to provide an incentive to low-income individuals to borrow. These mortgage loans frequently enabled purchasers to buy a house with only a 5\% down payment, in some cases even less, coupled with a repayment schedule that saw low repayments initially but higher repayments subsequently. The initial interest cost was so low in many of these mortgages that it was even lower than the `prime' rate -- the rate banks charge to their most prized customers. 

The crisis that resulted became known as the `sub-prime' mortgage crisis. In many cases loan officers got bonuses based on the total value of loans they oversaw, regardless of the quality or risk associated with the loan. The consequence was that they had the incentive to make loans to customers to whom they would not have lent, had these employees and managers been lending their own money, or had they been remunerated differently. The outcomes were disastrous for numerous lending institutions. When interest rates climbed, borrowers could not repay their loans. The construction industry produced a flood of houses that, combined with the sale of houses that buyers could no longer afford, sent housing prices through the floor. This in turn meant that recent house purchasers were left with negative value in their homes -- the value of their property was less than what they paid for it. Many such `owners' simply returned the keys to their bank, declared bankruptcy and walked away. Some lenders went bankrupt; some were bailed out by the government, others bought by surviving firms. This is a perfect example of the principal agent problem -- the managers of the lending institutions and their loan officers did not have the incentive to act in the interest of the owners of those institutions.

The broader consequence of this lending practice was a financial collapse greater than any since the Depression of the nineteen thirties. Assets of the world's commercial and investment banks plummeted in value. Their assets included massive loans and investments both directly and indirectly to the real estate market, and when real estate values fell, so inevitably did the value of the assets based on this sector. Governments around the world had to buy up bad financial assets from financial institutions, or invest massive amounts of taxpayer money in these same institutions. Otherwise the world's financial system might have collapsed, with unknowable consequences.

Taxpayers and shareholders together bore the burden of this disastrous investment policy. Shareholders in many banks saw their shares drop in value to just a few percent of what they had been worth a year or two prior to the collapse. 
\end{ApplicationBox}

\newhtmlpage

\subsection*{Economic and accounting profit}

Economists and accountants frequently differ in how they measure profits. An
accountant stresses the financial flows of corporate activity; the economist
is, in addition, concerned with opportunity cost. Imagine that Felicity has
just inherited \$250,000 and decides to pursue her dream by opening a
clothing boutique. She quits her job that pays her \$55,000 per annum,
invests her inheritance in the purchase of a small retail space on the high
street and launches her business. At the end of her first year she records
\$110,000 in clothing sales, which she purchased from the wholesaler for
\$50,000. She pays herself a salary of \$35,000 and has no other accounting
costs because she owns her physical capital -- the store. Her accounting
profit for the year is given by the margin returned between the buying and
selling price of her clothing (\$60,000) minus her incurred costs
(\$35,000) in salary. Her accounting profit is thus \$25,000. Should she be
content with this sum?

Felicity's economist friend, Prudence, informs Felicity that her enterprise
is not returning a profit by economic standards. Prudence points out that
Felicity could earn \$55,000 as an alternative to working in her own store,
hence there is an additional implicit cost of \$20,000 to be considered,
because Felicity only draws a salary of \$35,000. Furthermore, Felicity has
invested \$250,000 in her business to avoid rent. But that sum, invested at
the going interest rate of 4\%, could earn her \$10,000 per annum. That too
is a foregone income stream so it is an implicit cost. Altogether, the
additional implict costs, not included in the accounting flows, amount to
\$30,000, and these implicit costs exceed the `accounting profits'. Thus no
economic profits are being made, because the economist includes implicit
costs in her profit calculation. In economic terms Felicity would be better
off by returning to her job and investing her inheritance. That strategy
would generate an income of \$65,000, as opposed to the income of \$60,000
that she generates from the boutique -- a salary of \$35,000 plus an
accounting profit of \$25,000.

\newhtmlpage

We can summarize this: \terminology{Accounting profit} is the difference
between revenues and explicit costs. \terminology{Economic profit} is the
difference between revenue and the sum of explicit and implicit costs. 
\terminology{Explicit costs} are the measured financial costs;
\terminology{Implicit costs} represent the opportunity cost of the resources
used in production.

\begin{DefBox}
\textbf{Accounting profit}: is the difference between revenues and explicit costs. 

\textbf{Economic profit}: is the difference between revenue and the sum of explicit and implicit costs.

\textbf{Explicit costs}: are the measured financial costs.

\textbf{Implicit costs}: represent the opportunity cost of the resources used in production.
\end{DefBox}

We will return to these concepts in the following chapters. Opportunity
cost, or implicit costs, are critical in determining the long-run structure
of certain sectors in the economy.