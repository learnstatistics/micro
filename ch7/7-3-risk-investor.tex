\section{Risk and the investor}\label{sec:ch7sec3}

Firms cannot grow without investors. A successful firm's founder always
arrives at a point where more investment is required if her enterprise is to
expand. Frequently, she will not be able to secure a sufficiently large loan
for such growth, and therefore must induce outsiders to buy shares in her
firm. She may also realize that expansion carries risk, and she may want
others to share in this risk. Risk plays a central role in the life of the
firm and the investor. Most investors prefer to avoid risk, but are prepared
to assume a limited amount of it if the anticipated rewards are sufficiently
attractive.

A good illustration of risk-avoidance is to be seen in the purchase of home
insurance. Most owners of a house that has even a small probability of
burning down or being destroyed by lightning, purchase insurance. By doing
so they are avoiding risk. But how much are they willing to pay for such
insurance? If the house is worth \$500,000 and the probability of its being
destroyed is one in one thousand in a given year then, using an averaging
perspective, individuals should be willing to pay an insurance premium of
\$500 per annum. That insurance premium represents what actuaries call a
`fair' gamble: If the probability of disaster is one in one thousand, then
the `fair' premium should be one thousandth the value of the home that is
being insured. If the insurance company insures millions of homes, then on
average it will have to pay for the replacement of one house for every one
thousand houses it insures each year. So by charging homeowners a price that
exceeds \$500 the insurer will cover not only the replacement cost of homes,
but in addition cover her administrative costs and perhaps make a profit.
Insurers operate on the basis of what we sometimes call the `law of large
numbers'.

In fact however, most individuals are willing to pay more than this `fair'
amount, and actually do pay more. If the insurance premium is \$750 or
\$1,000 the home-owner is paying more than is actuarially `fair', but a
person who dislikes risk may be willing to pay such an amount in order to
avoid the risk of being uninsured.

Our challenge now is to explain why individuals who purchase home insurance
on terms that are less than actuarially `fair' in order to avoid risk are
simultaneously willing to invest their retirement savings into risky
companies. Companies, like homes, are risky; while they may not collapse or
implode in any given year, they can have good or bad returns in any given
year. Corporate returns are inherently unpredictable and therefore risky.
The key to understanding the willingness of risk-averse individuals to
invest in risky firms is to be found in the pooling of risks.