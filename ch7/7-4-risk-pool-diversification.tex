\section{Risk pooling and diversification}\label{sec:ch7sec4}

Firms carrying very high risk do succeed in attracting investment through
the \terminology{capital market} in the modern economy. In Silicon Valley
`angel' investors pour billions of dollars into startup companies that not
only have no cash flow, but may also have no product! These start-ups
frequently have no more than a good idea and are in need of financing in
order to turn the idea into a product, and then bring the product to market.

\begin{DefBox}
\textbf{Capital market}: a set of financial institutions that funnels financing from investors into bonds and stocks.
\end{DefBox}

How can we reconcile the fact that, while these firms carry extraordinary
uncertainty, investors are still willing to part with large sums of money to
fund development? And the investors are not only billionaires with a good
sense of the marketplace; private individuals who save for their retirement
also invest in these risky firms on the advice of their financial manager.
These firms might not be start-ups, but every firm that has publicly-traded
stocks is subject to major variations in its valuation, depending upon its
performance in any given time period.

The answer is that most individuals hold a \terminology{portfolio} of
investments, which is a combination of different stocks and bonds. By
investing in different stocks and bonds rather than concentrating in one
single investment or type of investment, an individual diversifies her
portfolio, which is to say she engages in \terminology{risk pooling}. A
rigourous theory underlies this ``don't put all of your eggs
in the one basket'' philosophy. The essentials of
diversification or pooling are illustrated in the example given in
Table~\ref{table:investmentstratriskyasset} below.\footnote{A different
form of risk management is defined by the idea of risk
spreading. Imagine that an oil supertanker has to be insured and the owner
approaches one insurer. In the event of the tanker being ship-wrecked the
damage caused by the resulting oil spill would be catastrophic -- both to
the environment and the insurance company. In this instance the insurance
company does not benefit from the law of large numbers -- it is not insuring
thousands of tankers and therefore would find it difficult to balance
the potential claims with the annual insurance premiums. As a consequence, 
insurers generally spread the potential cost among other insurers -- %
this is called risk spreading. The world's major insurers, such as
\textit{Lloyd's of London}, have hundreds of syndicates who each take on a small
proportion of a big risk. These syndicates may again choose to subdivide
their share among others, until the big risk becomes widely spread. In this
way it is possible to insure against almost any event or possibility, no
matter how large.}

\begin{DefBox}
\textbf{Portfolio}: a combination of assets that is designed to secure an income from investing and to reduce risk.

\textbf{Risk pooling}: Combining individual risks in such a way that the aggregate risk is reduced.
\end{DefBox}

There are two risky stocks here: Natural Gas (NG) and technology (Tech).
Each stock is priced at \$100, and over time it is observed that each yields
a \$10 return in good times and \$0 in bad times. The investor has \$200 to
invest, and each sector independently has a 50\% probability ($p=0.5$) of
good or bad times. This means that each stock should yield a \$5 return 
\textit{on average}. The challenge here is to develop an investment strategy
that minimizes the risk for the investor.

\newhtmlpage

At this point we need a specific working definition of \terminology{risk}.
We will define it in terms of how much variation a stock might experience in
its returns from year to year. Each of NG and Tech here have returns of
either \$0 or \$10, with equal probability. But what if the Tech returns
were either $+\$20$ or $-\$10$ with equal probability; or $+\$40$ or $-\$30$ with
equal probability? In each of these alternative scenarios the average
outcome remains the same: A positive average return of \$5. If the returns
profile to NG remains unchanged we would say that Tech is a riskier stock
(than NG) if its returns were defined by one of the alternatives here. Note
that the average return is unchanged, and we are defining risk in terms of
the greater spread in the possible returns around an unchanged average. The
key to minimizing risk in the investor's portfolio lies in exploring how the
variation in returns can be minimized by pooling risks.

\begin{DefBox}
\textbf{Risk}: A higher degree of risk is associated with increased variation in the possible returns around an unchanged mean return.
\end{DefBox}

\begin{Table}{caption={Investment strategies with risky assets \label{table:investmentstratriskyasset}}}
\begin{tabu} to \linewidth {|X[1,c]|X[1,c]X[1,c]X[1,c]|} \hline 
\cellcolor{rowcolour}\textbf{Strategy} & \multicolumn{3}{c|}{\cellcolor{rowcolour}\textbf{Expected returns with probabilities}} \\	\hline
\$200 in NG		& 220 ($p=0.5$)		&		& 200 ($p=0.5$) \\
\rowcolor{rowcolour}	\$200 in Tech	& 220 ($p=0.5$)		&		& 200 ($p=0.5$)\\
&	&	&	\\
\$100 in each	& 220 ($p=0.25$)	& 210 ($p=0.5$)	& 200 ($p=0.25$) \\ \hline 
\end{tabu}
\end{Table}

The outcomes from three different investment strategies are illustrated in
Table~\ref{table:investmentstratriskyasset}. By investing all of her \$200
in either NG or Tech, she will obtain \$220 half of the time and \$200 half
of the time, as indicated in the first two outcome rows. But by diversifying
through buying one of each stock, as illustrated in the final row, she
reduces the variability of her portfolio. To see why note that, since the
performance of each stock is independent, there is now only a one chance in
four that both stocks do well, and therefore there is a 25 percent
probability of earning \$220. By the same reasoning, there is a 25 percent
probability of earning \$200. But there is a 50 percent chance that one
stock will perform well and the other poorly. When that happens, she gets a
return of \$210. In contrast to the outcomes defined in rows 1 and 2, the 
\terminology{diversification} strategy in row 3 \textit{yields fewer extreme
potential outcomes and more potential outcomes that lie closer to the mean
outcome}.

\newhtmlpage

\begin{DefBox}
\textbf{Diversification} reduces the total risk of a portfolio by pooling risks across several different assets whose individual returns behave independently.
\end{DefBox}

Further diversification could reduce the variation in possible returns even
further. To see this, imagine that, rather than having a choice between
investing in one or two stocks, we could invest in four different stocks
with the same returns profile as the two given in the table above. In such a
case, the likelihood of getting extreme returns would be even lower than
when investing in two stocks. This is because, if the returns to each stock
are independent of the returns on the remaining stocks, it becomes
increasingly improbable that all, or almost all, of the stocks will
experience favorable (or unfavorable) returns in the same year. Now, imagine
that we had 8 stocks, or 16, or 32, or 64, etc. The ``magic''
of diversification is that the same average return
can be attained, yet variability can be reduced. If it can be reduced
sufficiently by adding ever more stocks to the portfolio, then even a highly
risk-averse individual can build a portfolio that is compatible with buying
into risky firms.

There is no longer a mystery surrounding why risk-averse individuals are
willing, at the same time, to pay a high home-insurance premium to avoid
risk, and simultaneously invest in ventures that have substantial variation
in the profile of the returns.

\newhtmlpage

% Application box 7.2
\begin{ApplicationBox}{caption={The value of a financial advisor \label{app:finadvisor}}}
	The modern economy has thousands of highly-trained financial advisors. The successful ones earn huge salaries. But there is a puzzle: Why do such advisors exist? Can they predict the behaviour of the market any better than an uninformed advisor? Two insights help us answer the question.
	
	First, Burton Malkiel wrote a best seller called \textit{A Random Walk down Wall Street}. He provided ample evidence that a portfolio chosen on the basis of a monkey throwing darts at a list of stocks would do just as well as the average portfolio constructed by your friendly financial advisor.
	
	Second, there are costs of transacting: An investor who builds a portfolio must devote time to the undertaking, and incur the associated financial trading cost. In recognizing this, investors may choose to invest in what they call mutual funds -- a diversified collection of stocks -- or may choose to employ a financial advisor who will essentially perform the same task of building a diversified portfolio. But, on average, financial advisors cannot beat the market, even though many individual investors would like to believe otherwise.
\end{ApplicationBox}

\newhtmlpage

At this point we may reasonably ask why individuals choose to invest any of
their funds in a ``safe'' asset -- perhaps
cash or Canadian Government bonds. After all, if their return to bonds is
lower on average than the return to stocks, and they can diversify away much
of the risk associated with stocks, why not get the higher average returns
associated with stocks and put little or nothing in the safer asset? The
reason is that it is impossible to fully diversify. When a recession hits,
for example, the \textit{whole stock market} may take a dive, because
profits fall across the whole economy. On account of this possibility, we
cannot ever arrive at a portfolio where the returns to the different stocks
are completely independent. As a consequence, the rational investor will
decide to put some funds in bonds in order to reduce this systematic risk
component that is associated with the whole market.

To see how the whole market can change dramatically students can go to any
publicly accessible financial data site -- such as \textit{Yahoo Finance}
and attempt to plot the TSX for the period 2005 -- present, or the NASDAQ
index from the mid-nineties to the present.

\section*{Conclusion}

We have now come full circle. We started this chapter by describing the key
role in economic development and growth played by firms and capital markets.
Capital markets channel the funds of individual investors to risk-taking
firms. Such firms---whether they are Dutch spice importers in the
seventeenth century, the Hudson's Bay Company in nineteenth-century Canada,
companies in Alberta's tar sands today or high tech start-ups in Silicon
Valley---are engines of growth and play a pivotal role in an economy's
development. Capital markets are what make it possible for these firms to
attract the savings of risk-averse individuals. By enabling individuals to
diversify their portfolios, capital markets form the link between
individuals and firms.

We next turn to examine decision making \textit{within} the firm. Firms must
make the right decisions if they are to grow and provide investors with a
satisfactory return. Firms that survive the growth process and ultimately
bring a product to market are the survivors of the uncertainty surrounding
product development.